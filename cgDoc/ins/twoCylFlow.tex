\subsection{Flow past two cylinders}\label{sec:twoCylFlow}

Figure~\ref{fig:twoCylFlow} shows the simulation of a high Reynolds number flow past two
cylinders.
The simulation used the command file {\tt cg/ins/cmd/tcilc.cmd}
and the grid was made with the ogen script {\tt Overture/sampleGrids/tcilc.cmd}. 
The radius of the cylinders was  $1/2$. The grid had about 22 million grid points. 
% 
{
\newcommand{\figWidthd}{6cm}
\newcommand{\trimfig}[2]{\trimPlot{#1}{#2}{.0}{.0}{.0}{.0}}
\begin{figure}[hbt]
\begin{center}
\begin{tikzpicture}[scale=1]
  \useasboundingbox (0,.75) rectangle (13.,19.5);  % set the bounding box (so we have less surrounding white space)
%
  \draw ( 0.0,13.) node[anchor=south west,xshift=-4pt,yshift=+0pt] {\trimfig{\cgDoc/ins/fig/tcilc64Vort1p0}{\figWidthd}};
  \draw ( 0.0,6.5) node[anchor=south west,xshift=-4pt,yshift=+0pt] {\trimfig{\cgDoc/ins/fig/tcilc64Vort1p5}{\figWidthd}};
  \draw ( 0.0,  0) node[anchor=south west,xshift=-4pt,yshift=+0pt] {\trimfig{\cgDoc/ins/fig/tcilc64Vort2p0}{\figWidthd}};
  \draw ( 6.5,13.) node[anchor=south west,xshift=-4pt,yshift=+0pt] {\trimfig{\cgDoc/ins/fig/tcilc64Vort2p5}{\figWidthd}};
  \draw ( 6.5,6.5) node[anchor=south west,xshift=-4pt,yshift=+0pt] {\trimfig{\cgDoc/ins/fig/tcilc64Vort3p0}{\figWidthd}};
  \draw ( 6.5, .0) node[anchor=south west,xshift=-4pt,yshift=+0pt] {\trimfig{\cgDoc/ins/fig/tcilc64Vort3p5}{\figWidthd}};
%
 % \draw (current bounding box.south west) rectangle (current bounding box.north east);
% grid:
% \draw[step=1cm,gray] (0,0) grid (13,19);
\end{tikzpicture}
\end{center}
\caption{Flow past two cylinders in a channel computed with Cgins. Contour plots of the vorticity.}\label{fig:twoCylFlow}
\end{figure}
}


%- 
%- {
%- \begin{figure}[H]
%- \psset{xunit=1.cm,yunit=1.cm,runit=1.cm}%
%- \newcommand{\figWidthd}{7.5cm}% 
%- \newcommand{\clipfigd}[2]{\clipFig{#1}{#2}{.0}{1.}{0.15}{1.}}
%- \begin{center}%
%- \begin{pspicture}(0,0)(17.,18.75)%
%- % \psgrid[subgriddiv=2]
%-  \rput(4.00,15.0){\clipfigd{\ovFigures/tcilc64Vort1p0.ps}{\figWidthd}}
%-  \rput(4.00, 8.5){\clipfigd{\ovFigures/tcilc64Vort1p5.ps}{\figWidthd}}
%-  \rput(4.00, 2.0){\clipfigd{\ovFigures/tcilc64Vort2p0.ps}{\figWidthd}}
%-  \rput(12.2,15.0){\clipfigd{\ovFigures/tcilc64Vort2p5.ps}{\figWidthd}}
%-  \rput(12.2, 8.5){\clipfigd{\ovFigures/tcilc64Vort3p0.ps}{\figWidthd}}
%-  \rput(12.2, 2.0){\clipfigd{\ovFigures/tcilc64Vort3p5.ps}{\figWidthd}}
%- \end{pspicture}%
%- \end{center}%
%- \caption{Flow past two cylinders in a channel computed with Cgins. Contour plots of the vorticity.}\label{fig:twoCylFlow}
%- \end{figure}
%- }
% -----------------------------------------------------------------------------------
