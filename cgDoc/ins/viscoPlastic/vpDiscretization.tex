\newcommand{\Es}{{\mathcal E}}
\newcommand{\ac}{k}
\newcommand{\Lrr}{{\mathcal L}_{rr}}
\newcommand{\Lrs}{{\mathcal L}_{rs}}
\newcommand{\dr}{{\Delta r}}
\newcommand{\ds}{{\Delta s}}
\newcommand{\hr}{h_r}
\newcommand{\hs}{h_s}
\newcommand{\Rand}{{\mathcal R}}
\newcommand{\SDm}{SD$m$}
\newcommand{\FDm}{FD$m$}
\newcommand{\TnSDm}[2]{T$#1$SD$#2$}
\newcommand{\TnFDm}[2]{T$#1$FD$#2$}
\newcommand{\Vlambda}{V_\lambda}

\clearpage
\subsection{Notes on the discretization used for the visco-plastic equations.}

% \sum_{j=1}^d

The divergence of a function $\fv$ can be written in in {\em conservation} form, or {\em self-adjoint} form, 
in general curvilinear coordinates as
\begin{align}
  \grad\cdot\fv &= \frac{1}{J} {\partial \over\partial r_m} \left( J~\grad_\xv r_m\cdot\fv \right) 
                 \qquad\text{(implied sum)}
     \label{eq:divCons} \\
   {\partial\over\partial x_j} f_j &=  {\partial \over\partial r_m} 
           \left(J~  {\partial r_m\over\partial x_j} f_j \right)
\end{align}
where $J$ denotes the determinant of the Jacobian matrix $[\partial x_i/\partial r_j]$.

Consider the generalized Laplace operator, $L$ defined by 
\begin{align}
  L w &= \grad\cdot( \eta (\grad w + \grad w^T) ) ~, \\
  L w_\mu &= {\partial \over \partial x_j}\left( \eta( \partial_j w_\mu + \partial_\mu w_j ) \right)
\end{align}
% 
% \sum_{m=1}^d\sum_{j=1}^d \sum_{m=1}^d\sum_{j=1}^d\sum_{k=1}^d 
The operator $L$ can be written in {\em conservation} form, or {\em self-adjoint} form, 
in general curvilinear coordinates as
\begin{align}
  L w_\mu &= \frac{1}{J}  {\partial \over\partial r_m}\Big(
                J~  {\partial r_m\over\partial x_j} ~\eta~[ \partial_j w_\mu + \partial_\mu w_j ]
                \Big)  
\end{align}
From the chain rule
\begin{align}
  {\partial w_\mu \over \partial x_j} &= {\partial r_k \over\partial x_j} {\partial w_\mu \over\partial r_k}
\end{align}
and thus
\begin{align}
  L w_\mu &= \frac{1}{J} {\partial \over\partial r_m}\Big(
                J~\eta~  {\partial r_m\over\partial x_j}
          \left[  {\partial r_k \over\partial x_j} {\partial w_\mu \over\partial r_k}
                    + {\partial r_k \over\partial x_\mu} {\partial w_j \over\partial r_k} \right]
                \Big)  \label{eq:generalizedLaplacianCons}
\end{align}
% 
Expression~\eqref{eq:generalizedLaplacianCons}
contains two types of second-order deriavtive terms, those with un-mixed derivatives
and those with mixed derivatives,
\begin{align}
   L_a w_\mu &= {\partial\over\partial r_m}\left( A {\partial w_\mu \over\partial r_m} \right), \qquad\text{(no implied sum)}, \\
   L_b w_\mu &= {\partial\over\partial r_m}\left( B {\partial w_\mu \over\partial r_n} \right), \qquad\text{$m\ne n$}.
\end{align}
% 
These terms are approximated to second-order as
\begin{align}
   L_a w_\mu &\approx D_{+m}\left( A_{i_m-\half} D_{-m} W^\mu_{\iv}\right), \label{eq:unMixedTerm} \\
   L_b w_\mu &\approx D_{0m}\left( B_\iv D_{0n} W^\mu_\iv\right) ,         \label{eq:mixedTerm}
\end{align}
with 
\begin{align}
   A_{i_m-\half} &= \half\left( A_{i_m} + A_{i_m-1}  \right) ~. \label{eq:ave}
\end{align}
Discretizing all the terms in~\eqref{eq:generalizedLaplacianCons} in this way leads to a compact self-adjoint
discrete approximation that is three points wide in each direction.

Note, however, that $\eta$ depends on the derivatives of $u$.
The approximations~\eqref{eq:unMixedTerm} and \eqref{eq:mixedTerm} will thus depend on 
a stencil of values for $W_\iv$ that is five points wide.




% \begin{align}
%  L w_i &=      &= 
%    \sum_{m=1}^d\sum_{n=1}^d\sum_{j=1}^d 
%             {\partial\over\partial r_m}\left( A_j^{mn} {\partial w_j \over\partial r_n} \right)
%            , \label{eq:LSymdef} \quad
%  A_j^{mn} = J \sum_{\mu=1}^d\sum_{\nu=1}^d\sum_{j=1}^d  ????
%             {\partial r_m \over\partial x_\mu}{\partial r_n \over\partial x_\nu} 
% \end{align}


