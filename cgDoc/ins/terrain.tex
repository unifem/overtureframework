% ========================================================================================================
\newcommand{\Gt}{\Gc_{t}}
\subsection{Flow over terrain}\label{sec:terrain}


To run this example see {\tt cg/ins/runs/terrain/Readme}.


We consider the flow over a terrain. The terrain elevation data is 
obtained from the USGS web site ...

The matlab script {\tt convertGIS.m} is used to convert the GIS data
into Cartesian coordinates (from lat-long). This matlab script
can also be used to optionally smooth the terrain, choose a sub-region,
or choose a two-dimensional cross section. 

The ogen script {\tt terrainGrid.cmd} takes the output from {\tt convertGIS.m}
and constructs an overlapping grid for the terrain.

Figure~\ref{fig:terrainGridAndFlow} shows a 2D and 3D grid and some preliminary
solutions.




{%%%
\newcommand{\figWidthc}{8.0cm}
\newcommand{\trimfigc}[2]{\trimPlotb{#1}{#2}{.0}{.0}{.2}{.2}}
\newcommand{\figWidtha}{7.5cm}
\newcommand{\trimfiga}[2]{\trimPlotb{#1}{#2}{.0}{.0}{.25}{.1}}
% 
\newcommand{\figWidthb}{7.5cm}
\newcommand{\trimfigb}[2]{\trimPlotb{#1}{#2}{.0}{.0}{.05}{.1}}
% 
% \newcommand{\figWidthb}{5.5cm}
% \newcommand{\trimfigb}[2]{\trimPlotb{#1}{#2}{.05}{.05}{.05}{.1}}
% \newcommand{\figWidthc}{6cm}
% \newcommand{\trimfigc}[2]{\trimPlotb{#1}{#2}{.025}{.025}{.095}{.095}}
% % -----------------------------------------------------------------------------------------------------------------------------------------
\begin{figure}[hbt]
\begin{center}
\begin{tikzpicture}[scale=1]
  \useasboundingbox (0,.75) rectangle (16.,12);  % set the bounding box (so we have less surrounding white space)
%
\draw (0.0,6.6)  node[anchor=south west,xshift=-4pt,yshift=+0pt] {\trimfigc{\cgDoc/ins/fig/site300Grid2de2}{\figWidthc}};
\draw (8.2,6.5)  node[anchor=south west,xshift=-4pt,yshift=+0pt] {\trimfiga{\cgDoc/ins/fig/site3002d32SL}{\figWidtha}};
%
\draw (0.0,0.0)  node[anchor=south west,xshift=-4pt,yshift=+0pt] {\trimfigb{\cgDoc/ins/fig/site300Gride2}{\figWidthb}};
\draw (8.0,0.0)  node[anchor=south west,xshift=-4pt,yshift=+0pt] {\trimfigb{\cgDoc/ins/fig/site300f2u}{\figWidthb}};
%

% \draw (current bounding box.south west) rectangle (current bounding box.north east);
% grid:
% \draw[step=1cm,gray] (0,0) grid (16,12.);
\end{tikzpicture}
\end{center}
 \caption{Flow over terrain.}
  \label{fig:terrainGridAndFlow}
\end{figure}
% -----------------------------------------------------------------------------------------------------------------------------------------------
%
}%

Figure~\ref{fig:terrainFlow} shows some results from the fourth-order accurate scheme AFS24 on grid $\Gt^{(4)}$ (10M pts).
The size the of the domain is approximately $1.5 km \times 1.5 km$ in the horizontal. 
The grid spacing near the surface is approximately $5m$ in the 
horizontal and $.5m$ for the grid cell next to the surface.

{%%%
\newcommand{\figWidthb}{7.5cm}
\newcommand{\trimfigb}[2]{\trimPlotb{#1}{#2}{.0}{.0}{.05}{.1}}
% 
% % -----------------------------------------------------------------------------------------------------------------------------------------
\begin{figure}[hbt]
\begin{center}
\begin{tikzpicture}[scale=1]
  \useasboundingbox (0,.75) rectangle (16.,6);  % set the bounding box (so we have less surrounding white space)
%
\draw (0.0,0.0)  node[anchor=south west,xshift=-4pt,yshift=+0pt] {\trimfigb{\cgDoc/ins/fig/site300g4o4velt100}{\figWidthb}};
\draw (8.0,0.0)  node[anchor=south west,xshift=-4pt,yshift=+0pt] {\trimfigb{\cgDoc/ins/fig/site300g4o4vort100}{\figWidthb}};
%

% \draw (current bounding box.south west) rectangle (current bounding box.north east);
% grid:
% \draw[step=1cm,gray] (0,0) grid (16,6.);
\end{tikzpicture}
\end{center}
 \caption{Flow over site300 terrain using grid $\Gt^{(4)}$ (10M pts) and scheme AFS24. Left: speed, right: enstrophy.
     The surface grid is coarsened by a factor of 2 for plotting purposes.}
  \label{fig:terrainFlow}
\end{figure}
% -----------------------------------------------------------------------------------------------------------------------------------------------
%
}%