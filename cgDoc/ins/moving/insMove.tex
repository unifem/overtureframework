%-----------------------------------------------------------------------
% Cgins : Moving Grid Studies
% 
%-----------------------------------------------------------------------
\documentclass{article}
\usepackage[bookmarks=true]{hyperref}

% \input documentationPageSize.tex
\hbadness=10000 
\sloppy \hfuzz=30pt

% \voffset=-.25truein
% \hoffset=-1.25truein
% \setlength{\textwidth}{7in}      % page width
% \setlength{\textheight}{9.5in}    % page height

\usepackage{calc}
% set the page width and height for the paper (The covers will have their own size)
\setlength{\textwidth}{7in}  
\setlength{\textheight}{9.5in} 
% here we automatically compute the offsets in order to centre the page
\setlength{\oddsidemargin}{(\paperwidth-\textwidth)/2 - 1in}
% \setlength{\topmargin}{(\paperheight-\textheight -\headheight-\headsep-\footskip)/2 - 1in + .8in }
\setlength{\topmargin}{(\paperheight-\textheight -\headheight-\headsep-\footskip)/2 - 1in -.2in }

\input homeHenshaw

% --------------------------------------------
% NOTE: trouble with tikz and program package ??
\usepackage{tikz}

\usepackage{amsmath}
\usepackage{amssymb}

\usepackage{verbatim}
% \usepackage{moreverb}

\usepackage{graphics}    
% \usepackage{epsfig}    
\usepackage{calc}
% \usepackage{ifthen}
% \usepackage{float}
% the next one cause the table of contents to disappear!
% * \usepackage{fancybox}

% *new* version of ``clipFig''
\input trimFig.tex

\newcommand{\bogus}[1]{}  % begin a section that will not be printed

% \usepackage{makeidx} % index
% \makeindex
% \newcommand{\Index}[1]{#1\index{#1}}


% ---- we have lemmas and theorems in this paper ----
\newtheorem{assumption}{Assumption}
\newtheorem{definition}{Definition}

\newcommand{\ovFigures}{\homeHenshaw/OvertureFigures}
\newcommand{\docFigures}{\homeHenshaw/OvertureFigures}
\newcommand{\figures}{\homeHenshaw/res/OverBlown/docFigures}
\newcommand{\obFigures}{\homeHenshaw/res/OverBlown/docFigures}  % note: local version for OverBlown

\newcommand{\Overture}{{\bf Overture\ }}

% -------------  -------------------
\newcommand{\Solver}{Cgins}
\newcommand{\solver}{cgins}

% ***************************************************************************
\begin{document}


% -----definitions-----
\input wdhDefinitions.tex

\newcommand{\tauv}{\boldsymbol{\tau}}
\newcommand{\sumi}{\sum_{i=1}^n}
\newcommand{\dt}{{\Delta t}}

\newcommand{\Cc}{{\mathcal C}}
\newcommand{\Rc}{{\mathcal R}}

\vglue 5\baselineskip
\begin{flushleft}
{\Large
Cgins : Moving Grid Studies \\
}
\vspace{2\baselineskip}
William D. Henshaw\footnote{This work was performed under the auspices of the U.S. Department of Energy (DOE) by
Lawrence Livermore National Laboratory under Contract DE-AC52-07NA27344 and by 
DOE contracts from the ASCR Applied Math Program.}  \\
Centre for Applied Scientific Computing  \\
Lawrence Livermore National Laboratory      \\
Livermore, CA, 94551.  \\
% henshaw@llnl.gov \\
% http://www.llnl.gov/casc/people/henshaw \\
% http://www.llnl.gov/casc/Overture\\
\vspace{\baselineskip}
\today\\
\vspace{\baselineskip}
LLNL-SM-455851

\vspace{4\baselineskip}

\noindent{\bf Abstract:}
This document presents Cgins results for moving and falling bodies.


\end{flushleft}

\clearpage
\tableofcontents
% \listoffigures

\clearpage
\input dropCyl

\clearpage
\input dropSphere

\clearpage
\input collide

%=================================================================================================
%\vfill\eject
%\bibliography{\homeHenshaw/papers/henshaw}
%\bibliographystyle{siam}

%\printindex

\end{document}


% ----------------------------------------------------------------------------------------------------------



