%-----------------------------------------------------------------------
% TEMPORARY: Cgins turbulence models
%-----------------------------------------------------------------------
\documentclass[11pt]{article}
\usepackage[bookmarks=true]{hyperref}  % this changes the page location !

% \input documentationPageSize.tex
\hbadness=10000 
\sloppy \hfuzz=30pt

% \voffset=-.25truein
% \hoffset=-1.25truein
% \setlength{\textwidth}{7in}      % page width
% \setlength{\textheight}{9.5in}    % page height

\usepackage{calc}
% set the page width and height for the paper (The covers will have their own size)
\setlength{\textwidth}{7in}  
\setlength{\textheight}{9.5in} 
% here we automatically compute the offsets in order to centre the page
\setlength{\oddsidemargin}{(\paperwidth-\textwidth)/2 - 1in}
% \setlength{\topmargin}{(\paperheight-\textheight -\headheight-\headsep-\footskip)/2 - 1in + .8in }
\setlength{\topmargin}{(\paperheight-\textheight -\headheight-\headsep-\footskip)/2 - 1in -.2in }

\input homeHenshaw

\input{pstricks}\input{pst-node}
\input{colours}

\usepackage{amsmath}
\usepackage{amssymb}

\usepackage{verbatim}
\usepackage{moreverb}

\usepackage{graphics}    
\usepackage{epsfig}    
\usepackage{calc}
\usepackage{ifthen}
\usepackage{float}
% the next one cause the table of contents to disappear!
% * \usepackage{fancybox}

\usepackage{makeidx} % index
\makeindex
\newcommand{\Index}[1]{#1\index{#1}}



% ---- we have lemmas and theorems in this paper ----
\newtheorem{assumption}{Assumption}
\newtheorem{definition}{Definition}

% \newcommand{\homeHenshaw}{/home/henshaw.0}

\newcommand{\Overture}{{\bf Over\-ture\ }}
\newcommand{\ogenDir}{\homeHenshaw/Overture/ogen}

\newcommand{\cgDoc}{\homeHenshaw/cgDoc}
\newcommand{\vpDir}{\homeHenshaw/cgDoc/ins/viscoPlastic}

\newcommand{\ovFigures}{\homeHenshaw/OvertureFigures}
\newcommand{\obFigures}{\homeHenshaw/res/OverBlown/docFigures}  % for figures
\newcommand{\convDir}{\homeHenshaw/cgDoc/ins/tables}

\begin{document}

\input wdhDefinitions.tex

\def\comma  {~~~,~~}
\newcommand{\uvd}{\mathbf{U}}
\def\ud     {{    U}}
\def\pd     {{    P}}
\def\calo{{\cal O}}

\newcommand{\mbar}{\bar{m}}
\newcommand{\Rbar}{\bar{R}}
\newcommand{\Ru}{R_u}         % universal gas constant
% \newcommand{\Iv}{{\bf I}}
% \newcommand{\qv}{{\bf q}}
\newcommand{\Div}{\grad\cdot}
\newcommand{\tauv}{\boldsymbol{\tau}}
\newcommand{\thetav}{\boldsymbol{\theta}}
% \newcommand{\omegav}{\mathbf{\omega}}
% \newcommand{\Omegav}{\mathbf{\Omega}}

\newcommand{\Omegav}{\boldsymbol{\Omega}}
\newcommand{\omegav}{\boldsymbol{\omega}}
\newcommand{\sigmav}{\boldsymbol{\sigma}}
\newcommand{\cm}{{\rm cm}}
\newcommand{\Jc}{{\mathcal J}}

\newcommand{\sumi}{\sum_{i=1}^n}
% \newcommand{\half}{{1\over2}}
\newcommand{\dt}{{\Delta t}}

\def\ff {\tt} % font for fortran variables

% define the clipFig commands:
\input clipFig.tex

\newcommand{\bogus}[1]{}  % removes is argument completely

\vspace{5\baselineskip}
\begin{flushleft}
{\Large
{\bf Cgins} Turbulence Models (This is a Temporary document) \\
}
\vspace{2\baselineskip}
William D. Henshaw\footnote{This work was performed under the auspices of the U.S. Department of Energy (DOE) by
Lawrence Livermore National Laboratory under Contract DE-AC52-07NA27344 and by 
DOE contracts from the ASCR Applied Math Program.}  \\
Centre for Applied Scientific Computing  \\
Lawrence Livermore National Laboratory      \\
Livermore, CA, 94551.  \\
\vspace{\baselineskip}
\today\\
\vspace{\baselineskip}
LLNL-SM-455871

\vspace{4\baselineskip}

\noindent{\bf\large Abstract:}

We describe turbulence models for Cgins.

\end{flushleft}

\clearpage
\tableofcontents
% \listoffigures

\vfill\eject


% ------------------------- TURBULENCE MODELS ----------------------------------
\clearpage
\input \cgDoc/ins/turbulenceModels/insTurbulenceModels.tex
% ------------------------------------------------------------------------------


% -------------------------------------------------------------------------------------------------
\vfill\eject
\bibliography{\homeHenshaw/papers/henshaw}
\bibliographystyle{siam}


\printindex


\end{document}
