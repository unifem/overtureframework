\subsection{Pitching and plunging airfoil}\label{sec:pitchingAirfoil}

This example shows the simulation of a pitching and plunging airfoil,
% -----------------------------------------------------------------------------------
The simulation used the command file {\tt cg/ins/cmd/wing2d.cmd}
and the grid was made with the ogen script {\tt Overture/sampleGrids/joukowsky2d.cmd}. 
The computation was performed in parallel and demonstrates the parallel moving grid
capabilities (starting with v24). The grid had $1.7M$ points.

% 
{
\newcommand{\figWidthd}{7.5cm}
\newcommand{\trimfig}[2]{\trimPlot{#1}{#2}{.0}{.0}{.13}{.11}}
\begin{figure}[hbt]
\begin{center}
\begin{tikzpicture}[scale=1]
  \useasboundingbox (0,.75) rectangle (16.,19.5);  % set the bounding box (so we have less surrounding white space)
%
  \draw ( 0.0,13.) node[anchor=south west,xshift=-4pt,yshift=+0pt] {\trimfig{\cgDoc/ins/fig/joukowskyPitchPlunge200}{\figWidthd}};
  \draw ( 0.0,6.5) node[anchor=south west,xshift=-4pt,yshift=+0pt] {\trimfig{\cgDoc/ins/fig/joukowskyPitchPlunge225}{\figWidthd}};
  \draw ( 0.0,  0) node[anchor=south west,xshift=-4pt,yshift=+0pt] {\trimfig{\cgDoc/ins/fig/joukowskyPitchPlunge250}{\figWidthd}};
  \draw ( 8.0,13.) node[anchor=south west,xshift=-4pt,yshift=+0pt] {\trimfig{\cgDoc/ins/fig/joukowskyPitchPlunge275}{\figWidthd}};
  \draw ( 8.0,6.5) node[anchor=south west,xshift=-4pt,yshift=+0pt] {\trimfig{\cgDoc/ins/fig/joukowskyPitchPlunge300}{\figWidthd}};
  \draw ( 8.0, .0) node[anchor=south west,xshift=-4pt,yshift=+0pt] {\trimfig{\cgDoc/ins/fig/joukowskyPitchPlunge325}{\figWidthd}};
%
 % \draw (current bounding box.south west) rectangle (current bounding box.north east);
% grid:
%  \draw[step=1cm,gray] (0,0) grid (16,19);
\end{tikzpicture}
\end{center}
 \caption{A pitching and plunging airfoil, computed in parallel with Cgins. Contour plots of the vorticity.}
\end{figure}
}



%- {
%- \begin{figure}[H]
%- \psset{xunit=1.cm,yunit=1.cm,runit=1.cm}%
%- \newcommand{\figWidthd}{8cm}% 
%- \newcommand{\clipfigd}[2]{\clipFig{#1}{#2}{.0}{1.}{0.17}{1.}}
%- \begin{center}%
%- \begin{pspicture}(0,0)(17.,18)%
%-  %\psgrid[subgriddiv=2]
%-  \rput(4.00,14.0){\clipfigd{\ovFigures/joukowskyPitchPlunge200.ps}{\figWidthd}}
%-  \rput(4.00, 8.0){\clipfigd{\ovFigures/joukowskyPitchPlunge225.ps}{\figWidthd}}
%-  \rput(4.00, 2.0){\clipfigd{\ovFigures/joukowskyPitchPlunge250.ps}{\figWidthd}}
%-  \rput(12.2,14.0){\clipfigd{\ovFigures/joukowskyPitchPlunge275.ps}{\figWidthd}}
%-  \rput(12.2, 8.0){\clipfigd{\ovFigures/joukowskyPitchPlunge300.ps}{\figWidthd}}
%-  \rput(12.2, 2.0){\clipfigd{\ovFigures/joukowskyPitchPlunge325.ps}{\figWidthd}}
%- \end{pspicture}%
%- \end{center}%
%- \caption{A pitching and plunging airfoil, computed in parallel with Cgins. Contour plots of the vorticity.}
%- \end{figure}
%- }


