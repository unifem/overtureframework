%-----------------------------------------------------------------------
% Solving multi-domain problems
%-----------------------------------------------------------------------
\documentclass[11pt]{article}
% \usepackage{times}  % for embeddable fonts, Also use: dvips -P pdf -G0

\input documentationPageSize.tex

\input homeHenshaw

\input{pstricks}\input{pst-node}
\input{colours}

\usepackage{amsmath}
\usepackage{amssymb}

\usepackage{verbatim}
\usepackage{moreverb}

\usepackage{graphics}    
\usepackage{epsfig}    
\usepackage{calc}
\usepackage{ifthen}
\usepackage{float}
% the next one cause the table of contents to disappear!
% * \usepackage{fancybox}

\usepackage{makeidx} % index
\makeindex
\newcommand{\Index}[1]{#1\index{#1}}



% ---- we have lemmas and theorems in this paper ----
\newtheorem{assumption}{Assumption}
\newtheorem{definition}{Definition}

% \newcommand{\homeHenshaw}{/home/henshaw.0}

\newcommand{\Overture}{{\bf Over\-ture\ }}
\newcommand{\ogenDir}{\homeHenshaw/Overture/ogen}

\newcommand{\cgDoc}{\homeHenshaw/cgDoc}
\newcommand{\vpDir}{\homeHenshaw/cgDoc/ins/viscoPlastic}

\newcommand{\obFigures}{\homeHenshaw/res/OverBlown/docFigures}  % for figures
\newcommand{\convDir}{.}

\begin{document}

\input wdhDefinitions.tex

\def\comma  {~~~,~~}
\newcommand{\uvd}{\mathbf{U}}
\def\ud     {{    U}}
\def\pd     {{    P}}
\def\calo{{\cal O}}

\newcommand{\mbar}{\bar{m}}
\newcommand{\Rbar}{\bar{R}}
\newcommand{\Ru}{R_u}         % universal gas constant
% \newcommand{\Iv}{{\bf I}}
% \newcommand{\qv}{{\bf q}}
\newcommand{\Div}{\grad\cdot}
\newcommand{\tauv}{\boldsymbol{\tau}}
\newcommand{\thetav}{\boldsymbol{\theta}}
% \newcommand{\omegav}{\mathbf{\omega}}
% \newcommand{\Omegav}{\mathbf{\Omega}}

\newcommand{\Omegav}{\boldsymbol{\Omega}}
\newcommand{\omegav}{\boldsymbol{\omega}}
\newcommand{\sigmav}{\boldsymbol{\sigma}}
\newcommand{\cm}{{\rm cm}}
\newcommand{\Jc}{{\mathcal J}}

\newcommand{\sumi}{\sum_{i=1}^n}
% \newcommand{\half}{{1\over2}}
\newcommand{\dt}{{\Delta t}}

\def\ff {\tt} % font for fortran variables

% define the clipFig commands:
\input clipFig.tex

\newcommand{\bogus}[1]{}  % removes is argument completely

\vspace{5\baselineskip}
\begin{flushleft}
{\Large
Notes on Solving Multi-Domain Problems \\
}
\vspace{2\baselineskip}
William D. Henshaw  \\
Centre for Applied Scientific Computing  \\
Lawrence Livermore National Laboratory      \\
Livermore, CA, 94551.  \\
henshaw@llnl.gov \\
% http://www.llnl.gov/casc/people/henshaw \\
http://www.llnl.gov/casc/Overture\\
\vspace{\baselineskip}
\today\\
\vspace{\baselineskip}
% UCRL-MA-134289

\vspace{4\baselineskip}

\noindent{\bf\large Abstract:}

This document holds notes and results on solving multi-domain problems.
We consider iteration strategies for solving the interface equations without
forming the full coupled implicit systems. 

\end{flushleft}

% \clearpage
\tableofcontents
% \listoffigures

\vfill\eject


\section{Solving Multi-Domain Problems by Iteration.}

% \input /home/henshaw.0/papers/th/materialInterfaceFigure

We are interested in solving multi-physics, multi-domain problems.
% 
One simple example is the solution to a problem of heat condition between
solids with different thermal conductivities. Another example is coupling
fluid flow with heat condition in adjacent solids.




\newcommand{\kappam}{\kappa^{(m)}}
\newcommand{\um}{u^{(m)}}
\newcommand{\uI}{u^{(1)}}
\newcommand{\uII}{u^{(2)}}
\newcommand{\kappaI}{\kappa^{(1)}}
\newcommand{\kappaII}{\kappa^{(2)}}
\newcommand{\jump}[1]{\Big[#1\Big]}
\newcommand{\up}{u^p}
\newcommand{\vp}{v^p}
\newcommand{\uh}{\hat{u}}
\newcommand{\vh}{\hat{v}}

\subsection{One-dimensional Interface Problem}

We begin by considering the solution to the one-dimensional two-domain problem for Poisson's equation,
on the domain $\Omega = [-a,b] = \Omega_1 \bigcup \Omega_2 $,  with $\Omega_1=[-a,0]$, $\Omega_2=[0,b]$,
and with interface $\Omega_I$ at $x=0$,
\begin{alignat}{3}
  \partial_x( \kappam \partial_x \um ) &= f &&\qquad m=1,2, ~~\text{for $x\in \Omega_m$}, \\
  \jump{ \um(0) } &= 0  &&\qquad \text{for $x\in \Omega_I$}, \\
  \jump{ \kappam \partial_x \um(0) } &= 0 &&\qquad \text{for $x\in \Omega_I$},\\
  \uI(-a)=g(a), \quad \uII(b) &=g(b)  ~.
\end{alignat}
Note that $\kappam>0$, $a>0$ and $b>0$. We also assume that $\kappam$ is constant. 


Suppose that we have a good solution method for each sub-domain and that we do not want to
solve the full coupled equations as a single system.
We can define an iteration to solve the coupled equations. Let $u^j \approx\uI$ and $v^j\approx \uII$.
Define the iteration
\begin{alignat}{3}
  \kappaI u_{xx}^j  &= f &  \kappaII v_{xx}^j &= f \\
    u^j(-a)&=g(a)      &  v^j(b)&=g(b) \\
  \kappaI u^j_x(0) &= \kappaII v^{j-1}_x(0)  \qquad  & v^j(0) &= u^{j}(0) 
\end{alignat}
where we have applied the {\em Neumann BC} on $u^j$ and the {\em Dirichlet} condition on $v^j$. 

{\bf Note:} Another option would be to use a mixed interface condition
\begin{align*}
  a_{11} \kappaI u^j_x(0) + a_{12} u^{j}(0) &= a_{11} \kappaII v^{j-1}_x(0) + a_{12} v^j(0), \\
  a_{21} \kappaII v^j_x(0) + a_{22} v^{j}(0) &= a_{21} \kappaI u^{j-1}_x(0) + a_{22} u^j(0), 
\end{align*}
where we require the coefficients $a_{ij}$ to form a non-singular matrix, $det(a_{ij})\ne 0$. 
Upon convergence we will have satisfied both jump conditions. {\bf This option still needs to be analysed.}
% 


We can analyze the convergence properties of this iteration. We first construct particular solutions, $u^p(x)$ and
$v^p(x)$ that satisfy the equations and boundary conditions, and satisfy some Dirichlet conditions
at the interface, 
\begin{alignat}{3}
  \kappaI \up_{xx}  &= f \qquad&  \kappaII \vp_{xx} &= f ,\\
    \up(-a)&=g(-a)      \qquad&  \vp(b)&=g(b) ,\\
   \up(0) &= 0 \qquad& \vp(0) &= 0 ~.
\end{alignat}
Letting $\uh^j=u^j-\up$ and $\vh^j=u^j-\vp$ then $(\uh^j,\vh^j)$ satisfy 
\begin{alignat}{3}
  \kappaI \uh_{xx}^j  &= 0 \qquad&  \kappaII \vh_{xx}^j &= 0 \\
    \uh^j(-a)&=0      \qquad&  \vh^j(b)&=0 \\
  \kappaI \uh^j_x(0) &= \kappaII \vh^{j-1}_x(0) - \jump{\kappa_m\up_x(0)}   \qquad& \vh^j(0) &= \uh^{j}(0) 
\end{alignat}
The solutions to the interior equations and boundary conditions 
are of the form 
\[ 
\uh^j = A_j (x+a) ~~,~~~~\text{and}~~ \vh^j = B_j(x-b),
\]
If we choose initial guesses as $u^0=\up$ and $v^0=\vp$ then $A_0=0$ and $B_0$=0. 
Substitution into the interface conditions gives 
\begin{align*}
   \kappaI A_j &= \kappaII B_{j-1} - \jump{\kappa_m\up_x(0)} \\
    B_j~(0-b) &= A_j~(0+a)
\end{align*}
and thus
\begin{align*}
   A_j &= - {a\over\kappaI}{\kappaII\over b} A_{j-1} - {1\over \kappaI} \jump{\kappa_m\up_x(0)} . 
\end{align*}
This iteration converges provided
\begin{equation}
   \left\vert  {a\over\kappaI}{\kappaII\over b} \right\vert <1 
\end{equation}
In general this is not a good situation since the iteration does not converge for general
values of $\kappaI/a$ and $\kappaII/b$. It will work fine if $\kappaI/a \gg \kappaII/b$. 
This result indicates that we should apply the {\em Neumann} interface BC on $\uI$ and the {\em Dirichlet} on $\uII$
if $\kappaI/a >\kappaII/b$,
but that we should apply the {\em Neumann} interface BC on $\uII$ and {\em Dirichlet} on $\uI$, if $\kappaI/a < \kappaII/b$. 

We can define an under-relaxed iteration with relaxation parameter $\omega$ by 
\begin{align*}
    A_j &= (1-\omega)A_{j-1} + \omega {1\over \kappaI}\Big(\kappaII B_{j-1} - \jump{\kappa_m\up_x(0)}\Big) \\
    B_j &= -{ a\over b} A_j
\end{align*}
in which case
\begin{align*}
   A_j &= \Big[ 1 -\omega\big(1+{a\over\kappaI}{\kappaII\over b}\big) \Big] A_{j-1} -
                         \omega{ {1\over \kappaI} \jump{\kappa_m\up_x(0)} } . 
\end{align*}
This iteration will converge provided $0< \omega < 2\omega_{\rm opt}$ where
the optimal value for $\omega$ is
\begin{align*}
   \omega_{\rm opt} &= {1 \over 1+{a\over\kappaI}{\kappaII\over b} }
\end{align*}
If $\omega=\omega_{\rm opt}$ then the iteration converges in one iteration to the solution
\begin{align*}
   A &=   - {a^{-1} \over \frac{\kappaI}{a} + \frac{\kappaII}{b} } \jump{\kappa_m\up_x(0)}, \\ 
   B &=     {b^{-1} \over \frac{\kappaI}{a} + \frac{\kappaII}{b} } \jump{\kappa_m\up_x(0)} .
\end{align*}

% ========================================================================
\clearpage 
\subsection{One-dimensional Interface Problem with Mixed Interface Conditions}

Consider now the situation when we use a mixture of the jump conditions on each side of the interface.
In this case the interface conditions are
\begin{align*}
  \alpha \kappaI u^j_x(0) + \beta u^{j}(0) &= \alpha \kappaII v^{j-1}_x(0) + \beta v^j(0), \\
  \beta \kappaII v^j_x(0) -\alpha v^{j}(0) &= \beta  \kappaI u^{j-1}_x(0) -\alpha u^j(0), 
\end{align*}
where $\beta=1-\alpha$. Proceeding as before we get 
\begin{align*}
   \alpha \kappaI A_j + \beta a A_j &= \alpha \kappaII B_{j-1} - b \beta B_{j-1} 
                   - \alpha \jump{\kappa_m\up_x(0)} \\
   \beta \kappaII B_j + \alpha b B_j &= \beta \kappaI A_j -\alpha a A_j +\beta \jump{\kappa_m\up_x(0)}
\end{align*}
and thus
\begin{align*}
   B_j &= \left[ {\beta\kappaI -\alpha a\over \beta\kappaII+\alpha b } \right] A_j + 
                        {\beta\over \beta\kappaII+\alpha b } \jump{\kappa_m\up_x(0)} \\
   A_j &= \left[ {\alpha\kappaII -\beta b \over \alpha\kappaI+\beta a } \right]
          \left[ {\beta\kappaI -\alpha a  \over \beta\kappaII+\alpha b} \right] A_{j-1}  
                   + \Big(  - \alpha  + \beta {\alpha\kappaII -\beta b \over \beta\kappaII+\alpha b}\Big)
                    {1\over \alpha\kappaI+\beta a} \jump{\kappa_m\up_x(0)} \\
       &= \left[ {\alpha\kappaII/b -\beta \over \beta\kappaII/b+\alpha}\right]
             \left[ {\beta\kappaI/a -\alpha  \over \alpha\kappaI/a+\beta }\right]A_{j-1} 
          - { (\alpha^2 + \beta^2) b \over [\alpha\kappaI+\beta a][\beta\kappaII+\alpha b] }\jump{\kappa_m\up_x(0)}
\end{align*}
Define the amplification factor
\begin{align*}
  \lambda &= \left[ {\alpha\kappaII/b -\beta \over \beta\kappaII/b+\alpha}\right]
             \left[ {\beta\kappaI/a -\alpha  \over \alpha\kappaI/a+\beta }\right]
\end{align*}
This iteration converges provided
\begin{equation}
   \left\vert \lambda \right\vert <1 
\end{equation}
{\bf Note 1:} If $\alpha=\beta=\half$ then the iteration will converge since
\begin{equation}
   \left\vert \lambda \right\vert = 
   \left\vert \left[ {\kappaII/b -1 \over \kappaII/b+1}\right]
              \left[ {\kappaI/a - 1 \over \kappaI/a+ 1 }\right]  \right\vert <1 
\end{equation}

{\bf Note 2:} If $\kappaI/a \gg \kappaII/b$ then we probably want to choose $\alpha=1$, $\beta=0$  since
\begin{equation}
   \left\vert \lambda \right\vert = 
   \left\vert \kappaII/b \over \kappaI/a  \right\vert \ll 1 
\end{equation}

{\bf Note 3:} If we use an under-relaxed iteration with parameter $\omega$ then the iteration 
will converge for $0< \omega < 2\omega_{\rm opt}$ where (*check this*)
\begin{equation}
   \omega_{\rm opt} = { 1 \over 1 - \lambda }
\end{equation}
with amplification factor
\begin{equation}
   \lambda(\omega) = 1 -\omega( 1 - \lambda) 
\end{equation}


% ========================================================================
\clearpage
\subsection{Two-dimensional Interface Problem}


Now consider a two dimension interface problem for two adjacent squares, $2\pi$-periodic in the $y$-direction,
on the region $\Omega = [-a,b]\times[0,2\pi] = \Omega_1 \bigcup \Omega_2 $, $\Omega_1=[-a,0]\times[0,2\pi]$, 
$\Omega_2=[0,b]\times[0,2\pi]$,
 with interface $\Omega_I$ at $x=0$,
\begin{alignat}{3}
  \partial_x( \kappam \partial_x \um ) + \partial_y( \kappam \partial_y \um ) &= f &&\qquad m=1,2, ~~\text{for $x\in \Omega_m$}, \\
  \jump{ \um(0,y) } &= 0  &&\qquad \text{for $x\in \Omega_I$}, \\
  \jump{ \kappam \partial_x \um(0,y) } &= 0 &&\qquad \text{for $x\in \Omega_I$},
  \uI(-a,y)=g(a), \quad \uII(b,y)=g(b)  ~.
\end{alignat}
If we Fourier transform in $y$, with dual variable $k$, and subtract out a particular solution, $(\up(x,y),\vp(x,y))$, ( as in
the one-dimensional case) we are led to the iteration
\begin{alignat}{3}
  \kappaI u_{xx}^j - \kappaI k^2 u^j  &= 0 &  \kappaII v_{xx}^j - \kappaII k^2 v^j &= 0 \\
    u^j(-a,y)&=0      &  v^j(b,y)&= 0 \\
  \kappaI u^j_x(0,y) &= \kappaII v^{j-1}_x(0,y) + f_I(y)  \qquad  & v^j(0,y) &= u^{j}(0,y)  \\
   f_I(y) &\equiv \jump{\kappa_m\up_x(0,y) }
\end{alignat}
The solution to these equations is of the form (for $k\ne 0$)
\begin{align*}
  u^j &= A_j(y)\half\big(e^{k(x+a)} - e^{-k(x+a)}\big) = A_j(y)\sinh(k(x+a)) \\
  v^j &= B_j(y)\half\big(e^{k(x-b)} - e^{-k(x-b)}\big) = B_j(y)\sinh(k(x-b))
\end{align*}
For $k=0$ the solution is of the same form as the one-dimensional problem.

Substitution into the interface conditions gives
\begin{align*}
   \kappaI A_j k \cosh(ka) &=   \kappaII B_{j-1} k \cosh(kb) +f_I(y)  \\
    B_j~\sinh(-kb) &= A_j~\sinh(ka)
\end{align*}
giving
\begin{align*}
   B_j &= - {\sinh(ka)\over\sinh(kb)} A_j 
\end{align*}
and 
\begin{align*}
   A_j &= - {\tanh(ka)\over\kappaI}{\kappaII\over\tanh(kb)} A_{j-1} + {1\over  k~\cosh(ka)\kappaI} f_I(y)
\end{align*}
We can define an under-relaxed iteration with relaxation parameter $\omega$ by 
\begin{align*}
   A_j &= (1-\omega)A_{j-1} + 
                  \omega \Big( - {\tanh(ka)\over\kappaI}{\kappaII\over\tanh(kb)} A_{j-1} + {1\over  k~\cosh(ka)\kappaI} f_I(y) \Big) \\
    B_j &= - {\sinh(ka)\over\sinh(kb)} A_j 
\end{align*}
in which case
\begin{align*}
   A_j &= \Big[ 1 -\omega\big(1+{\tanh(ka)\over\kappaI}{\kappaII\over\tanh(kb)}\big) \Big] A_{j-1} -
                         \omega{ {1\over  k~\cosh(ka)\kappaI} f_I(y) }.
\end{align*}
This iteration will converge provided $0< \omega < 2\omega_{\rm opt}$ where
the optimal value for $\omega$ is
\begin{align*}
   \omega_{\rm opt} &= {1 \over 1+{\tanh(ka)\over\kappaI}{\kappaII\over\tanh(kb)} }.
\end{align*}

For $ka\gg 1 $ and $kb\gg 1$,  the optimal value tends to $\omega_{\rm opt} \rightarrow 1/(1+\frac{a}{b}\frac{\kappaII}{\kappaI})$.
Thus the optimal $\omega$ becomes independent of $k$ for large enough $k$. 


Since in general we don't know $a$ or $b$ we could choose
\[
   \omega_0 = {1 \over 1+\frac{\kappaII}{\kappaI} }.
\]
This is a safe value provided $\omega_0< 2\omega_{\rm opt}$.
\begin{align*}
   \omega_0 &< 2 \omega_{\rm opt}\\
\Rightarrow {1 \over 1+\frac{\kappaII}{\kappaI} }   &< {2 \over 1+{\tanh(ka)\over\kappaI}{\kappaII\over\tanh(kb)} } \\
% \Rightarrow 1+{\frac{\tanh(ka)}{\tanh(kb)}}{\kappaII\over\kappaI} &< 2 + 2 \frac{\kappaII}{\kappaI} \\
% \Rightarrow\frac{\tanh(ka)}{\tanh(kb)}\frac{\kappaII}{\kappaI} &< 1 + 2 \frac{\kappaII}{\kappaI}\\
\Rightarrow\frac{\tanh(ka)}{\tanh(kb)} &< \frac{\kappaI}{\kappaII} + 2 
\end{align*}

Thus $\omega_0$ should be a reasonable value to choose provided $\kappaI/a \ge \kappaII/b$  where $a$ and 
$b$ are estimates of the linear size of the domain. 



% The interface conditions~(\ref{eq:solidFluidInterfaceA}-\ref{eq:solidFluidInterfaceB}) define
% the matching conditions at a solid interface. 
% These conditions are valid for $t>0$ and thus the time derivatives of these conditions are
% also valid. For a second-order accurate approximation we use the time derivative of~\eqref{eq:solidFluidInterfaceA},
% combined with the interior equations for the temperature to give the three interface equations
% \begin{align}
%   (T - \TS )_\Ic &= 0 , \nonumber \\
%   ( \kF \nv\cdot\grad T - \kS \nv\cdot\grad \TS)_\Ic &= 0, \nonumber \\
%   \Big( \grad\cdot( \kF \grad T) + f_T\Big)_\Ic ~-~ \Big(\grad\cdot( \kS \grad \TS) + f_S \Big)_\Ic &=0 
% \end{align}
% where we have used the condition that $\uv=0$ on the interface. 
% Note that these same conditions apply on a moving grid. 
% {\bf I could write down all the higher-order matching conditions ?}


%===============================================================================
\clearpage
\subsection{Iteration Strategy for Multiple Domains}

\newcommand{\aI}{a^{(1)}}
\newcommand{\aII}{a^{(2)}}

The previous analyses suggest a iteration strategy for solving a problem with many sub-domains. 
With multiple sub-domains we need to decide which interface conditions to apply on the two sides of each interface.

{\bf Stratgey:} For an interface between domains $\Omega_1$ and $\Omega_2$ we apply a Neumann boundary condition 
(discretizing the jump condition $[\kappa \partial_n u]=0$ ) on the side of the 
interface with larger value for $\kappa$ (we really want the larger value of "$\kappa/a$" but "$a$" is
not known). The other side of the interface should apply the Dirichlet condition (discretizing the
jump condition $[u]=0$). The relaxation parameter can be chosen as 
\[
 \omega_{\rm opt} \approx \Big(1+\frac{\aI}{\kappaI}\frac{\kappaII}{\aII} \Big)^{-1} , 
\]
where $\aI$ and $\aII$ are rough estimates of the linear size of the domains $\Omega_1$ and $\Omega_2$, respectively. 



Example: If we have heat conduction between a gas and a solid where $\kappa_{\rm gas} \ll \kappa_{\rm solid}$,
then at the interface we will apply the Neumann BC on the solid and the Dirichlet BC on the gas.  Thus,
during the iteration to solve the iterface equations, 
the solid is given the heat-flux from the gas, while the gas is given the temperature from the solid. 

Remark: The physical interpretation is that the heat flux condition ({\em Neumann} interface condition)
 should be applied on the region where {\em the temperature equilibrates most quickly}.
In the previous example with a gas and solid, if the gas region were tiny, $a_{\rm gas} \ll 1$, then it could be that 
$\kappa_{\rm gas}/a_{\rm gas} \ll \kappa_{\rm solid}/a_{\rm solid}$. In this case, the temperature in the gase 
would equilibrate most quickly and thus the heat-flux condition should be applied to the gas.

\section{stuff}
built upon the \Overture 
framework~\cite{Brown97},\cite{Henshaw96a},\cite{iscope97}. 

% -------------------------------------------------------------------------------------------------
\vfill\eject
\bibliography{\homeHenshaw/papers/henshaw}
\bibliographystyle{siam}


\printindex


\end{document}
