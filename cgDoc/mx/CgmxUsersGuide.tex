%-----------------------------------------------------------------------
% User's Guide for CGMX -- Maxwell's Equation Solver
% 
%-----------------------------------------------------------------------
\documentclass{article}
\usepackage[bookmarks=true,colorlinks=true,linkcolor=blue]{hyperref}


% \input documentationPageSize.tex
\hbadness=10000 
\sloppy \hfuzz=30pt

% \voffset=-.25truein
% \hoffset=-1.25truein
% \setlength{\textwidth}{7in}      % page width
% \setlength{\textheight}{9.5in}    % page height

\usepackage{calc}
\usepackage[lmargin=.75in,rmargin=.75in,tmargin=.75in,bmargin=.75in]{geometry}

\input homeHenshaw

\usepackage{amsmath}
\usepackage{amssymb}

\usepackage{verbatim}
\usepackage{moreverb}

% \usepackage{epsfig}    
% This next section will allow graphics files to be ps or pdf  -- from Jeff via Jeff
% \usepackage{ifpdf}
% \ifpdf
%     \usepackage[pdftex]{graphicx}
%     \usepackage{epstopdf}
%     \pdfcompresslevel=9
%     \pdfpagewidth=8.5 true in
%     \pdfpageheight=11 true in
%     \pdfhorigin=1 true in
%     \pdfvorigin=1.25 true in
% \else
%     \usepackage{graphicx}
% \fi

% \input{pstricks}\input{pst-node}
% \input{colours}

% define the clipFig commands:
\input trimFig.tex

% --------------------------------------------
\usepackage[usenames]{color} % e.g. \color{red}
\newcommand{\red}{\color{red}}
\newcommand{\blue}{\color{blue}}
\newcommand{\green}{\color{green}}
\newcommand{\jwb}[2]{{\color{red}(old: #1) }{\color{green} #2}}

\usepackage{tikz}


\usepackage{makeidx} % index
\makeindex
\newcommand{\Index}[1]{#1\index{#1}}


% ---- we have lemmas and theorems in this paper ----
\newtheorem{assumption}{Assumption}
\newtheorem{definition}{Definition}

% \newcommand{\homeHenshaw}{/home/henshaw.0}

\newcommand{\primer}{/users/henshaw/res/primer}
\newcommand{\GF}{/users/\-henshaw/\-res/\-gf}
\newcommand{\gf}{/users/henshaw/res/gf}
\newcommand{\mapping}{/users/henshaw/res/mapping}

\newcommand{\docFigures}{\homeHenshaw/OvertureFigures}
\newcommand{\figures}{\homeHenshaw/res/OverBlown/docFigures}
\newcommand{\obFigures}{\homeHenshaw/res/OverBlown/docFigures}  % note: local version for OverBlown
\newcommand{\maxDoc}{\homeHenshaw/res/maxwell/doc}

\newcommand{\OVERTUREOVERTURE}{/users/\-henshaw/\-Overture/\-Overture}
\newcommand{\OvertureOverture}{/users/henshaw/Overture/Overture}

\newcommand{\Overture}{{\bf Overture\ }}
\newcommand{\OverBlown}{{\bf OverBlown\ }}
\newcommand{\overBlown}{{\bf overBlown\ }}


% *** See http://www.eng.cam.ac.uk/help/tpl/textprocessing/squeeze.html
% By default, LaTeX doesn't like to fill more than 0.7 of a text page with tables and graphics, nor does it like too many figures per page. This behaviour can be changed by placing lines like the following before \begin{document}

\renewcommand\floatpagefraction{.9}
\renewcommand\topfraction{.9}
\renewcommand\bottomfraction{.9}
\renewcommand\textfraction{.1}   
\setcounter{totalnumber}{50}
\setcounter{topnumber}{50}
\setcounter{bottomnumber}{50}


\begin{document}


% -----definitions-----
\input wdhDefinitions.tex

\def\ud     {{    U}}
\def\pd     {{    P}}

\newcommand{\mbar}{\bar{m}}
\newcommand{\Rbar}{\bar{R}}
\newcommand{\Ru}{R_u}         % universal gas constant
% \newcommand{\Iv}{{\bf I}}
% \newcommand{\qv}{{\bf q}}
\newcommand{\Div}{\grad\cdot}
\newcommand{\tauv}{\boldsymbol{\tau}}
\newcommand{\sumi}{\sum_{i=1}^n}
% \newcommand{\half}{{1\over2}}
\newcommand{\dt}{{\Delta t}}
\newcommand{\eps}{\epsilon}

\vglue 10\baselineskip
\begin{flushleft}
{\Large
Cgmx User Guide: An Overture Solver for Maxwell's Equations on Composite Grids, \\
}
\vspace{2\baselineskip}
William D. Henshaw  \\
Department of Mathematical Sciences, \\
Rensselaer Polytechnic Institute, \\
Troy, NY, USA, 12180.
% Centre for Applied Scientific Computing  \\
% Lawrence Livermore National Laboratory      \\
% Livermore, CA, 94551.  \\
% \vspace{\baselineskip}
% LLNL-SM-523971 \\
% henshaw@llnl.gov \\
% http://www.llnl.gov/casc/people/henshaw \\
% http://www.llnl.gov/casc/Overture\\
\vspace{\baselineskip}
\today\\
\vspace{\baselineskip}
% UCRL-MA-134288

\vspace{4\baselineskip}

\noindent{\bf\large Abstract:}

Cgmx is a program that can be used to solve the time-dependent Maxwell's equations
of electromagnetics on composite overlapping grids in two and three space
dimensions. This document explains how to run Cgmx. Numerous examples are
given. The boundary conditions, initial conditions and forcing functions 
are described.
Cgmx solves Maxwell's equations in second-order form. Second-order
accurate and fourth-order accurate approximations are available. Cgmx can
accurately handle material interfaces.  Cgmx also implements a version of the
Yee scheme for the first-order system form of Maxwell's equations. The Yee
scheme can be run on a single Cartesian grid with material boundaries treated
with a stair-step approximation.

\end{flushleft}

\clearpage
\tableofcontents
% \listoffigures


\vfill\eject

\section{Introduction}

   Cgmx is a program that can be used to solve Maxwell's equations of electromagnetics 
on composite overlapping grids~\cite{max2006b}. It is built upon
the \Overture framework~\cite{Brown97},\cite{Henshaw96a},\cite{iscope97}.

More information about
{\bf Overture} can be found on the \Overture home page, {\tt overtureFramework.org}.


\noindent{\bf Cgmx features:}

\begin{itemize}
  \item solve the time-domain Maxwell equations on overlapping grids to second- and fourth-order accuracy in two and three space dimensions. 
  \item solve material interface problems to second- and fourth-order accuracy in two and three space dimensions. 
  \item solve the time-domain Maxwell equations on a Cartesian grid with variable $\mu$ and $\epsilon$ using the Yee scheme.
\end{itemize}


\noindent{\bf Installation:} To install cgmx you should follow the instructions for installing Overture and cg
from the Overture web page. Note that cgmx is NOT built by default when building the cg solvers so that
after installing cg you must go into the {\tt cg/mx} directory and type `make'. 
If you only wish to build cgmx and not any of the other cg solvers, then
after building Overture and unpacking cg, go to the {\tt cg/mx} and type make. 

\noindent The cgmx solver is found in the {\tt mx} directory in the {\bf cg} distribution and has
sub-directories
\begin{description}
 \item[{\tt bin}] : contains the executable, cgmx. You may want to put this directory in your path.
 \item[{\tt check}] : contains regression tests.
 \item[{\tt cmd}] : sample command files for running cgmx, see section (\ref{sec:demo}).
 \item[{\tt doc}] : documentation.
 \item[{\tt lib}] : contains the cgmx library, {\tt libCgmx.a}.
 \item[{\tt src}] : source files 
\end{description}


\subsection{Basic steps}\index{basic steps}
Here are the basic steps to solve a problem with cgmx.
\begin{enumerate}
  \item Generate an overlapping grid with ogen. 
  \item Run cgmx (found in the {\tt bin/cgmx} directory).
  \item Assign the boundary conditions and initial conditions.
  \item Choose the parameters for the PDE (such as material properties such as $\mu$, $\epsilon$))
  \item Choose run time parameters, time to integrate to, time stepping method etc.
  \item Compute the solution (optionally plotting the results as the code runs).
  \item When the code is finished you can look at the results (provided you saved a
     `show file') using {\tt plotStuff}.
\end{enumerate}
The commands that you enter to run cgmx can be saved in a \Index{command file} (by default
they are saved in the file `cgmx.cmd'). This command file can be used to re-run
the same problem by typing `cgmx file.cmd'. The command file can be edited to change parameters.

To get started you can run one of the demo's that come with cgmx, these are 
explained in section~(\ref{sec:demo}).
% 
For more information on the algorithms and approximations used in Cgmx see
\begin{enumerate}
  \item The {\sl Cgmx Reference Manual}~\cite{CgmxReferenceManual}.
  \item {\sl A High-Order Accurate Parallel Solver for {Maxwell}'s Equations on Overlapping Grids}
        \cite{max2006b}.
\end{enumerate}

  
% \begin{figure}[hbt]
% \begin{center}
%   \epsfig{file=\obFigures/OverBlownScreen.ps,width=.95\linewidth}  \\
% \caption{Snapshot of cgmx showing the run time dialog menu. 
%     The figure shows two falling bodies in an incompressible flow,
%     computed with the command file twoDrop.cmd.}
%   \end{center} 
%   \label{fig:screenDrops}
% \end{figure}

\clearpage
\section{Sample command files for running cgmx} \label{sec:demo}

Command files are supported throughout the Overture. They are files
that contain lists of commands. These commands can initially be saved
when the user is interactively choosing options.  The \Index{command files}
can then be used to re-run the job. Command files can be edited and
changed.

In this section we present a number of command files that can be used
to run cgmx.

\subsection{Running a command file} \label{sec:runningCommandFiles} 

Given a \Index{command file} for cgmx such as {\tt cyleigen.cmd}, found in {\tt
cmd/cyleigen.cmd}, one can type `{\tt cgmx cyleigen.cmd}' to run this command
file . You can also just type `{\tt cgmx cyleigen}, leaving off the {\tt
.cmd} suffix. Typing `{\tt cgmx -noplot cyleigen}' will run without
interactive graphics (unless the command file turns on graphics). Note that here
I assume that the {\tt bin} directory is in your path so that the {\tt
cgmx} command is found when you type it's name. The Cgmx sample
command files will automatically look for an overlapping grid in the {\tt
Overture/sampleGrids} directory, unless the grid is first found in the location
specified in the command file.

When you run a command file a graphics screen will appear and after some
processing the run-time dialog should appear and the initial conditions will be
plotted. The program will also print out some information about the problem
being solved. At this point choose {\tt continue} or {\tt movie
mode}. Section~(\ref{sec:runTimeDialog}) describes the options available in the
run time dialog.

When running in parallel it is convenient to define a shell variable for the parallel
version of cgmx.
For example in the tcsh shell you might use
\begin{verbatim}
  set cgmxp  = ${CGBUILDPREFIX}/mx/bin/cgmx
\end{verbatim} % $
An example of running the parallel version is then
\begin{verbatim}
  mpirun -np 2 $cgmxp cic.planeWaveBC -g=cice2.order4.hdf
\end{verbatim} % $


%------------------------------------------------------------------------------------------
\clearpage
\subsection{Scattering of a plane wave from a two-dimensional conducting cylinder} \label{sec:cyl2dScat}



The command file {\tt cic.planeWaveBC.cmd} can be used to compute the scattering
of a plane wave from a PEC (perfectly electric conducting) cylinder. The exact solution is known for this case
and the errors in the numerical solution are determined while the solution
is being computed.

% \noindent Example:

\noindent (1) Generate the grid using the command file {\tt cicArg.cmd} in {\tt Overture/sampleGrids}:
\begin{verbatim}
  ogen -noplot cicArg -order=4 -interp=e -factor=2
\end{verbatim}

\noindent (2) Run cgmx: 
\begin{verbatim}
  cgmx cic.planeWaveBC -g=cice2.order4.hdf
\end{verbatim}

\noindent Parallel: If you have built the parallel version then use 
\begin{verbatim}
  mpirun -np 2 $cgmxp cic.planeWaveBC -g=cice2.order4.hdf
\end{verbatim} % $
where {\tt \$cgmxp} is a variable that points to the parallel version of cgmx (see Section~\ref{sec:runningCommandFiles}).

{
\begin{figure}[hbt]
\newcommand{\figWidth}{5.5cm}
\newcommand{\trimfig}[2]{\trimFig{#1}{#2}{0.1}{0.05}{.05}{.05}}
\begin{center}
\begin{tikzpicture}[scale=1]
  \useasboundingbox (0,.5) rectangle (17,5.5);  % set the bounding box (so we have less surrounding white space)
  \draw ( 0.0, 0) node[anchor=south west] {\trimfig{figures/scatCyl2dExT1p0}{\figWidth}};
  \draw ( 5.7, 0) node[anchor=south west] {\trimfig{figures/scatCyl2dEyT1p0}{\figWidth}};
  \draw (11.4, 0) node[anchor=south west] {\trimfig{figures/scatCyl2dHzT1p0}{\figWidth}};
 % - labels
 %   \draw (\txa,4.75) node[draw,fill=white,anchor=east] {\scriptsize $t=0.5$};
 %   \draw (\txb,4.75) node[draw,fill=white,anchor=east] {\scriptsize $t=1.0$};
 %   \draw (\txc,4.75) node[draw,fill=white,anchor=east] {\scriptsize $t=1.5$};
 %  \draw (current bounding box.south west) rectangle (current bounding box.north east);
% grid:
%  \draw[step=1cm,gray] (0,0) grid (17.0,5);
\end{tikzpicture}
\end{center}
\caption{Scattering of a plane wave by a PEC cylinder. Computed solution at $t=1.0$, $E_x$, $E_y$ and $H_z$.}
\label{fig:cyl2dScat}
\end{figure}
}

\noindent{\bf Notes:}
\begin{enumerate}
  \item See the comments at the top of the command file for command line arguments and further examples.
  \item This case was run with fourth-order accuracy.
\end{enumerate}


% ------------------------------------------------------------------------------------------
% \clearpage
% \subsection{Eigenfunctions of a three-dimensional cylinder} \label{sec:cyl3d}


%------------------------------------------------------------------------------------------
\clearpage
\subsection{Scattering from a three-dimensional conducting sphere} \label{sec:sphere3dScat}

The command file {\tt sib.planeWaveBC.cmd} can be used to compute the scattering
of a plane wave from a PEC sphere. The exact solution is known for this case
and the errors in the numerical solution are determined while the solution
is being computed.

% \noindent Example:

\noindent (1) Generate the grid using the command file {\tt sibArg.cmd} in {\tt Overture/sampleGrids}:
\begin{verbatim}
  ogen -noplot sibArg -order=4 -interp=e -factor=4
\end{verbatim}

\noindent (2) Run cgmx: 
\begin{verbatim}
  cgmx sib.planeWaveBC -g=sibe4.order4.hdf
\end{verbatim}

\noindent Parallel: If you have built the parallel version then use 
\begin{verbatim}
  mpirun -np 2 $cgmxp sib.planeWaveBC -g=sibe4.order4.hdf
\end{verbatim} % $
where {\tt \$cgmxp} is a variable that points to the parallel version of cgmx, e.g.,
\begin{verbatim}
  set cgmxp  = ${CGBUILDPREFIX\}/mx/bin/cgmx}
\end{verbatim} % $

{
\begin{figure}[hbt]
\newcommand{\figWidth}{5.5cm}
\newcommand{\trimfig}[2]{\trimFig{#1}{#2}{0.1}{0.05}{.05}{.05}}
\begin{center}
\begin{tikzpicture}[scale=1]
  \useasboundingbox (0,.5) rectangle (17,5.5);  % set the bounding box (so we have less surrounding white space)
  \draw ( 0.0, 0) node[anchor=south west] {\trimfig{figures/scatSphereExT1p0}{\figWidth}};
  \draw ( 5.7, 0) node[anchor=south west] {\trimfig{figures/scatSphereEyT1p0}{\figWidth}};
  \draw (11.4, 0) node[anchor=south west] {\trimfig{figures/scatSphereEzT1p0}{\figWidth}};
 % - labels
 %   \draw (\txa,4.75) node[draw,fill=white,anchor=east] {\scriptsize $t=0.5$};
 %   \draw (\txb,4.75) node[draw,fill=white,anchor=east] {\scriptsize $t=1.0$};
 %   \draw (\txc,4.75) node[draw,fill=white,anchor=east] {\scriptsize $t=1.5$};
 %  \draw (current bounding box.south west) rectangle (current bounding box.north east);
% grid:
%  \draw[step=1cm,gray] (0,0) grid (17.0,5);
\end{tikzpicture}
\end{center}
\caption{Scattering of a plane wave by a PEC sphere. Computed solution at $t=1.0$, $E_x$, $E_y$ and $E_z$.}
\label{fig:cyl2dScat}
\end{figure}
}

\noindent{\bf Notes:}
\begin{enumerate}
  \item See the comments at the top of the command file for command line arguments and further examples.
  \item This case was run with fourth-order accuracy.
\end{enumerate}



%------------------------------------------------------------------------------------------
\clearpage
\subsection{Scattering of a plane wave from a two-dimensional dielectric cylinder} \label{sec:cyl2dDielectricScat}

The command file {\tt dielectricCyl.cmd} can be used to compute the scattering
of a plane wave from a dielectric cylinder. The exact solution is known for this case
and the errors in the numerical solution are determined while the solution
is being computed.

% \noindent Example:

\noindent (1) Generate the grid using the command file {\tt innerOuter.cmd} in {\tt Overture/sampleGrids}:
{\small
\begin{verbatim}
  ogen -noplot innerOuter -factor=4 -order=4 -deltaRad=.5 -interp=e -name="innerOutere4.order4.hdf"
\end{verbatim}
}
\noindent (2) Run cgmx: 
\begin{verbatim}
  cgmx dielectricCyl -g=innerOutere4.order4.hdf -kx=2 -eps1=.25 -eps2=1. -go=halt
\end{verbatim}

{
\begin{figure}[hbt]
\newcommand{\figWidth}{5.5cm}
\newcommand{\trimfig}[2]{\trimFig{#1}{#2}{0.1}{0.05}{.05}{.05}}
\begin{center}
\begin{tikzpicture}[scale=1]
  \useasboundingbox (0,.5) rectangle (17,5.5);  % set the bounding box (so we have less surrounding white space)
  \draw ( 0.0, 0) node[anchor=south west] {\trimfig{figures/cyl2dDieExT1p0}{\figWidth}};
  \draw ( 5.7, 0) node[anchor=south west] {\trimfig{figures/cyl2dDieEyT1p0}{\figWidth}};
  \draw (11.4, 0) node[anchor=south west] {\trimfig{figures/cyl2dDieHzT1p0}{\figWidth}};
 % - labels
 %   \draw (\txa,4.75) node[draw,fill=white,anchor=east] {\scriptsize $t=0.5$};
 %   \draw (\txb,4.75) node[draw,fill=white,anchor=east] {\scriptsize $t=1.0$};
 %   \draw (\txc,4.75) node[draw,fill=white,anchor=east] {\scriptsize $t=1.5$};
 %  \draw (current bounding box.south west) rectangle (current bounding box.north east);
% grid:
%  \draw[step=1cm,gray] (0,0) grid (17.0,5);
\end{tikzpicture}
\end{center}
\caption{Scattering of a plane wave from a dielectric cylinder. Computed solution at $t=1.0$, $E_x$, $E_y$ and $H_z$.}
\label{fig:cyl2dDielectricScat}
\end{figure}
}

\noindent{\bf Notes:}
\begin{enumerate}
  \item See the comments at the top of the command file for command line arguments and further examples.
  \item This case was run with fourth-order accuracy.
\end{enumerate}

%------------------------------------------------------------------------------------------
\clearpage
\subsection{Scattering of a plane wave from a two-dimensional dielectric cylinder - Yee scheme} \label{sec:cyl2dDielectricScat}

The command file {\tt dielectricCyl.cmd} can be used to compute the scattering
of a plane wave from a dielectric cylinder using the Yee scheme. The exact solution is known for this case
and the errors in the numerical solution are determined while the solution
is being computed.

% \noindent Example:

\noindent (1) Generate the grid using the command file {\tt bigSquare.cmd} in {\tt Overture/sampleGrids}:
{\small
\begin{verbatim}
  ogen noplot bigSquare -factor=8 -xa=-1. -xb=1. -ya=-1. -yb=1. -name="bigSquareSize1f8.hdf"
\end{verbatim}
}
\noindent (2) Run cgmx: 
\begin{verbatim}
  cgmx dielectricCyl -g=bigSquareSize1f8.hdf -kx=2 -eps1=.25 -eps2=1. -method=Yee -errorNorm=2 -go=halt
\end{verbatim}

{
\begin{figure}[hbt]
\newcommand{\figWidth}{5.5cm}
\newcommand{\trimfig}[2]{\trimFig{#1}{#2}{0.1}{0.05}{.05}{.05}}
\begin{center}
\begin{tikzpicture}[scale=1]
  \useasboundingbox (0,.5) rectangle (17,5.5);  % set the bounding box (so we have less surrounding white space)
  \draw ( 0.0, 0) node[anchor=south west] {\trimfig{figures/cyl2dDieYeeExT1p0}{\figWidth}};
  \draw ( 5.7, 0) node[anchor=south west] {\trimfig{figures/cyl2dDieYeeEyT1p0}{\figWidth}};
  \draw (11.4, 0) node[anchor=south west] {\trimfig{figures/cyl2dDieYeeHzT1p0}{\figWidth}};
 % - labels
 %   \draw (\txa,4.75) node[draw,fill=white,anchor=east] {\scriptsize $t=0.5$};
 %   \draw (\txb,4.75) node[draw,fill=white,anchor=east] {\scriptsize $t=1.0$};
 %   \draw (\txc,4.75) node[draw,fill=white,anchor=east] {\scriptsize $t=1.5$};
 %  \draw (current bounding box.south west) rectangle (current bounding box.north east);
% grid:
%  \draw[step=1cm,gray] (0,0) grid (17.0,5);
\end{tikzpicture}
\end{center}
\caption{Scattering of a plane wave from a dielectric cylinder using the Yee scheme. The solution is
computed on a single Cartesian grid and the cylinder is represented in a stair-step fashion. 
Computed solution at $t=1.0$, $E_x$, $E_y$ and $H_z$.}
\label{fig:cyl2dDielectricScatYee}
\end{figure}
}

\noindent{\bf Notes:}
\begin{enumerate}
  \item The material parameters are chosen using ...
\end{enumerate}


%------------------------------------------------------------------------------------------
\clearpage
\subsection{Scattering of a plane wave from a three-dimensional dielectric sphere} \label{sec:sphere3dDielectricScat}

The command file {\tt dielectricCyl.cmd} (yes, the same file as for the 2d dielectric cylinder) can be used to compute the scattering
of a plane wave from a dielectric cylinder. The exact solution is known for this case
and the errors in the numerical solution are determined while the solution
is being computed.

% \noindent Example:

\noindent (1) Generate the grid using the command file {\tt solidSphereInABox.cmd} in {\tt Overture/sampleGrids}:
{\small
\begin{verbatim}
  ogen noplot solidSphereInABox -order=4 -interp=e -factor=2
\end{verbatim}
}
\noindent (2) Run cgmx: 
{\small
\begin{verbatim}
  cgmx dielectricCyl -cyl=0 -g=solidSphereInABoxe2.order4 -kx=1 -eps1=.25 -eps2=1. -go=halt -tp=.01
\end{verbatim}
}
{
\begin{figure}[hbt]
\newcommand{\figWidth}{5.5cm}
\newcommand{\trimfig}[2]{\trimFig{#1}{#2}{0.1}{0.05}{.05}{.05}}
\begin{center}
\begin{tikzpicture}[scale=1]
  \useasboundingbox (0,.5) rectangle (17,5.5);  % set the bounding box (so we have less surrounding white space)
  \draw ( 0.0, 0) node[anchor=south west] {\trimfig{figures/scatDielectricSphereEx}{\figWidth}};
  \draw ( 5.7, 0) node[anchor=south west] {\trimfig{figures/scatDielectricSphereEy}{\figWidth}};
  \draw (11.4, 0) node[anchor=south west] {\trimfig{figures/scatDielectricSphereEz}{\figWidth}};
 % - labels
 %   \draw (\txa,4.75) node[draw,fill=white,anchor=east] {\scriptsize $t=0.5$};
 %   \draw (\txb,4.75) node[draw,fill=white,anchor=east] {\scriptsize $t=1.0$};
 %   \draw (\txc,4.75) node[draw,fill=white,anchor=east] {\scriptsize $t=1.5$};
 %  \draw (current bounding box.south west) rectangle (current bounding box.north east);
% grid:
%  \draw[step=1cm,gray] (0,0) grid (17.0,5);
\end{tikzpicture}
\end{center}
\caption{Scattering of a plane wave from a dielectric sphere. Computed solution at $t=1.0$, $E_x$, $E_y$ and $H_z$.}
\label{fig:cyl2dDielectricScat}
\end{figure}
}

\noindent{\bf Notes:}
\begin{enumerate}
  \item See the comments at the top of the command file for command line arguments and further examples.
  \item This case was run with fourth-order accuracy.
\end{enumerate}


%------------------------------------------------------------------------------------------
\clearpage
\subsection{Diffraction from a two-dimensional knife edge slit} \label{sec:knifeEdge2d}

The command file {\tt knifeEdgel.cmd} can be used to compute the scattering
of a plane wave from narrow slit.

\noindent (1) Generate the grid using the command file {\tt innerOuter.cmd} in {\tt Overture/sampleGrids}:
{\small
\begin{verbatim}
  ogen -noplot knifeEdge -interp=e -factor=2 -order=4 -yTop=1.1 -name="knifeSlit2.order4.hdf"
\end{verbatim}
}
\noindent (2) Run cgmx: 
\begin{verbatim}
  cgmx knifeEdge -g=knifeSlit2.order4 -kx=8 -tp=.05 -tf=2. -plotIntensity=1 -go=halt
\end{verbatim}

{
\begin{figure}[hbt]
\newcommand{\figWidth}{5.5cm}
\newcommand{\trimfig}[2]{\trimFig{#1}{#2}{0.1}{0.05}{.05}{.05}}
\begin{center}
\begin{tikzpicture}[scale=1]
  \useasboundingbox (0,.5) rectangle (17,5.5);  % set the bounding box (so we have less surrounding white space)
  \draw ( 0.0, 0) node[anchor=south west] {\trimfig{figures/knifeEdgeExT1p5}{\figWidth}};
  \draw ( 5.7, 0) node[anchor=south west] {\trimfig{figures/knifeEdgeEyT1p5}{\figWidth}};
  \draw (11.4, 0) node[anchor=south west] {\trimfig{figures/knifeEdgeIntensityT1p5}{\figWidth}};
 % - labels
 %   \draw (\txa,4.75) node[draw,fill=white,anchor=east] {\scriptsize $t=0.5$};
 %   \draw (\txb,4.75) node[draw,fill=white,anchor=east] {\scriptsize $t=1.0$};
 %   \draw (\txc,4.75) node[draw,fill=white,anchor=east] {\scriptsize $t=1.5$};
 %  \draw (current bounding box.south west) rectangle (current bounding box.north east);
% grid:
%  \draw[step=1cm,gray] (0,0) grid (17.0,5);
\end{tikzpicture}
\end{center}
\caption{Scattering of a plane wave from a slit. Computed solution at $t=1.5$, $E_x$, $E_y$ and intensity.}
\label{fig:knifeEdge2d}
\end{figure}
}

\noindent{\bf Notes:}
\begin{enumerate}
  \item Comment on initial conditions and initial conditions bounding box ...
  \item Adjust boundaries for incident field ...
\end{enumerate}



%------------------------------------------------------------------------------------------
\clearpage
\subsection{Scattering of a plane wave by a two-dimensional triangular body} \label{sec:scatTri}

The command file {\tt scat.cmd} can be used to compute the scattering
of a plane wave from a body.

\noindent (1) Generate the grid using the command file {\tt triangleArg.cmd} in {\tt Overture/sampleGrids}:
{\small
\begin{verbatim}
  ogen noplot triangleArg -factor=8 -order=4 -interp=e
\end{verbatim}
}
\noindent (2) Run cgmx: 
\begin{verbatim}
  cgmx scat -g=trianglee8.order4.hdf -bg=backGround
\end{verbatim}

{
\begin{figure}[hbt]
\newcommand{\figWidth}{5.5cm}
\newcommand{\trimfig}[2]{\trimFig{#1}{#2}{0.1}{0.05}{.05}{.05}}
\begin{center}
\begin{tikzpicture}[scale=1]
  \useasboundingbox (0,.5) rectangle (17,5.5);  % set the bounding box (so we have less surrounding white space)
  \draw ( 0.0, 0) node[anchor=south west] {\trimfig{figures/scatTriExT1p5}{\figWidth}};
  \draw ( 5.7, 0) node[anchor=south west] {\trimfig{figures/scatTriEyT1p5}{\figWidth}};
  \draw (11.4, 0) node[anchor=south west] {\trimfig{figures/scatTriHzT1p5}{\figWidth}};
 % - labels
 %   \draw (\txa,4.75) node[draw,fill=white,anchor=east] {\scriptsize $t=0.5$};
 %   \draw (\txb,4.75) node[draw,fill=white,anchor=east] {\scriptsize $t=1.0$};
 %   \draw (\txc,4.75) node[draw,fill=white,anchor=east] {\scriptsize $t=1.5$};
 %  \draw (current bounding box.south west) rectangle (current bounding box.north east);
% grid:
%  \draw[step=1cm,gray] (0,0) grid (17.0,5);
\end{tikzpicture}
\end{center}
\caption{Scattering of a plane wave from a triangular body. Computed solution at $t=1.5$, $E_x$, $E_y$ and $H_z$. The initial startup wave-front can
     be seen leaving the domain in the upper left and lower left.}
\label{fig:scatTri2d}
\end{figure}
}

\noindent{\bf Notes:}
\begin{enumerate}
  \item Computes the scattered field directly using the {\tt planeWaveBoundaryForcing} option.
  \item Boundary conditions on the outer square use the {\tt abcEM2} far-field conditions.
\end{enumerate}


%------------------------------------------------------------------------------------------
\clearpage
\subsection{Comparing far-field boundary conditions} \label{sec:compareFarField}

The command file {\tt rbc.cmd} can be used to compare far-field boundary conditions.
A Gaussian source~\ref{sec:gaussianSource} is placed in the middle of square or box and far field boundary 
conditions are applied to all boundaries. For comparison we also show results when the outer boundary
is taken as a perfect electrical conductor (which results in large reflections).

\noindent (1) Generate the grids (command files in Overture/sampleGrids)
{\small
\begin{verbatim}
  ogen -noplot squareArg -order=4 -nx=128
  ogen noplot boxArg -order=4 -xa=-1. -xb=1. -ya=-1. -yb=1. -za=-1. -zb=1. ...
                     -factor=4 -name="boxLx2Ly2Lz2Factor4.order4.hdf"
\end{verbatim}
}
\noindent (2a) Run cgmx with the square grid and different boundary conditions: 
{\small
\begin{verbatim}
  cgmx rbc -g=square128.order4.hdf -x0=.5 -y0=.5 -rbc=abcEM2 -go=halt
  cgmx rbc -g=square128.order4.hdf -x0=.5 -y0=.5 -rbc=abcPML -pmlWidth=21 -pmlStrength=50. -go=halt
  cgmx rbc -g=square128.order4.hdf -x0=.5 -y0=.5 -rbc=perfectElectricalConductor -go=halt  
\end{verbatim}
}
\noindent (2b) Run cgmx with the box grid and different boundary conditions: 
{\small
\begin{verbatim}
  cgmx rbc -g=boxLx2Ly2Lz2Factor4.order4.hdf -rbc=abcEM2 -x0=0. -y0=0. -z0=0. -go=halt
  cgmx rbc -g=boxLx2Ly2Lz2Factor4.order4.hdf -rbc=abcPML -x0=0. -y0=0. -z0=0. -pmlWidth=11 -go=halt
  cgmx rbc -g=boxLx2Ly2Lz2Factor4.order4.hdf -rbc=perfectElectricalConductor -x0=0. -y0=0. -z0=0. -go=halt
\end{verbatim}
}
{
\begin{figure}[hbt]
\newcommand{\figWidth}{5.5cm}
\newcommand{\trimfig}[2]{\trimFig{#1}{#2}{0.1}{0.05}{.05}{.05}}
\begin{center}
\begin{tikzpicture}[scale=1]
  \useasboundingbox (0,.5) rectangle (17,5.5);  % set the bounding box (so we have less surrounding white space)
  \draw ( 0.0, 0) node[anchor=south west] {\trimfig{figures/rbcSquareAbcEm2ExT1p0}{\figWidth}};
  \draw ( 5.7, 0) node[anchor=south west] {\trimfig{figures/rbcSquarePMLExT1p0}{\figWidth}};
  \draw (11.4, 0) node[anchor=south west] {\trimfig{figures/rbcSquarePECExT1p0}{\figWidth}};
 % - labels
 %   \draw (\txa,4.75) node[draw,fill=white,anchor=east] {\scriptsize $t=0.5$};
 %   \draw (\txb,4.75) node[draw,fill=white,anchor=east] {\scriptsize $t=1.0$};
 %   \draw (\txc,4.75) node[draw,fill=white,anchor=east] {\scriptsize $t=1.5$};
 %  \draw (current bounding box.south west) rectangle (current bounding box.north east);
% grid:
%  \draw[step=1cm,gray] (0,0) grid (17.0,5);
\end{tikzpicture}
\end{center}
\caption{Gaussian source in a square with different boundary conditions. $E_x$ at time $t=1.0$. Left bc={\tt abcEM2}, middle: bc={\tt abcPML} and
right bc={\tt perfectElectricalConductor}. }
\label{fig:rbcSquare}
\end{figure}
}
{
\begin{figure}[hbt]
\newcommand{\figWidth}{5.5cm}
\newcommand{\trimfig}[2]{\trimFig{#1}{#2}{0.1}{0.05}{.05}{.05}}
\begin{center}
\begin{tikzpicture}[scale=1]
  \useasboundingbox (0,.5) rectangle (17,5.5);  % set the bounding box (so we have less surrounding white space)
  \draw ( 0.0, 0) node[anchor=south west] {\trimfig{figures/rbcBoxAbcEM2ExT2p0}{\figWidth}};
  \draw ( 5.7, 0) node[anchor=south west] {\trimfig{figures/rbcBoxPMLExT2p0}{\figWidth}};
  \draw (11.4, 0) node[anchor=south west] {\trimfig{figures/rbcBoxPECExT2p0}{\figWidth}};
 % - labels
 %   \draw (\txa,4.75) node[draw,fill=white,anchor=east] {\scriptsize $t=0.5$};
 %   \draw (\txb,4.75) node[draw,fill=white,anchor=east] {\scriptsize $t=1.0$};
 %   \draw (\txc,4.75) node[draw,fill=white,anchor=east] {\scriptsize $t=1.5$};
 %  \draw (current bounding box.south west) rectangle (current bounding box.north east);
% grid:
%  \draw[step=1cm,gray] (0,0) grid (17.0,5);
\end{tikzpicture}
\end{center}
\caption{Gaussian source in a box with different boundary conditions. $E_x$ at time $t=1.0$. Left bc={\tt abcEM2}, middle: bc={\tt abcPML} and
right bc={\tt perfectElectricalConductor}. }
\label{fig:rbcBox}
\end{figure}
}

\noindent{\bf Notes:}
\begin{enumerate}
 \item Note that for the PML boundary condition the solution is damped in a region 
  next to the boundary (set by the option -pmlWidth={\em number-of-lines}). This explains why the solution decays near the boundaries.
 % \item The PML region 
 % \item The PML boundary condition seems to be broken (maybe only works in x-direction?)
 % \item The PML boundary condition seems to need a smaller time-step (lower cfl). The reason for this needs to be sorted out.
\end{enumerate}


%------------------------------------------------------------------------------------------
\clearpage
\subsection{Transmission of a plane wave through a bumpy glass-air interface} \label{sec:scatBump2d}

The command file {\tt afm.cmd} can be used compute the propagation of a plane wave through the 
two-dimensional interface between two materials (air and glass).

\noindent (1) Generate the grids (command files in Overture/sampleGrids)
{\small
\begin{verbatim}
  ogen noplot afm -interp=e -order=4 -factor=4
\end{verbatim}
}
\noindent (2) Run cgmx (specifying $\eps$ in the two domains and the wave number $k_y$ of the incident field), 
{\small
\begin{verbatim}
  cgmx afm -g=afme4.order4.hdf -eps1=2.25 -eps2=1. -ky=20 -diss=4. -tf=1.4 -tp=.2 -go=halt
\end{verbatim}
}
{
\begin{figure}[hbt]
\newcommand{\figWidth}{7.0cm}
\newcommand{\trimfig}[2]{\trimFig{#1}{#2}{0.25}{0.25}{.45}{.425}}
\begin{center}
\begin{tikzpicture}[scale=1]
  \useasboundingbox (0,.75) rectangle (14.5,15.);  % set the bounding box (so we have less surrounding white space)
  \draw ( 0.0, 0) node[anchor=south west] {\trimfig{figures/afm2dThreeBumpExT0p0}{\figWidth}};
  \draw ( 7.5, 0) node[anchor=south west] {\trimfig{figures/afm2dThreeBumpEyT0p0}{\figWidth}};
% 
  \draw ( 0.0,3.8) node[anchor=south west] {\trimfig{figures/afm2dThreeBumpExT0p2}{\figWidth}};
  \draw ( 7.5,3.8) node[anchor=south west] {\trimfig{figures/afm2dThreeBumpEyT0p2}{\figWidth}};
% 
  \draw ( 0.0,7.6) node[anchor=south west] {\trimfig{figures/afm2dThreeBumpExT0p4}{\figWidth}};
  \draw ( 7.5,7.6) node[anchor=south west] {\trimfig{figures/afm2dThreeBumpEyT0p4}{\figWidth}};
%
  \draw ( 0.0,11.4) node[anchor=south west] {\trimfig{figures/afm2dThreeBumpExT0p6}{\figWidth}};
  \draw ( 7.5,11.4) node[anchor=south west] {\trimfig{figures/afm2dThreeBumpEyT0p6}{\figWidth}};
%
% \draw (current bounding box.south west) rectangle (current bounding box.north east);
% grid:
% \draw[step=1cm,gray] (0,0) grid (14.0,11);
\end{tikzpicture}
\end{center}
\caption{Transmission of a plane wave through a bumpy glass air interface. Left $E_x$ and right $E_y$ at times (bottom to top) 
  $t=0$, $t=0.2$, $t=0.4$ and $t=0.6$}
\label{fig:scatBump2d}
\end{figure}
}

\noindent{\bf Notes:}
\begin{enumerate}
  \item This example uses the plane wave initial condition with the initial condition bounding box.
  \item The speed of light in the lower glass region is $c_g =1/\sqrt{2.25} = 2/3$ compared to the speed of light $c_a=1$ in the
      upper air domain. 
\end{enumerate}

%------------------------------------------------------------------------------------------
\clearpage
\subsection{Scattering of plane wave from an array of dielectric cylinders} \label{sec:dielectricCylArray}

The command file {\tt lattice.cmd} can be used to compute the scattering
of a plane wave from an array of dielectric cylinders. A plane wave with wave number $k_x$ 
moves from left to right past an
array of dielectric cylinders. Periodic boundary conditions are imposed on the top and bottom. 

% \noindent Example:

\noindent (1) Generate the grid: (using the command file in {\tt Overture/sampleGrids}):
{\small
\begin{verbatim}
  ogen -noplot lattice -order=4 -interp=e -nCylx=3 -nCyly=3 -factor=2 -name="lattice3x3yFactor2.order4.hdf"
\end{verbatim}
}
\noindent (2) Run cgmx: 
\begin{verbatim}
  cgmx -noplot lattice -g=lattice3x3yFactor4.order4 -eps1=.25 -eps2=1. -kx=4 ...
                       -plotIntensity=1 -xb=-2. -go=halt
\end{verbatim}

{
\begin{figure}[hbt]
\newcommand{\figWidth}{5.5cm}
\newcommand{\trimfig}[2]{\trimFig{#1}{#2}{0.225}{0.295}{.475}{.475}}
\begin{center}
\begin{tikzpicture}[scale=1]
  \useasboundingbox (0,.5) rectangle (17,11.5);  % set the bounding box (so we have less surrounding white space)
  \draw ( 0.0, 8) node[anchor=south west] {\trimfig{figures/lattice3x3yExT6p0}{\figWidth}};
  \draw ( 5.7, 8) node[anchor=south west] {\trimfig{figures/lattice3x3yEyT6p0}{\figWidth}};
  \draw (11.4, 8) node[anchor=south west] {\trimfig{figures/lattice3x3yIntensityT6p0}{\figWidth}};
% 
  \draw ( 0.0, 4) node[anchor=south west] {\trimfig{figures/lattice3x3ykx8ExT6p0}{\figWidth}};
  \draw ( 5.7, 4) node[anchor=south west] {\trimfig{figures/lattice3x3ykx8EyT6p0}{\figWidth}};
  \draw (11.4, 4) node[anchor=south west] {\trimfig{figures/lattice3x3ykx8IntensityT6p0}{\figWidth}};
% 
  \draw ( 0.0, 0) node[anchor=south west] {\trimfig{figures/lattice33kx4Eps4ExT6p0}{\figWidth}};
  \draw ( 5.7, 0) node[anchor=south west] {\trimfig{figures/lattice33kx4Eps4EyT6p0}{\figWidth}};
  \draw (11.4, 0) node[anchor=south west] {\trimfig{figures/lattice33kx4Eps4IntensityT6p0}{\figWidth}};
 % - labels
 %   \draw (\txa,4.75) node[draw,fill=white,anchor=east] {\scriptsize $t=0.5$};
 %   \draw (\txb,4.75) node[draw,fill=white,anchor=east] {\scriptsize $t=1.0$};
 %   \draw (\txc,4.75) node[draw,fill=white,anchor=east] {\scriptsize $t=1.5$};
%  \draw (current bounding box.south west) rectangle (current bounding box.north east);
% grid:
%  \draw[step=1cm,gray] (0,0) grid (17.0,5);
\end{tikzpicture}
\end{center}
\caption{Scattering of a plane wave from an array of dielectric cylinders. Computed solution at $t=6.0$, $E_x$, $E_y$ and Intensity.
  Top row: $k_x=4$, $eps_1=0.25$. Middle row: $k_x=8$, $eps_1=0.25$. Bottom row: $k_x=4$, $eps_1=4.0$.
}
\label{fig:dielectricCylArray}
\end{figure}
}

% \noindent{\bf Notes:}
% \begin{enumerate}
%   \item See the comments at the top of the command file for command line arguments and further examples.
%   \item This case was run with fourth-order accuracy.
% \end{enumerate}



% =======================================================================================================
\clearpage
\section{Boundary Conditions} \label{sec:bc}

The general format for setting a boundary condition in a command file is
\begin{verbatim}
  bc: name=bcName
\end{verbatim}
where "name" specifies the name of a grid face, or grid and "bcName" is the name of the boundary
condition (given below), 
\begin{verbatim}
  name = [all][gridName][gridName([0|1],[0|1|2])]
  bcName=[dirichlet][perfectElectricalConductor][planeWaveBoundaryCondition][abcEM2]...
\end{verbatim}
Here are some examples of setting boundary conditions in a command file,
\begin{verbatim}
  bc: all=perfectElectricalConductor
  bc: backGround=abcEM2
  bc: backGround(0,0)=planeWaveBoundaryCondition
  bc: square(1,0)=planeWaveBoundaryCondition
\end{verbatim}
\begin{enumerate}
  \item Note that {\em backGround} and {\em square} are the names of the grids (specified when the grid was constructed with Ogen).
  \item Note that {\em backGround(0,0)} specifies the {\em left} face of the  grid {\em backGround} using the standard Overture conventions for
   specifying the face of a grid as {\em gridName(side,axis)}.
  \item Note that the special name {\em all} means apply the boundary condition to all faces of all grids.
  \item The last boundary condition given to a face is the one that will be used. 
\end{enumerate}

\noindent The following boundary conditions are (more or less) available with cgmx,
\begin{description}
  \item[periodic]: chosen automatically to match the grid created with Ogen.
  \item[dirichlet]: a non-physical boundary condition where all components of the field are set to some known function.
                    For example, the known function could be an exact solution or a twilight-zone function.
  \item[perfectElectricalConductor]: PEC boundary condition, see Section~\ref{sec:perfectElectricalConductor}.
  \item[perfectMagneticConductor]: PMC boundary condition (NOT implemented yet).
  \item[planeWaveBoundaryCondition]: solution on the boundary is set equal to a plane wave solution, see Section~\ref{sec:planeWaveBoundaryCondition}. 
  \item[symmetry]: apply a symmetry boundary condition. 
  \item[interfaceBoundaryCondition]: for the interface between two regions with different properties
  \item[abcEM2]: absorbing BC, Engquist-Majda order 2, see Section~\ref{sec:EngquistMajdaABC}.
  \item[abcPML]: perfectly matched layer far field condition, see Section~\ref{sec:PML}.
%  \item[abc3]: future absorbing BC
%   \item[abc4]: future absorbing BC
%   \item[abc5]: future absorbing BC
  \item[rbcNonLocal]: radiation BC, non-local
  \item[rbcLocal]: radiation BC, local
\end{description}


% --------------------------------------------------------------------------------
\subsection{Perfect electrical conductor boundary condition}\label{sec:perfectElectricalConductor}

The perfect electrical conductor boundary condition sets the tangential components of the
electric field to zero
\begin{align}
   \tauv_m\cdot\Ev &= 0, \qquad \text{for $\xv \in \partial\Omega_{\rm pec}$}. 
\end{align}
Here $\tauv_m$, $m=1,2$, denote the tangent vectors to the boundary surface, $\partial\Omega_{\rm pec}$.

% --------------------------------------------------------------------------------
\subsection{Plane wave boundary condition}\label{sec:planeWaveBoundaryCondition}

The plane wave boundary condition set the electric (and magnetic) fields on the boundary to the plane wave 
solution~\eqref{eq:planeWaveE}-\eqref{eq:planeWaveH} defined in Section~\ref{sec:planeWaveIC},
\begin{align}
   \Ev(\xv,t) &= \Ev_{\rm pw}(\xv,t), \qquad \text{for $\xv \in \partial\Omega_{\rm pw}$}, \\
   \Hv(\xv,t) &= \Hv_{\rm pw}(\xv,t), \qquad \text{for $\xv \in \partial\Omega_{\rm pw}$}. 
\end{align}

% --------------------------------------------------------------------------------
\subsection{Engquist-Majda absorbing boundary conditions}\label{sec:EngquistMajdaABC}


The boundary condition {\tt abcEM2} uses the Engquist-Majda absorbing boundary 
condition (defined here of a boundary $x={\rm constant}$), 
\begin{align}
   \partial_t\partial_x u = \alpha \partial_x^2 u + \beta (\partial_y^2+\partial_z^2) u 
\end{align}
With $\alpha=c$ and $\beta=\half c$, this gives a {\em second-order accurate} approximation to 
a pseudo-differential operator that absorbs outgoing traveling waves. 
Here $u$ is any field which satisfies the second-order wave equation.

% --------------------------------------------------------------------------------
\subsection{Perfectly matched layer boundary condition}\label{sec:PML}

The boundary condition {\tt abcPML} imposes a perfectly matched layer boundary condition.
With this boundary condition, auxiliary equations are solved over a layer (of some number of
specified grid points) next to the boundary. The PML equations we solve
are those suggested in Hagstrom~\cite{Hagstrom1999} and given by (defined here for a boundary $x={\rm constant}$), (*check me*)
\begin{align}
  u_{tt} &= c^2 \Big( \Delta u - \partial_x v - w \Big), \\
   v_t &= \sigma( -v + \partial_x u ) , \\
   w_t &= \sigma ( -w  - \partial_x v + \partial_x^2 u ). 
\end{align}
Here $u$ is any field which satisfies the second-order wave equation and $v$ and $w$ are auxiliary variables
that only live in the layer domain. 
The PML damping function $\sigma_1(\xi)$ is given by 
\begin{align}
  \sigma(\xi) = a \xi^p
\end{align}
where $a$ is the strength, $p$ is the power and where $\xi$ varies from $0$ to $1$ through the layer.




% ------------------------------------------------------------------------------
% \subsection{Far field boundary conditions} \label{sec:farFieldBC}


% =======================================================================================================
\clearpage
\section{Initial conditions} \label{sec:ic}

\noindent The following initial conditions are currently available with cgmx,
\begin{description}
  \item[planeWaveInitialCondition]: See Section~\ref{sec:planeWaveIC}. 
  \item[gaussianPlaneWave]:
  \item[gaussianPulseInitialCondition]:
  \item[squareEigenfunctionInitialCondition]:  
  \item[annulusEigenfunctionInitialCondition]:
  \item[zeroInitialCondition]:
  \item[planeWaveScatteredFieldInitialCondition]:
  \item[planeMaterialInterfaceInitialCondition]:
  \item[gaussianIntegralInitialCondition]:   from Tom Hagstrom
  \item[twilightZoneInitialCondition]: for use with twilight-zone forcing. 
  \item[userDefinedInitialConditions] : see Section~\ref{sec:userDefinedInitialConditions}
\end{description}

{\bf To do:} The file {\tt userDefinedInitialConditions.C} can be used to define new initial conditions
for use with cgmx. 

% ------------------------------------------------------------------
\subsection{The plane-wave initial condition} \label{sec:planeWaveIC}

The plane wave initial condition is defined from the plane wave solution
\begin{align}
  \Ev &= \sin( 2\pi (\kv\cdot\xv - \omega t ))~\av, \label{eq:planeWaveE} \\
  \Hv &= \sin( 2\pi (\kv\cdot\xv - \omega t ))~\bv, \label{eq:planeWaveH} \\
  \omega & = c_m \vert \kv \vert, \\
  c_m &= { 1 \over \sqrt{ \epsilon \mu}}   \qquad \text{(speed of light in the material)} ~.
\end{align}
where $\kv=(k_x,k_y,k_z)$, $\vert \kv \vert = \sqrt{ k_x^2 + k_y^2 + k_z^2 }$,
$\av=(a_x,a_y,a_z)$, and $\bv=(b_x,b_y,b_z)$ satisfy 
\begin{align}
  \kv\cdot\av &=0, ~~\kv\cdot\bv=0, ~~\mbox{(from $\grad\cdot\Ev=0$ and $\grad\cdot\Hv=0$)}, \\
  \bv &= \sqrt{\frac{\epsilon}{\mu}}~{\kv\times\av \over \vert \kv \vert}, 
       ~~\mbox{(from $\mu \Hv_t = -\grad\times\Ev$)} .  \label{eq:bvFromav} 
\end{align}
Thus given $\av$ with $\kv\cdot\av =0$, $\bv$ is determined by~\eqref{eq:bvFromav}.
% 
% The plane wave initial condition is defined from the plane wave solution
% \begin{align}
%     \Ev_{\rm pw}(\xv,t) &= \av \sin( \kv\cdot\xv - c_{\rm pw} t ),  \label{eq:planeWaveE} \\
%     \Hv_{\rm pw}(\xv,t) &= \bv \sin( \kv\cdot\xv - c_{\rm pw} t )   \label{eq:planeWaveH}
% \end{align}
% The coefficient $\av$ should satisfy $\kv\cdot\av=0$. Given $\av$, $\bv$ can be determined from
% Maxwell's equations.
The parameters in the plane wave initial condition can be set with the commands:
\begin{verbatim}
  kx,ky,kz $kx $ky $kz
  plane wave coefficients $ax $ay $az $epsPW $muPW
\end{verbatim}
Here {\tt epsPW} and {\tt muPW} (which could be different from the values of $\eps$ and $\mu$ used
for the simulation) are used to define $c_{\rm pw}=1/\sqrt{ \eps_{\rm pw}\cdot \eps_{\rm pw}}$.

% -------------------------------------------------------------------
\subsection{The initial condition bounding box}

The initial condition bounding box can be used to restrict the region where the initial 
condition is assigned (see the example in Section~\ref{sec:knifeEdge2d}).
The command to define the box is 
\begin{verbatim}
  initial condition bounding box $xa $xb $ya $yb $za $zb
\end{verbatim}
This bounding box may only currently work with the plane wave initial condition.

The plane-wave initial conditions are turned off in a smooth way at the boundary of the box using the
function
\begin{align}
  g(\xv) = \half(1-\tanh(\beta 2\pi(k_x(x-x_0)+k_y(y-y_0)))).
\end{align}
where $x_0$ or $y_0$ a point on the edge of the box (usually the transition only happens on one face of the box and
it is assumed here that the  plane wave is parallel to this face). 
The coefficient $\beta$ in this function can be set with the command
\begin{verbatim}
  bounding box decay exponent $beta
\end{verbatim} // $

% -----------------------------------------------------------------------
\subsection{User defined initial conditions} \label{sec:userDefinedInitialConditions}

The file {\tt userDefinedInitialConditions.C} can be used to define new initial conditions
for use with Cgmx. Here are the steps to take to add your own initial conditions:
\begin{enumerate}
  \item Edit the file  {\tt cg/mx/src/userDefinedInitialConditions.C}.
  \item Add a new option to the function {\tt setupUserDefinedInitialConditions()}. Follow one of the previous
         examples. 
  \item Implement the initial condition option in the function {\tt userDefinedInitialConditions(...)}.
  \item Type {\em make} from the {\tt cg/mx} directory to recompile Cgmx.
  \item Run Cgmx and choose the {\em userDefinedInitalConditions} option from the {\em forcing options...} menu.
    (see the example command file {\tt mx/cmd/userDefinedInitialConditions.cmd}).
\end{enumerate}

{
\begin{figure}[hbt]
\newcommand{\figWidth}{5.5cm}
\newcommand{\trimfig}[2]{\trimFig{#1}{#2}{0.1}{0.05}{.05}{.05}}
\begin{center}
\begin{tikzpicture}[scale=1]
  \useasboundingbox (0,.5) rectangle (17,5.75);  % set the bounding box (so we have less surrounding white space)
  \draw ( 0.0, 0) node[anchor=south west] {\trimfig{figures/userDefinedInitialConditionGaussianPulsesBoxEx}{\figWidth}};
  \draw ( 5.7, 0) node[anchor=south west] {\trimfig{figures/userDefinedInitialConditionGaussianPulsesBoxEy}{\figWidth}};
  \draw (11.4, 0) node[anchor=south west] {\trimfig{figures/userDefinedInitialConditionGaussianPulsesBoxEz}{\figWidth}};
 % - labels
 %   \draw (\txa,4.75) node[draw,fill=white,anchor=east] {\scriptsize $t=0.5$};
 %   \draw (\txb,4.75) node[draw,fill=white,anchor=east] {\scriptsize $t=1.0$};
 %   \draw (\txc,4.75) node[draw,fill=white,anchor=east] {\scriptsize $t=1.5$};
 %  \draw (current bounding box.south west) rectangle (current bounding box.north east);
% grid:
%  \draw[step=1cm,gray] (0,0) grid (17.0,5);
\end{tikzpicture}
\end{center}
\caption{User defined initial conditions. Results for two Gaussian pulses in a box.}
\label{fig:userDefinedInitialConditionsGaussianPulses}
\end{figure}
}

% =======================================================================================================
\clearpage
\section{Forcing functions} \label{sec:forcings}

The forcing functions are added to the right-hand side of Maxwell's equations in
second-order form,
\begin{align}
  \partial_t^2 \Ev &= c^2~ \Delta \Ev  + \Fv_\Ev(\xv,t), \\
  \partial_t^2 \Hv &= c^2~ \Delta \Hv  + \Fv_\Hv(\xv,t).
\end{align}


\noindent The following forcing options are currently available with cgmx,
\begin{description}
  \item[noForcing]: the default is to have no forcing.
 % \item[magneticSinusoidalPointSource]:
  \item[gaussianSource]: See Section~\ref{sec:gaussianSource}.
  \item[twilightZoneForcing]: The forcing to make the twilight-zone function an exact solution.
  \item[planeWaveBoundaryForcing]: The scattered field from an incident plane wave can be computed directly using this forcing
         which is added to the right-hand side of the PEC boundary condition.
  \item[gaussianChargeSource]: See Section~\ref{sec:gaussianChargeSource}.
  \item[userDefinedForcing] : See Section~\ref{sec:userDefinedForcing}.
\end{description}



% -----------------------------------------------------------------
\subsection{Gaussian source} \label{sec:gaussianSource}

The Gaussian source $\Fv_\Ev$ in three dimensions is defined by 
\begin{align*}
   g(\xv,t) & = \beta^2 \cos(2 \pi \omega(t-t_0)) ~\exp\{ -\beta |\xv-\xv_0|^2 \}, \\
   F_{E_x} &= ((z-z_0) - (y-y_0) ) ~g(\xv,t), \\ 
   F_{E_y} &= ((x-x_0) - (z-z_0) ) ~g(\xv,t), \\ 
   F_{E_z} &= ((y-y_0) - (x-x_0) ) ~g(\xv,t). 
\end{align*}
Here $|\xv|$ denotes the usual length of a vector, $|\xv|^2= x_1^2 + x_2^2 + x_3^2$.

\noindent The Gaussian source in two-dimensions is defined in a somewhat different fashion (for some reason?)
\begin{align*}
   g(\xv,t) & = 2 \beta \sin(2 \pi \omega(t-t_0)) ~\exp\{ -\beta |\xv-\xv_0|^2 \}, \\
   F_{H_z} &= 2 \pi\omega \cos(2 \pi \omega(t-t_0)) ~\exp\{ -\beta |\xv-\xv_0|^2 \}, \\
   F_{E_x} &= -(y-y_0)~g(\xv,t), \\ 
   F_{E_y} &=  (x-x_0)~g(\xv,t) .
\end{align*}


% -----------------------------------------------------------------
\subsection{Gaussian charge source} \label{sec:gaussianChargeSource}

The Gaussian charge source is defined by a moving charge density 
\begin{align}
  \rho(\xv,t) &= a \exp\{ - \big[\beta| (\xv-\xv_0) - \vv t |\big]^p \}.
\end{align}
Here $\xv_0$ is the initial location of the charge, $\vv$ is the velocity of the charge, $a$ is the strength and $\beta$ and $p$ define the
shape of the pulse in space.
The forcing functions for Maxwell's equations are then defined from
\begin{align}
    \partial_t^2 \Ev &= c^2 [ \Delta \Ev - \grad(\rho/\eps) - \mu \Jv_t ], \\
    \partial_t^2 \Hv &= c^2 [ \Delta \Hv + \grad\times\Jv ], \\
     \Jv & = \rho \vv ,
\end{align}
which gives 
\begin{align}
   \Fv_\Ev &= c^2 [  - \grad(\rho/\eps) - \mu \Jv_t ].
\end{align}

% -----------------------------------------------------------------------
\subsection{User defined forcing} \label{sec:userDefinedForcing}



The file {\tt userDefinedForcing.C} can be used to define new forcing functions
for use with Cgmx. Here are the steps to take to add your own forcing:
\begin{enumerate}
  \item Edit the file  {\tt cg/mx/src/userDefinedForcing.C}.
  \item Add a new option to the function {\tt setupUserDefinedForcing()}. Follow one of the previous
         examples. 
  \item Implement the forcing option in the function {\tt userDefinedForcing(...)}.
  \item Type {\em make} from the {\tt cg/mx} directory to recompile Cgmx.
  \item Run Cgmx and choose the {\em userDefinedForcing} option from the {\em forcing options...} menu.
    (see the example command file {\tt mx/cmd/userDefinedForcing.cmd}).
\end{enumerate}

{
\begin{figure}[hbt]
\newcommand{\figWidth}{7.5cm}
\newcommand{\trimfig}[2]{\trimFig{#1}{#2}{0.1}{0.05}{.05}{.05}}
\begin{center}
\begin{tikzpicture}[scale=1]
  \useasboundingbox (0,.5) rectangle (15,7.75);  % set the bounding box (so we have less surrounding white space)
  \draw ( 0.0, 0) node[anchor=south west] {\trimfig{figures/userDefinedGaussianSourcesEx}{\figWidth}};
  \draw ( 7.5, 0) node[anchor=south west] {\trimfig{figures/userDefinedGaussianSourcesEy}{\figWidth}};
 % - labels
 %   \draw (\txa,4.75) node[draw,fill=white,anchor=east] {\scriptsize $t=0.5$};
 %   \draw (\txb,4.75) node[draw,fill=white,anchor=east] {\scriptsize $t=1.0$};
 %   \draw (\txc,4.75) node[draw,fill=white,anchor=east] {\scriptsize $t=1.5$};
%  \draw (current bounding box.south west) rectangle (current bounding box.north east);
% grid:
%  \draw[step=1cm,gray] (0,0) grid (17.0,5);
\end{tikzpicture}
\end{center}
\caption{User defined forcing example showing results from specifying two Gaussian sources. Left $E_x$ and right $E_y$}
\label{fig:userDefinedForcingGuassianSources}
\end{figure}
}

% =======================================================================================================
% \clearpage
% \section{Variable material properties} \label{sec:varMat}
% 
% Finish me...




% =======================================================================================================
\clearpage
\section{Options} \label{sec:option}

% ---------------------------------------------------------------------------
\subsection{Options affecting the scheme}

\begin{description}
  \item [\qquad cfl] $value$ : set the CFL number. The schemes are usually stable to CFL=1. The default is CFL=.9.
  \item [\qquad order of dissipation] $[2|4|6]$ : set the order of the dissipation.
  \item [\qquad dissipation] $value$ : set the coefficient of the dissipation (e.g. 1). Dissipation is usually needed on overlapping grids. 
  \item [\qquad coefficients] $\eps~\mu~gridName$: specify the values of $\eps$ and $\mu$ on a grid.
  \item [\qquad adjustFarFieldBoundariesForIncidentField] $[0|1]~[gridName|all]$ : subtract off the plane-wave solution before applying the
         far-field boundary condition.
\end{description}


% ---------------------------------------------------------------------------
\subsection{Run time options} \label{sec:runTimeDialog}

\begin{description}
  \item [\qquad tFinal] $value$ : solve the equations to this time.
  \item [\qquad tPlot] $value$ : time increment to save results and/or plot the solution.
  \item [\qquad debug] $value$ : an integer bit flag that turns on debugging information. For example, set debug=1 for some info, debug=3 (=1+2) for some
    more, debug=7 (=1=2+4)for even more. 
\end{description}

% ---------------------------------------------------------------------------
\subsection{Plotting options}

\begin{description}
  \item [\qquad error norm] $[0|1|2]$ : compute errors in the max-norm, $L_1$-norm or $L_2$-norm. (*check me*)
  \item [\qquad plot scattered field] $[0|1]$ : when computing the scattered field directly use this option to plot the scattered field.
  \item [\qquad plot total field] $[0|1]$ : when computing the scattered field directly use this option to plot the total field (i.e. add in the
            plane wave solution before plotting).
  \item [\qquad plot errors] $[0|1]$ :
  \item [\qquad check errors] $[0|1]$ :
  \item [\qquad plot intensity] $[0|1]$ : plot the intensity (really only make senses for fields that are time-harmonic). 
  \item [\qquad plot harmonic E field] $[0|1]$ : plot the components of the complex harmonic fields (for problems that are time-harmonic).
\end{description}

% ---------------------------------------------------------------------------
\subsection{Output options}

\begin{description}
  \item [\qquad specify probes] : specify a list of probe locations as $x$, $y$, $z$ values, one per line, finishing the
         list with 'done'. The solution values at these locations (actually the closest grid point to each location, with these
         values written to the screen) are
    written to a text file whose default name is "probeFile.dat". 
    In the following example we specify two probe locations, $(.2,.3,.1)$ and $(.4,.6,.3)$, 
\begin{verbatim}
specify probes
  .2 .3 .1.
  .4 .6 .3
done
\end{verbatim}
    Each line of the probe file contains the time followed
    by the three components of $\Ev$ (or in two-dimensions $E_x$, $E_y$ and $H_z$) for each probe location. For example,
    with two probes specified the file in three dimensions would contain data of the form 
\begin{verbatim}
   t1 Ex11 Ey11 Ez11  Ex12 Ey12 Ez12 
   t2 Ex21 Ey21 Ez21  Ex22 Ey22 Ez22 
   ...
\end{verbatim}
  \item [\qquad probe file:] $name$ : specify the name of the probe file.
  \item [\qquad probe frequency] $value$ : specify the frequency at which values are saved in the probe file. For example,
      if the probe frequency is set to 2 then the solution will be saved every 2nd time step to the probe file. 
\end{description}

% =======================================================================================================
\clearpage
\section{Maxwell's Equations} \label{sec:equations}


The time dependent Maxwell's equations for linear, isotropic and non-dispersive materials are
\begin{align}
  \partial_t \Ev &=  {1\over \eps} \grad\times\Hv - {1\over \eps}\Jv , \label{eq:FOS-Et}  \\
  \partial_t \Hv &= - {1\over \mu} \grad\times\Ev ,  \label{eq:FOS-Ht} \\
  \grad\cdot(\eps\Ev) &=\rho , ~~ \grad\cdot(\mu\Hv) = 0 , \label{eq:FOS-div}
\end{align}
Here $\Ev=\Ev(\xv,t)$ is the electric field, 
$\Hv=\Hv(\xv,t)$ is the magnetic field, $\rho=\rho(\xv,t)$ is the electric charge density,
$\Jv=\Jv(\xv,t)$ is the electric current density,
$\eps=\eps(\xv)$ is the electric permittivity, and $\mu=\mu(\xv)$ is the magnetic permeability.
This first-order system for Maxwell's equations can also be written in a
second-order form. By taking the time derivatives of~(\ref{eq:FOS-Ht}) and
(\ref{eq:FOS-Et}) and using (\ref{eq:FOS-div}) it follows that 
\begin{align}
 \eps\mu~\partial_t^2 \Ev &= \Delta \Ev + \grad\Big( \grad \ln\eps~\cdot\Ev \Big)
        +\grad\ln\mu\times\Big(\grad\times\Ev\Big) 
            -\grad(\frac{1}{\epsilon}\rho)- \mu \partial_t\Jv , \label{eq:waveEGen} \\
 \eps\mu~\partial_t^2 \Hv &= \Delta \Hv + \grad\Big( \grad \ln\mu~\cdot\Hv \Big)
                               +\grad\ln\eps\times\Big(\grad\times\Hv\Big) 
                     + \eps\grad\times(\frac{1}{\epsilon}\Jv ) \label{eq:waveHGen}.
\end{align}
It is evident that the equations for the electric and magnetic field are decoupled with each 
satisfying a vector wave equation with lower order terms.
In the case of constant $\mu$ and $\eps$ and no charges, $\rho=\Jv=0$, 
the equations simplify to the classical second-order wave equations,
\begin{align}
  \partial_t^2 \Ev = c^2~ \Delta \Ev , \qquad
  \partial_t^2 \Hv = c^2~ \Delta \Hv \label{eq:waveE}
\end{align}
where $c^2=1/(\eps\mu)$.
There are some advantages to solving the second-order form of the equations
rather than the first-order system. One advantage is that in some cases it is
only necessary to solve for one of the variables, say $\Ev$. 
If the other variable, $\Hv$ is required, it can be
determined by
integrating equation~\eqref{eq:FOS-Ht} as an ordinary differential equation
with known $\Ev$. Alternatively, as a post-processing step $\Hv$ can be computed from an
elliptic boundary value problem formed by taking the curl of equation~\eqref{eq:FOS-Et}.
Another advantage of the second-order form, which simplifies the implementation on
an overlapping grid, is that there is no need to use a staggered grid formulation. 
Many schemes approximating the first order system~(\ref{eq:FOS-Ht}-\ref{eq:FOS-div}) rely on a
staggered arrangement of the components of $\Ev$ and $\Hv$ such as the
popular Yee scheme~\cite{Yee66} for Cartesian grids. 


% ===============================================================================================
\input acknowledgments


\vfill\eject
\bibliography{\homeHenshaw/papers/henshaw,\homeHenshaw/papers/henshawPapers}
\bibliographystyle{siam}

\printindex

\end{document}

% ***************************************************************************************************




