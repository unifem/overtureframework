\section{Absorbing Boundary Conditions}\label{sec:abc}

In this section we discuss absorbing boundary conditions (ABC's).

\subsection{Engquist-Majda one-way wave equations}



Consider the wave equation
\[
   u_{tt} = u_{xx} + u_{yy}+ u_{zz}
\]

We consider a boundary at $x=0$. The wave equation can be formally factored using
pseudo-differential operators
\begin{align*}
   ( D_x - D_t\sqrt{ 1-S^2} )( D_x + D_t\sqrt{ 1-S^2}) &=0 
\end{align*}
where
\[
        S^2  = (D_y^2 + D_z^2)/D_t^2
\]
The operator $G^{-} =D_x - D_t\sqrt{ 1-S^2}$ only supports waves moving to
the left. Applying $G^{-}$ to a wave function $U$ will absorb waves
moving to left (at any angle).
To see this consider a wave that moves in the negative $x$-direction
\[
    U(\xv,t) = e^{ i(\kv\cdot\xv + \omega t)}
\]
with $k_x>0$ and $\omega=\sqrt{ \kv\cdot\kv}>0$.

In Fourier space
\begin{align*}
   {\mathcal F}\{ G^{-} U \} &= ik_x - i\omega\sqrt{ 1 - (k_y^2+k_z^2)/\omega^2}  \\
        &= i k_x - i\sqrt{ \omega^2 - {k_y^2+k_z^2} } \\
        &= i k_x - i |k_x| \\
        &= 0 
\end{align*}
thus giving $D_t G^{-} U=0$. 

We will apply $G^{-} =D_x - D_t\sqrt{ 1-S^2}$ as a non-reflecting boundary condition. If applied
exactly it will absorb (treat exactly without reflection) all outgoing plane waves. It will
not handle evanescent modes. 

{\bf Aside: Evanescent waves} When a plane wave hits a material interface there will be a reflected
wave and a refracted wave ( transmitted wave). If the refracted ray bends toward the normal we have
what is called external reflection. If it bends away from the normal it is called internal reflection.
At a critical angle $\theta_c$ the refracted wave is parallel to the interface. For angles
greater than $\theta_c$ there is no refracted wave 
and we have total internal reflection. There is however an evanescent wave that
travels parallel to the boundary and decays exponentially into the second medium. The evanescent wave
 ensures that
the tangential component of the electric field is continuous across the interface. 


If the wave approaching the boundary is nearly normal incidence then $ k_x^2  \gg k_y^2 + k_z^2$
and thus 
\begin{align*}
  \widehat{S^2} = (k_y^2 +  k_z^2)/\omega^2 = (k_y^2 +  k_z^2)/( k_x^2 + k_y^2 +  k_z^2) \ll 1 
\end{align*}
Thus $S^2$ is thought of as being small and we approximate
\begin{align*}
   \sqrt{ 1-S^2} \approx p_0 + p_2 S^2
\end{align*}
Whence
\begin{align*}
   D_t G^{-} &\approx D_tD_x - D_t^2(p_0 + p_2 S^2) \\
              &\approx D_t D_x -p_0  D_t^2 - p_2 (D_y^2+D_z^2) \\
              & = D_t D_x - p_0  D_x^2 - (p_0+p_2) (D_y^2+D_z^2)
\end{align*}
For $p_0=1$, $p_2=-1/2$ this gives the approximate (second-order) Engquist-Majda ABC,
\begin{align}
   L_{2}^{em}u  & = \partial_t\partial_x u -\partial_t^2 u + {1\over 2} (\partial_y^2+\partial_z^2)u = 0 \\
                  &= \partial_t\partial_x u -\partial_x^2 u -{1\over 2} (\partial_y^2+\partial_z^2)u = 0 
\end{align}
which is exact for waves impinging on the boundary in the normal direction (angle of incidence of zero).
Mur's scheme is a discretization of the above BC -- Mur centered his scheme at $t+\Delta/2$ and $\Delta x/2$
to give a second order approximation using only two time levels.



We can use a better approximation
\begin{align*}
   \sqrt{ 1-S^2} \approx { p_0 + p_2 S^2 \over q_0 + q_2 S^2 } 
\end{align*}
This give a third-order boundary condition
\begin{align}
   L_{3}^{em}u  & = \partial_x(q_0\partial_t^2 + q_2(\partial_y^2+\partial_z^2)) 
                         - \partial_t( p_0\partial_t^2 + p_2(\partial_y^2+\partial_z^2))\\
                & = q_0\partial_x \partial_t^2 + q_2 \partial_x(\partial_y^2+\partial_z^2)
                       - p_0 \partial_t^3 - p_2 \partial_t(\partial_y^2+\partial_z^2)
\end{align}
Engquist and Majda suggested $p_0=q_0=1$ and $p_2=-3/4$, $q_2=-1/4$ which is the Pad\'e approximation (minimizing
the error near $S=0$.

Trefethen and Halpern considered other possibilities such as a Chebyshev or least squares.
One could choose the coefficients
to make the approximation exact at other angles.


\newcommand{\Dxp}{D^{x}_+}
\newcommand{\Dxm}{D^{x}_-}
\newcommand{\Dxz}{D^{x}_0}
\newcommand{\Dyp}{D^{y}_+}
\newcommand{\Dym}{D^{y}_-}
\newcommand{\Dyz}{D^{y}_0}
\newcommand{\Dzp}{D^{z}_+}
\newcommand{\Dzm}{D^{z}_-}
\newcommand{\Dzz}{D^{z}_0}
\newcommand{\Dtp}{D^{t}_+}
\newcommand{\Dtm}{D^{t}_-}
\newcommand{\Dtz}{D^{t}_0}
\newcommand{\Avtp}{\mathcal{A}^t_+}
\subsection{Second-order accurate discretization}


Consider Engquist-Majda ABC,
\begin{align}
   \partial_t\partial_x u = \alpha \partial_x^2 u + \beta (\partial_y^2+\partial_z^2) u 
\end{align}
where, for example $\alpha=c$ and $\beta=\half c$, will give a second-order acuurate approximation. 
We can discretize this equation with the centered second-order accurate approximation
\begin{align}
   \Dtz\Dxz U_{ij}^n = \alpha \Dxp\Dxm U_{ij}^n +\beta ( \Dyp\Dym + \Dzp\Dzm) U_{ij}^n 
\end{align}
This equation will give the ghost point value $U_{-1j}^{n+1}$ given interior and boundary values at time $t^{n+1}$
and old values at time $t^n$. 

Here is another second-order approximation, centred at $t^{n+\half}$ that uses only two time levels,
\begin{align}
   \Dtp\Dxz U_{ij}^n &= \Avtp\Big( \alpha\Dxp\Dxm U_{ij}^n +\beta( \Dyp\Dym + \Dzp\Dzm) U_{ij}^n\Big) \label{eq:ABCtwoLevel}  \\
   \Avtp f^n &= \half( f^{n+1} + f^n) 
\end{align}
The value at the ghost point $U_{-1j}^{n+1}$ in equation~\eqref{eq:ABCtwoLevel} 
can be explicitly solved for given interior and boundary values at time $t^{n+1}$
and old values at time $t^n$. 

This gives the approximation
\begin{align}
\Big({1\over 2\dt\dx} + {\alpha\over 2\dx^2}\Big) U_{-1j}^{n+1} &= 
  \Dtp\Dxz U_{ij}^n + {1\over 2\dt\dx}U_{-1j}^{n+1}  \\
           & - \Avtp\Big( \alpha \Dxp\Dxm U_{ij}^n - \beta ( \Dyp\Dym + \Dzp\Dzm) U_{ij}^n\Big)
    + {\alpha\over 2\dx^2} U_{-1j}^{n+1}
\end{align}
or
\begin{align}
\Big( 1 + \alpha\frac{\dt}{\dx}\Big) U_{-1j}^{n+1} &= 2\dt\dx\Big[ 
  \Dtp\Dxz U_{ij}^n + {1\over 2\dt\dx}U_{-1j}^{n+1}  \\
           & - \Avtp\Big( \alpha \Dxp\Dxm U_{ij}^n - \beta ( \Dyp\Dym + \Dzp\Dzm) U_{ij}^n\Big)
    + {\alpha\over 2\dx^2} U_{-1j}^{n+1} \Big]
\end{align}
Note: The right-hand-side of the above expression does not depend on $U_{-1j}^{n+1}$. 

\subsection{Fourth-order accuracy}

 For fourth-order accuracy we have two ghost points and need an additional numerical
boundary condition.
We could use the normal derivative of one of the above approximations,
\begin{align*}
    \partial_x L_{2}^{em}u  &= D_t D_x^2 - p_0  D_x^3 - (p_0+p_2) D_x(D_y^2+D_z^2) \\
\end{align*}
It might be appropriate to use the first order approximation $L_{1}^{em}=\partial_t -\partial_x$
times the second-order
 \begin{align*}
    L_{1}^{em} L_{2}^{em}u  &= (\partial_t -\partial_x)\Big[D_t D_x - p_0  D_x^2 - (p_0+p_2) (D_y^2+D_z^2)\Big]
\end{align*}   


\subsection{Absorbing boundary conditions on a curvilinear grid}

Consider a rotated rectangular grid. The wave equation in transformed coordinates is 
\begin{align*}
   u_{tt} &= (r_x^2 + r_y^2 + r_z^2) u_{rr} + (s_x^2 + s_y^2+s_z^2) u_{ss} + ...  \\
          &= D_n^2 u + \Delta_\tau u \\
    D_n & = \| \grad_\xv r \| \partial_r 
\end{align*}   
where $D_n$ is the normal derivative to a boundary $r={\rm const}$. 
Following the same argument as before we can derive the (second-order) Engquist-Majda ABC as
\begin{align*}
  \partial_t D_n u &= \alpha D_n^2 u + \beta \Delta_\tau u
\end{align*} 


Now consider a general curvilinear grid. 
The wave equation in transformed coordinates is 
\begin{align*}
   u_{tt} &= L u \\
          &= (r_x^2 + r_y^2 + r_z^2) u_{rr} + (s_x^2 + s_y^2+s_z^2) u_{ss}  
                       + 2(r_x s_x + r_y s_y + r_z s_z ) u_{rs} \\
          &     ~~~        + (r_{xx}+r_{yy}+r_{zz})u_r + (s_{xx}+s_{yy}+s_{zz})u_s + ... \\
          &= D_n^2 u + \Delta_\tau u \\
    D_n & = \| \grad_\xv r \| \partial_r \\
    \Delta_\tau &=  (L - D_n^2 )  u
\end{align*} 
where we have arbitrarily chosen $D_n$ to be an approximation to the normal derivative. 
This gives the ABC
\begin{align*}
  \partial_t D_n u &= \alpha D_n^2 u + \beta \Delta_\tau u
\end{align*}
which may reasonably accurate for a nearly orthogonal grid. 

A more accurate approximation for non-orthogonal grids 
could be to set $D_n$ to the actual normal derivative:
\begin{align*}
D_n & = \nv\cdot\grad = { \grad_\xv r \over \vert \grad_\xv r \vert} \cdot\grad_\xv  \\
    & = \vert \grad_\xv r \vert \partial_r + (\nv\cdot \grad s) \partial_s 
\end{align*} 


\subsection{Non-reflecting Boundary Conditions and Incident Fields}


The non-reflecting boundary conditions were derived assuming that there are no incoming waves, just
waves leaving the domain. We can treat the case of an incident field arriving from outside the computational
domain in a couple of ways. 

{\bf Approach I:} In this approach 
 we write the total electric field in the neighbourhood of the boundary as the sum of a given incident
field $\Ev^i(\xv,t)$ and a scattered field $\Ev^s(\xv,t)$,
\[
  \Ev(\xv,t) = \Ev^i(\xv,t) + \Ev^s(\xv,t) ~.
\]
We assume that the incident field is an exact solution of Maxwell's equations near the boundary. 
We can apply the non-reflecting boundary condition to $\Ev^s(\xv,t)$ by subtracting $\Ev^i(\xv,t)$ from
the total field. In the discrete case, at each time step we can subtract $\Ev^i(\xv,t)$ from $\Ev(\xv,t)$ 
on a few points near the boundary, apply the NRBC, and then add back $\Ev^i(\xv,t)$. 

{\bf Approach II:} In this approach we change the NRBC or ABC to account for the incident field.
Given the NRBC,
\begin{align}
  \Lc \Ev^s = 0,
\end{align}
by substituting $ \Ev^s(\xv,t) = \Ev(\xv,t) - \Ev^i(\xv,t)$ 
we get the new condition for the total field
\begin{align}
  \Lc \Ev &=  \Lc \Ev^i
\end{align}
For example, 
\begin{align}
 \partial_t\partial_x \Ev -\Big( \alpha \partial_x^2 \Ev + \beta (\partial_y^2+\partial_z^2) \Ev \Big) = 
 \partial_t\partial_x \Ev^i -\Big( \alpha \partial_x^2 \Ev^i + \beta (\partial_y^2+\partial_z^2) \Ev^i  \Big)
\end{align}
Note: We could potentially set the RHS to $G^{-}\Ev^i$ if $\Lc \approx G^{-}$. 
