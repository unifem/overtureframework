\subsubsection{Show File Options Dialog (showfile options...)}\label{sec:showfileOptions}\index{showfile options}

Here is a description of the {\em Show File Options} dialog.

\noindent The pull down menu commands are
\begin{description}
  \item[\qquad show variables] : toggle on/off variables that should be saved in the show file.
\end{description}

\noindent The {\em mode} options are
\begin{description}
  \item[\qquad compressed] : a compressed file will be
         smaller (especially for AMR runs that create many grids) but a compressed file may not
         be readable by future versions. 
  \item[\qquad uncompressed] : 
\end{description}

\noindent The push button commands are
\begin{description}
  \item[\qquad open] : open a show file. You will be prompted for the name.
  \item[\qquad close] : close the show file.
\end{description}

\noindent The text commands are
\begin{description}
  \item[\qquad frequency to save] : By default the solution is saved in the show file
        as often as it is plotted according to {\tt 'times to plot'}. To save the solution less
        often set this integer value to be greater than 1. A value of 2 for example will save solutions
        every 2nd time the solution is plotted.
  \item[\qquad frequency to flush] : Save this many solutions in each show file so that multiple
      show files will be created (these are automatically handled by plotStuff). See section~(\ref{sec:flush})
      for why you might do this.  
  \item[\qquad frequency to save sequences] : specify how often to save sequences (such as the residual for
                steady state problems).
  \item[\qquad maximum number of parallel sub-files] : On a parallel machine the show file is split into
    parallel sub-files. If speed of the parallel writing depends on the number of processors and
      this value. A good value may be about 8 or 16 (depends on the number of I/O channels). 
\end{description}
