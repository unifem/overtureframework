\subsubsection{Initial Conditions Options dialog (initial conditions options...)}\label{sec:initialConditionsOptions}\index{initial conditions options}


\noindent The {\em plot component} option menu allows one to choose the solution
component to plot when displaying the initial conditions.

\noindent The {\em Forcing Options} options are
\begin{description}
  \item[\qquad no forcing] : 
  \item[\qquad showfile forcing] : supply a forcing function from a show file.
\end{description}


\noindent The push buttons are
\begin{description}
  \item[\qquad uniform flow...] : open the {\em Uniform Flow Parameters} dialog:
     \begin{description}
         \item[]
        \noindent The push button commands are
       \begin{description}
         \item[\qquad assign uniform state] : evaluate the initial conditions.
       \end{description}

       \noindent The text commands are
       \begin{description}
         \item[\qquad uniform state] : assign values to the components in the form {\tt p=0, u=1, v=2}.
       \end{description}
    \end{description}
     
  \item[\qquad step function...] : open the {\em Step Function Parameters} dialog.
    \begin{description}
         \item[]
        \noindent The push button commands are
       \begin{description}
         \item[\qquad assign step function] : evaluate the initial conditions.
       \end{description}

       \noindent The text commands are
       \begin{description}
         \item[\qquad state behind] : assign values to the components in the form {\tt p=0, u=2, v=1}.
         \item[\qquad state ahead] : assign values to the components in the form {\tt p=0, u=1, v=0}.
         \item[\qquad step: a*x+b*y+c*z=d] : supply $a,b,c,d$ for the equation of the plane.
         \item[\qquad step sharpness] : exponent in the $\tanh$ functions used to smooth out the step. Choose
                       a value of $-1$ for a true step function.
       \end{description}
    \end{description}

  \item[\qquad read from a show file...] : open the {\em Read From a Show File} dialog.
    \begin{description}
         \item[]
        \noindent The push button commands are
       \begin{description}
         \item[\qquad assign solution from show file] : evaluate the initial conditions.
         \item[\qquad choose file from menu...] : choose the show file from a file menu.
       \end{description}

        \noindent The toggle buttons are
       \begin{description}
         \item[\qquad use grid from show file] : if true (toggle on) then use the grid in the show file.
               If false (toggle off) then interpolate the solution from the grid in the show file.
       \end{description}

       \noindent The text commands are
       \begin{description}
         \item[\qquad show file name] : choose the name of a show file.
         \item[\qquad solution number] : choose a solution number in the show file.
       \end{description}
    \end{description}


  \item[\qquad twilight zone...] : open the {\em twilight zone} dialog (see section~\ref{sec:twilightZoneOptions}).
  \item[\qquad user defined...] : choose a user defined initial condition 
                (see section~\ref{sec:userDefinedInitialConditions}. 
  \item[\qquad change contour plot] : enter the contour plotter menu.
\end{description}

\noindent The text commands are
\begin{description}
  \item[\qquad initial time] : set the initial time. Normally the initial time is set to $t=0.$ unless a solution
        is read from a show file (restart) in which case th initial time equals that from the show file.  
\end{description}
