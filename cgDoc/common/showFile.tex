\subsection{The show file}

   The `show file' is a data base file of a particular format that
contains the solutions from \Solver. The post-processing routine
{\tt plotStuff}~\cite{PLOTSTUFF}
 knows how to read this file and find all the solutions
and the different grids if the grids are moving or adaptive. The show can
be looked at by typing {\tt `plotStuff file.show'} or just {\tt `plotStuff file'},
where {\tt file.show}
is the name that you gave to the show file when running cgins.
The program {\tt plotStuff} is found in {\tt Overture/bin}.

\subsubsection{Flushing the show file} \label{sec:flush}\index{show file!flushing}
  It is not possible to look at results in a show file while the program is 
running and the show file
is open and being written to. As a result, if the program crashes for some
reason you will not be able to look at the results. To overcome this 
problem it is possible to automatically save multiple show files, with each
show file containing one or more solutions. The number of solutions saved
in each show file is determined by the frequency the show file is flushed.
Use the {\tt `frequency to flush'}
option to specify how many solutions should be saved in each show file.
The files are named `file.show', `file.show1', `file.show2' etc. where
`file.show' was the name given to the show file.
The plotStuff program will automatically read all these different files if
you just type `plotStuff file.show'.

It is thus possible to look at the solutions when cgins crashes or
while it is still running. 
Only the most recent solutions that
belong to the most recent (open) show file will be unavailable.
