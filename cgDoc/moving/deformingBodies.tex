\newcommand{\rhos}{\bar{\rho}}
\newcommand{\ds}{\Delta s}
\newcommand{\xs}{\bar{x}}
\newcommand{\vs}{\bar{v}}
\newcommand{\xsv}{\mathbf{\xs}}
\newcommand{\vsv}{\mathbf{\vs}}
\section{DeformingBodyMotion Class}\label{sec:deformingBodies}


The DeformingBodyMotion Class defines various types of deforming bodies.


\begin{description}
  \item[advect body]: advect the surface with the fluid velocity. Use this option for a free surface.
  \item[elastic shell]: define a curve in 2D with properties of an elastic shell, see Section~\ref{sec:elasticShell}.
\end{description}


% --------------------------------------------------------------------------------
\subsection{Elastic Shell} \label{sec:elasticShell}

% // (rhoe*ds) dv/dt = p ds n - ke*( x-x0 ) + te*(x-x0).ss - be*v 

Consider a two-dimensional elastic {\em shell} (represented as a curve in 2D) that is immersed in a fluid.
Let $\xsv\in\Real^2$ with $\xsv=\xsv(s,t)$, $s\in[0,1]$  denote the position
of a point on the shell and $\vsv=\vsv(s,t)=\xsv_t$ denote the velocity of the shell. Let $\xsv_0(s)=\xsv(s,0)$ denote the
initial position of the shell. 

The equations of motion for the elastic shell are
\begin{align}
  \rhos \vsv_{t} &= -K (\xsv - \xsv_0) + \partial_s( T_e \partial_s(\xsv - \xsv_0) ) 
                   - \partial_s^2( B_e \partial_s^2 (\xsv-\xsv_0)) 
                   + A_e \ds^2\vsv_{ss} 
                   - D_e \vsv  
                   + \Fv, \label{eq:elasticShell}\\
    \xsv_t & = \vsv, 
\end{align}
where $\rhos$ is the line density of the shell (mass per unit length), 
$K_e$ is the coefficient of stiffness, $T_e$ the coefficient of tension, 
$B_e$ is the bending rigidity (NOT implemented yet), 
$D_e$ the damping coefficient,
$A_e$ the coefficient of artificial dissipation, $\Fv=\Fv(s,t)$ is the external forcing. 

For example, if the shell were immersed in an incompressible fluid the forcing term could be $\Fv=p\nv - \mu(\grad\vv+\grad\vv^T)\nv$. 

Note that the initial state of the shell, $\xsv(s,0)=\xsv_0(s)$,  is assumed to be in equilibirum (an un-stressed state)
and this is why the term $\xsv-\xsv_0$ appears in the equations (Is this the correct thing to do??). 

\paragraph{Elastic shell boundary conditions:} The available boundary conditions (for an end point $s_b=0$, or $s_b=1$), 
\begin{alignat}{3}
    \xsv(0,t) &= \xsv(1,t) ,  \qquad&& \text{periodic}, \\
    \xsv(s_b,t) &= \gv_d(t) , \qquad&& \text{Dirichlet (pinned or specified motion)}, \\
    \xsv_s(s_b,t) &= \gv_n(t) \qquad&& \text{Neumann, (given slope)}, \\  
    \xsv_{ss}(s_b,t) &= 0,    \qquad&& \text{Neumann, (free boundary)}, \\
    \nv_b\cdot\vsv(s_b,t) &= 0,    \qquad&& \text{Slide boundary, $\nv_b$=normal to boundary face)}, 
\end{alignat}
Note: In the case that $B_e\ne 0$, we will need an addition condition at the boundary in order to
define a well-posed problem.


Suppose that the elastic shell is periodic and defines a closed curve (e.g. ellipse or circle) that encloses some volume that
is not part of the computation domain.
If the gas or liquid inside this enclosed volume is assumed to be incompressible the the enclosed
volume should remain constant over time. To enforce this we add an additional penalty term
to~\eqref{eq:elasticShell} to conserve the enclosed volume, see~\cite{KimHuangShinSung2012}. 
This forcing term is given by
\begin{align}
  \Fv_V &=  \frac{\beta_V}{\dt} (1-V(t)/V_0) + \frac{\beta_V}{\dt} \int_0^t (1-V(\tau)/V_0)\, d\tau
\end{align}
where $V(t)$ is the current volume and $V_0$ is the initial volume. These two terms can be viewed as a 
a PI control function (P=proportional, I=integral). 


