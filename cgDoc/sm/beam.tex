\clearpage
\subsection{Vibrating Beam} \label{sec:beam}


A beam can be modeled using the bulk Cgsm solvers. The results can be compared to the Euler-Bernoulli beam theory.
See {\tt cg/sm/runs/beam}. 

A standing wave solution to the Euler-Bernoulli beam equation is
\begin{align*}
 &  y(x) = a\, \sin( k x )\,\sin( \omega t) , 
\end{align*} 
where the frequency of vibration of an Euler-Bernoulli beam  is
\begin{align*}
 &  \omega = \sqrt{ \frac{E I k^4}{\rho b h} }
\end{align*}
We take
\begin{align*}
&  \rho=\lambda=\mu=1, \rightarrow \nu = \frac{\lambda}{2(\lambda+\mu)} =\frac{1}{4}, ~~E=2\mu(1+\nu) = \frac{5}{2}, \\
&  \frac{I}{b} = \frac{h^3}{12} .
\end{align*}
For
\begin{align*}
  h=.04 ~~ T=\frac{2\pi}{\omega} \approx 17.43, \\
  h=.02 ~~ T=\frac{2\pi}{\omega} \approx 8.717
\end{align*}


Figure~\ref{fig:beamEvolution} shows results from a calculation for $h=.04$ and $k=2\pi$. 
Note that the {\em fibers} (lies normal to the neutral axis of the beam) remain nearly straight, of constant length and
perpendicular to the neutral axis. These are consistent with the assumptions of Euler-Bernoulli beam theory.
The approximate 
period of vibrartion is $T=8.7$ which agrees well with the Euler-Bernoulli beam theory.


{
\newcommand{\figWidtha}{16.cm}
\newcommand{\trimfiga}[2]{\trimPlotb{#1}{#2}{.0}{.125}{.4}{.4}}
\newcommand{\figWidthb}{10.cm}
\newcommand{\trimfigb}[2]{\trimPlotb{#1}{#2}{.05}{.1}{.05}{.1}}
\begin{figure}[hbt]
 \begin{center}
%
\begin{tikzpicture}[scale=1]
  \useasboundingbox (0,.75) rectangle (16.,12);  % set the bounding box (so we have less surrounding white spa  
  \draw ( 0.0,0.0) node[anchor=south west,xshift=-4pt,yshift=+0pt] {\trimfiga{\smDocDir/fig/beam_ar25_t0_u_wireFrame}{\figWidtha}};
  \draw ( 0.0,4.0) node[anchor=south west,xshift=-4pt,yshift=+0pt] {\trimfiga{\smDocDir/fig/beam_ar25_t4p3_u_wireFrame}{\figWidtha}};
  \draw ( 0.0,8.0) node[anchor=south west,xshift=-4pt,yshift=+0pt] {\trimfiga{\smDocDir/fig/beam_ar25_t8p7_u_wireFrame}{\figWidtha}};
%
 \draw ( 0.,4) node[anchor=north west,draw,fill=white,thick,xshift=5pt,yshift=-5pt] {\scriptsize $u_1$, $t=0.0$};
 \draw ( 0.,8) node[anchor=north west,draw,fill=white,thick,xshift=5pt,yshift=-5pt] {\scriptsize $u_1$, $t=4.3$};
 \draw ( 0.,12) node[anchor=north west,draw,fill=white,thick,xshift=5pt,yshift=-5pt] {\scriptsize $u_1$, $t=8.7$};
% grid:
  \draw[step=1cm,gray] (0,0) grid (16,12.);
\end{tikzpicture}
\end{center}
\caption{Vibrating beam simulated using the bulk solid solvers.}
\label{fig:beamEvolution}
\end{figure}
}





\subsubsection{Beam theories}

Here are some notes on beam theories. 

Ref. Graff. 

Euler-Bernoulli beam  (small deflections, low frequency respsonse)
\[
  \rho A \partial_t^2 y + \partial_x^2\big( E I \partial_x^2 y\big) = q(x,t) 
\]


Large rotation Euler-Bernoulli beam (using the von Karman strains, ref. wikipedia)
\[
  \rho A \partial_t^2 y + \partial_x^2\big( E I \partial_x^2 y\big) 
           - \frac{3}{2} E A (\partial_x y)^2  \partial_x^2 y  = q(x,y) 
\]


Timoshenko beam theory (accounts for shearing effects, high frequency respsonse)
\begin{align*}
  \rho A \partial_t^2 y + G A \kappa\Big(  \partial_x \psi - \partial_x^2 y \Big)  &= q(x,y)  \\
  \rho I \partial_t^2 \psi + G A \kappa\Big( \psi - \partial_x y \Big)  - E I \partial_x^2 \psi &= q(x,y) 
\end{align*}
or upon eliminating $\psi$
\begin{align*}
    \frac{EI}{\rho A}\partial_x^4 y - \frac{I}{A}\Big(1 +\frac{E}{G\kappa}\Big)\partial_x^2\partial_t^2 y + \partial_t^2 y 
             + \frac{\rho I}{G A \kappa}\partial_t^4 y = ...
\end{align*}
