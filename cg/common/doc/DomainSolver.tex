%-----------------------------------------------------------------------
% DomainSolver documentation.
%-----------------------------------------------------------------------
\documentclass[11pt]{article}
\usepackage{times}  % for embeddable fonts, Also use: dvips -P pdf -G0

\input documentationPageSize.tex

\input{pstricks}\input{pst-node}
\input{colours}

\usepackage{amsmath}
\usepackage{amssymb}

\usepackage{verbatim}
\usepackage{moreverb}

\usepackage{graphics}    
\usepackage{epsfig}    
\usepackage{calc}
\usepackage{ifthen}
\usepackage{float}
% the next one cause the table of contents to disappear!
% * \usepackage{fancybox}

\usepackage{makeidx} % index
\makeindex
\newcommand{\Index}[1]{#1\index{#1}}



% ---- we have lemmas and theorems in this paper ----
\newtheorem{assumption}{Assumption}
\newtheorem{definition}{Definition}

\newcommand{\Overture}{{\bf Over\-ture\ }}
\newcommand{\ogenDir}{/home/henshaw/Overture/ogen}

\begin{document}

\input wdhDefinitions.tex

\def\comma  {~~~,~~}
\newcommand{\uvd}{\mathbf{U}}
\def\ud     {{    U}}
\def\pd     {{    P}}
\def\calo{{\cal O}}

\newcommand{\mbar}{\bar{m}}
\newcommand{\Rbar}{\bar{R}}
\newcommand{\Ru}{R_u}         % universal gas constant
% \newcommand{\Iv}{{\bf I}}
% \newcommand{\qv}{{\bf q}}
\newcommand{\Div}{\grad\cdot}
\newcommand{\tauv}{\boldsymbol{\tau}}
\newcommand{\thetav}{\boldsymbol{\theta}}
% \newcommand{\omegav}{\mathbf{\omega}}
% \newcommand{\Omegav}{\mathbf{\Omega}}

\newcommand{\Omegav}{\boldsymbol{\Omega}}
\newcommand{\omegav}{\boldsymbol{\omega}}
\newcommand{\cm}{{\rm cm}}

\newcommand{\sumi}{\sum_{i=1}^n}
% \newcommand{\half}{{1\over2}}
\newcommand{\dt}{{\Delta t}}

\def\ff {\tt} % font for fortran variables

\vspace{5\baselineskip}
\begin{flushleft}
{\Large
{\bf DomainSolver}: The CG Base Class PDE Solver \\
}
\vspace{2\baselineskip}
Kyle K. Chand  \\
William D. Henshaw  \\
Centre for Applied Scientific Computing  \\
Lawrence Livermore National Laboratory      \\
Livermore, CA, 94551.  \\
henshaw@llnl.gov \\
http://www.llnl.gov/casc/people/henshaw \\
http://www.llnl.gov/casc/Overture\\
\vspace{\baselineskip}
\today\\
\vspace{\baselineskip}
% UCRL-MA-134289

\vspace{4\baselineskip}

\noindent{\bf\large Abstract:}

This document describes the {\bf DomainSolver} class.
\end{flushleft}

\clearpage
\tableofcontents
% \listoffigures

\vfill\eject


\section{Introduction}

Here is a description of the class DomainSolver. This class is the base class for PDE solvers such 
as Cgins, Cgcns, Cgad and Cgasf. This document is currently under development. 


\input DomainSolverInclude.tex



% -------------------------------------------------------------------------------------------------
\vfill\eject
\bibliography{/home/henshaw/papers/henshaw}
\bibliographystyle{siam}


\printindex


\end{document}
