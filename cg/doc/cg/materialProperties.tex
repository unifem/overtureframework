\section{Material Properties: Notes}\label{sec:materialProperties}



In this section we describe how variable material properties are
handled. A variable material property (VMP) is a coefficient
appearing in the governing equations that depends on its
spatial location $\xv$. For example, in elasticity the 
solid density, $\rho(\xv)$, and Lam\'e parameters  $\mu(\xv)$ and $\lambda(\xv)$ can be functions of
space. For heat conduction, the solid density $\rho(x)$, heat capacity $C_p(\xv)$ and thermal conductivity
$k(\xv)$ can also be functions of space.


{\bf General notes:}
\begin{enumerate}
  \item The VMP's can be plotted from the run-time-dialog using the "plot material properties"
  option.
  \item The interface to setting VMP's should support adaptive mesh refinement. In this case
      we need to dynamically re-evaluate the material properties as new grids are added.
  \item The functions associated with the implementation of VMP's mainly appear in the files
   {\tt GridMaterialProperties.C} and {\tt bodyForcing.C}.
\end{enumerate}


There are currently three ways to define VMP's. 
\begin{enumerate}
  \item User defined material
properties can be assigned using the functions {\tt setupUserDefinedMaterialProperties()}
and {\tt userDefinedMaterialProperties(...)}
that are found in the file {\tt cg/common/src/userDefinedMaterialProperties.C}.
You can edit these files to add new options. 

  \item VMP's can also be defined through the "body forcing..." option. This
option allows one to define one or more regions (e.g. currently boxes, but in the future more
general geometry such as cylinders, spheres, etc.) 
and to associate different material values with each region.

  \item VMP's can be defined as twilight-zone functions for use with twilight-zone computations.
\end{enumerate}



Internally in the code, VMP's are stored in the {\tt GridMaterialProperty} class. There 
is a separate {\tt GridMaterialProperty} object for each component grid. 
Currently there are two ways that are used to store a VMP on a grid. The choice of
storage option can be specified at run time (and will depend on whether the underyling
implementation supports both options).
The most general
way is to save the material property at each point on the grid using a 4 dimensional array
of real values:
\begin{align*}
    \rho_\iv &= \rm{matVal}(i_1,i_2,i_3,0), \\
    \mu_\iv &= \rm{matVal}(i_1,i_2,i_3,1), \\
    \lambda_\iv &= \rm{matVal}(i_1,i_2,i_3,2). 
\end{align*}
This storage option is known as {\tt GridMaterialProperties::variableMaterialProperties}. 
An alternative approach that saves storage when the material properties consist
of a few regions where the values are constant over each region is
\begin{align*}
    \rho_\iv &= \rm{matVal}(0,\rm{matIndex}(i_1,i_2,i_3)), \\
    \mu_\iv &= \rm{matVal}(1,\rm{matIndex}(i_1,i_2,i_3)), \\
    \lambda_\iv &= \rm{matVal}(2,\rm{matIndex}(i_1,i_2,i_3)). 
\end{align*}
This option is known as {\tt GridMaterialProperties::variableMaterialProperties}.
Note that this second option only requires a single integer to be stored
at each grid point. 
The \rm{matIndex}  and \rm{matVal} arrays are accessed from the {\tt GridMaterialProperty} object
using {\tt getMaterialIndexArray()} and {\tt getMaterialValuesArray()}.

The available material properties for a given domain solver are specified in an vector of 
strings. For example, for cgsm, the material properties are set in {\tt SmParameters.C} as
\begin{verbatim}
  std::vector<aString> & materialPropertyNames = 
                         dbase.get<std::vector<aString> >("materialPropertyNames");
  materialPropertyNames.push_back("rho");
  materialPropertyNames.push_back("mu");
  materialPropertyNames.push_back("lambda");
\end{verbatim}
The order of the names in the vector indicates how they are stored in the \rm{matVal} array above, i.e. rho 
was added first in the {\tt materialPropertyNames} vector and also appears first in the \rm{matVal} array. 
It is further assumed that the default value for each material property can be found in the
Parameters class data base, for example,
\begin{verbatim}
  real & rho = parameters.dbase.get<real>("rho");
  real & mu = parameters.dbase.get<real>("mu");
  real & lambda = parameters.dbase.get<real>("lambda");
\end{verbatim}



