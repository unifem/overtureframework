
 
\newlength{\setParametersIncludeArgIndent}

 Here is a desciption of the menu options available for changing parameters. This main parameter menu appears
 when \OverBlown is run and is found in the {\tt OverBlown::\-set\-Parameters\-Interactively()} function.
\begin{description}
  \item[continue] choose this item to exit this menu and continue on to the run-time dialog.
  \item[time stepping parameters...] : open the time stepping parameters dialog
    \begin{description}
      \item[time stepping method] : Not all schemes work for all PDEs. 
        \begin{description}
          \item[forwardEuler] : For CNS Godunov
          \item[adamsBashforth2] : For INS.
          \item[adamsPredictorCorrector2] : For INS.
          \item[variableTimeStepAdamsPredictorCorrector] for CNS.
          \item[midpoint] : For INS.
          \item[implicit] : For INS. Treat the viscous terms implicitly. One may optinally specify that some
             grids are integrated explicitly and some implicitly (see {\tt choose grids for implicit}).
          \item[all speed implicit] : For All-speed-flow.
          \item[linearized all speed implicit] :  For All-speed-flow. Linearize the implicit equations so that
             the implicit matrix is only formed and factored every ?? steps.
        \end{description}
      \item[final time] : Integrate to this time.
      \item[cfl] : Set the {\tt cfl} parameter. 
         The maximum time step based on stability is scaled by this factor.
         By default {\tt cfl=.9}. 
      \item[dtMax] : Restrict the time step to be no larger than this value.
      \item[implicit factor] :This value in $[0.,1.]$ is used with the implicit time-stepping. A value
         of $.5$ will correspond to a 2nd-order Crank-Nicolson approach for the viscous terms,
        a value of $1.$ will be backward-Euler and a value of $0.$ will be forward-Euler. See 
        the the reference manual for more details.
      \item[recompute dt every] : The time step, dt,  is recomputed every time the solution is plotted/saved.
              In addition you may specify the maximum number of steps that will be taken
              before dt is recomputed. Use this if the solution is not plotted very often.
      \item[slow start time] : Ramp the time step $\Delta t$ from a small value (determined by slow start cfl)
         to it maximum value (as  determined by the {\tt cfl} parameter over this time interval. 
      \item[slow start cfl]: The initial time step for the slow start option is determined by this cfl value, default$=.25$..
    \end{description}
  
 \item[pde options...] open the pde options menu. The dialog which opens depends on which PDE was chosen and is described below.
 \item[initial conditions options...] open the initial conditions dialog. This dialog is under
      construction.
    \begin{description}
      \item[read from a show file]: read initial conditions from a show file. This can either be
        a show file generated from OverBlown or one that you have built yourself.
      \item[read from a restart file]: read initial conditions from a restart file.  
      \item[uniform state] : specify a uniform state.
     \end{description}
 \item[show file options...] open the showfile dialog.
    \begin{description}
        \item[show variables] : toggle on/off variables that should be saved in the show file.
        \item[mode] : specify the mode as compressed or uncompressed. A compressed file will be
         smaller (especially for AMR runs that create many grids) but a compressed file will not
         be readable by future versions of OverBlown. 
        \item[open] : open a show file. You will be prompted for the name.
        \item[close] : close the show file.
      \item[frequency to save] : By default the solution is saved in the show file
          as often as it is plotted according to {\tt 'times to plot'}. To save the solution less
          often set this integer value to be greater than 1. A value of 2 for example will save solutions
          every 2nd time the solution is plotted.
      \item[frequency to flush] : Save this many solutions in each show file so that multiple
        show files will be created (these are automatically handled by plotStuff). See section~(\ref{sec:flush})
        for why you might do this.  
     \end{description}
 \item[display parameters] : print current values for parameters.
 \item[output options...] open the output options dialog.
  \item[output options] : Here are the output options.
    \begin{description}
      \item[plot option-menu] :
        \begin{description}
          \item[plot and wait first time] :
          \item[plot with no waiting] :
          \item[plot and always wait] : 
          \item[no plotting] : do not plot. If you want to turn off all graphics you must choose
            this option and also run overBlown with the noplot option.
        \end{description}
      \item[output periodically to a file] : output data to a file at each time step
      \item[times to plot] : Specify the time interval between plotting (and saving in a show file).
      \item[show file options...] open the show file options dialog.
      \item[save a restart file] : save or do not save a restart file.
      \item[allow user defined output] : call the userDefinedOutput routine at every step.
      \item[times to plot] : change the time interval between plotting (and output).
      \item[check file cutoffs] : used internally for regression tests.
 %     \item[read restart file] : read from a restart file.
 %     \item[restart file name (rsf=)] : name of the restart file to read from.
    \end{description}
 \item[boundary conditions...] open the boundary condition options dialog.
        This dialog is under construction
  \item[twilight zone options...] : open the twilight zone (method of analytic solutions) dialog.
    \begin{description}
     \item[type] : specify the type of analytic solution
     \begin{description}
       \item[polynomial] : 
         \begin{description}
           \item[turn on polynomial] : Make the twilight-zone function be a polynomial.
           \item[degree in space] : 0,1, or 2
           \item[degree in time] : 0,1, or 2
         \end{description}
       \item[trigonometric]
         \begin{description}
           \item[turn on trigonometric] : Make the twilight-zone function be a trigonometric polymoinal.
           \item[frequencies] : arguments to the trig functions are $\Pi$ ? times the frequency
                specified here. 
         \end{description}
      \end{description}
    \item[twilight zone flow] : toggle on or off. When this option is on the equations are forced so that the true solution is
         equal to some analytically defined function. This is used to test the accuracy of the code.
     \item[use 2D function in 3D] : use a 2D analytic function in 3D .
     \item[compare 3D run to 2D] : make adjustments so that an extruded 3D geoemtry can be compared to a 3D computation.
     \item[degree in space] : degree of the spatial polynomial
     \item[degree in time] : degree of the temporal polynomial
     \item[frequencies (x,y,z,t)] : frequencies to use with the trigonometric analytic solution.
    \end{description}
  \item[plot the grid] : plot the grid.
  \item[project initial conditions] : (popup menu)
    \begin{description}
      \item[project initial conditions] : Project initial conditions to nearly satisfy $\grad\cdot\uv=0$. This
        option applies to INS and ASF.
      \item[do not project initial conditions] : (popup menu)
    \end{description}

  \item[time stepping options] : Here are options that affect the time step.
    \begin{description}
      \item[choose grids for implicit] : For use with the {\tt implicit} time stepping option. Choose 
         which grids to integrate implicitly and which to integrate explicitly. Normally one should choose
         thoses grids with fine grid spacing (such as in boundary layers) to be implicit while a back-ground
       grid could be explicit. See section~(\ref{sec:implicitMenu}).
    \end{description}
  \item[boundary conditions] : Brings up a new menu described in section~(\ref{sec:bcMenu}).
  \item[data for boundary conditions] : Brings up the sub-menu described in section~(\ref{sec:bcDataMenu}).
  \item[initial conditions] : Brings up a new menu described in section~(\ref{sec:icMenu}).
  \item[pde parameters] : This brings up a new menu described below in section~(\ref{sec:pdeParams}).
  \item[axisymmetric flow] : solve an axisymmetric problem with cylindrical symmetry.
    \begin{description}
      \item[turn on axisymmetric flow] : The solution is assumed to have cylindrical symmetry about
                 the axis $y=0$ with the grid defined only in the region $y\ge 0$.
      \item[turn off axisymmetric flow] :
      % \item[cylindrical axis is x axis] : axis of symmetry is x=0
      % \item[cylindrical axis is y axis] : axis of symmetry is y=0
    \end{description}
   \item[adaptive grids] : use adaptive mesh refinement.
     \begin{description}
       \item[turn on adaptive grids] 
       \item[turn off adaptive grids] 
       \item[error threshold]
       \item[truncation error coefficient] : 
       \item[order of AMR interpolation] : 
       \item[regrid frequency] : 
       \item[change adaptive grid parameters] : change AMR regridding parameters (class Regrid).
       \item[change error estimator parameters] : change parameters in the error estimator (class ErrorEstimator).
       \item[show amr error function] : add the error function used for AMR regridding to the items that
         can be plotted.
     \end{description}
  \item[Debugging] :
    \begin{description}
       \item[debug file options] : turn out various output to the debug file, ob.debug.
         \begin{description}
           \item[print solution/errors] : print solution (or errors if known) at each time.
           \item[check error on ghost] : also check errors of ghost points.
           \item[print classify array] : print the classify array for sparse coefficient matrixes.
           \item[print sparse matrix] : print the sparse matrix generated by Oges (big).
        \end{description}
      \item[debug (debug=)] : This is a bit flag that turns on various messages. The more bits turned
        on, the more detailed the messages that appear. Thus a value of debug=3 (1+2) would have the first
        2 bits turned on and would display few messages. A value of debug=63 (1+2+4+8+16+32) would have 6
        bits turned on and would results in a lot of information.
      \item[Oges::debug (od=)] : bit flag debug variable for Oges.
      \item[Reactions::debug (rd=)] :
      \item[compare 3D run to 2D] : this option will adjust the equations and forcing so
         that a 3D run on an extruded 2D grid can be compared to the 2D computation. This includes setting
         the twilight-zone function to be 2D and changing the divergence damping (INS) to be two-dimensional 
         (otherwise it is scaled in the wrong way).
    \end{description}
  \item[reduce interpolation width] : specify a new interpolation width. For example, when solving the
      inviscid Navier-Stokes equations one may want to use linear interpolation (width=2) instead of
      of quadratic interpolation (width=3) since this may reduce wiggles. 
      If the grid was built with width=3 interpolation you can
      reduce the order of interpolation with this option.
  \item[sparse solver options] :
    \begin{description}
      \item[pressure solver options] : Choosing this item will allow you to change any {\tt Oges} related
         parameters as they apply to the elliptic equation for the pressure. See the {\tt Oges} documentation
         for a description of these parameters~\cite{OGES}.
      \item[implicit time step solver options] :Choosing this item will allow you to change any {\tt Oges} related
         parameters as they apply to the mplicit time stepping equations. See the {\tt Oges} documentation
         for a description of these parameters~\cite{OGES}.
    \end{description}
  \item[moving grids] : Options related to moving grids.
    \begin{description}
      \item[turn on moving grids] : Allow grids to move.
      \item[turn off moving grids] : do not allow grids to move.
      \item[specify grids to move] : indicate which grids move and how. You must also choose 
            {\tt`turn on moving grids'} if you really want these grids to move. See section~(\ref{sec:moveMenu}).
      \item[detect collisions] : detect collisions for some types of rigid bodies (wip)
      \item[do not detect collisions] : turn off collision detection.
      \item[minimum separation for collisions] : minimum allowed distance between colliding bodies. 
          This distance is in grid lines and should be chosen large enough so that a valid grid
          can still be generated. Usually value will be from 2 to 3 but may need to be more for
          some grids.
    \end{description}
  \item[plot the grid] : plot the grid. Useful to see if boundary conditions have been plotted correctly.
  \item[erase] : erase the graphics screen.
  \item[exit] : exit this menu and continue on (same as 'continue').
 \end{description}

 \subsubsection{Show file options}
   Here are the options related to show files, these options are from the {\tt updateShowFile} function
   in the {\tt OB\_Parameters} class.
 
 \input ShowFileOptionsInclude.tex

\subsubsection{Choosing grids for implicit time stepping.}\label{sec:implicitMenu}
 

  When the option {\tt `choose grids for implicit'} is chosen from the main parameter menu one can
  specify which grids should be treated implicitly or explicitly with the {\bf implicit} time stepping
  option. Type a line of the form
  \begin{verbatim}
           <grid name>=[explicit][implicit] 
  \end{verbatim}
  where {\tt <grid name>} is the name of a grid or {\tt `all'}. Type {\tt `help'} to see the names.
  Examples:
  \begin{verbatim}
       square=explicit
       all=implicit
       cylinder=implicit
  \end{verbatim}
  Type {\tt `done'} when finished.
\subsubsection{Initial condition options}\label{sec:icMenu}
 

 Here are the options for specifying initial conditions.
 This menu appears when {\tt `initial conditions'} is chosen from main parameter menu.
\begin{description}
  \item[uniform flow] : specify a uniform flow. Enter values in the form {\tt `p=1., u=2., ...'}.
     Variables not specified will get default values (usually zero).
  \item[step function] : specify two uniform conditions separted by a step
  \item[read from a show file] : read the initial conditions from a solution in a show file.
  \item[read from a restart file] : read the initial conditions from a solution in a restart file.
 \end{description}

\subsubsection{PDE parameters for INS}\label{sec:pdeParams}
 

 Here are the pde parameters that can be changed when solving the incompressible Navier-Stokes equations.
 This menu appears when {\tt `pde parameters'} is chosen from main menu.
\begin{description}
  \item[nu] : kinematic viscosity (constant).
  \item[divergence damping]
  \item[artificial diffusion] : see section~(\ref{AD}) for a description of the artificial diffusion terms.
    \begin{description}
      \item[second order artifical diffusion]
        \begin{description}
          \item[turn on second order artificial diffusion]
          \item[turn off second order artificial diffusion]
          \item[ad21 : coefficient of linear term]
          \item[ad22 : coefficient of non-linear term]
        \end{description}
      \item[fourth order artificial diffusion]
        \begin{description}
          \item[turn on fourth order artificial diffusion]
          \item[turn off fourth order artificial diffusion]
          \item[ad41 : coefficient of linear term]
          \item[ad42 : coefficient of non-linear term]
        \end{description}
    \end{description}
  \end{description}

\subsubsection{PDE parameters for CNS}\label{sec:cnsPdeParams}
 

 Here are the pde parameters that can be changed when solving the compressible Navier-Stokes equations.
 This menu appears when {\tt `pde parameters'} is chosen from main menu and you are solving the compressible
 Navier-Stokes equations. Normally one would specify either the {\tt Mach number} and {\tt Reynolds number}
  or alternatively one could specify values for {\tt mu}, and ...
\begin{description}
  \item[Mach number] : global Mach number.
  \item[Reynolds number] : global Reynolds number.
  \item[mu] : viscosity (currently constant)
  \item[Prandtl number] : 
  \item[kThermal] : thermal conductivity (currently constant).
  \item[Rg] : gas constant
  \item[gamma] : ratio of specific heats.
  \item[gravity] : a vector specifying the acceleration per unit mass due to gravity.
  \item[algorithms] :
    \begin{description}
      \item[conservative with artificial dissipation]: Use conservative differencing with a Jameson style
            artificial dissipation that mixes a second-order and fourth order dissipation.
      \item[non-conservative]: use a centered non-conservative scheme, not recommended if you have un-resolved
            shocks.
      \item[conservative Godunov] : Use a conservative Godunov Scheme by Don Schwendeman
    \end{description}
  \end{description}

\subsubsection{PDE parameters for ASF}\label{sec:asfPdeParams}
 

 Here are the pde parameters that can be changed when solving the all-speed flow version of the
  compressible Navier-Stokes equations.
 This menu appears when {\tt `pde parameters'} is chosen from main menu and you are solving the 
  {\tt allSpeedNavierStokes}. 
 Normally one would specify either the {\tt Mach number} and {\tt Reynolds number}
  or alternatively one could specify values for {\tt mu}, and ...
\begin{description}
  \item[Mach number] : global Mach number.
  \item[Reynolds number] : global Reynolds number.
  \item[mu] : viscosity (currently constant)
  \item[Prandtl number] : 
  \item[kThermal] : thermal conductivity (currently constant).
  \item[Rg] : gas constant
  \item[gamma] : ratio of specific heats.
  \item[gravity] : a vector specifying the acceleration per unit mass due to gravity.
  \item[nuRho] :
  \item[pressure level] : the constant background level of the pressure, normally determined automatically
     from the Mach number.
  \item[remove fast pressure waves (toggle)] : remove the $p_{tt}$ term from the pressure equation to
       eliminate sound waves with a fast time scale.
 \end{description}

\subsubsection{PDE parameters for INS}\label{sec:pdeParams}
 

 Here are the pde parameters that can be changed when solving the incompressible Navier-Stokes equations.
 This menu appears when {\tt `pde parameters'} is chosen from main menu.
\begin{description}
  \item[nu] : kinematic viscosity (constant).
  \item[divergence damping]
  \item[artificial diffusion] : see section~(\ref{AD}) for a description of the artificial diffusion terms.
    \begin{description}
      \item[second order artifical diffusion]
        \begin{description}
          \item[turn on second order artificial diffusion]
          \item[turn off second order artificial diffusion]
          \item[ad21 : coefficient of linear term]
          \item[ad22 : coefficient of non-linear term]
        \end{description}
      \item[fourth order artificial diffusion]
        \begin{description}
          \item[turn on fourth order artificial diffusion]
          \item[turn off fourth order artificial diffusion]
          \item[ad41 : coefficient of linear term]
          \item[ad42 : coefficient of non-linear term]
        \end{description}
    \end{description}
  \end{description}

\subsubsection{PDE parameters for CNS}\label{sec:cnsPdeParams}
 

 Here are the pde parameters that can be changed when solving the compressible Navier-Stokes equations.
 This menu appears when {\tt `pde parameters'} is chosen from main menu and you are solving the compressible
 Navier-Stokes equations. Normally one would specify either the {\tt Mach number} and {\tt Reynolds number}
  or alternatively one could specify values for {\tt mu}, and ...
\begin{description}
  \item[Mach number] : global Mach number.
  \item[Reynolds number] : global Reynolds number.
  \item[mu] : viscosity (currently constant)
  \item[Prandtl number] : 
  \item[kThermal] : thermal conductivity (currently constant).
  \item[Rg] : gas constant
  \item[gamma] : ratio of specific heats.
  \item[gravity] : a vector specifying the acceleration per unit mass due to gravity.
  \item[algorithms] :
    \begin{description}
      \item[conservative with artificial dissipation]: Use conservative differencing with a Jameson style
            artificial dissipation that mixes a second-order and fourth order dissipation.
      \item[non-conservative]: use a centered non-conservative scheme, not recommended if you have un-resolved
            shocks.
      \item[conservative Godunov] : Use a conservative Godunov Scheme by Don Schwendeman
    \end{description}
  \end{description}

\subsubsection{PDE parameters for ASF}\label{sec:asfPdeParams}
 

 Here are the pde parameters that can be changed when solving the all-speed flow version of the
  compressible Navier-Stokes equations.
 This menu appears when {\tt `pde parameters'} is chosen from main menu and you are solving the 
  {\tt allSpeedNavierStokes}. 
 Normally one would specify either the {\tt Mach number} and {\tt Reynolds number}
  or alternatively one could specify values for {\tt mu}, and ...
\begin{description}
  \item[Mach number] : global Mach number.
  \item[Reynolds number] : global Reynolds number.
  \item[mu] : viscosity (currently constant)
  \item[Prandtl number] : 
  \item[kThermal] : thermal conductivity (currently constant).
  \item[Rg] : gas constant
  \item[gamma] : ratio of specific heats.
  \item[gravity] : a vector specifying the acceleration per unit mass due to gravity.
  \item[nuRho] :
  \item[pressure level] : the constant background level of the pressure, normally determined automatically
     from the Mach number.
  \item[remove fast pressure waves (toggle)] : remove the $p_{tt}$ term from the pressure equation to
       eliminate sound waves with a fast time scale.
 \end{description}

