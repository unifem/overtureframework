\section{Projecting the initial
            conditions}  \label{sec:projectIC}
 
In this section we discuss the {\tt ``project initial conditions''} option
that is often used with Cgins in order to adjust the user specfied
initial conditions. 

 
The equation $\grad\cdot\uv=0$ imposes some compatibility constraints
on the initial and boundary conditions. For example,
the initial conditions should
satisfy $\grad\cdot\uv_0=0$. Imposing the divergence free condition
up to the boundary implies that
 that the normal component of
$\uv_0$ should equal the normal component of the velocity specified
on the boundary, $\nv\cdot\uv_0=\nv\cdot\uv_{\partial\Omega}$,
see~\cite{Gresho92}. A further constraint follows by integrating
$\grad\cdot\uv$ over $\Omega$ and using the
Gauss divergence theorem, giving
$\int_{\partial \Omega} \nv\cdot\uv ds = 0$.
 
In many cases it may not be easy to generate initial conditions that
satisfy the compatibility conditions. It is, however, well known how
to take a given function $\uv_I$ and to project this
function so that it is divergence free.
This projection is defined by
\begin{eqnarray*}
        \uv_0 &=& \uv_I + \grad \phi  ~~~\xv\in \Omega  \\
        \uv_0 &=& \uv_I               ~~~\xv\in \partial\Omega  \\
  \Delta \phi &=& - \grad\cdot\uv_I ~~~\xv\in\Omega\\
   \nv\cdot\grad\phi &=& 0          ~~~\xv\in\partial\Omega
\end{eqnarray*}
The function $\uv_0$ will satisfy
 $\grad\cdot\uv_0=0$ and $\nv\cdot\uv_0$
will equal $\nv\cdot\uv_I$ on the boundary.
Note, however, that this projection does not force the
tangential components of $\uv_0$ to be continuous at the boundary.
 
In the discrete case this projection is easy to compute since the
function $\phi$ satisfies the same equation as the pressure, but with
different data.  However, the discrete projection operator does not
make the discrete divergence exactly zero for the same reason that the
discrete divergence is not exactly zero in the overall scheme.  Thus
if the initial function, $\uv_I$, is not smooth then it may be
necessary to first smooth $\uv_I$ before applying the projection.  In
practice a sequence of smoothing and projection steps is applied until
the discrete divergence nolonger decreases significantly.