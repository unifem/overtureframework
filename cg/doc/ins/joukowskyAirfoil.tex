

% ================================================================================================================
\subsection{Flow past an airfoil at different Reynolds numbers}\label{sec:airfoilWithDifferentReynolds}


We simulate the flow past an airfoil using the second- and fourth-order accurate versions of the AFS scheme.
The second-order and fourth-order SSLES turbulence models are used. The results indicate how the flow
character changes as the effective Reynolds number increases.

% 
{
\newcommand{\figWithCaption}[4]{
\begin{scope}[yshift=#1cm]
  \draw ( 0.0,0.0) node[anchor=south west,xshift=-4pt,yshift=+0pt] {\trimfiga{#2}{\figWidtha}};
  \draw ( 8.0,.10) node[anchor=south west,xshift=-4pt,yshift=+0pt] {\trimfigb{#3}{\figWidthb}};
  \draw ( 0.0,0.3 ) node[draw,fill=white,anchor=south west,xshift=+1pt,yshift=-4pt] {\scriptsize #4};
\end{scope}
}% end figWithCaption
% 
\newcommand{\figWidtha}{7.5cm}
\newcommand{\figWidthb}{8.0cm}
\newcommand{\trimfiga}[2]{\trimPlotb{#1}{#2}{.15}{.15}{.3}{.3}}
\newcommand{\trimfigb}[2]{\trimPlotb{#1}{#2}{.0}{.12}{.275}{.275}}
% -----------------------------------------------------------------------------------------------------------------------------------------
\begin{figure}[hbt]
\begin{center}
\begin{tikzpicture}[scale=1]
  \useasboundingbox (0,.75) rectangle (16.,9.3);  % set the bounding box (so we have less surrounding white space)
%
\figWithCaption{4.75}{\cgDoc/ins/fig/joukowsky4Order2NoMoveVort10p0}{\cgDoc/ins/fig/joukowsky4Order2NoMoveVortCloseup10p0}{AFS2, $\Gc^{(4)}$, vorticity $[-30,30]$}
%
\figWithCaption{0}{\cgDoc/ins/fig/joukowsky4Order4NoMoveVort10p0}{\cgDoc/ins/fig/joukowsky4Order4NoMoveVortCloseup10p0}{AFS4, $\Gc^{(4)}$, vorticity $[-30,30]$}
%
 % \draw (current bounding box.south west) rectangle (current bounding box.north east);
% grid:
% \draw[step=1cm,gray] (0,0) grid (16,9.);
\end{tikzpicture}
\end{center}
 \caption{Flow past an airfoil on grid $\Gc^{(4)}$. Contour plots of the vorticity at $t=10$. 
         }
\end{figure}
% -----------------------------------------------------------------------------------------------------------------------------------------
\begin{figure}[hbt]
\begin{center}
\begin{tikzpicture}[scale=1]
  \useasboundingbox (0,.75) rectangle (16.,9.3);  % set the bounding box (so we have less surrounding white space)
%
\figWithCaption{4.75}{\cgDoc/ins/fig/joukowsky8Order2NoMoveVort10p0}{\cgDoc/ins/fig/joukowsky8Order2NoMoveVortCloseup10p0}{AFS2, $\Gc^{(8)}$, vorticity $[-30,30]$}
%
\figWithCaption{0}{\cgDoc/ins/fig/joukowsky8Order4NoMoveVort10p0}{\cgDoc/ins/fig/joukowsky8Order4NoMoveVortCloseup10p0}{AFS4, $\Gc^{(8)}$, vorticity $[-30,30]$}
%
 % \draw (current bounding box.south west) rectangle (current bounding box.north east);
% grid:
% \draw[step=1cm,gray] (0,0) grid (16,9.);
\end{tikzpicture}
\end{center}
 \caption{Flow past an airfoil on grid $\Gc^{(8)}$. Contour plots of the vorticity at $t=10$. 
         }
\end{figure}
% -----------------------------------------------------------------------------------------------------------------------------------------
\begin{figure}[hbt]
\begin{center}
\begin{tikzpicture}[scale=1]
  \useasboundingbox (0,.75) rectangle (16.,9.3);  % set the bounding box (so we have less surrounding white space)
%
\figWithCaption{4.75}{\cgDoc/ins/fig/joukowsky16Order2NoMoveVort10p0}{\cgDoc/ins/fig/joukowsky16Order2NoMoveVortCloseup10p0}{AFS2, $\Gc^{(16)}$, vorticity $[-30,30]$}
%
\figWithCaption{0}{\cgDoc/ins/fig/joukowsky16Order4NoMoveVort10p0}{\cgDoc/ins/fig/joukowsky16Order4NoMoveVortCloseup10p0}{AFS4, $\Gc^{(16)}$, vorticity $[-50,50]$}
%
 % \draw (current bounding box.south west) rectangle (current bounding box.north east);
% grid:
% \draw[step=1cm,gray] (0,0) grid (16,9.);
\end{tikzpicture}
\end{center}
 \caption{Flow past an airfoil  on grid $\Gc^{(16)}$. Contour plots of the vorticity at $t=10$. 
         }
\end{figure}
% -----------------------------------------------------------------------------------------------------------------------------------------------
}

% -----------------------------------------------------------------------------------
The simulations used the command file {\tt cg/ins/cmd/wing2d.cmd}
and the grid was made with the ogen script {\tt Overture/sampleGrids/joukowsky2d.cmd}. 
AFS2 denotes the second-order accurate AFS scheme and AFS4 is the fourth-order accurate. 
Grid $\Gc^{(j)}$ has a background grid spacing of about $1/(20 j)$. The chord length of
th airfoil is 1. 



%- {
%- \begin{figure}[H]
%- \psset{xunit=1.cm,yunit=1.cm,runit=1.cm}%
%- \newcommand{\figWidthd}{8cm}% 
%- \newcommand{\clipfigd}[2]{\clipFig{#1}{#2}{.0}{1.}{0.17}{1.}}
%- \begin{center}%
%- \begin{pspicture}(0,0)(17.,18)%
%-  %\psgrid[subgriddiv=2]
%-  \rput(4.00,14.0){\clipfigd{\ovFigures/joukowskyPitchPlunge200.ps}{\figWidthd}}
%-  \rput(4.00, 8.0){\clipfigd{\ovFigures/joukowskyPitchPlunge225.ps}{\figWidthd}}
%-  \rput(4.00, 2.0){\clipfigd{\ovFigures/joukowskyPitchPlunge250.ps}{\figWidthd}}
%-  \rput(12.2,14.0){\clipfigd{\ovFigures/joukowskyPitchPlunge275.ps}{\figWidthd}}
%-  \rput(12.2, 8.0){\clipfigd{\ovFigures/joukowskyPitchPlunge300.ps}{\figWidthd}}
%-  \rput(12.2, 2.0){\clipfigd{\ovFigures/joukowskyPitchPlunge325.ps}{\figWidthd}}
%- \end{pspicture}%
%- \end{center}%
%- \caption{A pitching and plunging airfoil, computed in parallel with Cgins. Contour plots of the vorticity.}
%- \end{figure}
%- }



% ========================================================================================================
\clearpage
\subsection{Flow past a pitching-plunging airfoil at different Reynolds numbers}\label{sec:pitchingPlungingAirfoilWithDifferentReynolds}


We simulate the flow past a pitching-plunging airfoil using the second- and
fourth-order accurate versions of the AFS scheme.  The second-order and
fourth-order SSLES turbulence models are used. The results indicate how the flow
character changes as the effective Reynolds number increases.
{%%%
% 
{
\newcommand{\figWithCaption}[4]{
\begin{scope}[yshift=#1cm]
  \draw ( 0.0,0.0) node[anchor=south west,xshift=-4pt,yshift=+0pt] {\trimfiga{#2}{\figWidtha}};
  \draw ( 8.0,.10) node[anchor=south west,xshift=-4pt,yshift=+0pt] {\trimfigb{#3}{\figWidthb}};
  \draw ( 0.0,0.3 ) node[draw,fill=white,anchor=south west,xshift=+1pt,yshift=-4pt] {\scriptsize #4};
\end{scope}
}% end figWithCaption
\newcommand{\figWidtha}{7.5cm}
\newcommand{\figWidthb}{7.0cm}
\newcommand{\trimfiga}[2]{\trimPlotb{#1}{#2}{.15}{.15}{.27}{.27}}
\newcommand{\trimfigb}[2]{\trimPlotb{#1}{#2}{.0}{.05}{.15}{.2}}
% -----------------------------------------------------------------------------------------------------------------------------------------
\begin{figure}[hbt]
\begin{center}
\begin{tikzpicture}[scale=1]
  \useasboundingbox (0,.75) rectangle (16.,9.3);  % set the bounding box (so we have less surrounding white space)
%
% \figWithCaption{4.75}{\cgDoc/ins/fig/joukowsky16Order2NoMoveVort10p0}{\cgDoc/ins/fig/joukowsky16Order2NoMoveVortCloseup10p0}{AFS2, $\Gc^{(16)}$, vorticity $[-30,30]$}
%
\figWithCaption{0}{\cgDoc/ins/fig/joukowsky8Order4MoveVort5p0}{\cgDoc/ins/fig/joukowsky8Order4MoveVortCloseup5p0}{AFS4, $\Gc^{(8)}$, vorticity $[-150,150]$}
%
 % \draw (current bounding box.south west) rectangle (current bounding box.north east);
% grid:
%% \draw[step=1cm,gray] (0,0) grid (16,9.);
\end{tikzpicture}
\end{center}
 \caption{Flow past a pitching-plunging airfoil on grid $\Gc^{(8)}$. Contour plots of the vorticity at $t=5$. 
         }
\end{figure}
% ----------------------------------------------------------------------------------------------------------------------------------------------
% 
}%%%
