% ========================================================================================================
\clearpage
\subsection{Flow past a moving sphere}\label{sec:movingSphere}


We simulate the flow past a moving sphere.

The grid for this problem was generated from the ogen command file {\tt sphereInABox.cmd}.
The solution was computed with the Cgins command file {\tt cg/ins/cmd/sphereMove.cmd}.

The geometry for the problem consists of a sphere of radius $r=1.0$, centered at the 
origin, in a channel $[-3,6]\times[-3,3]\times[-3,3]$. 
Let $\Gc^{(j)}$ denote the composite grid for this geometry. The target grid spacing is $\ds=1/(10 j)$.
The grid spacing is stretched in the normal direction to the cylinder so that the boundary layer
spacing is $\dsbl$. 




Figure~\ref{fig:sphereOscillate} shows the solution for a sphere that oscillates up and down
in the y-direction with a sinusoidal motion given by 
\begin{align*}
   y(t) &= a_0 \sin( 2\pi f_0 t) ,
\end{align*}
where the amplitude is taken as $a_0=0.25$ and the frequency is $f_0=0.5$.
The incoming flow is in the $x$-direction with $u=1$.
%The grid is $\Gc^{(2)}$ and the boundary layer spacing is 5 times smaller than the target grid spacing, with $\dsbl/\ds=1/5$.
%Contours of the enstrophy $\xi$, (magnitude of the vorticity vector, $\xi=\| \grad\times \uv\|$) are shown.
%The solution was computed with the scheme AFS4 and the SSLES4 turbulence model ($\nu=10^{-3}$). 
%The results show that the flow develops into a fully three-dimensional wake with vortices being shed.


{%%%
% 
{
\newcommand{\figWithCaption}[5]{
\begin{scope}[yshift=#1cm]
  \draw ( 0.0,0.0) node[anchor=south west,xshift=-4pt,yshift=+0pt] {\trimfiga{#2}{\figWidtha}};
  \draw ( 8.0,.0) node[anchor=south west,xshift=-4pt,yshift=+0pt] {\trimfiga{#3}{\figWidtha}};
  \draw ( 0.0,0.3 ) node[draw,fill=white,anchor=south west,xshift=+1pt,yshift=-4pt] {\scriptsize #4};
  \draw ( 8.0,0.3 ) node[draw,fill=white,anchor=south west,xshift=+1pt,yshift=-4pt] {\scriptsize #5};
\end{scope}
}% end figWithCaption
\newcommand{\figWidtha}{7.5cm}
\newcommand{\trimfiga}[2]{\trimPlotb{#1}{#2}{.1}{.05}{.1}{.15}}
% -----------------------------------------------------------------------------------------------------------------------------------------
\begin{figure}[hbt]
\begin{center}
\begin{tikzpicture}[scale=1]
  \useasboundingbox (0,.75) rectangle (16.,14);  % set the bounding box (so we have less surrounding white space)
%
cp \figWithCaption{7.0}{./fig/sphereInABoxGrid}{./fig/sphereMoveEnstrophyt14p0}{Grid $\Gc^{(2)}$}{$t=14$}
%
\figWithCaption{0}{./fig/sphereMoveEnstrophyt15p0}{./fig/sphereMoveEnstrophyt16p0}{$t=15$}{$t=16$}
%
 % \draw (current bounding box.south west) rectangle (current bounding box.north east);
% grid:
% \draw[step=1cm,gray] (0,0) grid (16,14);
\end{tikzpicture}
\end{center}
 \caption{Flow past an oscillating sphere using scheme IM2 on grid $\Gc^{(4)}$ with $\nu=2\times10^{-4}$. Contour plots of the enstrophy.
  The contour levels are scaled to $[0,??]$. The cylinder is moving the $y-direction$. }
  \label{fig:sphereOscillate}
\end{figure}
% -----------------------------------------------------------------------------------------------------------------------------------------------
%
}%%%
