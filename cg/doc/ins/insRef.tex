%-----------------------------------------------------------------------
% Cgins: Incompressible Navier Stokes Solver
% 
%        Reference Manual
% 
%-----------------------------------------------------------------------
\documentclass[12pt]{article}
\usepackage{times}  % for embeddable fonts, Also use: dvips -P pdf -G0

\input documentationPageSize.tex

\input{pstricks}\input{pst-node}
\input{colours}

\usepackage{amsmath}
\usepackage{amssymb}

\usepackage{verbatim}
\usepackage{moreverb}

\usepackage{graphics}    
\usepackage{epsfig}    
\usepackage{calc}
\usepackage{ifthen}
\usepackage{float}
% the next one cause the table of contents to disappear!
% * \usepackage{fancybox}

\usepackage{makeidx} % index
\makeindex
\newcommand{\Index}[1]{#1\index{#1}}



% ---- we have lemmas and theorems in this paper ----
\newtheorem{assumption}{Assumption}
\newtheorem{definition}{Definition}

\newcommand{\Overture}{{\bf Over\-ture\ }}
\newcommand{\ogenDir}{/home/henshaw/Overture/ogen}
\newcommand{\obFigures}{/home/henshaw/res/OverBlown/docFigures}  % for figures
\newcommand{\convDir}{.}

\begin{document}

\input wdhDefinitions.tex

\def\comma  {~~~,~~}
\newcommand{\uvd}{\mathbf{U}}
\def\ud     {{    U}}
\def\pd     {{    P}}
\def\calo{{\cal O}}

\newcommand{\mbar}{\bar{m}}
\newcommand{\Rbar}{\bar{R}}
\newcommand{\Ru}{R_u}         % universal gas constant
% \newcommand{\Iv}{{\bf I}}
% \newcommand{\qv}{{\bf q}}
\newcommand{\Div}{\grad\cdot}
\newcommand{\tauv}{\boldsymbol{\tau}}
\newcommand{\thetav}{\boldsymbol{\theta}}
% \newcommand{\omegav}{\mathbf{\omega}}
% \newcommand{\Omegav}{\mathbf{\Omega}}

\newcommand{\Omegav}{\boldsymbol{\Omega}}
\newcommand{\omegav}{\boldsymbol{\omega}}
\newcommand{\sigmav}{\boldsymbol{\sigma}}
\newcommand{\cm}{{\rm cm}}

\newcommand{\sumi}{\sum_{i=1}^n}
% \newcommand{\half}{{1\over2}}
\newcommand{\dt}{{\Delta t}}

\def\ff {\tt} % font for fortran variables

\vspace{5\baselineskip}
\begin{flushleft}
{\Large
{\bf Cgins}: A Solver for the Incompressible Navier--Stokes Equations \\
    Reference Manual \\
}
\vspace{2\baselineskip}
William D. Henshaw  \\
Centre for Applied Scientific Computing  \\
Lawrence Livermore National Laboratory      \\
Livermore, CA, 94551.  \\
henshaw@llnl.gov \\
http://www.llnl.gov/casc/people/henshaw \\
http://www.llnl.gov/casc/Overture\\
\vspace{\baselineskip}
\today\\
\vspace{\baselineskip}
% UCRL-MA-134289

\vspace{4\baselineskip}

\noindent{\bf\large Abstract:}

This document describes {\bf Cgins}, a solver written using the \Overture framework
to solve the incompressible Navier-Stokes (INS).  
Cgins can be used to the solve time-dependent Navier-Stokes equations to
second and fourth-order accuracy. There is also a pseudo-steady line implicit solver with
a nonlinear second- or fourth-order artificial dissipation.

\end{flushleft}

\clearpage
\tableofcontents
% \listoffigures

\vfill\eject


\section{Introduction}

This document is currently under development. 


Cgins solves the incompressible Navier-Stokes equations
on overlapping grids and is built upon the \Overture 
framework~\cite{Brown97},\cite{Henshaw96a},\cite{iscope97}. 



\section{The Equations}


The incompressible Navier-Stokes equations are
\begin{align}
   \uv_t + (\uv\cdot\grad)\uv + \grad p &= \nu \Delta \uv, \\
   \grad\cdot\uv &= 0.
\end{align}

We solve the incompressible Navier-Stokes equations written in the
form (\Index{pressure-poisson system})
\begin{eqnarray}
 \left. \begin{array}{rcl}
  \uv_t + (\uv\cdot\grad)\uv + \grad p -\nu \Delta \uv -\fv &=&0
                                                \\
  \Delta p -(\grad u\cdot\uv_x + \grad v\cdot\uv_y + \grad w\cdot \uv_z)
     -C_d(\nu) \grad\cdot\uv    - \grad\cdot\fv  &=&0
        \end{array}\right\} && \xv\in\Omega       \\
 \left. \begin{array}{rcl}
        B(\uv,p) &=& 0   \\
   \grad\cdot\uv &=& 0
        \end{array}\right\} && \xv\in\partial\Omega  \nonumber \\
   \uv(\xv,0)  =  \uv_0(\xv)   ~&&~~~~\mbox{ at } t=0  \nonumber
\end{eqnarray}
There are $n_d$ boundary conditions, $B(\uv,p)=0$, where $n_d$ is
the number of space dimensions. On a no-slip wall, for example,
$\uv=0$. In addition, a boundary condition is required for the
pressure. The boundary condition $\grad\cdot\uv=0$ is added.
With this extra boundary condition it follows that the above problem
is equivalent to the formulation with the Poisson equation for
the pressure replaced by $\grad\cdot\uv=0$ everywhere. The term
$C_d(\nu) \grad\cdot\uv$ appearing in the equation for the
pressure is used to damp the divergence~\cite{INSDIV}.
For further details see also~\cite{ICNS}


\newcommand{\Jc}{{\mathcal J}}
\subsection{Addition of Temperature and Buoyancy: The Boussinesq Approximation}


The effects of temperature and buoyancy can be modeled with the Boussinesq approximation
given by
\begin{align*}
  \uv_t + (\uv\cdot\grad)\uv + \grad p -\nu \Delta \uv +\alpha \gv T  -\fv &=0 ,\\
  \Delta p - \Jc(\grad\uv) -\alpha (\gv\cdot\grad) T - \grad\cdot\fv &=0,  \\
  T_t + (\uv\cdot\grad) T - k \Delta T - f_T &=0 , \\
  \Jc(\grad\uv) \equiv (\grad u\cdot\uv_x + \grad v\cdot\uv_y + \grad w\cdot \uv_z) &~.
\end{align*}
Here $T$ represents a temperature perturbation, $\gv$ is the acceleration due to gravity,
$\alpha$ is the coefficient of thermal expansivity and $k$ is a coefficient of
thermal conductivity, $k=\lambda/(\rho C_p)$. 


The Rayleigh number and Prandtl number are defined
as
\[
   Ra = \alpha g \Delta T d^3 /(\nu\kappa), \qquad Pr = \nu/\kappa 
\]
where $\Delta T$ is the representative temperature difference and $d$ is a length scale. 



\subsection{A Visco-Plastic Flow Model}

A model for visco-plastic flows (with possible buoyancy effects) is
\begin{align*}
  \uv_t + (\uv\cdot\grad)\uv + \grad p - \grad\cdot \sigmav' +\alpha \gv T  -\fv &=0 ,\\
  \Delta p - \Jc(\grad\uv) -\grad\cdot(\grad\cdot \sigmav') 
                -\alpha (\gv\cdot\grad) T - \grad\cdot\fv &=0,  \\
  T_t + (\uv\cdot\grad) T - k \Delta T - f_T &=0 , \\
  \Jc(\grad\uv) \equiv (\grad u\cdot\uv_x + \grad v\cdot\uv_y + \grad w\cdot \uv_z) &~.
\end{align*}
Here $T$ represents a temperature perturbation, $\gv$ is the acceleration due to gravity,
$\alpha$ is the coefficient of thermal expansivity and $k$ is a coefficient of
thermal conductivity, $k=\lambda/(\rho C_p)$. 

% \newcommand{\esr}{\| \overline{ {\dot \ev} } \|}
\newcommand{\esr}{\overline{ {\dot \ev} }}

The quantity $\sigmav'$ is the stress deviator. One model for visco-plastic models with
a yield stress $\sigma_y$ is given by
\begin{align*}
   \sigmav' &= \eta {\dot \ev}, \\
   \eta &= 2\Big( \eta_p(T) + {\sigma_y(T) \over \esr }( 1 - \exp(-m\esr) \Big), \\
    {\dot \ev}_{ij} &= \half( {\partial u_i\over\partial x_j} + {\partial u_j\over\partial x_i}), \\
    \esr &= \sqrt{ {2\over 3} {\dot \ev}_{ij}{\dot \ev}_{ij} } . 
\end{align*}


Here $\eta=\eta(\esr,T)$ is the effective coefficient of viscosity that depends on
the temperature and the effective strain rate, $\esr$. $\eta_p$ is the coefficient
of viscosity in the plastic regime. 


The limiting behaviours of the viscosity coefficient $\eta$ for high and low strain rates are
\begin{alignat*}{3}
\eta &\sim 2 \eta_p(T), &\qquad& \text{for $\esr \gg \sigma_y$}, \\
\eta &\sim 2\Big( \eta_p(T) + m \sigma_y(T) \Big), &\qquad& \text{for $\esr \ll \sigma_y$} . 
\end{alignat*}
The value of $m$ is choosen to be large and positive ($m \approx 100-1000$) so that
$\eta$ is a large value when the strain rates are small (corresponding to a nearly solid state
with almost no flow).
When the strain rates become sufficiently large the fluid can flow with an effective
coefficient of viscosity of $2\eta_p$. 

Note that 
\begin{align*}
  (\grad\cdot \sigmav')_{i} &= 
     \partial_{x_j} \big( \eta \half( \partial_{x_j} u_i + \partial_{x_i} u_j ) \big) \\
          &= \half\eta \partial_{x_j}\partial_{x_j} u_i + 
                 \half (\partial_{x_j}\eta)(\partial_{x_j} u_i + \partial_{x_i} u_j )
\end{align*}
where the incompressibility condition, $\partial_{x_j} u_j = 0$, has been used.
It also follows that
\begin{align*}
\grad\cdot(\grad\cdot \sigmav') &= 
    \partial_{x_i}\Big( \half\eta \partial_{x_j}\partial_{x_j} u_i + 
                 \half (\partial_{x_j}\eta)(\partial_{x_j} u_i + \partial_{x_i} u_j ) \Big) \\
   &= \half(\partial_{x_i}\eta)\partial_{x_j}\partial_{x_j} u_i +
        \half (\partial_{x_i}\partial_{x_j}\eta)(\partial_{x_j} u_i + \partial_{x_i} u_j )  +
        \half (\partial_{x_j}\eta)(\partial_{x_i}\partial_{x_i} u_j ) \\ 
   &= (\partial_{x_i}\eta)\partial_{x_j}\partial_{x_j} u_i +
        \half (\partial_{x_i}\partial_{x_j}\eta)(\partial_{x_j} u_i + \partial_{x_i} u_j )
\end{align*}

In two dimensions
\begin{align*}
 \esr^2 &=  {2\over 3} \Big( u_x^2 + \half ( u_y + v_x )^2 + v_y^2 \Big)  \\
 (\grad\cdot \sigmav')_1 &= \half\eta\Delta u 
         +\half\Big( \eta_x(2u_x) + \eta_y(u_y+v_x)\Big) \\
 (\grad\cdot \sigmav')_2 &= \half\eta\Delta v 
         +\half\Big( \eta_x(u_y+v_x) + \eta_y(2v_y)\Big) \\
 \grad\cdot(\grad\cdot \sigmav') &= \eta_x\Delta u + \eta_y\Delta v +
    \Big( \eta_{xx}u_x+ \eta_{xy}(u_y+v_x) + \eta_{yy}v_y \Big)
\end{align*}


\section{ Discretization}\index{discretization!incompressible Navier-Stokes}

\newcommand{\id}{i}
\def\Fs {{\cal F}}
Let $\Vv_i$ and $\pd_\id$ denote the discrete approximations to
$\uv$ and $p$ so that
\[
      \Vv_i \approx \uv(\xv_{\id})  \comma
      \pd_{\id} \approx p(\xv_{\id})  ~.
\]
Here $\Vv_i=(V_{1\id},V_{2\id},V_{3\id})$ and
$\id=(i_1,i_2,i_3)$ is a multi-index.
After discretizing in space the equations we solve are of the form
$$
 \left. \begin{array}{rcl}
  {d\over dt} \Vv_i + (\Vv_i\cdot\grad_h)\Vv_i + \grad_h \pd_i
       - \nu \Delta_h \Vv_i -\fv(\xv_i,t) &=&0
                                                \\
  \Delta_h \pd_i - \sum_m \grad_h V_{m,i} \cdot D_{m,h} \Vv_i
   - C_{d,i} \grad_h\cdot\Vv_i
   - \grad_h\cdot\fv(\xv_i,t)  &=&0
        \end{array}\right\} ~~~ \xv\in\Omega
$$
$$
 \left. \begin{array}{rcl}
        B(\Vv_i,\pd_i) &=& 0   \\
   \grad_h\cdot\Vv_i &=& 0
        \end{array}\right\} ~~~ \xv_i\in\partial\Omega_h
$$
$$
   \Vv(\xv_i,0) = \uvd_0(\xv_i)   ~~~~\mbox{ at } t=0
$$
where the divergence damping coefficient, $C_{d,i}$ is defined below.
The subscript ``h'' denotes a second or fourth-order centred difference
approximation,
$$
  D_{m,h} \approx {\partial \over \partial x_m} ~~~,~~
  \grad_h = (D_{1,h}, D_{2,h}, D_{3,h} ) ~~~,~~
  \Delta_h \approx \sum_m {\partial^2 \over \partial x_m^2}
$$
Extra numerical boundary conditions are also added, see~\cite{ICNS}
\cite{BCNS} for further details. An artificial diffusion term can be
added to the momentum equations. This is described in section~(\ref{AD}).

\def\Ps {{\cal P}}
When discretized in space on an overlapping grid this system of PDEs
can be thought of as a large system of ODEs of the form
$$
    {d \Uv \over dt} = \Fs(t,\Uv,\Pv)
$$
where $\Uv$ is a vector of all solution values at all grid points.
For the purpose of discussing time-stepping methods it is often
convenient to think of the pressure as simply a function of $\Uv$,
$\Pv=\Ps(\Uv)$
There are also interpolation equations that need to be satisfied but
this causes no difficulties.

\clearpage
\section{Dimensional Units and Non-dimensional Equations}


When you solve the incompressible Navier-Stokes equations with Cgins
it solves the following equations
\begin{align}
   \uv_t + (\uv\cdot\grad)\uv + \grad p &= \nu \Delta \uv, \label{eq:uvc} \\
   \grad\cdot\uv &= 0.
\end{align}
There is only one parameter that you specify and that is $\nu$, the kinematic 
viscosity. Cgins does NOT scale the input values that you specify for
the initial conditions and boundary conditions. Cgins does not scale the
the dimensions of the grid.
Also note that the pressure $p$ is only determined up to a constant. 

\newcommand{\tp}{t^d}
\newcommand{\xvp}{\xv^d}
\newcommand{\uvp}{\uv^d}
\newcommand{\pp}{p^d}
\newcommand{\rhop}{\rho^d}
\newcommand{\Deltap}{\Delta^d}
\newcommand{\gradp}{\grad^d}

Let's relate these equations to the equations in dimensional form,
\begin{align}
   \uvp_{\tp} + (\uvp\cdot\gradp)\uvp + {1\over\rhop} \gradp \pp &= \frac{\mu}{\rhop} \Deltap \uvp, 
                          \label{eq:uvp}\\
   \gradp\cdot\uvp &= 0.
\end{align}
Here the density $\rhop$ is
a constant and $\mu$ is the constant viscosity.
The superscript $d$ denotes dimensional units. For example, the components of  $\uvp$ might be
measured in $m/s$, lengths in $m$, the pressure $\pp$ in $N/m^2$ and the viscosity coefficient
$\mu$ in $kg/(m~s)$.  

Suppose that you want to use dimensional units, for example MKS units, to specify a problem for Cgins.
The grid should then be built with dimensions in $m$. The inflow values
for the velocity should be given in $m/s$. The kinematic viscosity should be
set equal to $\nu=\mu/\rhop$ which has units $m^2/s$. 
Comparing equation~(\ref{eq:uvp}) to equation~(\ref{eq:uvc})
we see that the pressure computed by Cgins will be $p=\pp/\rhop$. Thus you should
specify boundary conditions and initial conditions for $p$ from $p=\pp/\rhop$ (which has
units of $m^2/s^2$). 



\newcommand{\tnd}{t^n}
\newcommand{\xvt}{\xv^n}
\newcommand{\uvt}{\uv^n}
\newcommand{\pt}{p^n}
\newcommand{\rhot}{\rho^n}
\newcommand{\Deltat}{\Delta^n}
\newcommand{\gradt}{\grad^n}
In general one may want to non-dimensionalize the equations before solving them
with Cgins. 
Let us choose appropriate values, $U$, $L$, $R$,
to scale the length, velocity and density,
and define non-dimensional variables
\begin{alignat}{3}
   \xvt & = \xvp/L  ~, &\qquad&  \rhot &= \rhop/R  ~,\\
   \tnd & = \tp/(L/U) ~, &&  \uvt &= {\uvp/U} ~, \\
   \pt &= \pp/(R U^2)~.
\end{alignat}
The Navier-Stokes equations then become
\begin{align}
   \uvt_{\tnd} + (\uvt\cdot\gradt)\uvt + {1\over\rhot} \gradt \pt &= \frac{\mu}{\rhot R U L} \Deltat \uvt, 
                          \label{eq:uvt}\\
   \gradt\cdot\uvt &= 0.
\end{align}
In this case we will generate the grid with lengths that have been scaled by $L$. We will specify inflow
velocities with values scaled by $U$, the pressure will be scaled by $R U^2$,
the kinematic viscosity will be defined by
$\nu=\mu/(\rhot R U L)$ and the non-dimensional pressure $\pt$ will be related to
the computed value $p$ by $p=\pt/\rhot$.

\clearpage
\section{Divergence Damping} \index{divergence damping}

The divergence damping term, $C_{d,i} \grad_h\cdot\Vv_i$, appears in the pressure equation.
In simplified terms, the coefficient $C_d$ is taken proportional to the inverse
of the time step, $C_d \sim {1\over \Delta t}$. In practice we have found better results
by taking $C_d \sim {\nu \over \Delta x ^2}$. For explicit time stepping theses are
very similar since the explicit time step restriction is something like ${\nu \Delta t \over\Delta x ^2 }<C$.
To allow for the case $\nu=0$ we use the minumum grid spacing, $h_{\rm min}$,  instead of $\nu$, if
$h_{\rm min} > \nu$.
The size of $C_d$ affects the time step, the stability condition is proportional to $C_d \Delta t$.
As a result we do not want $C_d$ to be much larger than $1/\Delta t$,
and thus it is limited by ${C_t \over \Delta t}$ where $C_t$ is a constant with default value of $0.25$. 
(Note that we don't actually know the true $\Delta t$ at this point, it depends on $C_d$, so we just use a guess).

Here is the actual formula for the divergence damping coefficient:
\[ 
   C_{d,i} = \min( {\cal D}_i , {C_t \over \Delta t} )
\]
where
\begin{align*}
  {\cal D}_i &=   C_0~ \max( \nu, h_{\rm min} ) ~\left(
        {1\over (\Delta_{0,r_1} x_{1,i})^{2}} +
        {1\over (\Delta_{0,r_2} x_{2,i})^{2}} +
        {1\over (\Delta_{0,r_3} x_{3,i})^{2}} \right)   \\
  \Delta_{0,r_1} x_{m,i} &=  {1\over2} ( x_{m,i_1+1} - x_{m,i_1-1} ) \qquad \mbox{(undivided second difference)}\\
  h_{\rm min} &= \min_i ( \| \Delta_{+,r_1} \xv_i \|, \Delta_{+,r_2} \xv_i \|, \Delta_{+,r_3} \xv_i \| ) 
        \qquad \mbox{(minimum grid spacing)} \\
\end{align*}
and where $C_0=1.$ by default.

\section{Artificial Diffusion} \label{AD}\index{artificial diffusion}

Cgins implements an
artificial diffusion based on a second-order undivided
difference or a fourth-order undivided difference.
The second-order artificial diffusion is
\begin{equation} \label{AD2}
   \dv_{2,i} =
    \left( {\ff ad21} + {\ff ad22} | \grad_h \Vv_i |_1
    \right) \sum_{m=1}^{n_d} \Delta_{m+}\Delta_{m-} \Vv_i
\end{equation}
while the in the fourth-order one is
\begin{equation} \label{AD4}
   \dv_{4,i} =
  - \left( {\ff ad41} + {\ff ad42} | \grad_h \Vv_i |_1
    \right) \sum_{m=1}^{n_d} \Delta_{m+}^2\Delta_{m-}^2 \Vv_i
\end{equation}
Here $|\grad_h \Vv_i |_1$ is the magnitude of the
gradient of the velocity and $\Delta_{m\pm}$ are the forward and backward
undivided difference operators in direction $m$
$$
\begin{array}{rcl}
  | \grad_h \Vv_i |_1 &=& n_d^{-2}
          \sum_{m=1}^{n_d} \sum_{n=1}^{n_d} | D_{m,h} V_{n i} | \\
  \Delta_{1+}\Vv_i &=& \Vv_{i_1+1} - \Vv_i \\
  \Delta_{1-}\Vv_i &=& \Vv_i - \Vv_{i_1-1}   \\
  \Delta_{2+}\Vv_i &=& \Vv_{i_2+1} - \Vv_i \\
  \Delta_{2-}\Vv_i &=& \Vv_i - \Vv_{i_2-1}  ~~~\mbox{etc.} \\
\end{array}
$$
The artificial diffusion is added to the momentum equations
$$
  {d\over dt} \Vv_i + (\Vv_i\cdot\grad_h)\Vv_i + \grad_h \pd_i
       - \nu \Delta_h \Vv_i -\fv(\xv_i,t) - \dv_{m,i}  = 0
$$
but does not change the pressure equation.  Typical choices for the
constants ${\ff ad21}={\ff ad41}=1$ and ${\ff ad22}={\ff
ad42}=.5$. These artificial diffusions should not affect the order of
accuracy of the method.  With the artificial diffusion turned on to a
sufficient degree, the real viscosity can be set at low as zero, ${\ff
nu}=0$.  

This form of the artficial diffusion is based on a theoretical
result~\cite{HKR1}\cite{HKR2} that states that the minimum scale,
$\lambda_{\rm min}$, of solutions to the incompressible Navier-Stokes
equations is proportional to the square root of the kinematic
viscosity divided by the square root of the maximum velocity gradient:
$$
   \lambda_{\rm min} \propto \sqrt{ \nu \over | \grad\uv | + c} ~.
$$
This result is valid locally in space so that $|\grad\uv|$ measures
the local value of the velocity gradient.  The minimum scale measures
the size of the smallest eddy or width of the sharpest shear layer as
a function of the viscosity and the size of the gradients of $\uv$.
Scales smaller than the minimum scale are in the exponentially small
part of the spectrum.

This result can be used to tell us the smallest value that we can
choose for the (artificial) viscosity, $\nu_A$, and still obtain a
reasonable numerical solution.  We require that the artificial
viscosity be large enough so that the smallest (but still significant)
features of the flow are resolved on the given mesh. If the local grid
spacing is $h$, then we need
$$
     h  \propto \sqrt{ \nu_A \over | \grad\uv | + c}  ~.
$$
This gives
$$
   \nu_A = (c_1 + c_2 |\grad\uv|) h^2
$$
and thus we can choose an artificial diffusion of
$$
      (c_1 + c_2 |\grad\uv|) h^2 \Delta \uv
$$
which is just the form (\ref{AD2}).

In the fourth-order case we wish to add an artificial
diffusion of the form
$$
      - \nu_A \Delta^2  \uv
$$
since, as we will see, this will lead to $\nu_A \propto h^4$.
In this case, if we consider solutions to the incompressible
Navier-Stokes equations with the diffusion term $\nu\Delta\uv$ replaced
by $ -\nu_A \Delta^2\uv$ then
the minimum scale would be
$$
  \lambda_{\rm min} \propto \left( \nu_A \over | \grad\uv | \right)^{1/4}
$$
Following the previous argument leads us to choose an artificial
diffusion of the form
$$
     -  (c_1 + c_2 |\grad\uv|) h^4 \Delta^2 \uv
$$
which is just like (\ref{AD4}).

\section{Boundary Conditions}\index{boundary conditions}


The boundary conditions for method INS are
\begin{align*}
  \text{noSlipWall} &= \begin{cases}
                           \uv = \gv & \text{velocity specified} \\
                           \grad\cdot\uv=0 & \text{divergence zero} 
                       \end{cases}  \\
  \text{slipWall}   &=
     \begin{cases}
       \nv\cdot\uv = g & \text{normal velocity specified} \\
       \partial_n(\tv_m\cdot\uv) =0 & \text{normal derivative of tangential velcity is zero} \\
     \grad\cdot\uv=0 & \text{divergence zero} 
     \end{cases}  \\
  \text{inflowWithVelocityGiven} &= \begin{cases}
                           \uv = \gv & \text{velocity specified} \\
                           \partial_n p =0 & \text{normal derivative of the pressure zero.}
                       \end{cases}  \\
  \text{outflow} &= \begin{cases}
                        \text{extrapolate~} \uv & \text{} \\
                        \alpha p + \beta \partial_n p = g & \text{mixed derivative of p given.}
                       \end{cases} \\
  \text{symmetry} &= \begin{cases}
                           \nv\cdot\uv \text{: odd}, \tv_m\cdot\uv \text{: even}  & \text{vector symmetry} \\
                           \partial_n p =0 & \text{normal derivative of the pressure zero.}
                       \end{cases}  \\
  \text{dirichletBoundaryCondition} &= \begin{cases}
                           \uv = \gv & \text{velocity specified} \\
                           p = P     & \text{pressure given} 
                       \end{cases}  
\end{align*}


% --------------------------------------------------------------------------------------------
\clearpage

\newcommand{\nd}{n_d}
\newcommand{\PF}{\sum_{m=1}^{\nd} \grad_4 \ud_{m,i} \cdot D_{4 x_m} \uvd_i}
\newcommand{\Ds}{{\mathcal D}}
\newcommand{\Extrap}{D_{+m}^4}

\section{Boundary conditions for the fourth-order method}

Here are the analytic and numerical conditions that we impose at a boundary inorder
to determine the values of $\uv$ at the two ghost points.

{\bf noSlipWall:} 
Analytic boundary conditions
\[
  \uv = \uv_B(\xv,t)
\]
plus numerical boundary conditions
\begin{align*}
   \tv_\mu\cdot\Big\{ \nu\Delta\uv -\grad p - (\uv\cdot\grad)\uv -\uv_t \Big\} & = 0 \\
   \mbox{Extrapolate~~} \tv_\mu\cdot\uv &=0 \\
   \grad\cdot\uv &= 0 \\
   \partial_n(\grad\cdot\uv) &= 0 \\
\end{align*}


{\bf inflowWithVelocityGiven or outflow:}
Analytic boundary conditions for inflow are
\[
  \uv = \uv_I(\xv,t) \qquad\mbox{(inflow)}
\]
For outflow the equation is used on the boundary.
The numerical boundary conditions are
\begin{align*}
   \tv_\mu\cdot( \uv_{nn} ) &=0 \\
   \mbox{Extrapolate~~} \tv_\mu\cdot\uv &=0 \\
   \grad\cdot\uv &= 0 \\
   \partial_n(\grad\cdot\uv) &= 0 \\
\end{align*}



{\bf slipWall:}
Analytic boundary conditions are
\begin{align*}
   \nv\cdot\uv  = \nv\cdot\uv_B 
\end{align*}
The numerical boundary conditions are
\begin{align*}
    \tv_\mu\cdot\Big\{ \nu\Delta\uv -\grad p - (\uv\cdot\grad)\uv -\uv_t \Big\} & = 0 
                \quad\mbox{determines $\tv_\mu\cdot\uv$ on the boundary} \\  
   \tv_\mu\cdot( \uv_n ) &= 0\\
   \tv_\mu\cdot( \uv_{nnn} ) &= 0\\
   \grad\cdot\uv &= 0 \\
   \partial_n(\grad\cdot\uv) &= 0 \\
\end{align*}




{\bf Discretizing the Boundary conditions:}
For the purposes of this discussion assume that the boundary condition
for $\uv$ is of the form $\uv(\xv,t)=\uv_B(\xv,t)$ for
$\xv\in\partial\Omega$.  More general boundary conditions on $\uv$ and
$p$, such as extrapolation conditions, can also be dealt with although
some of the details of implementation may vary.
\def\ib {n_{m,a}}
At a boundary the following conditions are applied
\begin{eqnarray*}
   \left. \begin{array}{rcl}
 \uvd_{\id}  -\uv_B(\xv_\id) &=& 0  \\
 \grad_4 \cdot \uvd_{\id}  &=& 0    \\
 D_{4n} ( \grad_4\cdot \uvd_{\id} ) &=& 0  \\
 {d\over dt} \uvd_{\id}
 + (\uvd_{\id}\cdot\grad_4) \uvd_{\id}
  +\grad_4 \pd_{\id}  -\nu\Delta_4 \uvd_{\id}-\fv_{\id}  &=& 0   \\
\Delta_4\pd_{\id} + \PF - \grad_4 \cdot \fv_{\id}   &=& 0
   \end{array} \right\}
     &&~~~\mbox{for $\id\in$ Boundary}                  \\
   \left. \begin{array}{rcl}
  \tv_\mu\cdot \Extrap  \uvd_{\id}     &=& 0  \\
   \Extrap \pd_\id                      &=& 0
   \end{array} \right\}
       &&~~~\mbox{for $\id\in$ 2nd fictitious line}
\end{eqnarray*}
where $\tv_\mu$, $\mu=1,\nd-1$ are linearly independent
vectors that are tangent to the boundary. In the extrapolation conditions
either $D_{+m}$ or $D_{-m}$ should be chosen, as appropriate.
Thus at each point along the boundary
there are $12$ equations for the $12$
unknowns $(\uvd_{\id},\pd_{\id})$ located on the boundary
and the 2 lines of fictitious points.
Note that two of the numerical boundary conditions couple the pressure
and velocity.  In order to advance the velocity with an explicit time
stepping method it convenient to decouple the solution of the pressure
equation from the solution of the velocity.  A procedure to accomplish
this is described in the next section on time stepping.


{\bf Edges and Vertices:}
An important special case concerns obtaining solution values at
points that lie near edges and vertices of grids
(or corners of grids in 2D).  Define a
{\it boundary edge} to be the edge that is formed at the intersection
of adjacent
faces of the unit cube where both faces are boundaries of the
computational domain.
Along a boundary edge, values of the solution are required at the
fictitious points in the region exterior to both boundary
faces.  For example, suppose that the edge defined by
$i_1=n_{1,a}$, $i_2=n_{2,a}$ and $i_3=n_{3,a},\ldots,n_{3,b}$
is a boundary edge.
Values must be determined at the exterior points
$\id=(n_{1,a}+m,n_{2,a}+n,i_3)$ for $m,n=-2,-1$.


\newcommand{\ra}{{r_1}}
\newcommand{\rb}{{r_2}}
\newcommand{\rc}{{r_3}}
\newcommand{\trunc}{O(|\rv|^6)}
Here we derive a more accurate formula than was in my paper. These expressions will be 
exact for polynomials of degree 4. 
By Taylor series,
\begin{align*}
  u(\ra,\rb)&= u(0,0) + \Ds_1(\ra,\rb) + \Ds_2(\ra,\rb) + \Ds_3(\ra,\rb) + \Ds_4(\ra,\rb) + \trunc
\end{align*}
where
\begin{align*}
  \Ds_1(\ra,\rb) &= (\ra \partial_\ra + \rb \partial_\rb ) u(0,0) \\
  \Ds_2(\ra,\rb) &= {1\over2}( \ra^2\partial_\ra^2+\rb^2\partial_\rb^2+ 2\ra\rb\partial_\ra\partial_\rb  )u(0,0)  \\
  \Ds_3(\ra,\rb) &= {1\over3!}( \ra^3\partial_r^3 + \rb^3\partial_\rb^3 
                   +3\ra^2\rb\partial_\ra^2\partial_\rb + 3\ra\rb^2\partial_\ra\partial_\rb^2  )u(0,0)  
\end{align*}
We also have 
\begin{align}
 u(-\ra,-\rb) &= 2 u(0,0) - u(\ra,\rb) + 2\Ds_2(\ra,\rb) + 2\Ds_4(\ra,\rb)+ \trunc \label{taylor1} \\
  u(2\ra,2\rb)&= u(0,0) + 2\Ds_1(\ra,\rb) + 4\Ds_2(\ra,\rb) + 8\Ds_3(\ra,\rb) + 16\Ds_4(\ra,\rb) + \trunc\\
 8 u(\ra,\rb) - u(2\ra,2\rb) &= 7u(0,0) + 6\Ds_1(\ra,\rb) + 4 \Ds_2(\ra,\rb) -8 \Ds_4(\ra,\rb) + \trunc \label{taylor2} 
\end{align}
From equation~(\ref{taylor2}) we can solve for $\Ds_4(\ra,\rb)$,
\[
  \Ds_4(\ra,\rb) = {7\over8}u(0,0) - u(\ra,\rb) + {1\over 8}u(2\ra,2\rb) + {3\over4}\Ds_1(\ra,\rb) + {1\over2} \Ds_2(\ra,\rb)
                             + O(|r|^5+|s|^5)
\]
and substitute into equation~(\ref{taylor1}) 
\begin{equation}
  u(-\ra,-\rb) = {15\over4} u(0,0) - 3 u(\ra,\rb) + {1\over 4}u(2\ra,2\rb) + {3\over2}\Ds_1(\ra,\rb) 
                          + 3\Ds_2(\ra,\rb) +  \trunc \label{eq:corner1}
\end{equation}
We will use this last equation to determine $u$ at the first corner ghost point, $\uvd_{-1,-1,i_3}$.
Proceeding in a similar way it follows that
\begin{align}
 u(-2\ra,-\rb) &= {15\over4} u(0,0) - 3 u(2\ra,\rb) + {1\over 4}u(4\ra,2\rb) + {3\over2}\Ds_1(2\ra,\rb) 
                          + 3\Ds_2(2\ra,\rb) +  \trunc \trunc \label{eq:corner2}\\
 u(-\ra,-2\rb) &= {15\over4} u(0,0) - 3 u(\ra,2\rb) + {1\over 4}u(2\ra,4\rb) + {3\over2}\Ds_1(\ra,2\rb) 
                          + 3\Ds_2(\ra,2\rb) +  \trunc \trunc \label{eq:corner3}\\
 u(-2\ra,-2\rb) &=30 u(0,0) -32 u(\ra,\rb) + 3u(2\ra,2\rb) + 24 \Ds_1(\ra,\rb) + 24 \Ds_2(\ra,\rb) +  \trunc\trunc \label{eq:corner4}
\end{align}
from which we will determine the ghost points values at $\uvd_{-2,-2,i_3}$, $\uvd_{-2,-1,i_3}$ and
 $\uvd_{-1,-2,i_3}$. By symmetry we obtain formulae for ghost points outside all other edges
in three-dimensions, $\uvd_{-2:-1,i_2,-2:-1}$, $\uvd_{i_1,-2:-1,-2:-1}$. For ghost points outside
the vertices in three-dimensions we have
\begin{align}
  u(-\ra,-\rb,-\rc) &= {15\over4} u(0,0) - 3 u(\ra,\rb,\rc) + {1\over 4}u(2\ra,2\rb,2\rc)\\
    &  + {3\over2}\Ds_1(\ra,\rb,\rc) 
                          + 3\Ds_2(\ra,\rb,\rc) + \trunc\trunc \label{eq:corner5}
\end{align}
where
\begin{align*}
  \Ds_1(\ra,\rb,\rc) &= \Big(\ra \partial_\ra + \rb \partial_\rb + \rc\partial_\rc \Big) u(0,0,0) \\
  \Ds_2(\ra,\rb,\rc) &= {1\over2}\Big( \ra^2\partial_\ra^2+ \rb^2\partial_\rb^2+ \rc^2\partial_\rc^2
      +2\ra \rb\partial_\ra\partial_\rb + 2\ra \rc\partial_\ra\partial_\rc +2\rb \rc\partial_\rb\partial_\rc 
                       \Big)u(0,0,0)  \\
  \Ds_3(\ra,\rb,\rc) &= {1\over3!}\sum_{m_1=1}^3\sum_{m_2=1}^3\sum_{m_3=1}^3
                         r_{m_1} r_{m_2} r_{m_3} \partial_{m_1}\partial_{m_2}\partial_{m_3} u(0,0,0) \\
                     &={1\over3!}\Big( \ra^3\partial_r^3+ \rb^3\partial_\rb^3+ \rc^3\partial_\rc^3
   +3 \ra^2 \rb\partial_\ra^2\partial_\rb + 3 \ra\rb^2\partial_\ra\partial_\rb^2  \\
  & +3 \ra^2 \rc\partial_\ra^2\partial_\rc + 3 \ra\rc^2\partial_\ra\partial_\rc^2 
   +3 \rb^2 \rc\partial_\rb^2\partial_\rc+ 3 \rb\rc^2\partial_\rb\partial_\rc^2 
   +6 \ra\rb\rc \partial_\ra\partial_\rb\partial_\rc  \Big) u(0,0,0)  
\end{align*}


In order to evaluate the formulae~(\ref{eq:corner1},\ref{eq:corner2},\ref{eq:corner3},
\ref{eq:corner4},\ref{eq:corner5}) we need to evaluate the derivatives appearing in 
$\Ds_1$ and $\Ds_2$. All the non-mixed derivatives $\partial^m u(0,0)/\partial_{r_n}^m$, $m=1,2$,
can be evaluated using the boundary values since these are all tangential derivatives. 
The second-order mixed derivative term such as $u_{r_1 r_2}$
requires a bit more work. In two-dimensions we evaluate this term by taking the parametric
derivatives of the divergence, $\partial_{r_m} \grad\cdot\uv=0$.
Since
\[
    \grad\cdot\uv = \sum_{n=1}^2 (\partial_x r_n)~ u_{r_n} ~+~  \sum_{n=1}^2 (\partial_y r_n)~v_{r_n}
\]
then $\partial_{r_m} \grad\cdot\uv=0$ gives
\begin{align*}
    (\partial_x r_2) ~ u_{r_1 r_2} + (\partial_y r_2) ~v_{r_1 r_2} =& 
      (\partial_x r_2)_{r_1} ~ u_{r_2} +
      (\partial_x r_1) ~u_{r_1 r_1} + (\partial_x r_1)_{r_1}~\partial_{r_1} u + \\
     & (\partial_y r_2)_{r_1} ~ v_{r_2} +(\partial_y r_1) ~v_{r_1 r_1} + (\partial_y r_1)_{r_1}~\partial_{r_1} v \\
    (\partial_x r_1) ~u_{r_1 r_2}  + (\partial_y r_1) ~v_{r_1 r_2} =& 
        (\partial_x r_2) ~u_{r_2 r_2} + (\partial_x r_2)_{r_2}~\partial_{r_2} u       ... 
\end{align*}
These last two equations can be solved for $u_{r_1 r_2}$ and $v_{r_1 r_2}$
in terms of known tangential derivatives.

In three dimensions taking the two parametric derivatives of the divergence, 
\begin{align}
    (\partial_x r_2) ~ u_{r_1 r_2} + (\partial_y r_2) ~v_{r_1 r_2} + (\partial_z r_2) ~w_{r_1 r_2} =& 
                     ...  \label{eq:div3a}\\
    (\partial_x r_1) ~u_{r_1 r_2}  + (\partial_y r_1) ~v_{r_1 r_2} + (\partial_z r_1) ~w_{r_1 r_2} =&
                     ...  \label{eq:div3b}
\end{align}
gives only two equations for the three unknowns, $u_{r_1 r_2},v_{r_1 r_2}$ and $w_{r_1 r_2}$.
We therefore add an extra condition by extrapolating the tangential component of the velocity,
\begin{equation}
\tv_3 \cdot D_{+,1,2}^6  \uvd_{i_1-1,i_2-1,i_3} = 0  \label{eq:extrapTan}
\end{equation}
Solve the last equation for $\tv_3\cdot \uvd_{i_1-1,i_2-1,i_3}$ gives
\begin{equation}
\tv_3\cdot \uvd_{i_1-1,i_2-1,i_3} = {\mathcal E}_{+,1,2}^6 \tv_3\cdot\uvd_{i_1-1,i_2-1,i_3}  \label{eq:extrapTan2}
\end{equation}
where we have introduced the operator ${\mathcal E}_{+,1,2}^6$.
By substituting this last equation~(\ref{eq:extrapTan2}) into $\tv_3\cdot$ equation~(\ref{eq:corner1}),
\[
 \tv_3\cdot \uv(-\ra,-\rb) =\tv_3\cdot\left( {15\over4} \uv(0,0) - 3 \uv(\ra,\rb) + {1\over 4}\uv(2\ra,2\rb) 
                 + {3\over2}\Ds_1(\ra,\rb)\uv(0)
                 + 3\Ds_2(\ra,\rb)\uv(0) \right) +  \trunc 
\]
we can eliminate $\tv_3 \cdot\uvd_{i_1-1,i_2-1,i_3}$ and obtain an equation for 
$\tv_3 \cdot \uv_{r_1 r_2}$ in terms of known quanitites,
\begin{align*}
\tv_3 \cdot \left( {\mathcal E}_{+,1,2}^6 \uvd_{i_1-1,i_2-1,i_3} \right) 
  &= \tv_3\cdot\Big( {15\over4} \uv(0,0) - 3 \uv(\ra,\rb) + {1\over 4}\uv(2\ra,2\rb) + {3\over2}\Ds_1(\ra,\rb)\uv(0) \\
  & + {3\over2}( \ra^2\partial_\ra^2+\rb^2\partial_\rb^2 +\rc^2\partial_\rc^2
  + 2\ra\rb\partial_\ra\partial_\rb + 2\ra\rc\partial_\ra\partial_\rc + 2\rb\rc\partial_\rb\partial_\rc  )\uv(0,0)
                \Big)
\end{align*}
or re-written as
\begin{align}
  \tv_3\cdot \uv_{r_1 r_2}(0,0) &= {1\over 3}\Big\{
               \tv_3 \cdot \left( {\mathcal E}_{+,1,2}^6 \uvd_{i_1-1,i_2-1,i_3} \right) 
        -  \tv_3\cdot\Big( {15\over4} \uv(0,0) - 3 \uv(\ra,\rb) + {1\over 4}\uv(2\ra,2\rb)
             + {3\over2}\Ds_1(\ra,\rb)\uv(0)  \label{eq:div3c} \\
  & + {3\over2}( \ra^2\partial_\ra^2+\rb^2\partial_\rb^2 +\rc^2\partial_\rc^2
  + 2\ra\rc\partial_\ra\partial_\rc + 2\rb\rc\partial_\rb\partial_\rc
    )\uv(0,0)     \Big\}\nonumber
\end{align}

To summarize we solve equations~(\ref{eq:div3a},\ref{eq:div3b},\ref{eq:div3c}) for the
three unknowns $u_{r_1 r_2},v_{r_1 r_2}$ and $w_{r_1 r_2}$.

\vskip2\baselineskip

% The following conditions are imposed
% \begin{eqnarray}
%   {\partial \over \partial r_m} \Big(\grad\cdot\uv\Big) &=& 0 ~~~m=1,2
%                                            \label{edge1}  \\
%     \tv_3 \cdot D_{+,1,2}^6  \uvd_{i_1-1,i_2-1,i_3} &=& 0
%                                            \label{edge2}
% \end{eqnarray}
% Here $\tv_3$ is the unit vector in the direction of the edge.  Recall
% that $D_{+,1,2}\uvd_\id = \uvd_{i_1+1,i_2+1,i_3}-\uvd_\id$ and thus
% the condition~(\ref{edge2}) is an extrapolation into the region of the
% component of the velocity that is parallel to the edge.
% Equations~(\ref{edge1}),(\ref{edge2}) supply sufficient information
% to determine the values of the fictitious points outside the edge,
% as will now be shown.
% 
% % --------------------- old ----------------------
% \vskip3\baselineskip
% 
% 
% 
% By expanding
% $\uv(-r_1,-r_2,r_3)$ and $\uv(+r_1,+r_2,r_3)$
% in a Taylor series about $(0,0,0)$
% it follows that
% \begin{align*}
%   \uv(-r_1,-r_2,r_3) &= 2 \uv(0,0,r_3) - \uv(r_1,r_2,r_3)  + 2 \Ds_2  + 2 \Ds_4 + O(|r_1|^6+|r_2|^6) \label{edge3}\\
%    \Ds_2 &= {1\over2}( r_1^2\partial_{r_1}^2+r_1 r_2\partial_{r_1}\partial_{r_2} +
%                               r_2^2\partial_{r_2}^2 )\uv(0,0,r_3)                         \nonumber \\
%    \Ds_4 &= {1\over 4!}( r_1^4\partial_{r_1}^4 + r_1^3 r_2 \partial_{r_1}^3\partial_{r_2} 
%                +6 r_1^2 r_2^2 \partial_{r_1}^2\partial_{r_2}^2
%             +  4 r_1 r_2^3\partial_{r_1}\partial_{r_2}^3 + r_2^4 \partial_{r_2}^4  )\uv(0,0,r_3) \nonumber
% \end{align*}
% The derivatives $\uv_{r_1 r_1}$ and $\uv_{r_2 r_2}$ are tangential
% derivatives (on the appropriate face) and can be computed from the
% given boundary data.  Here it is assumed that the given boundary data
% are compatible at edges.  The mixed derivative term, $\uv_{r_1 r_2}$,
% remains to be determined.  When expanded by the chain rule
% equations~(\ref{edge1}) can be written as
% $$
%   \sum_{l,n=1}^3 {\partial r_n \over \partial x_l}
%                {\partial^2 u_l \over \partial r_m \partial r_n}
%  + \sum_{l,n=1}^3 {\partial^2 r_n \over \partial r_m \partial x_l}
%                {\partial u_l \over \partial r_n} = 0
% $$
% for $m=1,2$. The only term in these equations
% that is not known from the boundary
% data is the mixed derivative term $\uv_{r_1 r_2}$; and thus there are two
% equations for the three unknown components of $\uv_{r_1 r_2}$.
% To get a third equation for $\uv_{r_1 r_2}$
%  the extrapolation condition~(\ref{edge2}) is combined with
% the equation formed when the tangent vector $\tv_3$ is dotted
% into~(\ref{edge3}) (with $r_1=-\Delta r_1$, $r_2=-\Delta r_2$).
% After solving for $\uv_{r_1 r_2}$, (\ref{edge3}) gives
% a fourth order accurate approximation to the 4 solution
% values that lie outside the boundary edge.
%  
% In two space dimensions, the values outside a corner are determined
% in a similar manner, although the extrapolation condition is
% not required.
%  
% \def\zero {{\bf 0}}
% At a vertex in 3D it follows from Taylor series that
% $$
%   \uv(-\rv) = 2\uv(\zero) -\uv(\rv)
%    + \sum_{mn} r_m r_n
%     {\partial^2 \uv \over \partial r_m \partial r_n}(\zero) + O(|\rv|^4) ~.
% $$
% All of the second order derivatives $\uv_{r_m r_n}$ are tangential
% derivatives on one of the faces that meets at the vertex and
% thus are known in terms of the given
% boundary values. Thus the value of $\uvd_\id$ at the
% $8$ points which lie outside a vertex can be computed.
 
{\bf Solving the numerical boundary equations:}
The numerical boundary conditions~(\ref{nbc}) define the
values of $\uvd$ on two lines of fictitious points in terms
of values of the velocity on the boundary and the interior.
The equations couple the unknowns in the tangential direction
to the boundary so that in principle a system of equations for
all boundary points must be solved. However, when the grid
is nearly orthogonal to the boundary there is a much more
efficient way to solve the boundary conditions.
The first step in the algorithm is to solve for the tangential
components of the velocity from
\begin{eqnarray*}
  \left. \begin{array}{rcl}
 \uvd_{\id}(t)  -\uv_B(\xv_\id,t) &=& 0 \\
\tv_\mu\cdot\Big\{
 {d\over dt} \uvd_{\id}(t)
+ (\uvd_{\id}(t)\cdot\grad_4)\uvd_{\id}(t)
  +\grad_4 \pd^*(t) -\nu\Delta_4 \uvd_{\id}(t)-\fv\Big\} &=& 0
  \end{array} \right \}
     &&  \mbox{for $\id\in$ Boundary}        \\
  \left. \begin{array}{rcl}
  \tv_\mu\cdot D_{+m}^6( \uvd_{\id}(t) )  &=& 0
  \end{array}  \right \}
     && \mbox{for $\id\in$ Second fictitious line}     \nonumber
\end{eqnarray*}
If the grid is orthogonal to the boundary then the
discrete Laplacian applied at boundary
will not have any mixed derivative terms. Therefore the only
fictitious points appearing in the equation applied at the
the boundary point $(i_1,i_2,i_3)$
will be the two points $(i_1,i_2,i_3-n)$
$n=1,2$ (here we assume that $i_3$ is in the normal direction
to the boundary).
Thus for each point on the boundary $(i_1,i_2,i_3)$
the values of $\tv_\mu\cdot\uv$ can be determined at
the fictitious points $(i_1,i_2,i_3-1)$ and $(i_1,i_2,i_3-2)$.
There is no coupling between adjacent boundary points so no
large system of equations need be solved.
The tangential components of the velocity are
determined for all fictitious points on the entire boundary.
The second step is to determine the
the normal component of
the velocity at the fictitious points from
\begin{eqnarray*}
  \left. \begin{array}{rcl}
 \uvd_{\id}(t)  -\uv_B(\xv_\id,t) &=& 0 \\
 \grad_4 \cdot \uvd_{\id}(t)  &=& 0     \\
 D_{4n} ( \grad_4\cdot \uvd_{\id}(t) ) &=& 0
  \end{array} \right \}
     &&  \mbox{for $\id\in$ Boundary}
\end{eqnarray*}
If the grid is orthogonal to the boundary then the divergence on the
boundary can be written in the form
$$
    \grad\cdot\uv = {1\over e_1 e_2 e_3} \left\{
          {\partial \over \partial n} ( e_2 e_3 \nv\cdot\uv )
        + {\partial \over \partial t_1} ( e_1 e_3 \tv_1 \cdot\uv )
        + {\partial \over \partial t_2} ( e_1 e_2 \tv_2 \cdot\uv )
              \right \}
$$
where the $e_m$ are functions of $\partial \xv / \partial \rv$.
Note that only normal
derivatives of $\nv\cdot\uv$ appear in the expression for the divergence.
Thus, at a boundary point, $(i_1,i_2,i_3)$, the stencil for
$\grad_4\cdot\uvd$
will only involve the fictitious points at
$(i_1,i_2,i_3-n)$, $n=1,2$.
Similarly, the stencil for
$D_{4n}(\grad_4\cdot\uvd)$ at a boundary
will only involve the fictitious points at
$(i_1,i_2,i_3-n)$, $n=1,2$.
Thus there is no coupling between adjacent boundary points and the
unknown values for $\nv\cdot\uv$ can be easily determined.  Note that
the equations for $D_{4n}(\grad_4\cdot\uvd)$ will couple values for
$\tv_\mu\cdot\uv$ at fictitious points along the boundary
but these values have already been determined in the first step.
 
In practice the boundary conditions are solved in a correction mode --
some initial guess is assumed for the values at the fictitious points
and a correction is computed.  If the grid is orthogonal or nearly
orthogonal to the boundary then the first correction will give an
accurate answer to the boundary conditions.  If the grid is not
orthogonal to the boundary then the solution procedure can repeated
one or more times until a desired accuracy is achieved. This
iteration should converge quickly provided that the grid is not overly
skewed.

% ------------------------- TURBULENCE MODELS ----------------------------------
\clearpage
\input insTurbulenceModels.tex
% ------------------------------------------------------------------------------

% ------------------------- LINE SOLVER ----------------------------------------
\clearpage
\input lineSolver.tex
% ------------------------------------------------------------------------------

\clearpage
\section{Convergence results}\index{convergence results!INS}

This section details the results of various convergence tests. 
Convergence results are run using the {\bf twilight-zone} option, also
known less formally as the {\bf method of analytic solutions}.
In this case the equations are forced so the the solution will
be a known analytic function.

The tables show the maximum errors in the solution components. The rate shown is estimated convergence rate, $\sigma$,
assuming ${\rm error} \propto h^\sigma$. The rate is estimated by a least squares fit to the data.

The 2D trigonometric solution used as a twilight zone function is
\begin{align*}
    u &= \half \cos(\pi \omega_0 x) \cos(\pi \omega_1 y) \cos( \omega_3 \pi t) + \half\\
    v &= \half \sin(\pi \omega_0 x) \sin(\pi \omega_1 y) \cos( \omega_3 \pi t) + \half\\
    p &=       \cos(\pi \omega_0 x) \cos(\pi \omega_1 y) \cos( \omega_3 \pi t)  + \half
\end{align*}
The 3D trigonometric solution is
\begin{align*}
    u &=       \cos(\pi \omega_0 x) \cos(\pi \omega_1 y) \cos(\pi \omega_2 z) \cos( \omega_3 \pi t) \\
    v &= \half \sin(\pi \omega_0 x) \sin(\pi \omega_1 y) \cos(\pi \omega_2 z) \cos( \omega_3 \pi t) \\
    w &= \half \sin(\pi \omega_0 x) \sin(\pi \omega_1 y) \sin(\pi \omega_2 z) \cos( \omega_3 \pi t) \\
    p &= \half \sin(\pi \omega_0 x) \cos(\pi \omega_1 y) \cos(\pi \omega_2 z) \sin( \omega_3 \pi t) 
\end{align*}
When $\omega_0=\omega_1=\omega_2$ it follows that $\grad\cdot\uv=0$.
There are also algebraic polynomial solutions of different orders.


Tables~(\ref{table:ins.square}-\ref{table:ins.box}) show results from running Cgins on various
grids. 

\input \convDir/ins.square.order2.table.tex
\input \convDir/ins.square.order4.table.tex
\input \convDir/ins.cic.order2.table.tex
\input \convDir/ins.cic.order4.table.tex
\input \convDir/ins.box.order2.table.tex
\input \convDir/ins.box.order4.table.tex
\input \convDir/ins.sib.order2.table.tex
\input \convDir/ins.sib.order4.table.tex

\begin{figure}[htb]
  \begin{center}
   \epsfig{file=\obFigures/ins.cic3.tz.ps,width=.7\linewidth} 
  \caption{Incompressible N-S, twilight zone solution for convergence test} \label{fig:ins.cic.tz}
  \end{center}
\end{figure}

% ==============================================================================================================
\clearpage
\section{Some interesting examples}

Here is a collection of interesting examples computed with the Cgins incompressible solver.

\subsection{Incompressible flow past a mast and sail}

Figure (\ref{fig:mastSail2d}) shows incompressible flow past a sail on a mast (grid created
with {\tt Overture/\-sampleGrids/\-mastSail2d.cmd}.

\begin{figure}[hbt]
\begin{minipage}{.5\linewidth}
  \begin{center}
   \epsfig{file=\obFigures/ins.mastSail2d.p.ps,width=\linewidth}
  \end{center}
\end{minipage}\\
\begin{minipage}{.5\linewidth}
  \begin{center}
   \epsfig{file=\obFigures/ins.mastSail2d.sl.ps,width=\linewidth} 
  \end{center}
\end{minipage}
\caption{Incompressible flow past a mast and sail.} \label{fig:mastSail2d}
\end{figure}


\clearpage
\subsection{Two falling bodies in an incompressible flow}

Figure (\ref{fig:twoDrop}) shows two rigid bodies failing under the influence
of gravity in an incompressible flow.

\begin{figure}[hbt]
  \begin{center}
   \epsfig{file=\obFigures/twoDrop0.p.ps,width=.45\linewidth}
   \epsfig{file=\obFigures/twoDrop2.0.p.ps,width=.45\linewidth} \\
   \epsfig{file=\obFigures/twoDrop4.0.p.ps,width=.45\linewidth}
   \epsfig{file=\obFigures/twoDrop6.0.p.ps,width=.45\linewidth}
  \end{center}
\caption{Two falling bodies in an incompressible flow.} \label{fig:twoDrop}
\end{figure}

\clearpage
\subsection{Incompressible flow past a truck}

Figure (\ref{fig:cabTender}) shows a computation of the incompressible Navier-Stokes
equations for flow past the cab of a truck. The steady state line solver was used for this
computation.

\renewcommand{\figWidth}{.45\linewidth}

\newcommand{\clipfig}[1]{\psclip{\psframe[linewidth=2pt](0.,.0)(8.1,7.0)}\epsfig{#1}\endpsclip}
\newcommand{\clipfigb}[1]{\psclip{\psframe[linewidth=2pt](0.,2.5)(14.,10)}\epsfig{#1}\endpsclip}
% \newcommand{\clipfigb}[1]{\psclip{\psframe[linecolor=white](0.,1.)(8.1,8.2)}\epsfig{#1}\endpsclip}
% \newcommand{\clipfigb}[1]{\psclip{\psframe[linecolor=black](0.,1.)(8.1,8.2)}\epsfig{#1}\endpsclip}

\psset{xunit=1.cm,yunit=1.cm,runit=1.cm}
\begin{figure}[htb]
\begin{center}
\begin{pspicture}(0,-7.)(17.5,7.0)
\rput( 4.5, 3.75){\clipfig{file=\obFigures/truckCabCad.ps,width=\figWidth}}
\rput(13.25, 3.75){\clipfig{file=\obFigures/cabTenderGrids.ps,width=\figWidth}}
\rput(9.,-3.5){\clipfigb{file=\obFigures/cabTenderTracers3.ps,width=.75\linewidth}}
% turn on the grid for placement
% \psgrid[subgriddiv=2]
% \rput*(.5,-1.6){\makebox(0,0)[l]{Cut plane is used to generate starting curve.}}
% \psline[linewidth=1.pt]{->}(2.,-1.5)(3.,1.)
% \psline[linewidth=1.pt]{->}(7.25,-1.45)(12.1,3.)
% \rput*(8.5,-3.5){\makebox(0,0)[r]{Stretching is added after marching.}}
% \psline[linewidth=1.pt]{->}(8.6,-3.5)(10.6,-2.75)
% \rput*(8.5,-4.5){\makebox(0,0)[r]{Bumper indentation smoothed.}}
% % \psline[linewidth=1.pt]{->}(8.5,-4.5)(5.,2.5)
% \psline[linewidth=1.pt]{->}(8.6,-4.5)(11.,-4.75)
% %\rput(9.5, 5){\makebox(0,0)[c]{\small Pressure, $t=1.75$}}
\end{pspicture}
\end{center}
\caption{Incompressible flow past the cab of a truck. Shown are the CAD geometry, the grids and
     some tracer particles.} \label{fig:cabTender}
\end{figure}

% -------------------------------------------------------------------------------------------------------
\renewcommand{\clipfig}[1]{\psclip{\psframe[linewidth=2pt](.2,-.2)(9.25,8.75)}\epsfig{#1}\endpsclip}
\renewcommand{\figWidth}{.5\linewidth}
\newcommand{\figWidthb}{.53\linewidth}
\clearpage
\subsection{Incompressible flow past a city scape}

Figure (\ref{fig:cityScape}) shows a computation of the incompressible Navier-Stokes
equations for flow past a city scape. The steady state line solver was used for this
computation. The command file for generating this grid is {\tt Overture/sampleGrids/multiBuildings.cmd}
and the Cgins command file is {\tt ins/cmd/multiBuildings.cmd}.

\begin{center}
\begin{pspicture}(0,-2.0)(10.,8.0)
\rput(.5,3.50){\clipfig{file=\obFigures/multiBuildingsTracersAndCut.ps,width=\figWidth}}
\rput(9.6,3.75){\clipfig{file=\obFigures/multiBuildings.ps,width=\figWidthb}}
\rput*(1.0,-1.7){\makebox(0,0)[c]{\bf 3D flow past a city scape.}}
\rput*(9.,-1.7){\makebox(0,0)[c]{\bf Overlapping grid for a city scape.}}
% turn on the grid for placement
% \psgrid[subgriddiv=2]
\end{pspicture}
\end{center}\label{fig:cityScape}
Notes:
\begin{flushleft}
 $\blue\diamondsuit$ pseudo steady-state line implicit solver, 4th-order dissipation, \\
 $\blue\diamondsuit$ local time-stepping (spatially varying dt)\\
 $\blue\diamondsuit$ requires 1.4GB of memory, \\
 $\blue\diamondsuit$ cpu = 29s/step, \\
 $\blue\diamondsuit$ 2.2 GHz Xeon, 2 GB of memory
\end{flushleft}


% -------------------------------------------------------------------------------------------------
\vfill\eject
\bibliography{/home/henshaw/papers/henshaw}
\bibliographystyle{siam}


\printindex


\end{document}
