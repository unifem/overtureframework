\subsection{Artificial Diffusion} \label{AD}\index{artificial diffusion}

% For the second-order accurate discretization 
Cgins implements 
artificial diffusions based either on a second-order undivided
difference or a fourth-order undivided difference.

The artificial diffusions are
\begin{equation} \label{AD2}
   \dv_{2,i} =
    \left( {\ff ad21} + {\ff ad22} | \grad_h \Vv_i |_1
    \right) \sum_{m=1}^{n_d} \Delta_{m+}\Delta_{m-} \Vv_i
\end{equation}
in the second-order case and
\begin{equation} \label{AD4}
   \dv_{4,i} =
  - \left( {\ff ad41} + {\ff ad42} | \grad_h \Vv_i |_1
    \right) \sum_{m=1}^{n_d} \Delta_{m+}^2\Delta_{m-}^2 \Vv_i
\end{equation}
in the fourth-order case. Here $|\grad_h \Vv_i |_1$ is the magnitude of the
gradient of the velocity and $\Delta_{m\pm}$ are the forward and backward
undivided difference operators in direction $m$
$$
\begin{array}{rcl}
  | \grad_h \Vv_i |_1 &=& n_d^{-2}
          \sum_{m=1}^{n_d} \sum_{n=1}^{n_d} | D_{m,h} V_{n i} | \\
  \Delta_{1+}\Vv_i &=& \Vv_{i_1+1} - \Vv_i \\
  \Delta_{1-}\Vv_i &=& \Vv_i - \Vv_{i_1+1}   \\
  \Delta_{2+}\Vv_i &=& \Vv_{i_2+1} - \Vv_i \\
  \Delta_{2-}\Vv_i &=& \Vv_i - \Vv_{i_2+1}  ~~~\mbox{etc.} \\
\end{array}
$$
The artificial diffusion is added to the momentum equations
$$
  {d\over dt} \Vv_i + (\Vv_i\cdot\grad_h)\Vv_i + \grad_h \pd_i
       - \nu \Delta_h \Vv_i -\fv(\xv_i,t) - \dv_{m,i}  = 0
$$
but does not change the pressure equation.
Typical choices for the constants  ${\ff ad21}={\ff ad41}=1$
and ${\ff ad22}={\ff ad42}=1.$. These artificial diffusions
should not affect the order of accuracy of the method.
With the artificial diffusion turned on to a sufficient degree, the
real viscosity can be set at low as zero, ${\ff nu}=0$.
% (See also the note about {\ff cdv} when ${\ff nu}=0$ in
% section (\ref{Running}).)

In difficult cases you may need to increase the coefficients 
${\ff ad21}$ and ${\ff ad22}=1.$ to keep the solution stable. 
I suggest trying ${\ff ad21}={\ff ad22}=2.$ then $4.$ etc. until
the solution doesn't blow up. I don't think I have ever needed
a value larger than $10.$. 

This form of the artificial diffusion is based on a theoretical
result~\cite{HKR1}\cite{HKR2}
that states that the minimum scale,  $\lambda_{\rm min}$,
 of solutions to the incompressible
Navier-Stokes equations
is proportional to the square
root of the kinematic viscosity divided by the square root of the
maximum velocity gradient:
$$
   \lambda_{\rm min} \propto \sqrt{ \nu \over | \grad\uv | + c} ~.
$$
This result is valid locally in space so that $|\grad\uv|$ measures
the local value of the velocity gradient.
The \Index{minimum scale} measures the
size of the smallest eddy or width of the sharpest shear layer as a
function of the viscosity and the size of the gradients of $\uv$.
Scales smaller than the minimum scale are in the exponentially small
part of the spectrum.

This result can be used to tell us the smallest value that we
can choose for the
(artificial) viscosity, $\nu_A$, and still obtain a reasonable
numerical  solution.
We require that the artificial viscosity
be large enough so that the smallest (but still significant)
features of the flow are resolved
on the given mesh. If the local grid spacing is $h$, then we need
$$
     h  \propto \sqrt{ \nu_A \over | \grad\uv | + c}  ~.
$$
This gives
$$
   \nu_A = (c_1 + c_2 |\grad\uv|) h^2
$$
and thus we can choose an artificial diffusion of
$$
      (c_1 + c_2 |\grad\uv|) h^2 \Delta \uv
$$
which is just the form (\ref{AD2}).
% With this choice for the artificial diffusion
% we are assured that the solution to the resulting equations will be


In the fourth-order accurate case we wish to add an artificial
diffusion of the form
$$
      - \nu_A \Delta^2  \uv
$$
since, as we will see, this will lead to $\nu_A \propto h^4$.
In this case, if we consider solutions to the incompressible
Navier-Stokes equations with the diffusion term $\nu\Delta\uv$ replaced
by $ -\nu_A \Delta^2\uv$ then
the minimum scale would be
$$
  \lambda_{\rm min} \propto \left( \nu_A \over | \grad\uv | \right)^{1/4}
$$
Following the previous argument leads us to choose
an artificial diffusion of the form
$$
     -  (c_1 + c_2 |\grad\uv|) h^4 \Delta^2 \uv
$$
which is just like (\ref{AD4}).

