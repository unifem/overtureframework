\subsection{Incompressible flow past a city scape}\label{sec:flowPastBuildings}

Figure (\ref{fig:cityScape}) shows a computation of the incompressible Navier-Stokes
equations for flow past a city scape. The steady state line solver was used for this
computation. The command file for generating this grid is {\tt Overture/sampleGrids/multiBuildings.cmd}
and the Cgins command file is {\tt ins/cmd/multiBuildings.cmd}.


{
\newcommand{\figWidthd}{8.0cm}
\newcommand{\trimfig}[2]{\trimPlotb{#1}{#2}{.0}{.0}{.000}{.00}}
\newcommand{\figWidtha}{10cm}
\newcommand{\trimfiga}[2]{\trimPlotb{#1}{#2}{.0}{.0}{.0}{.0}}
\begin{figure}[hbt]
\begin{center}
\begin{tikzpicture}[scale=1]
  \useasboundingbox (0,.75) rectangle (16.,8.5);  % set the bounding box (so we have less surrounding white space)
%
  \draw ( 0.0,0.0) node[anchor=south west,xshift=-4pt,yshift=+0pt] {\trimfig{\cgDoc/ins/fig/multiBuildingsTracersAndCut}{\figWidthd}};
  \draw ( 8.5,0.0) node[anchor=south west,xshift=-4pt,yshift=+0pt] {\trimfig{\cgDoc/ins/fig/multiBuildings}{\figWidthd}};
%
 % \draw (current bounding box.south west) rectangle (current bounding box.north east);
% grid:
% \draw[step=1cm,gray] (0,0) grid (16,8);
\end{tikzpicture}
\end{center}
\caption{Incompressible flow through a city scape} \label{fig:cityScape}
\end{figure}
}



Notes:
\begin{flushleft}
 $\blue\diamondsuit$ pseudo steady-state line implicit solver, 4th-order dissipation, \\
 $\blue\diamondsuit$ local time-stepping (spatially varying dt)\\
 $\blue\diamondsuit$ requires 1.4GB of memory, \\
 $\blue\diamondsuit$ cpu = 29s/step, \\
 $\blue\diamondsuit$ 2.2 GHz Xeon, 2 GB of memory
\end{flushleft}


%- 
%- \renewcommand{\clipfig}[1]{\psclip{\psframe[linewidth=2pt](.2,-.2)(9.25,8.75)}\epsfig{#1}\endpsclip}
%- \renewcommand{\figWidth}{.5\linewidth}
%- \newcommand{\figWidthb}{.53\linewidth}
%- \begin{center}
%- \begin{pspicture}(0,-2.0)(10.,8.0)
%- \rput(.5,3.50){\clipfig{file=\obFigures/multiBuildingsTracersAndCut.ps,width=\figWidth}}
%- \rput(9.6,3.75){\clipfig{file=\obFigures/multiBuildings.ps,width=\figWidthb}}
%- \rput*(1.0,-1.7){\makebox(0,0)[c]{\bf 3D flow past a city scape.}}
%- \rput*(9.,-1.7){\makebox(0,0)[c]{\bf Overlapping grid for a city scape.}}
%- % turn on the grid for placement
%- % \psgrid[subgriddiv=2]
%- \end{pspicture}
%- \end{center}\label{fig:cityScape}
