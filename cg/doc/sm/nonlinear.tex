%
%  Nonlinear Solid Mechanics - Notes 
%
\documentclass[11pt]{article}
% \usepackage[bookmarks=true]{hyperref}  % this changes the page location !
\usepackage[bookmarks=true,colorlinks=true,linkcolor=blue]{hyperref}

% \input documentationPageSize.tex
\hbadness=10000 
\sloppy \hfuzz=30pt

% \voffset=-.25truein
% \hoffset=-1.25truein
% \setlength{\textwidth}{7in}      % page width
% \setlength{\textheight}{9.5in}    % page height

\usepackage{calc}
\usepackage[lmargin=.75in,rmargin=.75in,tmargin=.75in,bmargin=.75in]{geometry}

\input homeHenshaw

% --------------------------------------------
% \input{pstricks}\input{pst-node}
% \input{colours}

\usepackage{tikz}
\usepackage{pgfplots}
\input trimFig.tex

% The amssymb package provides various useful mathematical symbols
\usepackage{amsmath}
\usepackage{amssymb}

\newcommand{\Largebf}{\sffamily\bfseries\Large}
\newcommand{\largebf}{\sffamily\bfseries\large}
\newcommand{\largess}{\sffamily\large}
\newcommand{\Largess}{\sffamily\Large}
\newcommand{\bfss}{\sffamily\bfseries}
\newcommand{\smallss}{\sffamily\small}

\newcommand{\beq}{\begin{equation}}
\newcommand{\eeq}{\end{equation}}
\newcommand{\Omegav}{\boldsymbol{\Omega}}
\newcommand{\omegav}{\boldsymbol{\omega}}

\input wdhDefinitions.tex
\newcommand{\mbar}{\bar{m}}
\newcommand{\Rbar}{\bar{R}}
\newcommand{\Ru}{R_u}         % universal gas constant
% \newcommand{\grad}{\nabla}
\newcommand{\Div}{\grad\cdot}
\newcommand{\tauv}{\boldsymbol{\tau}}
\newcommand{\sigmav}{\boldsymbol{\sigma}}
\newcommand{\phiv}{\boldsymbol{\phi}}
\newcommand{\sumi}{\sum_{i=1}^n}
\newcommand{\dt}{\Delta t}


\newcommand{\Pc}{{\mathcal P}}
\newcommand{\Hc}{{\mathcal H}}
\newcommand{\Ec}{{\mathcal E}}

\newcommand{\mw}{W}  % molecular weight
\newcommand{\mwBar}{\overline{W}}  % molecular weight of the mixture
\newcommand{\Dc}{\mathcal{D}}

\newcommand{\tr}{\text{tr}}% trace

% \usepackage{verbatim}
% \usepackage{moreverb}
% \usepackage{graphics}    
% \usepackage{epsfig}    
% \usepackage{fancybox}    

% *** See http://www.eng.cam.ac.uk/help/tpl/textprocessing/squeeze.html
% By default, LaTeX doesn't like to fill more than 0.7 of a text page with tables and graphics, nor does it like too many figures per page. This behaviour can be changed by placing lines like the following before \begin{document}

\renewcommand\floatpagefraction{.9}
\renewcommand\topfraction{.9}
\renewcommand\bottomfraction{.9}
\renewcommand\textfraction{.1}   
\setcounter{totalnumber}{50}
\setcounter{topnumber}{50}
\setcounter{bottomnumber}{50}


% ==========================================================================================================================
\begin{document}
 
\title{Notes on Nonlinear Solid Mechanics}

\author{
Bill Henshaw \\
\  \\
Centre for Applied Scientific Computing, \\
Lawrence Livermore National Laboratory, \\
henshaw@llnl.gov }
 
\maketitle

\tableofcontents

\section{Nomenclature}
\begin{align}
  \rho & \qquad \mbox{density} \\
  u_i & \qquad \mbox{displacement vector} \\
  \epsilon_{ij} & \qquad \mbox{strain tensor} \\
%   \omega_{ij} & \qquad \mbox{rotation tensor} \\
  \sigma_{ij} & \qquad \mbox{Cauchy stress tensor} \\
  \lambda & \qquad \mbox{ Lam\'e constant} \\
  \mu & \qquad \mbox{Lam\'e constant, shear modulus, $G$}\\
  \nu = \lambda/(2(\lambda+\nu) &  \qquad \mbox{Poisson's ratio}\\
  K  = \lambda + 2\mu/3 &  \qquad \mbox{Bulk modulus}
\end{align}

\clearpage 
\section{Governing Equations}

These notes are based on {\em Nonlinear Finite Elements for Continua and Structures} by
T. Belytschko, W.K. Liu and B. Moran~\cite{Belytschko2005}.

Deformation and Motion (Section 3.2)
\begin{align}
  \Omega_0 & \qquad \mbox{reference configuration} \\
  \Omega   & \qquad \mbox{current configuration} \\
  \Xv=\sum_i X_i \ev_i & \qquad \mbox{position of a material pt in $\Omega_0$} \\
  \xv=\sum_i x_i \ev_i & \qquad \mbox{position of a pt in $\Omega$} \\
  \xv=\phiv(\Xv,t) & \qquad \mbox{motion of the body} \\
  \uv(\Xv,t) =\phiv(\Xv,t)-\Xv = \xv-\Xv &\qquad \mbox{displacement vector} \\
  \vv(\Xv,t) = \partial_t \phiv(\Xv,t) = \partial_t \uv(\Xv,t) &\qquad \mbox{velocity of a material pt} \\
  \av(\Xv,t) = \partial_t \vv(\Xv,t)  &\qquad \mbox{acceleration of a material pt} \\
  D_t \vv(\xv,t) = \partial_t \vv(\xv,t) + \vv\cdot\grad \vv &\qquad \mbox{acceleration (Eulerian)} \\
  \Fv = \partial \phiv/\partial \Xv = \partial x_i/ \partial X_j  &\qquad \mbox{deformation gradient} \\
  d\xv = \Fv \cdot d\Xv , & \\
  ~\Fv = \Iv + \partial \uv/\partial \Xv  & \\
  J = \det(\Fv),~~ \partial_t J(\Xv,t) = J \grad\cdot\vv  &\qquad \mbox{Jacobian determinant} \\
  \int_\Omega f(\xv,t)d\Omega = \int_{\Omega_0} f(\Xv,t) J d\Omega_0 &\qquad \mbox{integrals} 
\end{align}


Rigid Body Motion (3.2.8)
\begin{align}
  \xv_t(t) & \qquad \mbox{translation} \\
  \Rv(t) & \qquad \mbox{rotation tensor, $\Rv^T\Rv = I$} \\
  x_{RB}(\Xv,t) = \Rv(t) \Xv + \xv_t(t) & \qquad \mbox{rigid body motion} 
\end{align}

Strain Measures (3.3), use $d\xv = \Fv \cdot d\Xv$, $ds^2=d\xv\cdot d\xv$, $dS^2=d\Xv\cdot d\Xv$, 
\begin{align}
  ds^2 - dS^2 = 2 d\Xv \cdot \Ev \cdot d\Xv & \qquad \mbox{Green strain, $\Ev$} \\
  \Ev = \half(\Fv^T \cdot \Fv - \Iv) & \qquad \mbox{Green strain} \\
  \Cv = \Fv^T \cdot \Fv & \qquad \mbox{right Cauchy-Green deformation tensor} \\
  \Bv = \Fv \cdot \Fv^T & \qquad \mbox{left Cauchy-Green deformation tensor} \\
  \Ev = \half((\grad_0\uv)^T + \grad_0\uv + \grad_0\uv\cdot(\grad_0\uv)^T) & \qquad \mbox{Green strain} \\
  \Ev = \half( \partial u_i/ \partial X_j +  \partial u_j/ \partial X_i 
                       + \partial u_k/ \partial X_i \partial u_k/ \partial X_j) & \qquad \mbox{Green strain} \\
  \grad_0 = \partial/\partial X_i & \qquad \mbox{(left) material gradient}
\end{align}


Rate of deformation (3.3.2)
 \begin{align}
  \Lv = \partial \vv(\xv,t) /\partial \xv = (\grad \vv)^T  = \Dv + \Wv & \qquad \mbox{velocity gradient, $\Lv$ (Eulerian)} \\
  d\vv = \Lv \cdot d \xv \qquad \mbox{} \\ 
  \Dv = \half( \Lv + \Lv^T) = \half( \partial v_i/ \partial x_j +  \partial v_j/ \partial x_i ) & \qquad \mbox{rate of deformation (velocity strain)} \\
  \Wv = \half( \Lv - \Lv^T) = \half( \partial v_i/ \partial x_j -  \partial v_j/ \partial x_i ) & \qquad \mbox{spin tensor} \\
  \partial_t ( ds^2) = 2~d\xv\cdot \Dv \cdot d\xv & \qquad \mbox{} \\
  \Lv = \partial \vv(\xv,t) /\partial \Xv ~ \partial \Xv/\partial\xv = \dot{\Fv} \Fv^{-1} & \qquad \mbox{} \\
  \dot{\Ev} = \Fv^T\cdot\Dv\cdot\Fv,  & \qquad \mbox{{\em pull back operation:} $\xv \rightarrow \Xv$} \\
  \Dv =  \Fv^{-T}\cdot \dot{\Ev} \cdot\Fv^{-1} & \qquad \mbox{{\em push forward operation:} $\Xv \rightarrow \xv$} 
\end{align}


Stress Measures (3.4)
 \begin{align}
  \sigmav =  J^{-1} \Fv\cdot\Pv ~= J^{-1} \Fv\cdot\Sv\cdot\Fv^{T} & \qquad \mbox{Stress: Cauchy: $\sigmav$, PK2: $\Sv$, nominal: $\Pv$} \\
  \Pv = \Sv \Fv^T & \qquad \mbox{$\Sv$ : 2nd Piola-Kirchhoff (PK2) stress} \\
  \nv\cdot\sigmav d\Gamma =\tv d\Gamma = d\fv  & \qquad \mbox{ traction $\tv$} \\
  \nv_0\cdot\Pv d\Gamma_0 =\tv_0 d\Gamma_0 = d\fv  & \qquad \mbox{} \\
  \nv_0\cdot\Sv d\Gamma_0 = \Fv^{-1} \tv_0 d\Gamma_0 = \Fv^{-1}d\fv  & \qquad \mbox{} \\
  \nv d\Gamma = J \nv_0 \cdot\Fv^{-1} d\Gamma_0 & \qquad \mbox{Nanson's relation} \\
\end{align}


Material time derivatives of integrals and Reynold's transport theorem (3.5.3) (for any material region $\Omega$)
 \begin{align}
  D_t \int_\Omega f d\Omega = \int_{\Omega_0}\partial_t( f(\Xv,t) J(\Xv,t) d\Omega_0 & \qquad \mbox{} \\
  D_t \int_\Omega f d\Omega = \int_\Omega (f_t + \grad\cdot( f\vv) ) d\Omega  & \qquad \mbox{ Reynold's transport theorem} \\
\end{align}



Eulerian Conservation Equations (3.5)
 \begin{align}
  D_t \rho + \rho\grad\cdot(\vv)=0 & \qquad \mbox{Mass conservation} \\
  \rho D_t \vv = \grad\cdot( \sigmav) + \rho \bv  & \qquad \mbox{Linear momentum} \\
  \sigmav = \sigmav^T  & \qquad \mbox{Angular momentum} \\
  \rho D_t w^{int} = \Dv:\sigmav -\grad\cdot q + \rho s  & \qquad \mbox{Energy} 
\end{align}

Lagrangian Conservation Equations (3.6) ($\tilde{q}=J^{-1}\Fv^T\cdot\qv$) **check these**
 \begin{align}
  \rho(\Xv,t) J(\Xv,t) = \rho_0(\Xv) & \qquad \mbox{Mass conservation} \\
  \rho_0 \partial_t \vv(\Xv,t) = \grad_0\cdot \Pv + \rho_0 \bv  & \qquad \mbox{Linear momentum} \\
  \Fv\Pv = \Pv^T\Fv^T & \qquad \mbox{Angular momentum} \\
  \rho \partial_t w^{int}(\Xv,t) = \dot{\Fv}^T:\Pv -\grad_0\cdot\tilde{q} + \rho s  & \qquad \mbox{Energy} 
\end{align}


Constitutive Models (5)
 \begin{align}
    w=w(E)  & \qquad \mbox{$w$ : elastic strain energy (potential)} \\
    w(E) = \psi(2\Ev+I) = \psi(\Cv)  & \qquad \mbox{$\psi$ : stored energy potential (Hyper-elastic materials), $\Cv=\Fv^T\Fv$)} \\
\end{align}

% ------------------------------------------------------------------------------------------------
\subsection{Nanson's relation}

Here is a derivation of Nanson's relation. 
We start from the transformation between volume elements, 
\begin{align*}
  &   dv = J\, dV , 
\end{align*}
where $dv= d x_1 d x_2 d x_3$, $dV=d X_1 d X_2 d X_3$ and $J=\det(F)$.
Suppose the volume element $dv$ is formed from the dot product of an oriented area $d\av= da\,\nv$ and
a line element $d\lv$, (and similarly for $dV$), then
\begin{align*}
  & dv = d\av^T d\lv = da\,\nv^T d\lv, \qquad \text{(x-volume element in terms of area element and line element)}  \\
  & dV = d\Av^T d\Lv = dA\, \Nv^T d\Lv,\qquad \text{(X-volume element in terms of area element and line element)}  
\end{align*}
Usng the transformation rule for line elements, 
\begin{align*}
  & d\lv = F d\Lv  \qquad\text{(transformation between line elements)}
\end{align*}
it follows that $dv = J dV $ implies
\begin{align*}
  & da\,\nv^T F \,d\Lv = J\, dA\, \Nv^T \,d\Lv
\end{align*}
and thus
\begin{align*}
  & da\,\nv^T F = J\, dA\, \Nv^T , 
\end{align*}
Defining $\beta=dA/da$ gives the relations
\begin{align*}
  & F^T \nv = \beta J\, \Nv , \\
  & \beta = J^{-1}\,  \Nv^T F^T \nv = J^{-1}\, \nv^T F \Nv \\
  & \nv = \beta J\, F^{-T} \Nv ,
\end{align*}

% ------------------------------------------------------------------------------------------------
\subsection{Time derivatives of the Jacobian determinant}

Consider an transformation $\xv=\gv(\rv,t)$ from $\Real^n\rightarrow \real^n$m and let $H=[h_{ij}]$, $h_{ij}=\partial g_i/\partial r_j$ be the Jacobian matrix and
$J=\det(H)$ be the Jacobian (determinant). 
We wish to compute $\partial J/\partial t$. 
The determinant is given by the Leibnitz formula, 
\begin{align*}
    J =  \det(H) = \sum_{\sigma\in S_n} \text{sgn}(\sigma) \, h_{1,\sigma(1)}\, h_{2,\sigma(2)}\, \ldots \, h_{n,\sigma(n)} 
\end{align*}
where the sum is over all permutations $\sigma$ of $\{ 1,\, 2,\, 3,\ldots\, n\}$. 
Thus the time derivative is 
\begin{align}
   \frac{\partial J}{\partial t} &=   
        \sum_{\sigma\in S_n} \text{sgn}(\sigma) \, \partial_t h_{1,\sigma(1)}\, h_{2,\sigma(2)}\, \ldots \, h_{n,\sigma(n)}  \label{eq:detExpansion} \\
      & +\sum_{\sigma\in S_n} \text{sgn}(\sigma) \,  h_{1,\sigma(1)}\, \partial_t h_{2,\sigma(2)}\, \ldots \, h_{n,\sigma(n)} \\
      & \ldots \\
      & + \sum_{\sigma\in S_n} \text{sgn}(\sigma) \,  h_{1,\sigma(1)}\, h_{2,\sigma(2)}\, \ldots \, \partial_t h_{n,\sigma(n)} . 
\end{align}
Letting $w_i=\partial g_i/\partial t$, then by the chain rule
\begin{align*}
  \frac{\partial h_{ij}}{\partial t} &= \frac{\partial w_i}{\partial r_j} ~
                                     = \sum_k \frac{\partial w_i}{\partial x_k} \, \frac{\partial x_k}{\partial r_j} ~
                                     = \sum_k \frac{\partial w_i}{\partial x_k} \, h_{kj}
\end{align*}
Use this last expression in the first term (other terms will be similiar) in the expansion~\eqref{eq:detExpansion}
\begin{align}
&    \sum_{\sigma\in S_n} \text{sgn}(\sigma) \, \frac{\partial h_{1,\sigma(1)}}{\partial t}\, h_{2,\sigma(2)}\, \ldots \, h_{n,\sigma(n)} 
 = \sum_{\sigma\in S_n} \text{sgn}(\sigma) \, \big(\sum_k \frac{\partial w_1}{\partial x_k} \, h_{k\sigma(1)}\big) \, h_{2,\sigma(2)}\, \ldots \, h_{n,\sigma(n)} , \\
&\qquad =  \sum_k \frac{\partial w_1}{\partial x_k} \Big\{ \sum_{\sigma\in S_n} \text{sgn}(\sigma) \, h_{k\sigma(1)} \, h_{2,\sigma(2)}\, \ldots \, h_{n,\sigma(n)} \Big\} , \label{eq:detSum} \\
&\qquad = \frac{\partial w_1}{\partial x_1} \sum_{\sigma\in S_n} \text{sgn}(\sigma) \, h_{1\sigma(1)} \, h_{2,\sigma(2)}\, \ldots \, h_{n,\sigma(n)} . \\
&\qquad = \frac{\partial w_1}{\partial x_1} J 
\end{align}
where we have used the fact that the determinant is zero when two rows are equal and thus only the term $k=1$ remains in ~\eqref{eq:detSum}. 
Therefore
\begin{align}
\frac{\partial J}{\partial t} &= \Big(\sum_i \frac{\partial w_i}{\partial x_i} \Big) J  ~= (\grad_{\xv}\cdot\wv) \, J
\end{align}
% ------------------------------------------------------------------------------------------------
\subsection{General transformation}

Consider the continuity and momentum equations for the solid (we drop the bars here) in the Eulerian frame
\begin{align*}
  & \frac{\partial\rho}{\partial t} + v_j\frac{\partial\rho}{\partial x_j} + \rho \frac{\partial v_j}{\partial x_j} = 0 , \\
  & \rho \Big[\frac{\partial v_i }{\partial t} + v_j\frac{\partial v_i}{\partial x_j}\Big] = 
      \frac{\partial \sigma_{ji}}{\partial x_j} 
\end{align*}
with summation convention. Now we make a general moving coordinate transformation, $\xv=\gv(\rv,t)$. 
Under this transformation the equations become
\begin{align*}
  & \frac{\partial\rho}{\partial t} + (v_j-w_j)\frac{\partial r_k}{\partial x_j}\frac{\partial\rho}{\partial r_k} 
     + \frac{\rho}{J} \frac{\partial}{\partial r_j}\Big( J \frac{\partial r_j}{\partial x_k} v_k\Big) = 0 , \\
  & \rho \Big[\frac{\partial v_i }{\partial t} 
     + (v_j-w_j)\frac{\partial r_k}{\partial x_j}\frac{\partial v_i}{\partial r_k}\Big] =
   \frac{1}{J}\frac{\partial}{\partial r_j}\Big( J \frac{\partial r_j}{\partial x_k} \sigma_{ki} \Big) 
\end{align*}
where $\wv=\partial \gv/\partial t$ is the grid velocity. We could also write this in fully
conservative form ...



We also have *CHECK*
\begin{align*}
   & \frac{\partial J(\rv,t)}{\partial t} = J \frac{\partial w_j(\rv,t)}{\partial x_j} \, = J \grad_{\xv}\cdot \wv, \\
   & \frac{\partial J(\xv,t)}{\partial t} + w_j\frac{\partial J(\xv,t)}{\partial x_j}  = J \frac{\partial w_j(\xv,t)}{\partial x_j} , \\
\end{align*}

Let $G=\partial\xv/\partial \rv$, $G_{ij}=\partial x_i/\partial r_j$,  be the Jacobian matrix of the transformation, then
{
\newcommand{\mstrut}{\rule{0pt}{15pt}}% strutt to make table height bigger
\newcommand{\ds}{\displaystyle}
\newcommand{\dsm}{\displaystyle\mstrut}
\begin{align*}
    G &= \begin{bmatrix} \ds\frac{\partial\xv}{\partial r_1} &  \ds\frac{\partial\xv}{\partial r_2} &  \ds\frac{\partial\xv}{\partial r_3}\end{bmatrix}  \\
 G^{-1} &= \begin{bmatrix} \ds\frac{\partial\rv}{\partial x_1} &  \ds\frac{\partial\rv}{\partial x_2} &  \ds\frac{\partial\rv}{\partial x_3}\end{bmatrix}
   ~= \begin{bmatrix} \dsm\grad_{\xv} r_1^T \\ \dsm\grad_{\xv} r_2^T \\ \dsm\grad_{\xv} r_3^T \\ \end{bmatrix}
   ~= \begin{bmatrix}  \dsm\alpha_1 \nv_1^T  \\ \dsm\alpha_2\nv_2^T \\ \dsm\alpha_3 \nv_3^T \\ \end{bmatrix}, \quad \alpha_i=1/|\grad_{\xv} r_i|.
\end{align*}
The columns of $G$, $\frac{\partial\xv}{\partial r_i}$ are vectors in the directions of the tangents, $\tv_i$, to the coordinate directions. 
The rows of $G^{-1}$ are vectors in the directions of the normals, $\nv_i$, to the coordinate planes, i.e. $\grad_{\xv} r_i$ is
proportional to the normal to the coordinate plane $r_i=\text{constant}$. 
}

% *NO* We also have $D J/Dt = J \grad_{\xv}\cdot\vv$, and $D(\rho J)/Dt=0$, 
% \begin{align*}
%    & \frac{\partial J}{\partial t} + v_j\frac{\partial J}{\partial x_j}  - J \frac{\partial v_j}{\partial x_j} = 0. 
% \end{align*}
% ------------------------------------------------------------------------------------------------
\clearpage
\input KirchoffMaterial

% ------------------------------------------------------------------------------------------------
\clearpage
\input ConstituitiveModels


% --------------------------------------------------------------------------------------------------
\clearpage
\input hemp.tex







% -------------------------------------------------------------------------------------------------
\vfill\eject
\bibliography{\homeHenshaw/papers/henshaw}
\bibliographystyle{siam}

\end{document}

