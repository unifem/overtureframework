\section{Higher-order time stepping with the Modified Equation approach}

The modified equation approach is based on the taylor series expression
\[
   u^{n\pm1}= u^n + \dt \partial_t u + {\dt^2\over 2} \partial_t^2 u
                                     + {\dt^3\over 3!} \partial_t^3 u 
                                     + {\dt^4\over 4!} \partial_t^4 u + \ldots
\]
giving
\[
   u^{n+1} -2 u ^n + u^{n-1}  =  2{\dt^2\over 2!} \partial_t^2 u + 2 {\dt^4\over 4!} \partial_t^4 u
                    + 2 {\dt^6\over 6!} \partial_t^6 u + \ldots
\]
If we are solving the (forced) second-order wave equation
\[
    u_{tt} = c^2 \Delta u + f 
\] 
then
\begin{align*}
  \partial_t^4 u &= (c^2 \Delta )^2 u + c^2 \Delta f  + f_{tt} \\
  \partial_t^6 u &= (c^2 \Delta )^3 u + (c^2 \Delta)^2 f + c^2 \Delta f_{tt} + \partial_t^4 f \\
  \partial_t^6 u &= (c^2 \Delta )^4 u + (c^2 \Delta)^3 f + (c^2 \Delta)^2 f_{tt} 
                 + c^2 \Delta \partial_t^4 f + \partial_t^6 f
\end{align*}

then we can derive the expression
\begin{align*}
  u^{n+1} -2 u ^n + u^{n-1} &= 2{\dt^2\over 2!} \big(c^2 \Delta u +f \big) \\
   & + 2 {\dt^4\over 4!} \Big\{ (c^2 \Delta)^2 u + c^2 \Delta f + f_{tt}\Big\} \\
   & + 2 {\dt^6\over 6!} \Big\{ (c^2 \Delta)^3 u + (c^2 \Delta)^2 f + c^2 \Delta f_{tt} + \partial_t^4 f \Big\} \\
   & + 2 {\dt^8\over 8!} \Big\{ (c^2 \Delta )^4 u+ (c^2 \Delta)^3 f + (c^2 \Delta)^2 f_{tt} 
                 + c^2 \Delta \partial_t^4 f + \partial_t^6 f \Big\}  + \ldots
\end{align*}
% Define the wave operator
% \[
%   \square = \partial_t^2 + \Delta
% \]
Here's a fourth-order accurate approximation
\begin{align}
  U^{n+1} -2 U ^n + U^{n-1} &= \dt^2 \big(c^2 \Delta_{4h} U + f \big) \\
                   & + {\dt^4\over 12} \big( (c^2 \Delta_{2h})^2 U + c^2 \Delta f  + f_{tt} \big) \label{eq:modifiedOrder4}
\end{align}
where $\Delta_{mh}$ is a $m^{th}$-order approximation.

Here's a sixth-order scheme,
\begin{align*}
  U^{n+1} -2 U ^n + U^{n-1} &= \dt^2 \big(c^2 \Delta_{6h} U + f \big) \\
                 & + {\dt^4\over 12} \big( (c^2 \Delta_{4h})^2 U + c^2 \Delta f  + f_{tt} \big) \\
                 & + {\dt^6\over 360} \big( (c^2 \Delta_{2h})^3 U + (c^2 \Delta)^2 f + c^2 \Delta f_{tt} + \partial_t^4 f \big)
\end{align*}
Here's a eighth-order scheme,
\begin{align*}
   U^{n+1} -2 U ^n + U^{n-1} & = \dt^2 \big(c^2 \Delta_{8h} U + f \big) \\
                  &  + {\dt^4\over 12} \big( (c^2 \Delta_{6h})^2 U + c^2 \Delta f  + f_{tt}\big) \\
                  & + {\dt^6\over 360} \big( (c^2 \Delta_{4h})^3 U + (c^2 \Delta)^2 f + c^2 \Delta f_{tt} + \partial_t^4 f \big) \\
                  & + {\dt^8\over 20160}\big( (c^2 \Delta_{2h})^4 U + F_4 \big)
\end{align*}

% ----------------------------------------------------------------------------------------
\subsection{Modified Equation Time Stepping and Complex Index of Refraction}

Consider the case of Maxwell's equations in a lossy media (complex index of refraction)
\begin{align*}
    u_{tt} &= c^2 \Delta u - \sigma(\xv) u_t + f  ~,\\
    u_{tt} + \sigma(\xv) u_t &= c^2 \Delta u + f  ~.
\end{align*}
where $\sigma u_t$ is the loss term with $\sigma$ the electric conductivity
Let us derive a fourth-order accurate modified equation approximation.
Using
\begin{align*}
  {u^{n+1} - u^{n-1} \over 2 \dt } &= u_t + {\dt^2\over 6} \partial_t^3 u + O( \dt^4 )
\end{align*}
it follows that (we will treat the $\sigma u_t$ term implicitly )
\begin{align*}
 {  u^{n+1} -2 u ^n + u^{n-1} \over \dt^2} + \sigma { u^{n+1} - u^{n-1} \over 2\dt } &=
      \Big[ u_{tt} + \sigma u_t \Big] + \frac{\dt^2}{12}\Big[ \partial_t^4 u + 2\sigma \partial_t^3 u \Big] + O( \dt^4 ) \\
   &= \Big[ c^2 \Delta u + f \Big] + 
            \frac{\dt^2}{12}\Big[ (\partial_t^2 +\sigma\partial_t)^2 u - \sigma^2\partial_t^2 u \Big] + O( \dt^4) \\
   & = \Big[c^2 \Delta u + f \Big] + 
          \frac{\dt^2}{12}\Big[ (\partial_t^2 +\sigma\partial_t)^2 u - \sigma^2( c^2 \Delta u - \sigma u_t + f) \Big] + O( \dt^4) 
\end{align*}
Now (there will some extra terms here if $\sigma$ depends on $t$)
\begin{align*}
 (\partial_t^2 +\sigma\partial_t)^2 u &= (\partial_t^2 +\sigma\partial_t)\big[ c^2 \Delta u + f \big] \\
       &= c^2 \Delta\big[ c^2 \Delta u +f \big] + f_{tt} +\sigma f_t \\
       &= c^4 \Delta^2 u + c^2 \Delta f + f_{tt} +\sigma f_t
\end{align*}
and whence
\begin{align}
 {  u^{n+1} -2 u ^n + u^{n-1} \over \dt^2} + \sigma\Big[1-\frac{(\sigma\dt)^2}{12}\Big] { u^{n+1} - u^{n-1} \over 2\dt } &=
    c^2 \Big[1-\frac{(\sigma\dt)^2}{12}\Big] \Delta u + \frac{\dt^2}{12} c^4 \Delta^2 u \\
     & + \Big[1-\frac{(\sigma\dt)^2}{12}\Big] f + \frac{\dt^2}{12}\big[  c^2 \Delta f + f_{tt} +\sigma f_t \big]+ O( \dt^4) .
\end{align}
This approximation requires relatively simple changes to the method with $\sigma=0$, equation~\eqref{eq:modifiedOrder4}.
%
We probably need 
\[
  \sigma\dt < \sqrt{12}
\]
for this scheme to be stable, but maybe not (?). 

\clearpage
% ----------------------------------------------------------------------------------------
\subsection{Modified Equation Time Stepping and Divergence Cleaning}\label{sec:modifiedDivergenceCleaning}


A possible way to damp the divergence in the solution to Maxwell's equations is to solve the following
coupled systems for $\Ev$ and $\Hv$
\begin{align}
   \Ev_{tt} & = c^2 \Delta \Ev + \fv_E -\alpha( \Ev_t - \eps^{-1} \grad\times \Hv -\gv_E ), \label{eq:Edamped}\\
   \Hv_{tt} & = c^2 \Delta \Hv + \fv_H -\alpha( \Hv_t + \mu^{-1} \grad\times \Ev  -\gv_H ) , \label{eq:Hdamped}
\end{align}
where $\alpha>0$ is a positive damping constant.
Then the divergence of the electric (or magnetic) field, $\delta = \grad\cdot\Ev$, satisfies the damped wave equation
\begin{align*}
   \delta_{tt} & = c^2 \Delta \delta - \alpha \delta_t. 
\end{align*}
Ignoring boundaries, this equation will damp all spatial modes (except the constant mode) to zero as $t\rightarrow \infty$.


To approximate equations~\eqref{eq:Edamped}-\eqref{eq:Hdamped} with the modified equation method
we can either proceed directly or instead start from modified equation approximations for the
second-order and first order systems (which seems easier):
\begin{align}
   \Ev_{tt} & = c^2 \Delta \Ev + \fv_E , \\
   \Hv_{tt} & = c^2 \Delta \Hv + \fv_H ,\\
   \Ev_t &= \eps^{-1} \grad\times \Hv +\gv_E, \\
   \Hv_t &= -\mu^{-1} \grad\times \Ev +\gv_H .
\end{align}
Fourth-order modified equation approximations are 
\begin{align}
 { \Ev^{n+1} -2 \Ev ^n + \Ev^{n-1}\over  \dt^2} &= c^2 \Delta_{4h} \Ev + \fv_E 
                    + {\dt^2\over 12} \big( (c^2 \Delta_{2h})^2 \Ev + c^2 \Delta \fv_E  + \fv_{E,tt} \big), \\
 {\Ev^{n+1} - \Ev^{n-1} \over 2\dt } &=  \partial_t \Ev + {\dt^2\over 6} \partial_t^3 \Ev \\
          &= \eps^{-1}\grad\times\Hv + \eps^{-1} c^2 {\dt^2\over 6} \grad\times\Delta\Hv 
        + \gv_E + \eps^{-1}{\dt^2\over 6}\Big[ c^2  \Big]   + ...(\fv_H,\gv_E) ...
\end{align}
which can be combined
\begin{align}
 { \Ev^{n+1} -2 \Ev ^n + \Ev^{n-1}\over  \dt^2} &+ \alpha {\Ev^{n+1} - \Ev^{n-1} \over 2\dt } = \\
 & \big(c^2 \Delta_{4h} \Ev + \fv_E \big) + {\dt^2\over 12} \big( (c^2 \Delta_{2h})^2 \Ev 
                                                     + c^2 \Delta \fv_E  + \fv_{E,tt} \big) \\
 & + \alpha\Big[ \eps^{-1}\grad\times\Hv + \eps^{-1} c^2 {\dt^2\over 6} \grad\times\Delta\Hv + ... \Big] 
\end{align}
%
For {\bf Cartesian grids} (or rectanglar tensor product grids), if we define the
discrete divergence and discrete curl operators using the same fourth order
approximations
\begin{align}
  {\rm div}_{4h} \Ev &= \grad_{4h} \cdot \Ev = (D_{4x},D_{4y},D_{4z})\cdot\Ev, \\
  {\rm curl}_{4h} \Ev &= \grad_{4h} \times \Ev = (D_{4x},D_{4y},D_{4z})\times\Ev,
\end{align}
then it will follow that the divergence of the curl will be zero for the discrete operators
\begin{align}
   {\rm div}_{4h}{\rm curl}_{4h} \uv &= \grad_{4h} \cdot  \grad_{4h} \times \uv =0 ,
\end{align}
since all the discrete approximates commute, $D_{4x}D_{4y}=D_{4y}D_{4x}$.
Unfortunately, 
if we use the fourth order approximation ${\rm curl}_{4h}$ in the approximation to $c^2\grad\times\Delta\Hv$,
we will end up with at least a 7 point stencil. If we a 5 point stencil to approximate $c^2\grad\times\Delta\Hv$
then it will no longer be true that the discrete divergence of this approximation will be zero. The equation
for the discrete divergence of the electric field, $\delta_{4h}=\grad_{4h} \cdot \Ev$, 
would then have a source term in it and $\delta_{4h}$ would not 
converge to zero in time. 

To avoid this we can instead use the
approximation 
\begin{align}
   \partial_t^3 \Ev &= c^2 \partial_t\Delta \Ev \\
         &= c^2 \Big[ {\Delta \Ev^n -  \Delta\Ev^{n-1} \over \dt} + \dt \partial_t^2 \Delta \Ev \Big] +O(\dt^2) \\
         &= c^2 \Big[ {\Delta \Ev^n -  \Delta\Ev^{n-1} \over \dt} + c^2 \dt \Delta^2 \Ev^n \Big] +O(\dt^2)
\end{align}
With this approximation the discrete divergence should go to zero.


\newcommand{\xrmHat}{\widehat{\xv_{r_m}}}
\newcommand{\rmHat}{\hat{r}_m}
\newcommand{\rnHat}{\hat{r}_n}
For {\bf curvilinear grids}, some care is required in approximating the divergence and curl so that
the discrete div of the discrete curl is zero. 
In this case it is important to approximate both the div and curl with consistent {\em conservative} approximations.
At the continuous level the continuous approximations are
\begin{align*}
  \grad\cdot\fv &= {1\over J}\sum_{m=1}^{d} {\partial\over \partial r_m}\big( J\grad_\xv r_m \cdot \fv\big),\\
  \grad\times\fv  &= \sum_m \frac{1}{J} {\partial\xv\over\partial r_{m}}
                          \Big[ {\partial\over\partial r_{m+1}}\Big( {\partial\xv\over\partial r_{m+2}}\cdot\fv \Big) -
                                {\partial\over\partial r_{m+2}}\Big( {\partial\xv\over\partial r_{m+1}}\cdot\fv \Big) \Big] ,
%  \rmHat &= {\partial \xv \over \partial r_m} / \left\vert {\partial \xv \over \partial r_m} \right\vert
\end{align*}
% Here $\rmHat$ is the unit vector in the direction of increasing $r_m$ and 
where the subscripts $m+1$, $m+2$ are
to be taken modulo $d$. Note that
\begin{align}
   \grad_\xv r_m \cdot {\partial\xv\over\partial r_{n}} = 
   \sum_k {\partial r_m \over\partial x_k}{\partial x_k \over\partial r_{n}} =
   {\partial r_m \over\partial r_{n}} = \delta_{mn}. \label{eq:ortho}
\end{align}
% since $\grad_\xv r_m$ is orthogonal to the $\hat{r}_{m+1}$ and $\hat{r}_{m+2}$
Thus
\begin{align*}
  \grad\cdot(\grad\times\fv) &= 
           {1\over J}\sum_{m=1}^{d} {\partial\over \partial r_m}\Big( J\grad_\xv r_m \cdot (\grad\times\fv)\Big)\\
   &= {1\over J}\sum_{m=1}^{d} 
  {\partial\over \partial r_m}\Big[ {\partial\over\partial r_{m+1}}\Big( {\partial\xv\over\partial r_{m+2}}\cdot\fv \Big) -
                                {\partial\over\partial r_{m+2}}\Big( {\partial\xv\over\partial r_{m+1}}\cdot\fv \Big) \Big] \\
      &= 0 
\end{align*}
In the discrete case we define difference approximations for the unit cube derivatives
\begin{align*}
  D_{r_m} &\approx {\partial\over \partial r_m} 
\end{align*}
%           &= D_{0r_m}\Big( 1 - \alpha(\Delta r_m)^2 
(say a centered second or fourth order accurate approximation) and
then define discrete conservative approximations to div and curl as
\begin{align*}
  \grad_h\cdot\fv &\equiv {1\over J}\sum_{m=1}^{d} D_{r_m}\big( J\grad_\xv r_m \cdot \fv\big),\\
  \grad_h\times\fv  &\equiv \sum_m \frac{1}{J} {\partial\xv\over\partial r_{m}}
                           \Big[ D_{r_{m+1}} \Big( {\partial\xv\over\partial r_{m+2}} \cdot\fv \Big) -
                                 D_{r_{m+2}} \Big( {\partial\xv\over\partial r_{m+1}} \cdot\fv \Big) \Big] ~.
\end{align*}
These approximations will satisfy $\grad_h\cdot(\grad_h\times\fv)=0$ provided~\eqref{eq:ortho} is
satisfied at the discrete level since 
\begin{align*}
  \grad_h\cdot(\grad_h\times\fv) &= 
           {1\over J}\sum_{m=1}^{d} D_{r_m}\Big( J\grad_\xv r_m \cdot (\grad_h\times\fv)\Big)\\
   &= {1\over J}\sum_{m=1}^{d} 
         D_{r_m}\Big[ D_{r_{m+1}}\Big( {\partial\xv\over\partial r_{m+2}}\cdot\fv \Big) -
                      D_{r_{m+2}}\Big( {\partial\xv\over\partial r_{m+1}}\cdot\fv \Big) \Big] \\
      &= 0 .
\end{align*}
Here we have used the fact that $D_{r_m}$ commutes with $D_{r_n}$, $D_{r_m}D_{r_n} = D_{r_n}D_{r_m}$. 

For general {\bf curvilinear grids}, it will {\bf not} be true that the divergence of the Laplacian is equal to
the Laplacian of the divergence. The equation for the discrete divergence of $\Ev$ for
the second-order accurate scheme will be 
\begin{align*}
{\delta_{h}^{n+1} - 2\delta_{h}^{n} + \delta_{h}^{n-1} \over \dt^2 }
  + \alpha {\delta_{h}^{n+1} -  \delta_{h}^{n-1} \over 2 \dt } &= 
                c^2 \grad_{h}\cdot\Delta_{h} \Ev \\
   & = c^2 \Delta_{h}\delta_{h} + c^2 (\grad_{h}\cdot\Delta_{h} \Ev - \Delta_{h}\grad_{h}\cdot \Ev), \\
   & = c^2 \Delta_{h}\delta_{h} + c^2 O( h^2 ) .
\end{align*}
The divergence satisfies a damped wave equation with a forcing function of $O( h^2 )$.
In this case we expect that $\delta_{h} = O(c^2 h^2/\alpha)$. We can thus reduce the divergence by increasing $\alpha$,
although we likely want to keep $\alpha$ smaller than $O(1/\dt)$. 
Note that if the term $\alpha\grad_{h}\cdot\grad_h\times\Hv$ had not been exactly zero then we could only expect
 $\delta_{h} = O(h^2)$ (without the exact factor of $1/\alpha$) and there would have been less of a benefit of
increasing $\alpha$ (This needs to be verified numerically).

For the fourth-order accurate scheme we have
\begin{align*}
{\delta_{h}^{n+1} - 2\delta_{h}^{n} + \delta_{h}^{n-1} \over \dt^2 }
  + \alpha {\delta_{h}^{n+1} -  \delta_{h}^{n-1} \over 2 \dt } &= 
                c^2(1 +\alpha\dt/(6\epsilon))\grad_{h}\cdot\Delta_{h} \Ev^{n} 
                + {c^4 \dt^2 \over 12}(1 +\alpha\dt/\epsilon)\grad_{h}\cdot\Delta_{2h}^2 \Ev^{n} \\
                &- c^2 \alpha\dt/(6\epsilon) \grad_{h}\cdot\Delta_{2h}\Ev^{n-1}
\end{align*}
or 
\begin{align*}
{\delta_{h}^{n+1} - 2\delta_{h}^{n} + \delta_{h}^{n-1} \over \dt^2 }
  + \alpha {\delta_{h}^{n+1} -  \delta_{h}^{n-1} \over 2 \dt } &= 
                c^2(1 +\alpha\dt/(6\epsilon))\Delta_{h} \delta_h^{n} 
                + {c^4 \dt^2 \over 12}(1 +\alpha\dt/\epsilon)\Delta_{2h}^2 \delta_h^{n} \\
                &- c^2 \alpha\dt/(6\epsilon) \Delta_{2h}\delta_h^{n-1} + O( c^2 h^4+ c^4 \dt^2 h^2) (1+\alpha\dt)
\end{align*}
In this case we expect that $\delta_{h} = O( c^2 h^4+ c^4 \dt^2 h^2) (1+\alpha\dt)/\alpha$. 


The divergence equation is of the form
\begin{align*}
  y_{tt} + \alpha y_t + k^2 y =  f 
\end{align*}
where $y \approx \delta$ and $k$ is the wave-number of a spatial mode. The solution of the homogenous ODE is
\begin{align*}
  y &= A e^{\lambda_1 t } + B e^{\lambda_2 t } , \\
  \lambda_m &= -{\alpha\over2} \pm \sqrt{ (\alpha/2)^2 - k^2} , \\
  \lambda_m & \sim -{\alpha\over2} \pm i \beta \qquad\text{if $\alpha < 2k$ (high frequencies)} , \\
            & \sim -{\alpha\over2} \pm {\alpha\over2}(1-2k^2/\alpha^2) \qquad\text{if $\alpha > 2k$ (low frequencies)} .
\end{align*}
The general solution is 
\begin{align*}
  y &=  A e^{\lambda_1 t } + B e^{\lambda_2 t } 
         + {1\over \lambda_2 -\lambda_1} \int_0^t \big( e^{\lambda_2(t-\tau)} - e^{\lambda_1(t-\tau)} \big) f(\tau) ~d\tau, \\
    &= A e^{\lambda_1 t } + B e^{\lambda_2 t } 
         + {1\over 2\sqrt{ (\alpha/2)^2 - k^2}} \int_0^t \big( e^{\lambda_2(t-\tau)} - e^{\lambda_1(t-\tau)} \big) f(\tau) ~d\tau.
\end{align*}
Thus high frequencies will be damped at a rate $e^{-(\alpha/2) t }$ while low frequencies should be $O(f/\alpha)$.
