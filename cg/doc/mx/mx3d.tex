%-----------------------------------------------------------------------
% Notes on elements of the 3D Maxwell Solver
% --------------------------------------------
\documentclass[11pt]{article} 

\input documentationPageSize.tex
%% \pagestyle{empty}

% \addtolength{\oddsidemargin}{-.975in}
% \addtolength {\textwidth} {2.0in}

% \addtolength{\topmargin}{-1.0in}
% \addtolength {\textheight} {1.5in}

% \voffset=-1.25truein
% \hoffset=-1.truein
% \setlength{\textwidth}{6.75in}      % page width
% \setlength{\textheight}{9.5in}    % page height

\input homeHenshaw

\input{pstricks}\input{pst-node}
\input{colours}

% or use the epsfig package if you prefer to use the old commands
\usepackage{epsfig}
\usepackage{calc}
\input clipFig.tex

% The amssymb package provides various useful mathematical symbols
\usepackage{amsmath}
\usepackage{amssymb}

\newcommand{\Largebf}{\sffamily\bfseries\Large}
\newcommand{\largebf}{\sffamily\bfseries\large}
\newcommand{\largess}{\sffamily\large}
\newcommand{\Largess}{\sffamily\Large}
\newcommand{\bfss}{\sffamily\bfseries}
\newcommand{\smallss}{\sffamily\small}

\newcommand{\beq}{\begin{equation}}
\newcommand{\eeq}{\end{equation}}
\newcommand{\Omegav}{\boldsymbol{\Omega}}
\newcommand{\omegav}{\boldsymbol{\omega}}

\input wdhDefinitions.tex
\newcommand{\mbar}{\bar{m}}
\newcommand{\Rbar}{\bar{R}}
\newcommand{\Ru}{R_u}         % universal gas constant
% \newcommand{\grad}{\nabla}
\newcommand{\Div}{\grad\cdot}
\newcommand{\tauv}{\boldsymbol{\tau}}
\newcommand{\sigmav}{\boldsymbol{\sigma}}
\newcommand{\sumi}{\sum_{i=1}^n}


\newcommand{\Gc}{{\mathcal G}}
\newcommand{\Pc}{{\mathcal P}}
\newcommand{\Hc}{{\mathcal H}}
\newcommand{\Ec}{{\mathcal E}}
\newcommand{\Ic}{{\mathcal I}}

\newcommand{\eps}{\epsilon}
\newcommand{\kappav}{{\boldsymbol\kappa}}

\newcommand{\dt}{{\Delta t}}
% \newcommand{\figWidth}{5cm}

\newcommand{\ee}{{\rm e}}
\newcommand{\maxNorm}[1]{\|#1\|_\infty} 
\newcommand{\curl}{\grad\times}

\newcommand{\bogus}[1]{}

\newcommand{\tableFont}{\footnotesize}
% \usepackage{verbatim}
% \usepackage{moreverb}
% \usepackage{graphics}    
% \usepackage{epsfig}    
% \usepackage{fancybox}    


\begin{document}
 
\title{Notes On Solving Maxwell's Equations with Three-Dimensional Interfaces}

\author{
Bill Henshaw \\
% \  \\
% Centre for Applied Scientific Computing, \\
% Lawrence Livermore National Laboratory, \\
% henshaw@llnl.gov 
}
 
\maketitle

\tableofcontents

\section{Introduction}

These notes discuss the enhancements made to the cgmx solver for Maxwell's equations
to solve three-dimensional problems with material interfaces in parallel. 
The approach used in cgmx is described in detail in the paper~\cite{max2006b}.

The main tasks of the current work are
\begin{itemize}
  \item implement the interface jump conditions for Maxwell's equations in three-dimensions.
  \item parallelize the solution to the interface equations.
  \item verify the 3D interface approximations by comparing to known solutions.
  \item apply the new capabilities to modeling the transmission through a wavy interface between glass and air.
  \item build surface grids for 3D AFM data (requires smoothing of the AFM data).
\end{itemize}



The initial conditions and boundary conditions often use the plane wave solution
given by 
\begin{align*}
  \Ev &= \sin( 2\pi (\kv\cdot\xv - \omega t ))~\av, \\
  \Hv &= \sin( 2\pi (\kv\cdot\xv - \omega t ))~\bv, \\
  \omega & = c_m \vert \kv \vert, \\
  c_m &= { 1 \over \sqrt{ \epsilon \mu}}   \qquad \text{(speed of light in the material)} ~.
\end{align*}
where $\kv=(k_x,k_y,k_z)$, $\vert \kv \vert = \sqrt{ k_x^2 + k_y^2 + k_z^2 }$,
$\av=(a_x,a_y,a_z)$, and $\bv=(b_x,b_y,b_z)$ satisfy 
\begin{align}
  \kv\cdot\av &=0, ~~\kv\cdot\bv=0, ~~\mbox{(from $\grad\cdot\Ev=0$ and $\grad\cdot\Hv=0$)}, \\
  \bv &= \sqrt{\frac{\epsilon}{\mu}}~{\kv\times\av \over \vert \kv \vert}, 
       ~~\mbox{(from $\mu \Hv_t = -\grad\times\Ev$)} .  \label{eq:bvFromav} 
\end{align}
Thus given $\av$ with $\kv\cdot\av =0$, $\bv$ is determined by~\eqref{eq:bvFromav}.
The intensity (defined below) for this plane wave solution is
\begin{align*}
  \Ic &= \frac{1}{4}~c_m~( \epsilon \vert \av\vert^2 + \mu \vert \bv\vert^2 ) 
               ~= \frac{1}{2}~c_m~\epsilon \vert \av\vert^2. 
\end{align*}


% 
The energy density $\Ec=\Ec(\xv,t)$ (energy per unit volume) is 
\begin{equation}
  \Ec = \half \epsilon \vert \Ev \vert^2  + \half \mu \vert \Hv \vert^2 ~.
\end{equation}
The intensity $\Ic=\Ic(\xv)$ is the time averaged energy density times $c$, 
\begin{align*}
  \Ic &= {1\over P} \int_0^P \half c\eps \vert \Ev \vert^2 + \half c \mu \vert \Hv \vert^2  ~dt ~~
       = \half c \eps \overline{\vert \Ev \vert^2} + \half c \mu \overline{\vert \Hv \vert^2} ~~
       = \half\sqrt{ \eps\over \mu}~ \overline{\vert \Ev \vert^2} + 
         \half\sqrt{ \mu\over\eps}~ \overline{\vert \Hv \vert^2},
\end{align*}
where $P$ is the period. The units of intensity are power per unit area $[J/(s~m^2)]=[W/m^2]$.


%-------------------------------------------------------------------------------------------------
\section{Interface Jump Conditions}


The basic jump conditions at a material interface are 
\begin{alignat}{3}
  [ \epsilon \nv\cdot\Ev] & =0 &\qquad& [\mu \nv\cdot\Hv] =0   \label{eq:jumpN0}\\
  [ \tau\cdot \Ev] &=0         &\qquad& [ \tau\cdot\Hv ] =0    \label{eq:jumpT0}
\end{alignat}
Here $\tau$ represents a tangent to the material interface. Since there are two linearly independent
tangents, $\tau_m$, $m=1,2$, there will be two linear independent conditions
$[\tau_m\cdot \Ev]=0$, $m=1,2$.

By taking time derivatives of the governing equations it follows that for $m=0,1,2,3,\ldots$, 
\begin{alignat}{3}
  [ \epsilon\nv\cdot\Delta^m \Ev/(\epsilon\mu)^m] & =0  
          &\qquad& [\mu \nv\cdot\Delta^m\Hv/(\epsilon\mu)^m] =0 \label{eq:jumpNm}\\
  [ \tau\cdot\Delta^m \Ev/(\mu\epsilon)^m] &=0 
          &\qquad& [ \tau\cdot\Delta^m\Hv/(\mu\epsilon)^m ] =0    \label{eq:jumpTm} \\
  [ \Div(\Delta^m\Ev)] & =0               &\qquad& [\Div(\Delta^m\Hv)] =0  \\
  [ \mu^{-1}\tau\cdot\curl\Delta^m\Ev/(\mu\epsilon)^m] &=0    
          &\qquad& [ \epsilon^{-1}\tau\cdot\curl\Delta^m\Hv/(\mu\epsilon)^m ] =0  \label{eq:jumpTmII}
\end{alignat}
These interface jump conditions impose conditions for each spatial derivative of the solution.

Another way to write the jump conditions~\eqref{eq:jumpNm} through ~\eqref{eq:jumpTmII} that
doesn't involve the tangent vectors $\tauv$ is as 
\begin{align}
  \Big[ \big(\Delta^m \Ev  + ( (\epsilon-1) \nv\cdot\Delta^m\Ev )~\nv\big)/(\epsilon\mu)^m \Big] &=0, \\
  \Big[ \big(\mu^{-1}(\curl\Delta^m\Ev - (\nv\cdot \curl\Delta^m\Ev)~\nv)
             + \Div(\Delta^m\Ev)~\nv  \big)/(\epsilon\mu)^m \Big] &=0,
\end{align}
and 
\begin{align}
  \Big[ \big(\Hv  + ( (\mu-1) \nv\cdot\Hv )~\nv\big)/(\epsilon\mu)^m \Big] &=0 , \\
  \Big[ \big(\epsilon^{-1}(\curl\Delta^m\Hv - (\nv\cdot \curl\Delta^m\Hv)~\nv )
             + \Div(\Delta^m\Hv)~\nv  \big)/(\epsilon\mu)^m \Big] &=0 ~.
\end{align}
The former equations~\eqref{eq:jumpNm}-\eqref{eq:jumpTmII} follow by taking the
dot product of the above equations with $\nv$ or $\tauv$.  This latter form is 
used when discretizing the equations since there is no need to define
tangent vectors.


%-------------------------------------------------------------------------------------------------
\clearpage 
\input magneticFromElectricField




\clearpage
\input scatPlaneInterface

% ----------------------------------------------------------------------------------
\clearpage
\section{Scattering by a dielectric sphere}
\input scatDielectricSphere

\clearpage
\input mx3dParallel


\clearpage
\input scatBump3d

% -------------------------------------------------------------------------------
\input ../nif/afm/afm3d


\clearpage
\bibliography{\homeHenshaw/papers/henshaw}
\bibliographystyle{siam}

\end{document}


% ----------------------------------------------------------------------------------------------------------



