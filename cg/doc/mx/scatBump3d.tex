\section{Scattering from an 3D interface with a bump}\label{sec:scatBump3d}

\newcommand{\ppw}{{\rm PPW}}
% --------------------------------------------------------------------------
\subsection{Results from the second-order accurate code CGFD2}

\input scatBump3dFig


In this section we present results for a plane wave traveling through an interface between glass and
vacuum. The interface has a smooth bump.

The equation for the bump height is defined in terms of a cosine, 
\begin{align*}
  h(x,y) &= \begin{cases}
               \alpha ( 1 + \cos(2\pi r) ) & \text{if $\vert r\vert <.5$} \\
                 0 & \text{otherwise}
             \end{cases}, \\
  r^2 &= ((x-x_0)/w_0)^2 + ((y-y_0)/w_0)^2. 
\end{align*}
where $w_0=1$ and $\alpha=.1$. 

The overlapping grid for the domain is shown in figure~\ref{fig:scatBump3dGridFig}.

The grid with resolution $m$, denoted by $\Gc^{(m)}$, has a grid spacing of about $\Delta s = 1/(20 m)$. 
For a wave with wave-number $k_z$ the number of points per wavelength will thus be $\ppw=20 m/k_z$.


Figure~\ref{fig:scatBump3dFig} shows results from a plane wave traveling from glass ($\eps=2.25$, bottom)
to vacuum ($\eps=1.$, top) through an interface with a bump using the second-order method CGFD2. 
Grid $\Gc^{(8)}$ was used so that $\ppw=32$ for $k_z=5$ and $\ppw=16$ for $k_z=5$. 
Using only $\ppw=16$ for a second-order accurate scheme is probably somewhat under-resolved.

The incident field was $\Ev=\av~e^{2\pi i(\kv\cdot\xv-\omega t)}$ with $\av=(\sqrt{2/3},\sqrt{2/3},0)$.
Tne intensity of the incident field is $I=\half \sqrt{\eps/\mu} \vert \av \vert ^2 = 1.$ for $\eps=2.25$, $\mu=1$. 


The wave can be seen to focus
above the bump. 
The grid for this problem had about $40$ million grid points. These results were run on
a single processor and required about $4.2$ Gbytes of memory.



% -----------------------------------------------------------------------------------
% define an arrow from top to bottom with two labels
\newcommand{\upArrow}[2]{%
\psline[linewidth=1.pt]{->}(0,-1)(0,1)
\rput[l](0.1,-1){\smallss #1}
\rput[l](0.1,+1){\smallss #2}
}
% define an arrow from bottom to top with two labels
\newcommand{\downArrow}[2]{%
\psline[linewidth=1.pt]{<-}(0,-1)(0,1)
\rput[l](0.1,-1){\smallss #2}
\rput[l](0.1,+1){\smallss #1}
}
\clearpage
\subsection{Results from the fourth-order accurate code CGFD4}


In this section we present further results on the scattering of a plane wave
by a three-dimensional material interface with a bump.


We begin by considering the case when the materials are the same on both sides of the interface.
In this case the plane wave should pass through the interface with no change. Numerically there
will be some effects that result from the interface.
Results are shown in figure~\ref{fig:bump3dFigEps1}. In both cases the intensity is nearly
constant as it should be. For the fine grid case, apart from effects from the far field top boundary (these need to be fixed),
the computed intensity is in the range $[.992,1.005]$. The fine grid case which uses $\ppw=32$ is
thus quite accurate. For the coarse grid case the intensity is approximately in the range $[.977,1.006]$. 
There is a general trend for the intensity to decrease as the wave propagates. This is caused by the artificial dissipation. 

\input bump3dConvexFig

Figure~\ref{fig:bump3dFigGlassToAirConvexKz20} shows results for $k_z=20$ on grid $\Gc^{(16)}$, which 
has about 300M grid points. This computation was run on 32 nodes (256 processors) on zeus for 2400 steps 
and took about 100 min.


Figures~\ref{fig:bump3dFigGlassToAirConvex} and~\ref{fig:bump3dFigGlassToAirConvex} show results for a
plane wave passing from glass to air through a convex glass bump. 
The convex glass bump acts to focus the beam in the air along the axis of symmetry. The beam is also focussed, to
a lesser degree, in the glass, along the axis, due to reflections from the interface which acts as a concave mirror. 
Results are shown for different incident wave numbers $k_z$.



Figure~\ref{fig:bump3dFigAirToGlassConvex} shows results for a plane wave passing from air to glass through a
convex glass bump. The beam is focussed in the glass along the axis.


We now consider the case when the glass acts as a concave lens.
Figures~\ref{fig:bump3dFigGlassToAirConcave} and~\ref{fig:bump3dFigAirToGlassConcave} shows
results for this case.


\input bump3dConcaveFig

% ----------------------------------------------------------------------------
\clearpage
\subsection{Results from a billion point computation}

Figure~\ref{fig:bump3dFigGlassToAirConvexBig} shows result from a computation
with 1 billion grid points. This computation was run on 
256 processors (32 nodes, 8 processors per node) on the Zeus linux cluster
(nodes are AMD Opteron with 8 cores (2.4 GHz) and 16 GB memory) . The computation took about
$7$ seconds/step or 8.3 hours for 4160 steps.


% results from runs/cgmx/bump3d/bump3dO4f24kz30Eps2p25.out
{\footnotesize
\begin{verbatim}
              ---------Maxwell Summary------- 
                       Sat Jul 11 21:33:29 2009
               Grid:   interfaceBump3d1bumpe24.order4 
  ==== numberOfStepsTaken =     4160, grids=4, gridpts =1.02195e+09, interp pts=8531528, processors=256 ==== 
  ==== memory per-proc: [min=504.438,ave=592.309,max=707.031](Mb), max-recorded=870.312 (Mb), total=151631 (Mb)
   Timings:         (ave-sec/proc:)   seconds    sec/step   sec/step/pt     %     [max-s/proc] [min-s/proc]
total time..........................  2.96e+04    7.11e+00    6.96e-09   100.000   2.959e+04   2.959e+04
setup and initialize................  2.24e+01    5.38e-03    5.26e-12     0.076   2.735e+01   1.847e+01
initial conditions..................  2.02e+01    4.84e-03    4.74e-12     0.068   2.019e+01   2.014e+01
advance.............................  2.89e+04    6.95e+00    6.80e-09    97.732   2.892e+04   2.892e+04
  advance rectangular grids.........  7.31e+03    1.76e+00    1.72e-09    24.694   8.400e+03   4.804e+03
  advance curvilinear grids.........  1.11e+03    2.68e-01    2.62e-10     3.765   2.701e+03   3.850e-01
   (advOpt).........................  8.42e+03    2.02e+00    1.98e-09    28.448   1.085e+04   5.182e+03
  add dissipation...................  3.70e+03    8.88e-01    8.69e-10    12.490   4.043e+03   3.089e+03
  boundary conditions...............  2.77e+03    6.65e-01    6.51e-10     9.356   2.989e+03   2.117e+03
  interface bc......................  6.81e+02    1.64e-01    1.60e-10     2.302   2.263e+03   1.399e+00
  interpolation.....................  6.61e+03    1.59e+00    1.55e-09    22.331   7.701e+03   5.086e+03
  update ghost (parallel)...........  6.05e+03    1.45e+00    1.42e-09    20.437   8.512e+03   4.036e+03
compute intensity...................  3.04e+03    7.31e-01    7.15e-10    10.276   3.234e+03   2.836e+03
compute dt..........................  3.53e-02    8.50e-06    8.31e-15     0.000   3.545e-02   3.528e-02
plotting............................  1.27e+01    3.06e-03    2.99e-12     0.043   1.281e+01   1.267e+01
showFile............................  6.11e+02    1.47e-01    1.44e-10     2.064   6.112e+02   6.105e+02
\end{verbatim}
}

\input bump3dBigFig
