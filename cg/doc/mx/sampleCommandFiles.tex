\section{Sample command files for running cgmx} \label{sec:demo}

Command files are supported throughout the Overture. They are files
that contain lists of commands. These commands can initially be saved
when the user is interactively choosing options.  The \Index{command files}
can then be used to re-run the job. Command files can be edited and
changed.

In this section we present a number of command files that can be used
to run cgmx.

\subsection{Running a command file} \label{sec:runningCommandFiles} 

Given a \Index{command file} for cgmx such as {\tt cyleigen.cmd}, found in {\tt
cmd/cyleigen.cmd}, one can type `{\tt cgmx cyleigen.cmd}' to run this command
file . You can also just type `{\tt cgmx cyleigen}, leaving off the {\tt
.cmd} suffix. Typing `{\tt cgmx -noplot cyleigen}' will run without
interactive graphics (unless the command file turns on graphics). Note that here
I assume that the {\tt bin} directory is in your path so that the {\tt
cgmx} command is found when you type it's name. The Cgmx sample
command files will automatically look for an overlapping grid in the {\tt
Overture/sampleGrids} directory, unless the grid is first found in the location
specified in the command file.

When you run a command file a graphics screen will appear and after some
processing the run-time dialog should appear and the initial conditions will be
plotted. The program will also print out some information about the problem
being solved. At this point choose {\tt continue} or {\tt movie
mode}. Section~(\ref{sec:runTimeDialog}) describes the options available in the
run time dialog.

When running in parallel it is convenient to define a shell variable for the parallel
version of cgmx.
For example in the tcsh shell you might use
\begin{verbatim}
  set cgmxp  = ${CGBUILDPREFIX}/mx/bin/cgmx
\end{verbatim} % $
An example of running the parallel version is then
\begin{verbatim}
  mpirun -np 2 $cgmxp cic.planeWaveBC -g=cice2.order4.hdf
\end{verbatim} % $


%------------------------------------------------------------------------------------------
\clearpage
\subsection{Scattering of a plane wave from a two-dimensional conducting cylinder} \label{sec:cyl2dScat}



The command file {\tt cic.planeWaveBC.cmd} can be used to compute the scattering
of a plane wave from a PEC (perfectly electric conducting) cylinder. The exact solution is known for this case
and the errors in the numerical solution are determined while the solution
is being computed.

% \noindent Example:

\noindent (1) Generate the grid using the command file {\tt cicArg.cmd} in {\tt Overture/sampleGrids}:
\begin{verbatim}
  ogen -noplot cicArg -order=4 -interp=e -factor=2
\end{verbatim}

\noindent (2) Run cgmx: 
\begin{verbatim}
  cgmx cic.planeWaveBC -g=cice2.order4.hdf
\end{verbatim}

\noindent Parallel: If you have built the parallel version then use 
\begin{verbatim}
  mpirun -np 2 $cgmxp cic.planeWaveBC -g=cice2.order4.hdf
\end{verbatim} % $
where {\tt \$cgmxp} is a variable that points to the parallel version of cgmx (see Section~\ref{sec:runningCommandFiles}).

{
\begin{figure}[hbt]
\newcommand{\figWidth}{5.5cm}
\newcommand{\trimfig}[2]{\trimFig{#1}{#2}{0.1}{0.05}{.05}{.05}}
\begin{center}
\begin{tikzpicture}[scale=1]
  \useasboundingbox (0,.5) rectangle (17,5.5);  % set the bounding box (so we have less surrounding white space)
  \draw ( 0.0, 0) node[anchor=south west] {\trimfig{figures/scatCyl2dExT1p0}{\figWidth}};
  \draw ( 5.7, 0) node[anchor=south west] {\trimfig{figures/scatCyl2dEyT1p0}{\figWidth}};
  \draw (11.4, 0) node[anchor=south west] {\trimfig{figures/scatCyl2dHzT1p0}{\figWidth}};
 % - labels
 %   \draw (\txa,4.75) node[draw,fill=white,anchor=east] {\scriptsize $t=0.5$};
 %   \draw (\txb,4.75) node[draw,fill=white,anchor=east] {\scriptsize $t=1.0$};
 %   \draw (\txc,4.75) node[draw,fill=white,anchor=east] {\scriptsize $t=1.5$};
 %  \draw (current bounding box.south west) rectangle (current bounding box.north east);
% grid:
%  \draw[step=1cm,gray] (0,0) grid (17.0,5);
\end{tikzpicture}
\end{center}
\caption{Scattering of a plane wave by a PEC cylinder. Computed solution at $t=1.0$, $E_x$, $E_y$ and $H_z$.}
\label{fig:cyl2dScat}
\end{figure}
}

\noindent{\bf Notes:}
\begin{enumerate}
  \item See the comments at the top of the command file for command line arguments and further examples.
  \item This case was run with fourth-order accuracy.
\end{enumerate}


%------------------------------------------------------------------------------------------
\clearpage
\input scatteringChirped


% ------------------------------------------------------------------------------------------
% \clearpage
% \subsection{Eigenfunctions of a three-dimensional cylinder} \label{sec:cyl3d}


%------------------------------------------------------------------------------------------
\clearpage
\subsection{Scattering from a three-dimensional conducting sphere} \label{sec:sphere3dScat}

The command file {\tt sib.planeWaveBC.cmd} can be used to compute the scattering
of a plane wave from a PEC sphere. The exact solution is known for this case
and the errors in the numerical solution are determined while the solution
is being computed.

% \noindent Example:

\noindent (1) Generate the grid using the command file {\tt sibArg.cmd} in {\tt Overture/sampleGrids}:
\begin{verbatim}
  ogen -noplot sibArg -order=4 -interp=e -factor=4
\end{verbatim}

\noindent (2) Run cgmx: 
\begin{verbatim}
  cgmx sib.planeWaveBC -g=sibe4.order4.hdf
\end{verbatim}

\noindent Parallel: If you have built the parallel version then use 
\begin{verbatim}
  mpirun -np 2 $cgmxp sib.planeWaveBC -g=sibe4.order4.hdf
\end{verbatim} % $
where {\tt \$cgmxp} is a variable that points to the parallel version of cgmx, e.g.,
\begin{verbatim}
  set cgmxp  = ${CGBUILDPREFIX\}/mx/bin/cgmx}
\end{verbatim} % $

{
\begin{figure}[hbt]
\newcommand{\figWidth}{5.5cm}
\newcommand{\trimfig}[2]{\trimFig{#1}{#2}{0.1}{0.05}{.05}{.05}}
\begin{center}
\begin{tikzpicture}[scale=1]
  \useasboundingbox (0,.5) rectangle (17,5.5);  % set the bounding box (so we have less surrounding white space)
  \draw ( 0.0, 0) node[anchor=south west] {\trimfig{figures/scatSphereExT1p0}{\figWidth}};
  \draw ( 5.7, 0) node[anchor=south west] {\trimfig{figures/scatSphereEyT1p0}{\figWidth}};
  \draw (11.4, 0) node[anchor=south west] {\trimfig{figures/scatSphereEzT1p0}{\figWidth}};
 % - labels
 %   \draw (\txa,4.75) node[draw,fill=white,anchor=east] {\scriptsize $t=0.5$};
 %   \draw (\txb,4.75) node[draw,fill=white,anchor=east] {\scriptsize $t=1.0$};
 %   \draw (\txc,4.75) node[draw,fill=white,anchor=east] {\scriptsize $t=1.5$};
 %  \draw (current bounding box.south west) rectangle (current bounding box.north east);
% grid:
%  \draw[step=1cm,gray] (0,0) grid (17.0,5);
\end{tikzpicture}
\end{center}
\caption{Scattering of a plane wave by a PEC sphere. Computed solution at $t=1.0$, $E_x$, $E_y$ and $E_z$.}
\label{fig:cyl2dScat}
\end{figure}
}

\noindent{\bf Notes:}
\begin{enumerate}
  \item See the comments at the top of the command file for command line arguments and further examples.
  \item This case was run with fourth-order accuracy.
\end{enumerate}



%------------------------------------------------------------------------------------------
\clearpage
\subsection{Scattering of a plane wave from a two-dimensional dielectric cylinder} \label{sec:cyl2dDielectricScat}

The command file {\tt dielectricCyl.cmd} can be used to compute the scattering
of a plane wave from a dielectric cylinder. The exact solution is known for this case
and the errors in the numerical solution are determined while the solution
is being computed.

% \noindent Example:

\noindent (1) Generate the grid using the command file {\tt innerOuter.cmd} in {\tt Overture/sampleGrids}:
{\small
\begin{verbatim}
  ogen -noplot innerOuter -factor=4 -order=4 -deltaRad=.5 -interp=e -name="innerOutere4.order4.hdf"
\end{verbatim}
}
\noindent (2) Run cgmx: 
\begin{verbatim}
  cgmx dielectricCyl -g=innerOutere4.order4.hdf -kx=2 -eps1=.25 -eps2=1. -go=halt
\end{verbatim}

{
\begin{figure}[hbt]
\newcommand{\figWidth}{5.5cm}
\newcommand{\trimfig}[2]{\trimFig{#1}{#2}{0.1}{0.05}{.05}{.05}}
\begin{center}
\begin{tikzpicture}[scale=1]
  \useasboundingbox (0,.5) rectangle (17,5.5);  % set the bounding box (so we have less surrounding white space)
  \draw ( 0.0, 0) node[anchor=south west] {\trimfig{figures/cyl2dDieExT1p0}{\figWidth}};
  \draw ( 5.7, 0) node[anchor=south west] {\trimfig{figures/cyl2dDieEyT1p0}{\figWidth}};
  \draw (11.4, 0) node[anchor=south west] {\trimfig{figures/cyl2dDieHzT1p0}{\figWidth}};
 % - labels
 %   \draw (\txa,4.75) node[draw,fill=white,anchor=east] {\scriptsize $t=0.5$};
 %   \draw (\txb,4.75) node[draw,fill=white,anchor=east] {\scriptsize $t=1.0$};
 %   \draw (\txc,4.75) node[draw,fill=white,anchor=east] {\scriptsize $t=1.5$};
 %  \draw (current bounding box.south west) rectangle (current bounding box.north east);
% grid:
%  \draw[step=1cm,gray] (0,0) grid (17.0,5);
\end{tikzpicture}
\end{center}
\caption{Scattering of a plane wave from a dielectric cylinder. Computed solution at $t=1.0$, $E_x$, $E_y$ and $H_z$.}
\label{fig:cyl2dDielectricScat}
\end{figure}
}

\noindent{\bf Notes:}
\begin{enumerate}
  \item See the comments at the top of the command file for command line arguments and further examples.
  \item This case was run with fourth-order accuracy.
\end{enumerate}


%------------------------------------------------------------------------------------------
\clearpage
\subsection{Scattering of a plane wave from a two-dimensional dielectric cylinder - Yee scheme} \label{sec:cyl2dDielectricScat}

The command file {\tt dielectricCyl.cmd} can be used to compute the scattering
of a plane wave from a dielectric cylinder using the Yee scheme. The exact solution is known for this case
and the errors in the numerical solution are determined while the solution
is being computed.

% \noindent Example:

\noindent (1) Generate the grid using the command file {\tt bigSquare.cmd} in {\tt Overture/sampleGrids}:
{\small
\begin{verbatim}
  ogen noplot bigSquare -factor=8 -xa=-1. -xb=1. -ya=-1. -yb=1. -name="bigSquareSize1f8.hdf"
\end{verbatim}
}
\noindent (2) Run cgmx: 
\begin{verbatim}
  cgmx dielectricCyl -g=bigSquareSize1f8.hdf -kx=2 -eps1=.25 -eps2=1. -method=Yee -errorNorm=2 -go=halt
\end{verbatim}

{
\begin{figure}[hbt]
\newcommand{\figWidth}{5.5cm}
\newcommand{\trimfig}[2]{\trimFig{#1}{#2}{0.1}{0.05}{.05}{.05}}
\begin{center}
\begin{tikzpicture}[scale=1]
  \useasboundingbox (0,.5) rectangle (17,5.5);  % set the bounding box (so we have less surrounding white space)
  \draw ( 0.0, 0) node[anchor=south west] {\trimfig{figures/cyl2dDieYeeExT1p0}{\figWidth}};
  \draw ( 5.7, 0) node[anchor=south west] {\trimfig{figures/cyl2dDieYeeEyT1p0}{\figWidth}};
  \draw (11.4, 0) node[anchor=south west] {\trimfig{figures/cyl2dDieYeeHzT1p0}{\figWidth}};
 % - labels
 %   \draw (\txa,4.75) node[draw,fill=white,anchor=east] {\scriptsize $t=0.5$};
 %   \draw (\txb,4.75) node[draw,fill=white,anchor=east] {\scriptsize $t=1.0$};
 %   \draw (\txc,4.75) node[draw,fill=white,anchor=east] {\scriptsize $t=1.5$};
 %  \draw (current bounding box.south west) rectangle (current bounding box.north east);
% grid:
%  \draw[step=1cm,gray] (0,0) grid (17.0,5);
\end{tikzpicture}
\end{center}
\caption{Scattering of a plane wave from a dielectric cylinder using the Yee scheme. The solution is
computed on a single Cartesian grid and the cylinder is represented in a stair-step fashion. 
Computed solution at $t=1.0$, $E_x$, $E_y$ and $H_z$.}
\label{fig:cyl2dDielectricScatYee}
\end{figure}
}

\noindent{\bf Notes:}
\begin{enumerate}
  \item The material parameters are chosen using ...
\end{enumerate}



%------------------------------------------------------------------------------------------
\clearpage
\subsection{Scattering of a plane wave from a three-dimensional dielectric sphere} \label{sec:sphere3dDielectricScat}

The command file {\tt dielectricCyl.cmd} (yes, the same file as for the 2d dielectric cylinder) can be used to compute the scattering
of a plane wave from a dielectric cylinder. The exact solution is known for this case
and the errors in the numerical solution are determined while the solution
is being computed.

% \noindent Example:

\noindent (1) Generate the grid using the command file {\tt solidSphereInABox.cmd} in {\tt Overture/sampleGrids}:
{\small
\begin{verbatim}
  ogen noplot solidSphereInABox -order=4 -interp=e -factor=2
\end{verbatim}
}
\noindent (2) Run cgmx: 
{\small
\begin{verbatim}
  cgmx dielectricCyl -cyl=0 -g=solidSphereInABoxe2.order4 -kx=1 -eps1=.25 -eps2=1. -go=halt -tp=.01
\end{verbatim}
}
{
\begin{figure}[hbt]
\newcommand{\figWidth}{5.5cm}
\newcommand{\trimfig}[2]{\trimFig{#1}{#2}{0.1}{0.05}{.05}{.05}}
\begin{center}
\begin{tikzpicture}[scale=1]
  \useasboundingbox (0,.5) rectangle (17,5.5);  % set the bounding box (so we have less surrounding white space)
  \draw ( 0.0, 0) node[anchor=south west] {\trimfig{figures/scatDielectricSphereEx}{\figWidth}};
  \draw ( 5.7, 0) node[anchor=south west] {\trimfig{figures/scatDielectricSphereEy}{\figWidth}};
  \draw (11.4, 0) node[anchor=south west] {\trimfig{figures/scatDielectricSphereEz}{\figWidth}};
 % - labels
 %   \draw (\txa,4.75) node[draw,fill=white,anchor=east] {\scriptsize $t=0.5$};
 %   \draw (\txb,4.75) node[draw,fill=white,anchor=east] {\scriptsize $t=1.0$};
 %   \draw (\txc,4.75) node[draw,fill=white,anchor=east] {\scriptsize $t=1.5$};
 %  \draw (current bounding box.south west) rectangle (current bounding box.north east);
% grid:
%  \draw[step=1cm,gray] (0,0) grid (17.0,5);
\end{tikzpicture}
\end{center}
\caption{Scattering of a plane wave from a dielectric sphere. Computed solution at $t=1.0$, $E_x$, $E_y$ and $H_z$.}
\label{fig:cyl2dDielectricScat}
\end{figure}
}

\noindent{\bf Notes:}
\begin{enumerate}
  \item See the comments at the top of the command file for command line arguments and further examples.
  \item This case was run with fourth-order accuracy.
\end{enumerate}


%------------------------------------------------------------------------------------------
\clearpage
\subsection{Diffraction from a two-dimensional knife edge slit} \label{sec:knifeEdge2d}

The command file {\tt knifeEdgel.cmd} can be used to compute the scattering
of a plane wave from narrow slit.

\noindent (1) Generate the grid using the command file {\tt innerOuter.cmd} in {\tt Overture/sampleGrids}:
{\small
\begin{verbatim}
  ogen -noplot knifeEdge -interp=e -factor=2 -order=4 -yTop=1.1 -name="knifeSlit2.order4.hdf"
\end{verbatim}
}
\noindent (2) Run cgmx: 
\begin{verbatim}
  cgmx knifeEdge -g=knifeSlit2.order4 -kx=8 -tp=.05 -tf=2. -plotIntensity=1 -go=halt
\end{verbatim}

{
\begin{figure}[hbt]
\newcommand{\figWidth}{5.5cm}
\newcommand{\trimfig}[2]{\trimFig{#1}{#2}{0.1}{0.05}{.05}{.05}}
\begin{center}
\begin{tikzpicture}[scale=1]
  \useasboundingbox (0,.5) rectangle (17,5.5);  % set the bounding box (so we have less surrounding white space)
  \draw ( 0.0, 0) node[anchor=south west] {\trimfig{figures/knifeEdgeExT1p5}{\figWidth}};
  \draw ( 5.7, 0) node[anchor=south west] {\trimfig{figures/knifeEdgeEyT1p5}{\figWidth}};
  \draw (11.4, 0) node[anchor=south west] {\trimfig{figures/knifeEdgeIntensityT1p5}{\figWidth}};
 % - labels
 %   \draw (\txa,4.75) node[draw,fill=white,anchor=east] {\scriptsize $t=0.5$};
 %   \draw (\txb,4.75) node[draw,fill=white,anchor=east] {\scriptsize $t=1.0$};
 %   \draw (\txc,4.75) node[draw,fill=white,anchor=east] {\scriptsize $t=1.5$};
 %  \draw (current bounding box.south west) rectangle (current bounding box.north east);
% grid:
%  \draw[step=1cm,gray] (0,0) grid (17.0,5);
\end{tikzpicture}
\end{center}
\caption{Scattering of a plane wave from a slit. Computed solution at $t=1.5$, $E_x$, $E_y$ and intensity.}
\label{fig:knifeEdge2d}
\end{figure}
}

\noindent{\bf Notes:}
\begin{enumerate}
  \item Comment on initial conditions and initial conditions bounding box ...
  \item Adjust boundaries for incident field ...
\end{enumerate}



%------------------------------------------------------------------------------------------
\clearpage
\subsection{Scattering of a plane wave by a two-dimensional triangular body} \label{sec:scatTri}

The command file {\tt scat.cmd} can be used to compute the scattering
of a plane wave from a body.

\noindent (1) Generate the grid using the command file {\tt triangleArg.cmd} in {\tt Overture/sampleGrids}:
{\small
\begin{verbatim}
  ogen noplot triangleArg -factor=8 -order=4 -interp=e
\end{verbatim}
}
\noindent (2) Run cgmx: 
\begin{verbatim}
  cgmx scat -g=trianglee8.order4.hdf -bg=backGround
\end{verbatim}

{
\begin{figure}[hbt]
\newcommand{\figWidth}{5.5cm}
\newcommand{\trimfig}[2]{\trimFig{#1}{#2}{0.1}{0.05}{.05}{.05}}
\begin{center}
\begin{tikzpicture}[scale=1]
  \useasboundingbox (0,.5) rectangle (17,5.5);  % set the bounding box (so we have less surrounding white space)
  \draw ( 0.0, 0) node[anchor=south west] {\trimfig{figures/scatTriExT1p5}{\figWidth}};
  \draw ( 5.7, 0) node[anchor=south west] {\trimfig{figures/scatTriEyT1p5}{\figWidth}};
  \draw (11.4, 0) node[anchor=south west] {\trimfig{figures/scatTriHzT1p5}{\figWidth}};
 % - labels
 %   \draw (\txa,4.75) node[draw,fill=white,anchor=east] {\scriptsize $t=0.5$};
 %   \draw (\txb,4.75) node[draw,fill=white,anchor=east] {\scriptsize $t=1.0$};
 %   \draw (\txc,4.75) node[draw,fill=white,anchor=east] {\scriptsize $t=1.5$};
 %  \draw (current bounding box.south west) rectangle (current bounding box.north east);
% grid:
%  \draw[step=1cm,gray] (0,0) grid (17.0,5);
\end{tikzpicture}
\end{center}
\caption{Scattering of a plane wave from a triangular body. Computed solution at $t=1.5$, $E_x$, $E_y$ and $H_z$. The initial startup wave-front can
     be seen leaving the domain in the upper left and lower left.}
\label{fig:scatTri2d}
\end{figure}
}

\noindent{\bf Notes:}
\begin{enumerate}
  \item Computes the scattered field directly using the {\tt planeWaveBoundaryForcing} option.
  \item Boundary conditions on the outer square use the {\tt abcEM2} far-field conditions.
\end{enumerate}


%------------------------------------------------------------------------------------------
\clearpage
\subsection{Comparing far-field boundary conditions} \label{sec:compareFarField}

The command file {\tt rbc.cmd} can be used to compare far-field boundary conditions.
A Gaussian source~\ref{sec:gaussianSource} is placed in the middle of square or box and far field boundary 
conditions are applied to all boundaries. For comparison we also show results when the outer boundary
is taken as a perfect electrical conductor (which results in large reflections).

\noindent (1) Generate the grids (command files in Overture/sampleGrids)
{\small
\begin{verbatim}
  ogen -noplot squareArg -order=4 -nx=128
  ogen noplot boxArg -order=4 -xa=-1. -xb=1. -ya=-1. -yb=1. -za=-1. -zb=1. ...
                     -factor=4 -name="boxLx2Ly2Lz2Factor4.order4.hdf"
\end{verbatim}
}
\noindent (2a) Run cgmx with the square grid and different boundary conditions: 
{\small
\begin{verbatim}
  cgmx rbc -g=square128.order4.hdf -x0=.5 -y0=.5 -rbc=abcEM2 -go=halt
  cgmx rbc -g=square128.order4.hdf -x0=.5 -y0=.5 -rbc=abcPML -pmlWidth=21 -pmlStrength=50. -go=halt
  cgmx rbc -g=square128.order4.hdf -x0=.5 -y0=.5 -rbc=perfectElectricalConductor -go=halt  
\end{verbatim}
}
\noindent (2b) Run cgmx with the box grid and different boundary conditions: 
{\small
\begin{verbatim}
  cgmx rbc -g=boxLx2Ly2Lz2Factor4.order4.hdf -rbc=abcEM2 -x0=0. -y0=0. -z0=0. -go=halt
  cgmx rbc -g=boxLx2Ly2Lz2Factor4.order4.hdf -rbc=abcPML -x0=0. -y0=0. -z0=0. -pmlWidth=11 -go=halt
  cgmx rbc -g=boxLx2Ly2Lz2Factor4.order4.hdf -rbc=perfectElectricalConductor -x0=0. -y0=0. -z0=0. -go=halt
\end{verbatim}
}
{
\begin{figure}[hbt]
\newcommand{\figWidth}{5.5cm}
\newcommand{\trimfig}[2]{\trimFig{#1}{#2}{0.1}{0.05}{.05}{.05}}
\begin{center}
\begin{tikzpicture}[scale=1]
  \useasboundingbox (0,.5) rectangle (17,5.5);  % set the bounding box (so we have less surrounding white space)
  \draw ( 0.0, 0) node[anchor=south west] {\trimfig{figures/rbcSquareAbcEm2ExT1p0}{\figWidth}};
  \draw ( 5.7, 0) node[anchor=south west] {\trimfig{figures/rbcSquarePMLExT1p0}{\figWidth}};
  \draw (11.4, 0) node[anchor=south west] {\trimfig{figures/rbcSquarePECExT1p0}{\figWidth}};
 % - labels
 %   \draw (\txa,4.75) node[draw,fill=white,anchor=east] {\scriptsize $t=0.5$};
 %   \draw (\txb,4.75) node[draw,fill=white,anchor=east] {\scriptsize $t=1.0$};
 %   \draw (\txc,4.75) node[draw,fill=white,anchor=east] {\scriptsize $t=1.5$};
 %  \draw (current bounding box.south west) rectangle (current bounding box.north east);
% grid:
%  \draw[step=1cm,gray] (0,0) grid (17.0,5);
\end{tikzpicture}
\end{center}
\caption{Gaussian source in a square with different boundary conditions. $E_x$ at time $t=1.0$. Left bc={\tt abcEM2}, middle: bc={\tt abcPML} and
right bc={\tt perfectElectricalConductor}. }
\label{fig:rbcSquare}
\end{figure}
}
{
\begin{figure}[hbt]
\newcommand{\figWidth}{5.5cm}
\newcommand{\trimfig}[2]{\trimFig{#1}{#2}{0.1}{0.05}{.05}{.05}}
\begin{center}
\begin{tikzpicture}[scale=1]
  \useasboundingbox (0,.5) rectangle (17,5.5);  % set the bounding box (so we have less surrounding white space)
  \draw ( 0.0, 0) node[anchor=south west] {\trimfig{figures/rbcBoxAbcEM2ExT2p0}{\figWidth}};
  \draw ( 5.7, 0) node[anchor=south west] {\trimfig{figures/rbcBoxPMLExT2p0}{\figWidth}};
  \draw (11.4, 0) node[anchor=south west] {\trimfig{figures/rbcBoxPECExT2p0}{\figWidth}};
 % - labels
 %   \draw (\txa,4.75) node[draw,fill=white,anchor=east] {\scriptsize $t=0.5$};
 %   \draw (\txb,4.75) node[draw,fill=white,anchor=east] {\scriptsize $t=1.0$};
 %   \draw (\txc,4.75) node[draw,fill=white,anchor=east] {\scriptsize $t=1.5$};
 %  \draw (current bounding box.south west) rectangle (current bounding box.north east);
% grid:
%  \draw[step=1cm,gray] (0,0) grid (17.0,5);
\end{tikzpicture}
\end{center}
\caption{Gaussian source in a box with different boundary conditions. $E_x$ at time $t=1.0$. Left bc={\tt abcEM2}, middle: bc={\tt abcPML} and
right bc={\tt perfectElectricalConductor}. }
\label{fig:rbcBox}
\end{figure}
}

\noindent{\bf Notes:}
\begin{enumerate}
 \item Note that for the PML boundary condition the solution is damped in a region 
  next to the boundary (set by the option -pmlWidth={\em number-of-lines}). This explains why the solution decays near the boundaries.
 % \item The PML region 
 % \item The PML boundary condition seems to be broken (maybe only works in x-direction?)
 % \item The PML boundary condition seems to need a smaller time-step (lower cfl). The reason for this needs to be sorted out.
\end{enumerate}


%------------------------------------------------------------------------------------------
\clearpage
\subsection{Transmission of a plane wave through a bumpy glass-air interface} \label{sec:scatBump2d}

The command file {\tt afm.cmd} can be used compute the propagation of a plane wave through the 
two-dimensional interface between two materials (air and glass).

\noindent (1) Generate the grids (command files in Overture/sampleGrids)
{\small
\begin{verbatim}
  ogen noplot afm -interp=e -order=4 -factor=4
\end{verbatim}
}
\noindent (2) Run cgmx (specifying $\eps$ in the two domains and the wave number $k_y$ of the incident field), 
{\small
\begin{verbatim}
  cgmx afm -g=afme4.order4.hdf -eps1=2.25 -eps2=1. -ky=20 -diss=4. -tf=1.4 -tp=.2 -go=halt
\end{verbatim}
}
{
\begin{figure}[hbt]
\newcommand{\figWidth}{7.0cm}
\newcommand{\trimfig}[2]{\trimFig{#1}{#2}{0.25}{0.25}{.45}{.425}}
\begin{center}
\begin{tikzpicture}[scale=1]
  \useasboundingbox (0,.75) rectangle (14.5,15.);  % set the bounding box (so we have less surrounding white space)
  \draw ( 0.0, 0) node[anchor=south west] {\trimfig{figures/afm2dThreeBumpExT0p0}{\figWidth}};
  \draw ( 7.5, 0) node[anchor=south west] {\trimfig{figures/afm2dThreeBumpEyT0p0}{\figWidth}};
% 
  \draw ( 0.0,3.8) node[anchor=south west] {\trimfig{figures/afm2dThreeBumpExT0p2}{\figWidth}};
  \draw ( 7.5,3.8) node[anchor=south west] {\trimfig{figures/afm2dThreeBumpEyT0p2}{\figWidth}};
% 
  \draw ( 0.0,7.6) node[anchor=south west] {\trimfig{figures/afm2dThreeBumpExT0p4}{\figWidth}};
  \draw ( 7.5,7.6) node[anchor=south west] {\trimfig{figures/afm2dThreeBumpEyT0p4}{\figWidth}};
%
  \draw ( 0.0,11.4) node[anchor=south west] {\trimfig{figures/afm2dThreeBumpExT0p6}{\figWidth}};
  \draw ( 7.5,11.4) node[anchor=south west] {\trimfig{figures/afm2dThreeBumpEyT0p6}{\figWidth}};
%
% \draw (current bounding box.south west) rectangle (current bounding box.north east);
% grid:
% \draw[step=1cm,gray] (0,0) grid (14.0,11);
\end{tikzpicture}
\end{center}
\caption{Transmission of a plane wave through a bumpy glass air interface. Left $E_x$ and right $E_y$ at times (bottom to top) 
  $t=0$, $t=0.2$, $t=0.4$ and $t=0.6$}
\label{fig:scatBump2d}
\end{figure}
}

\noindent{\bf Notes:}
\begin{enumerate}
  \item This example uses the plane wave initial condition with the initial condition bounding box.
  \item The speed of light in the lower glass region is $c_g =1/\sqrt{2.25} = 2/3$ compared to the speed of light $c_a=1$ in the
      upper air domain. 
\end{enumerate}

%------------------------------------------------------------------------------------------
\clearpage
\subsection{Scattering of plane wave from an array of dielectric cylinders} \label{sec:dielectricCylArray}

The command file {\tt lattice.cmd} can be used to compute the scattering
of a plane wave from an array of dielectric cylinders. A plane wave with wave number $k_x$ 
moves from left to right past an
array of dielectric cylinders. Periodic boundary conditions are imposed on the top and bottom. 

% \noindent Example:

\noindent (1) Generate the grid: (using the command file in {\tt Overture/sampleGrids}):
{\small
\begin{verbatim}
  ogen -noplot lattice -order=4 -interp=e -nCylx=3 -nCyly=3 -factor=2 -name="lattice3x3yFactor2.order4.hdf"
\end{verbatim}
}
\noindent (2) Run cgmx: 
\begin{verbatim}
  cgmx -noplot lattice -g=lattice3x3yFactor4.order4 -eps1=.25 -eps2=1. -kx=4 ...
                       -plotIntensity=1 -xb=-2. -go=halt
\end{verbatim}

{
\begin{figure}[hbt]
\newcommand{\figWidth}{5.5cm}
\newcommand{\trimfig}[2]{\trimFig{#1}{#2}{0.225}{0.295}{.475}{.475}}
\begin{center}
\begin{tikzpicture}[scale=1]
  \useasboundingbox (0,.5) rectangle (17,11.5);  % set the bounding box (so we have less surrounding white space)
  \draw ( 0.0, 8) node[anchor=south west] {\trimfig{figures/lattice3x3yExT6p0}{\figWidth}};
  \draw ( 5.7, 8) node[anchor=south west] {\trimfig{figures/lattice3x3yEyT6p0}{\figWidth}};
  \draw (11.4, 8) node[anchor=south west] {\trimfig{figures/lattice3x3yIntensityT6p0}{\figWidth}};
% 
  \draw ( 0.0, 4) node[anchor=south west] {\trimfig{figures/lattice3x3ykx8ExT6p0}{\figWidth}};
  \draw ( 5.7, 4) node[anchor=south west] {\trimfig{figures/lattice3x3ykx8EyT6p0}{\figWidth}};
  \draw (11.4, 4) node[anchor=south west] {\trimfig{figures/lattice3x3ykx8IntensityT6p0}{\figWidth}};
% 
  \draw ( 0.0, 0) node[anchor=south west] {\trimfig{figures/lattice33kx4Eps4ExT6p0}{\figWidth}};
  \draw ( 5.7, 0) node[anchor=south west] {\trimfig{figures/lattice33kx4Eps4EyT6p0}{\figWidth}};
  \draw (11.4, 0) node[anchor=south west] {\trimfig{figures/lattice33kx4Eps4IntensityT6p0}{\figWidth}};
 % - labels
 %   \draw (\txa,4.75) node[draw,fill=white,anchor=east] {\scriptsize $t=0.5$};
 %   \draw (\txb,4.75) node[draw,fill=white,anchor=east] {\scriptsize $t=1.0$};
 %   \draw (\txc,4.75) node[draw,fill=white,anchor=east] {\scriptsize $t=1.5$};
%  \draw (current bounding box.south west) rectangle (current bounding box.north east);
% grid:
%  \draw[step=1cm,gray] (0,0) grid (17.0,5);
\end{tikzpicture}
\end{center}
\caption{Scattering of a plane wave from an array of dielectric cylinders. Computed solution at $t=6.0$, $E_x$, $E_y$ and Intensity.
  Top row: $k_x=4$, $eps_1=0.25$. Middle row: $k_x=8$, $eps_1=0.25$. Bottom row: $k_x=4$, $eps_1=4.0$.
}
\label{fig:dielectricCylArray}
\end{figure}
}

% \noindent{\bf Notes:}
% \begin{enumerate}
%   \item See the comments at the top of the command file for command line arguments and further examples.
%   \item This case was run with fourth-order accuracy.
% \end{enumerate}


%------------------------------------------------------------------------------------------
\clearpage
\subsection{Scattering of a plane wave by a three-dimensional re-entry vehicle} \label{sec:scatCRV}

The command file {\tt cg/mx/runs/scattering/scattering.cmd} can be used to compute the scattering
of a plane wave from a three-dimensional body.

\noindent (1) Generate the grid for the crew reentry vehicle using the command file {\tt crv.cmd} in {\tt Overture/sampleGrids}:
{\small
\begin{verbatim}
  ogen -noplot crv -order=4 -blSpacingFactor=2 -prefix=crvbl2 -factor=4
\end{verbatim}
}
\noindent (2) Run cgmx (here in parallel): 
\begin{verbatim}
  mpirun -np 4 $cgmxp -noplot scattering.cmd -g=crvbl2e4.order4 -bg=backGround -rbc=abcPML 
                      -boundaryForcing=1 -kx=2 -tf=5. -tp=.5 -show=crv4.show -go=go 
\end{verbatim}

{
\begin{figure}[hbt]
\newcommand{\figWidth}{7.5cm}
\newcommand{\trimfig}[2]{\trimFig{#1}{#2}{0.1}{0.05}{.05}{.05}}
\begin{center}
\begin{tikzpicture}[scale=1]
  \useasboundingbox (0,.5) rectangle (16,16);  % set the bounding box (so we have less surrounding white space)
  \draw ( 0.0, 0) node[anchor=south west] {\trimfig{figures/scatteringReenrtyVehichleG8Ex}{\figWidth}};
  \draw ( 8.0, 0) node[anchor=south west] {\trimfig{figures/scatteringReenrtyVehichleG8Ey}{\figWidth}};
  \draw ( 0.0, 8) node[anchor=south west] {\trimfig{figures/scatteringReenrtyVehichleG4Ex}{\figWidth}};
  \draw ( 8.0, 8) node[anchor=south west] {\trimfig{figures/scatteringReenrtyVehichleG4Ey}{\figWidth}};
%   \draw (11.4, 0) node[anchor=south west] {\trimfig{figures/scatTriHzT1p5}{\figWidth}};
 % - labels
 %   \draw (\txa,4.75) node[draw,fill=white,anchor=east] {\scriptsize $t=0.5$};
 %   \draw (\txb,4.75) node[draw,fill=white,anchor=east] {\scriptsize $t=1.0$};
 %   \draw (\txc,4.75) node[draw,fill=white,anchor=east] {\scriptsize $t=1.5$};
 %  \draw (current bounding box.south west) rectangle (current bounding box.north east);
% grid:
% \draw[step=1cm,gray] (0,0) grid (16.0,16);
\end{tikzpicture}
\end{center}
\caption{Scattering of a plane wave from a re-entry vehicle.}
\label{fig:scatCRV}
\end{figure}
\noindent{\bf Notes:}
\begin{enumerate}
  \item The computation on grid $\Gc^{(8)}$ (27M pts, 1800 steps) took about $36$ (min) on 16 processors. 
\end{enumerate}

}

