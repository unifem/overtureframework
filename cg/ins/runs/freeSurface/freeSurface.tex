%-----------------------------------------------------------------------
%    Free surface notes
%-----------------------------------------------------------------------
\documentclass[11pt]{article}
% \usepackage[bookmarks=true]{hyperref}  % this changes the page location !
\usepackage[bookmarks=true,colorlinks=true,linkcolor=blue]{hyperref}

% \input documentationPageSize.tex
\hbadness=10000 
\sloppy \hfuzz=30pt

% \voffset=-.25truein
% \hoffset=-1.25truein
% \setlength{\textwidth}{7in}      % page width
% \setlength{\textheight}{9.5in}    % page height

\usepackage{calc}
\usepackage[lmargin=.75in,rmargin=.75in,tmargin=.75in,bmargin=.75in]{geometry}

% \input homeHenshaw
\newcommand{\CgTexDir}{../../../common/tex}% Here are some common tex files

\newcommand{\grad}{\nabla}
\newcommand{\citeCount}[1]{}% for citation counts papers

\newcommand{\blue}{\color{blue}}
\newcommand{\green}{\color{green}}
\newcommand{\red}{\color{red}}
\newcommand{\black}{\color{black}}

\newcommand{\tableFont}{\small}
\newcommand{\num}[2]{#1e#2} % Use this macro to define the format of the numbers in the table
\newcommand{\errFormat}[1]{$E_j^{(#1)}$}


\usepackage{amsmath}
\usepackage{amssymb}

\usepackage{verbatim}
% \usepackage{moreverb}

\usepackage{graphics}    
% \usepackage{ifthen}
% \usepackage{float}

\usepackage{tikz}
% \usepackage{pgfplots}
\input \CgTexDir/trimFig.tex


% ---- we have lemmas and theorems in this paper ----
%\newtheorem{assumption}{Assumption}
%\newtheorem{definition}{Definition}

\usepackage{listings}
\lstset{
basicstyle=\small\ttfamily,
columns=flexible,
breaklines=true
}

% \newcommand{\homeHenshaw}{/home/henshaw.0}

% \newcommand{\Overture}{{\bf Over\-ture\ }}
% \newcommand{\ogenDir}{\homeHenshaw/Overture/ogen}
% 
% \newcommand{\cgDoc}{\homeHenshaw/cgDoc}
% \newcommand{\vpDir}{\homeHenshaw/cgDoc/ins/viscoPlastic}
% 
% \newcommand{\ovFigures}{\homeHenshaw/OvertureFigures}
% \newcommand{\obFigures}{\homeHenshaw/res/OverBlown/docFigures}  % for figures
% \newcommand{\convDir}{\homeHenshaw/cgDoc/ins/tables}
% \newcommand{\insDocDir}{\homeHenshaw/cgDoc/ins}

% *** See http://www.eng.cam.ac.uk/help/tpl/textprocessing/squeeze.html
% By default, LaTeX doesn't like to fill more than 0.7 of a text page with tables and graphics, nor does it like too many figures per page. This behaviour can be changed by placing lines like the following before \begin{document}

\renewcommand\floatpagefraction{.99}
\renewcommand\topfraction{.99}
\renewcommand\bottomfraction{.99}
\renewcommand\textfraction{.01}   
\setcounter{totalnumber}{50}
\setcounter{topnumber}{50}
\setcounter{bottomnumber}{50}

\begin{document}

\input \CgTexDir/wdhDefinitions.tex

\newcommand{\dt}{{\Delta t}}
\newcommand{\rhos}{{\rho_b}}
\newcommand{\rhob}{{\rho_b}}

\newcommand{\Bc}{{\mathcal B}}
\newcommand{\Dc}{{\mathcal D}}
\newcommand{\Ec}{{\mathcal E}}
\newcommand{\Fc}{{\mathcal F}}
\newcommand{\Gc}{{\mathcal G}}
\newcommand{\Hc}{{\mathcal H}}
\newcommand{\Ic}{{\mathcal I}}
\newcommand{\Jc}{{\mathcal J}}
\newcommand{\Lc}{{\mathcal L}}
\newcommand{\Nc}{{\mathcal N}}
\newcommand{\Pc}{{\mathcal P}}
\newcommand{\Rc}{{\mathcal R}}
\newcommand{\Sc}{{\mathcal S}}

\newcommand{\bogus}[1]{}  % removes is argument completely

\vspace{5\baselineskip}
\begin{flushleft}
{\LARGE
Notes on free surface calculations with deforming grids using Cgins\\
}
\vspace{2\baselineskip}
William D. Henshaw, \\
% 
\smallskip
% \vspace{1\baselineskip}
% 
Department of Mathematical Sciences, \\
Rensselaer Polytechnic Institute (RPI), \\
Troy, NY, USA, 12180. \\
\vspace{\baselineskip}
\today\\

\vspace{4\baselineskip}

\noindent{\bf\large Abstract:}

This document describes results for simulating free surface flows with deforming grids using the Cgins solver
from Overture. 

\end{flushleft}

% \clearpage
\tableofcontents
% \listoffigures

\clearpage
% ================================================================================================================
\section{Introduction}

This document describes results for simulating free surface flows with deforming grids using the Cgins solver
from Overture. 

% rising disk in a counter flow
\input tex/submergedCylinder


% \clearpage
% \bibliography{henshaw,henshawPapers}
% \bibliographystyle{siam}



\end{document}
% **************************************************************************************************************
% **************************************************************************************************************
% **************************************************************************************************************
