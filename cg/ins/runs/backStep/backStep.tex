%-----------------------------------------------------------------------
% Flow past a backward facing step -- notes
%-----------------------------------------------------------------------
\documentclass[11pt]{article}
% \usepackage[bookmarks=true]{hyperref}  % this changes the page location !
\usepackage[bookmarks=true,colorlinks=true,linkcolor=blue]{hyperref}

% \input documentationPageSize.tex
\hbadness=10000 
\sloppy \hfuzz=30pt

% \voffset=-.25truein
% \hoffset=-1.25truein
% \setlength{\textwidth}{7in}      % page width
% \setlength{\textheight}{9.5in}    % page height

\usepackage{calc}
\usepackage[lmargin=.75in,rmargin=.75in,tmargin=.75in,bmargin=.75in]{geometry}

% \input homeHenshaw
\newcommand{\grad}{\nabla}
\newcommand{\citeCount}[1]{}% for citation counts papers

% \input{pstricks}\input{pst-node}
% \input{colours}
\newcommand{\blue}{\color{blue}}
\newcommand{\green}{\color{green}}
\newcommand{\red}{\color{red}}
\newcommand{\black}{\color{black}}


\usepackage{amsmath}
\usepackage{amssymb}

\usepackage{verbatim}
\usepackage{moreverb}

\usepackage{graphics}    
\usepackage{epsfig}    
\usepackage{calc}
\usepackage{ifthen}
\usepackage{float}
% the next one cause the table of contents to disappear!
% * \usepackage{fancybox}

\usepackage{makeidx} % index
\makeindex
\newcommand{\Index}[1]{#1\index{#1}}

\usepackage{tikz}
% \usepackage{pgfplots}
\input trimFig.tex


% ---- we have lemmas and theorems in this paper ----
\newtheorem{assumption}{Assumption}
\newtheorem{definition}{Definition}

% \newcommand{\homeHenshaw}{/home/henshaw.0}

% \newcommand{\Overture}{{\bf Over\-ture\ }}
% \newcommand{\ogenDir}{\homeHenshaw/Overture/ogen}
% 
% \newcommand{\cgDoc}{\homeHenshaw/cgDoc}
% \newcommand{\vpDir}{\homeHenshaw/cgDoc/ins/viscoPlastic}
% 
% \newcommand{\ovFigures}{\homeHenshaw/OvertureFigures}
% \newcommand{\obFigures}{\homeHenshaw/res/OverBlown/docFigures}  % for figures
% \newcommand{\convDir}{\homeHenshaw/cgDoc/ins/tables}
% \newcommand{\insDocDir}{\homeHenshaw/cgDoc/ins}

% *** See http://www.eng.cam.ac.uk/help/tpl/textprocessing/squeeze.html
% By default, LaTeX doesn't like to fill more than 0.7 of a text page with tables and graphics, nor does it like too many figures per page. This behaviour can be changed by placing lines like the following before \begin{document}

\renewcommand\floatpagefraction{.99}
\renewcommand\topfraction{.99}
\renewcommand\bottomfraction{.99}
\renewcommand\textfraction{.01}   
\setcounter{totalnumber}{50}
\setcounter{topnumber}{50}
\setcounter{bottomnumber}{50}

\begin{document}

\input wdhDefinitions.tex

\def\comma  {~~~,~~}
\newcommand{\uvd}{\mathbf{U}}
\def\ud     {{    U}}
\def\pd     {{    P}}
\def\calo{{\cal O}}

\newcommand{\mbar}{\bar{m}}
\newcommand{\Rbar}{\bar{R}}
\newcommand{\Ru}{R_u}         % universal gas constant
% \newcommand{\Iv}{{\bf I}}
% \newcommand{\qv}{{\bf q}}
\newcommand{\Div}{\grad\cdot}
\newcommand{\tauv}{\boldsymbol{\tau}}
\newcommand{\thetav}{\boldsymbol{\theta}}
% \newcommand{\omegav}{\mathbf{\omega}}
% \newcommand{\Omegav}{\mathbf{\Omega}}

\newcommand{\Omegav}{\boldsymbol{\Omega}}
\newcommand{\omegav}{\boldsymbol{\omega}}
\newcommand{\sigmav}{\boldsymbol{\sigma}}
\newcommand{\cm}{{\rm cm}}

\newcommand{\ds}{\Delta s}
\newcommand{\dsbl}{\ds_{\rm bl}}


\newcommand{\sumi}{\sum_{i=1}^n}
% \newcommand{\half}{{1\over2}}
\newcommand{\dt}{{\Delta t}}

\def\ff {\tt} % font for fortran variables

% define the clipFig commands:
%% \input clipFig.tex

\newcommand{\Bc}{{\mathcal B}}
\newcommand{\Dc}{{\mathcal D}}
\newcommand{\Ec}{{\mathcal E}}
\newcommand{\Fc}{{\mathcal F}}
\newcommand{\Gc}{{\mathcal G}}
\newcommand{\Hc}{{\mathcal H}}
\newcommand{\Ic}{{\mathcal I}}
\newcommand{\Jc}{{\mathcal J}}
\newcommand{\Lc}{{\mathcal L}}
\newcommand{\Nc}{{\mathcal N}}
\newcommand{\Pc}{{\mathcal P}}
\newcommand{\Rc}{{\mathcal R}}
\newcommand{\Sc}{{\mathcal S}}

\newcommand{\bogus}[1]{}  % removes is argument completely

\vspace{5\baselineskip}
\begin{flushleft}
{\LARGE
Simulating Airflow Past a Backward Facing Step \\
Using Overture's Cgins Solver\\
}
\vspace{2\baselineskip}
Jeffrey W. Banks,  \\
William D. Henshaw, \\
Donald W. Schwendeman \\
% 
\vspace{2\baselineskip}
% 
Department of Mathematical Sciences, \\
Rensselaer Polytechnic Institute (RPI), \\
Troy, NY, USA, 12180. \\
\vspace{\baselineskip}
\today\\

\vspace{4\baselineskip}

\noindent{\bf\large Abstract:}

This document describes results for simulating flow past a backward facing step.
In some cases there are moving bodies situated in the wake.

\end{flushleft}

% \clearpage
\tableofcontents
% \listoffigures

\clearpage


\section{Introduction}

Cgins is an incompressible flow solver built upon the Overture framework.
Cgins solves the incompressible Navier-Stokes equations (with Boussinesq approimation
for temperature dependent buoyant flows). See~\cite{CginsUserGuide} and~\cite{CginsReferenceManual} for further 
details. See~\cite{ICNS}~\cite{splitStep2003} for further details of the basic numerical scheme. 

The steps for simulating flow over some specified geometry are
\begin{enumerate}
  \item Use Ogen, the overlapping grid generator, to generate the initial grid using on of the provided command files (e.g. {\tt backStepGrid.cmd}).
  \item Use one of the Cgins command files,  ({\tt e.g. backStep.cmd}) to simulate the flow using Cgins.
\end{enumerate}
% Note the relevant scripts  can be found in 
% the directory {\tt cg/ins/runs/backStep} of the CG distribution. 

The default Cgins scheme being used in these computations is 
\begin{enumerate}
  \item AFS24-MG : Approximate factored scheme, 2nd-order in time, 4th-order in space, multigrid solver for the pressure equation,
    typcially run at a CFL number of $2-4$. 
\end{enumerate}

Notes:
\begin{enumerate}
  \item Cgins uses a non-linear dissipation to stabilize under-resolved flows. For the fourth-order accurate method this is
    a non-linear fourth-order eddy viscosity known as SSLES4 (smallest-scale large eddy simulation model, order 4).
  The effective dissipation term in the momentum equations is thus of the form
  \begin{align*}
     \Dc & =  \nu \Delta \uv  + \Big(\text{ad41} + \text{ad42} \vert \grad\uv\vert \Big) ~h^4~ \Delta^2 \uv , 
  \end{align*}
  where ad41 and ad42 are the coefficients of the SSLES viscosity (e.g. ad41=1, ad42=1) and $h$ is the approximate grid spacing.
\end{enumerate}




% =================================================================================================================================
% =================================================================================================================================
% =================================================================================================================================
\section{Flow past a two-dimensional backward facing step}

The geometry of the backward facing step is shown in~\ref{fig:backStepGrids}. 
An inlet region starting at $x=x_a$ reaches the step at $x=0$. The step has height of $1$ and the bottom
of the step is located at $x=0$, $y=0$. The outlet region extends to $x=x_b$. 

The boundary conditions are chosen as inflow on the left at $x=x_a$. The inflow profile is a parabola
over some small height and then uniform above this. An outflow condition is used on the right.
The top wall is either a slip wall (symmetry) or an outflow or zero-traction condition $p=0$. 
The lower wall is a no-slip wall.


The overlapping grid is denoted by $\Gc^{(j)}$ where $j$ denotes the grid resolution factor: the grid spacing in the 
grids in vicinity of the step have a a grid spacing of approximately $\Delta s = 1/(10 j)$. The grids are stretched
in the direction normal to the boundary by a factor of about $5$.

Figure~\ref{fig:backStepGrids} shows a basic overlapping grid for the backward facing step.
The step is rounded and represented with the SmoothedPolygon mapping.
Two background Cartesian grids fill the inlet and outlet regions.

Figure~\ref{fig:backStepRefineGrids} shows an {\em improved} version of the grid including far-field grids
to treat a larger domain and a stretched boundary layer grid on the bottom wall that follows the step.

{
\begin{figure}[hbt]
\newcommand{\figWidth}{10cm}
\newcommand{\trimfig}[2]{\trimFigb{#1}{#2}{0.025}{.5}{.375}{.375}}
\newcommand{\figWidtha}{5.5cm}
\newcommand{\trimfiga}[2]{\trimFigb{#1}{#2}{0.0}{.0}{.0}{.0}}
\begin{center}\small
% ------------------------------------------------------------------------------------------------
\begin{tikzpicture}[scale=1]
  \useasboundingbox (0,0.25) rectangle (16,5.5);  % set the bounding box (so we have less surrounding white space)
  \draw ( 0, 0) node[anchor=south west,xshift=-4pt,yshift=-4pt] {\trimfig{fig/backStepGrid}{\figWidth}};
  \draw (10.5, 0) node[anchor=south west,xshift=-4pt,yshift=-4pt] {\trimfiga{fig/backStepGridZoom}{\figWidtha}};
%  \draw[step=1cm,gray] (0,0) grid (16,5);
\end{tikzpicture}
% ----------------------------------------------------------------------------------------
\caption{
Overlapping grid (version 1) 
for the backward facing step. Grid $\Gc^{(4)}$ (resolution factor $j=4$, grid for fourth-order accuracy). Right: zoomed view near rounded step.
Interpolation points (two layers) are marked as black squares.
% Two-dimensional overlapping grid for a cross-section through the terrain. Grid $\Gc^{(4)}$ (second-order accuracy) is shown
%    along with a magnified view of the grid near the surface.
}
\label{fig:backStepGrids}
\end{center}
\end{figure}
}


%- {
%- \begin{figure}[hbt]
%- \newcommand{\figWidth}{10cm}
%- \newcommand{\trimfig}[2]{\trimFigb{#1}{#2}{0.025}{.5}{.375}{.375}}
%- \newcommand{\figWidtha}{5.5cm}
%- \newcommand{\trimfiga}[2]{\trimFigb{#1}{#2}{0.0}{.0}{.0}{.0}}
%- \begin{center}\small
%- % ------------------------------------------------------------------------------------------------
%- \begin{tikzpicture}[scale=1]
%-   \useasboundingbox (0,0.25) rectangle (16,5.5);  % set the bounding box (so we have less surrounding white space)
%-   \draw ( 0, 0) node[anchor=south west,xshift=-4pt,yshift=-4pt] {\trimfig{fig/backStepRefineGrid}{\figWidth}};
%-   \draw (10.5, 0) node[anchor=south west,xshift=-4pt,yshift=-4pt] {\trimfiga{fig/backStepRefineGridZoom}{\figWidtha}};
%- %  \draw[step=1cm,gray] (0,0) grid (16,5);
%- \end{tikzpicture}
%- % ----------------------------------------------------------------------------------------
%- \caption{
%- Overlapping grid for the backward facing step with far-field grids. Grid $\Gc^{(4)}$ (fourth-order accuracy). Right: zoomed view near rounded step.
%- Interpolation points (two layers) are marked as black squares.
%- % Two-dimensional overlapping grid for a cross-section through the terrain. Grid $\Gc^{(4)}$ (second-order accuracy) is shown
%- %    along with a magnified view of the grid near the surface.
%- }
%- \label{fig:backStepRefineGrids}
%- \end{center}
%- \end{figure}
%- }


{
\begin{figure}[hbt]
\newcommand{\figWidth}{16cm}
\newcommand{\trimfig}[2]{\trimFigb{#1}{#2}{0.01}{.01}{.275}{.275}}
\newcommand{\figWidtha}{5.5cm}
\newcommand{\trimfiga}[2]{\trimFigb{#1}{#2}{0.0}{.0}{.0}{.0}}
\begin{center}\small
% ------------------------------------------------------------------------------------------------
\begin{tikzpicture}[scale=1]
  \useasboundingbox (0,0.25) rectangle (16,10.);  % set the bounding box (so we have less surrounding white space)
  \draw ( 0, 5.6) node[anchor=south west,xshift=-4pt,yshift=-4pt] {\trimfig{fig/backStepRefineGrid}{\figWidth}};
  \draw ( 5, 0.0) node[anchor=south west,xshift=-4pt,yshift=-4pt] {\trimfiga{fig/backStepRefineGridZoom}{\figWidtha}};
%   \draw (10.5, 0) node[anchor=south west,xshift=-4pt,yshift=-4pt] {\trimfiga{fig/backStepRefineGrid2Zoom}{\figWidtha}};
%% \draw[step=1cm,gray] (0,0) grid (16,10.);
\end{tikzpicture}
% ----------------------------------------------------------------------------------------
\caption{
Overlapping grid (version 2) for the backward facing step with far-field grids. Grid $\Gc^{(4)}$ (fourth-order accuracy). Bottom: zoomed view near rounded step.
Interpolation points (two layers) are marked as black squares.
% Two-dimensional overlapping grid for a cross-section through the terrain. Grid $\Gc^{(4)}$ (second-order accuracy) is shown
%    along with a magnified view of the grid near the surface.
}
\label{fig:backStepRefineGrids}
\end{center}
\end{figure}
}

Figure~\ref{fig:backStepResolutions} shows how the solution depends on the grid resolution.
As the grid is refined, finer scale features of the flow can be resolved.
The maximum vorticity also increases as the grid is refined. 
In this case the true viscosity plays no role as the nonlinear SSLES model provides the dissipation.

{
\begin{figure}[hbt]
\newcommand{\figWidth}{11cm}
\newcommand{\trimfig}[2]{\trimFigb{#1}{#2}{0.0}{.0}{.33}{.33}}
\begin{center}\small
% ------------------------------------------------------------------------------------------------
\begin{tikzpicture}[scale=1]
  \useasboundingbox (0,0.25) rectangle (11,13.);  % set the bounding box (so we have less surrounding white space)
  \draw ( 0,0.0) node[anchor=south west,xshift=-4pt,yshift=-4pt] {\trimfig{fig/backStepRefineG32Vort20}{\figWidth}};
  \draw ( 0,4.5) node[anchor=south west,xshift=-4pt,yshift=-4pt] {\trimfig{fig/backStepRefineG16Vort20}{\figWidth}};
  \draw ( 0,9.0) node[anchor=south west,xshift=-4pt,yshift=-4pt] {\trimfig{fig/backStepRefineG8Vort20}{\figWidth}};
%   \draw (10.5, 0) node[anchor=south west,xshift=-4pt,yshift=-4pt] {\trimfiga{fig/backStepRefineGridZoom}{\figWidtha}};
%  \draw[step=1cm,gray] (0,0) grid (11,13.);
\end{tikzpicture}
% ----------------------------------------------------------------------------------------
\caption{
Vorticity at time $t=20$ for grids of different resolutions using the SSLES4 turbulence model.
Solution in the near wake region is shown. The max and min contours of the vorticity are limited to the range $[-20,20]$.
Top: coarse grid $\Gc^{(8)}$. Middle: medium grid $\Gc^{(16)}$. Bottom: fine grid $\Gc^{(32)}$ (approx. $2.3M$ grid pts).  
}
\label{fig:backStepResolutions}
\end{center}
\end{figure}
}


\clearpage
% =================================================================================================================================
% =================================================================================================================================
% =================================================================================================================================
\section{Flow past a two-dimensional backward facing step with moving cylinder in the wake}

Figure~\ref{fig:backStepAndBodyGrids} shows a basic overlapping grid for backward facing step
with a cylinder in the wake. The cylinder oscillates up and down. 

Figure~\ref{fig:backStepAndBodyMoveCylinder} shows the solution at two instants in time
computed on grid $\Gc^{(16}$ (approx. $750,000$ grid points) using AFS24-MG. 
{
\begin{figure}[hbt]
\newcommand{\figWidth}{15cm}
\newcommand{\trimfig}[2]{\trimFigb{#1}{#2}{0.0}{.25}{.27}{.275}}
\begin{center}\small
% ------------------------------------------------------------------------------------------------
\begin{tikzpicture}[scale=1]
  \useasboundingbox (0,0.25) rectangle (16,6.5);  % set the bounding box (so we have less surrounding white space)
  \draw ( 0, 0) node[anchor=south west,xshift=-4pt,yshift=-4pt] {\trimfig{fig/backStepAndBodyGrid}{\figWidth}};
%   \draw (10.5, 0) node[anchor=south west,xshift=-4pt,yshift=-4pt] {\trimfiga{fig/backStepRefineGridZoom}{\figWidtha}};
% \draw[step=1cm,gray] (0,0) grid (16,6);
\end{tikzpicture}
% ----------------------------------------------------------------------------------------
\caption{
Overlapping grid for the backward facing step with a cylinderical body in the wake, grid $\Gc^{(4)}$. 
}
\label{fig:backStepAndBodyGrids}
\end{center}
\end{figure}
}

{
\begin{figure}[hbt]
\newcommand{\figWidth}{11cm}
\newcommand{\trimfig}[2]{\trimFigb{#1}{#2}{0.0}{.0}{.27}{.3}}
\begin{center}\small
% ------------------------------------------------------------------------------------------------
\begin{tikzpicture}[scale=1]
  \useasboundingbox (0,0.25) rectangle (11,10.25);  % set the bounding box (so we have less surrounding white space)
  \draw ( 0, 0) node[anchor=south west,xshift=-4pt,yshift=-4pt] {\trimfig{fig/backStepAndBodyG16-Vor-t31}{\figWidth}};
  \draw ( 0, 5) node[anchor=south west,xshift=-4pt,yshift=-4pt] {\trimfig{fig/backStepAndBodyG16-Vor-t33}{\figWidth}};
%   \draw (10.5, 0) node[anchor=south west,xshift=-4pt,yshift=-4pt] {\trimfiga{fig/backStepRefineGridZoom}{\figWidtha}};
% \draw[step=1cm,gray] (0,0) grid (11,10.25);
\end{tikzpicture}
% ----------------------------------------------------------------------------------------
\caption{
Vorticity at two times for flow past a backward facing step and a moving cylinder in the wake. 
The cylinder oscillates up and down. 
Solution computed with AFS24-MG on grid $\Gc^{(16)}$.
}
\label{fig:backStepAndBodyMoveCylinder}
\end{center}
\end{figure}
}


% =================================================================================================================================
% =================================================================================================================================
% =================================================================================================================================
\section{Flow past a two-dimensional backward facing step in a channel - comparison to experiment }

The paper by Kim, Kline, and Johnston [1980] presents experiement results for flow past a
backward facing step in a channel in a wind tunnel. 

The {\em reference case} has parameters: $H=3.81 cm (1.5 in)$ (step height), $w_1 = 7.62cm (3 in)$ inlet height,
reference speed $U=18.2 m/s$ , $\nu=1.56 e-5 m^2/s$ (air at 300k). This gives
a Reynolds number based on the step height of
\begin{align*}
 &  Re(H=3.81 cm) = \frac{U H}{\nu} = \frac{18.2 \times .0381}{1.568 \times 10^{-5}} \approx  4.445 \times 10^4, \\
 &  \frac{1}{Re} \approx 2.25 \times 10^{-5}. 
\end{align*}

The problem is non-dimensionalized using the length scale $L=H$ and reference speed $U$. 
The grid for the geometry is shown in figure~\ref{fig:backStepInChannelGrid} where the non-dimensional
step height is $1$ and inflow velocity is $1$ (with a small parbolic correction at the top and bottom
no-slip walls of the inlet. The nondimensional kinematic viscosity is then $\nu= 1/Re = 2.25 \times 10^{-5}$. 

{\bf Smallest scale:} We can estimate how fine the mesh must be to perform a resolved DNS.
The smallest scale of the flow is~\cite{HKR1}
\begin{align}
    \lambda = \sqrt{\frac{\nu}{|\grad\uv|}}.  \label{eq:smallestScale}
\end{align}
If we guess that the non-dimensional vorticity is on the order of $Re^{1/2} \approx 200$ and use $|\grad\uv|=200$ in ~\eqref{eq:smallestScale}
then we arrive at the estimate 
\begin{align*}
    \lambda \approx  \sqrt{\frac{2.25 \times 10^{-5}}{200}} \approx 3.4e-4
\end{align*}
Thus the grid spacing in the boundary layer (where the vorticity is largest) show be roughly $\ds\approx 3.4e-4$.


Grid $\Gc^{(j)}$ has nominal spacing $\ds^{(j)}=1/(10 j)$ and boundary layer spacing of $\ds_{BL}=1/(50 j)$. 
For $j=32$ this gives a spacing $\ds_{BL}^{(32)} = 6.25e-4$ while for $j=64$  we have $\ds_{BL}^{(32)} = 3.125e-4$
{\bf We thus guess that $\Gc^{(16)}$ or $\Gc^{(32)}$ will be of sufficient resolution to perform an accurate DNS simulation.}

\bigskip
Figure Fig.~\ref{fig:backStepInChannelGrid} shows some results for grid $\Gc^{(16)}$. The flow is highly non-steady.

Figure Fig.~\ref{fig:backStepInChannelTimeAverage} shows the time averaged solution.


{
\begin{figure}[hbt]
\newcommand{\figWidth}{16cm}
\newcommand{\trimfig}[2]{\trimFigb{#1}{#2}{0.01}{.01}{.275}{.275}}
\newcommand{\figWidtha}{5.5cm}
\newcommand{\trimfiga}[2]{\trimFigb{#1}{#2}{0.0}{.0}{.0}{.0}}
\begin{center}\small
% ------------------------------------------------------------------------------------------------
\begin{tikzpicture}[scale=1]
  \useasboundingbox (0,0.25) rectangle (16,10.);  % set the bounding box (so we have less surrounding white space)
  \draw ( 0, 5.6) node[anchor=south west,xshift=-4pt,yshift=-4pt] {\trimfig{fig/backStepInChannelGrid}{\figWidth}};
  \draw ( 5, 0.0) node[anchor=south west,xshift=-4pt,yshift=-4pt] {\trimfiga{fig/backStepInChannelGridZoom}{\figWidtha}};
%   \draw (10.5, 0) node[anchor=south west,xshift=-4pt,yshift=-4pt] {\trimfiga{fig/backStepRefineGrid2Zoom}{\figWidtha}};
%% \draw[step=1cm,gray] (0,0) grid (16,10.);
\end{tikzpicture}
% ----------------------------------------------------------------------------------------
\caption{
Overlapping grid for the backward facing step in a channel (upper wall is no-slip). Very coarse grid $\Gc^{(2)}$ (second accuracy). Bottom: zoomed view near rounded step.
Interpolation points (two layers) are marked as black squares.
% Two-dimensional overlapping grid for a cross-section through the terrain. Grid $\Gc^{(4)}$ (second-order accuracy) is shown
%    along with a magnified view of the grid near the surface.
}
\label{fig:backStepInChannelGrid}
\end{center}
\end{figure}
}

{
\begin{figure}[hbt]
\newcommand{\figWidth}{17cm}
\newcommand{\trimfig}[2]{\trimFigb{#1}{#2}{0.0}{.0}{.29}{.29}}
\begin{center}\small
% ------------------------------------------------------------------------------------------------
\begin{tikzpicture}[scale=1]
  \useasboundingbox (0,0.25) rectangle (16,6.);  % set the bounding box (so we have less surrounding white space)
  \draw ( 0,3.0) node[anchor=south west,xshift=-4pt,yshift=-4pt] {\trimfig{fig/backStepRefineG16Vort10t50}{\figWidth}};
  \draw ( 0,0.0) node[anchor=south west,xshift=-4pt,yshift=-4pt] {\trimfig{fig/backStepRefineG16SLt50}{\figWidth}};
%   \draw (10.5, 0) node[anchor=south west,xshift=-4pt,yshift=-4pt] {\trimfiga{fig/backStepRefineGridZoom}{\figWidtha}};
%  \draw[step=1cm,gray] (0,0) grid (11,13.);
\end{tikzpicture}
% ----------------------------------------------------------------------------------------
\caption{
Vorticity and streamlines at time $t=50$ for grid $\Gc^{(16)}$.
% Solution in the near wake region is shown. The max and min contours of the vorticity are limited to the range $[-20,20]$.
% Top: coarse grid $\Gc^{(8)}$. Middle: medium grid $\Gc^{(16)}$. Bottom: fine grid $\Gc^{(32)}$ (approx. $2.3M$ grid pts).  
}
\label{fig:backStepInChannel}
\end{center}
\end{figure}
}



{
\begin{figure}[hbt]
\newcommand{\figWidth}{17cm}
\newcommand{\trimfig}[2]{\trimFigb{#1}{#2}{0.0}{.0}{.26}{.32}}
\begin{center}\small
% ------------------------------------------------------------------------------------------------
\begin{tikzpicture}[scale=1]
  \useasboundingbox (0,0.25) rectangle (16,9.8);  % set the bounding box (so we have less surrounding white space)
  \draw ( 0,6.4) node[anchor=south west,xshift=-4pt,yshift=-4pt] {\trimfig{fig/backStepInChannelAve25to50SL}{\figWidth}};
  \draw ( 0,3.2) node[anchor=south west,xshift=-4pt,yshift=-4pt] {\trimfig{fig/backStepInChannelAve25to50u}{\figWidth}};
  \draw ( 0,0.0) node[anchor=south west,xshift=-4pt,yshift=-4pt] {\trimfig{fig/backStepInChannelAve25to50v}{\figWidth}};
%   \draw (10.5, 0) node[anchor=south west,xshift=-4pt,yshift=-4pt] {\trimfiga{fig/backStepRefineGridZoom}{\figWidtha}};
%  \draw[step=1cm,gray] (0,0) grid (11,13.);
\end{tikzpicture}
% ----------------------------------------------------------------------------------------
\caption{
Time averaged streamlines, $u$ and $v$ for grid $\Gc^{(16)}$.
Averaged over $t\in[25,50]$.
% Solution in the near wake region is shown. The max and min contours of the vorticity are limited to the range $[-20,20]$.
% Top: coarse grid $\Gc^{(8)}$. Middle: medium grid $\Gc^{(16)}$. Bottom: fine grid $\Gc^{(32)}$ (approx. $2.3M$ grid pts).  
}
\label{fig:backStepInChannelTimeAverage}
\end{center}
\end{figure}
}


\vfill\eject
\bibliography{henshaw,henshawPapers}
\bibliographystyle{siam}



\end{document}
% *************************************************************************************************************************************
% *************************************************************************************************************************************
% *************************************************************************************************************************************
% *************************************************************************************************************************************


