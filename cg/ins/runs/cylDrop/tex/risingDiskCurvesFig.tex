{% ------ BODY ACCELERATION CURVES ------
\newcommand{\figWidth}{8.cm}
\newcommand{\trimfig}[2]{\trimw{#1}{#2}{.0}{.0}{.0}{.0}}
% zoom: 
\newcommand{\figWidthz}{7.5cm}
\newcommand{\trimfigz}[2]{\trimw{#1}{#2}{.0}{.0}{.0}{.0}}
\begin{figure}[htb]
\begin{center}
\resizebox{11cm}{!}{% START resize box
\begin{tikzpicture}[scale=1]
  \useasboundingbox (0.,1.3) rectangle (16.,21);  % set the bounding box (so we have less surrounding white space)
%
  \draw( 0.0,14.0) node[anchor=south west,xshift=-4pt,yshift=+0pt] {\trimfig{fig/risingDropPosition}{\figWidth}};
  \draw(8.25,14.25) node[anchor=south west,xshift=-4pt,yshift=+0pt] {\trimfigz{fig/risingDropThetaZoom}{\figWidthz}};
%
  \draw( 0.0,7.0) node[anchor=south west,xshift=-4pt,yshift=+0pt] {\trimfig{fig/risingDropVelocity}{\figWidth}};
  \draw(8.25,7.25) node[anchor=south west,xshift=-4pt,yshift=+0pt] {\trimfigz{fig/risingDropVelocityZoom}{\figWidthz}};
%
  \draw(0.0,0.0) node[anchor=south west,xshift=-4pt,yshift=+0pt] {\trimfig{fig/risingDropAcceleration}{\figWidth}};
  \draw(8.25,0.25) node[anchor=south west,xshift=-4pt,yshift=+0pt] {\trimfigz{fig/risingDropAccelerationZoom}{\figWidthz}};
%
%
% grid:
% \draw[step=1cm,gray] (0,0) grid (16,21);
\end{tikzpicture}
}% end resize box
\end{center}
  \caption{Rising disk in a counter-flow. Time history of the position and angular displacement of a light rigid body (top),
  its translational and angular velocities (middle), and its accelerations (bottom).  The plots on the right show selected enlarged views of the corresponding plots on the left.
%  \dws{(I suggest changing $\nu=1$ to $\mu=1$ to be
%     consistent with the \lq\lq viscosity'' used earlier, and change $\theta_3$ to $\theta_b$ and $\omega_3$ to $\omega_b$ to be consistent with earlier changes :))}
   % Results for grid G4 are in colour. Results for grid G8 are in black. 
%    Horizontal components of the displacement, $x_1$, velocity $v_1$ and acceleration $a_1$ of the rigid body.
%     Grid $\Gc^{(2)}$. 
     }
  \label{fig:risingDropCurves}
\end{figure}
}
