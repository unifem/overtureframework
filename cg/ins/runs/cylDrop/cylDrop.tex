%-----------------------------------------------------------------------
%    Dropping cylinder notes
%-----------------------------------------------------------------------
\documentclass[11pt]{article}
% \usepackage[bookmarks=true]{hyperref}  % this changes the page location !
\usepackage[bookmarks=true,colorlinks=true,linkcolor=blue]{hyperref}

% \input documentationPageSize.tex
\hbadness=10000 
\sloppy \hfuzz=30pt

% \voffset=-.25truein
% \hoffset=-1.25truein
% \setlength{\textwidth}{7in}      % page width
% \setlength{\textheight}{9.5in}    % page height

\usepackage{calc}
\usepackage[lmargin=.75in,rmargin=.75in,tmargin=.75in,bmargin=.75in]{geometry}

% \input homeHenshaw
\newcommand{\CgTexDir}{../../../common/tex}% Here are some common tex files

\newcommand{\grad}{\nabla}
\newcommand{\citeCount}[1]{}% for citation counts papers

\newcommand{\blue}{\color{blue}}
\newcommand{\green}{\color{green}}
\newcommand{\red}{\color{red}}
\newcommand{\black}{\color{black}}

\newcommand{\tableFont}{\small}
\newcommand{\num}[2]{#1e#2} % Use this macro to define the format of the numbers in the table
\newcommand{\errFormat}[1]{$E_j^{(#1)}$}


\usepackage{amsmath}
\usepackage{amssymb}

\usepackage{verbatim}
% \usepackage{moreverb}

\usepackage{graphics}    
% \usepackage{ifthen}
% \usepackage{float}

\usepackage{tikz}
% \usepackage{pgfplots}
\input \CgTexDir/trimFig.tex


% ---- we have lemmas and theorems in this paper ----
%\newtheorem{assumption}{Assumption}
%\newtheorem{definition}{Definition}

\usepackage{listings}
\lstset{
basicstyle=\small\ttfamily,
columns=flexible,
breaklines=true
}

% \newcommand{\homeHenshaw}{/home/henshaw.0}

% \newcommand{\Overture}{{\bf Over\-ture\ }}
% \newcommand{\ogenDir}{\homeHenshaw/Overture/ogen}
% 
% \newcommand{\cgDoc}{\homeHenshaw/cgDoc}
% \newcommand{\vpDir}{\homeHenshaw/cgDoc/ins/viscoPlastic}
% 
% \newcommand{\ovFigures}{\homeHenshaw/OvertureFigures}
% \newcommand{\obFigures}{\homeHenshaw/res/OverBlown/docFigures}  % for figures
% \newcommand{\convDir}{\homeHenshaw/cgDoc/ins/tables}
% \newcommand{\insDocDir}{\homeHenshaw/cgDoc/ins}

% *** See http://www.eng.cam.ac.uk/help/tpl/textprocessing/squeeze.html
% By default, LaTeX doesn't like to fill more than 0.7 of a text page with tables and graphics, nor does it like too many figures per page. This behaviour can be changed by placing lines like the following before \begin{document}

\renewcommand\floatpagefraction{.99}
\renewcommand\topfraction{.99}
\renewcommand\bottomfraction{.99}
\renewcommand\textfraction{.01}   
\setcounter{totalnumber}{50}
\setcounter{topnumber}{50}
\setcounter{bottomnumber}{50}

\begin{document}

\input \CgTexDir/wdhDefinitions.tex

\newcommand{\dt}{{\Delta t}}
\newcommand{\rhos}{{\rho_b}}
\newcommand{\rhob}{{\rho_b}}

\newcommand{\Bc}{{\mathcal B}}
\newcommand{\Dc}{{\mathcal D}}
\newcommand{\Ec}{{\mathcal E}}
\newcommand{\Fc}{{\mathcal F}}
\newcommand{\Gc}{{\mathcal G}}
\newcommand{\Hc}{{\mathcal H}}
\newcommand{\Ic}{{\mathcal I}}
\newcommand{\Jc}{{\mathcal J}}
\newcommand{\Lc}{{\mathcal L}}
\newcommand{\Nc}{{\mathcal N}}
\newcommand{\Pc}{{\mathcal P}}
\newcommand{\Rc}{{\mathcal R}}
\newcommand{\Sc}{{\mathcal S}}

\newcommand{\bogus}[1]{}  % removes is argument completely

\vspace{5\baselineskip}
\begin{flushleft}
{\LARGE
Notes on a cylinder dropping in a channel \\
}
\vspace{2\baselineskip}
William D. Henshaw, \\
% 
\smallskip
% \vspace{1\baselineskip}
% 
Department of Mathematical Sciences, \\
Rensselaer Polytechnic Institute (RPI), \\
Troy, NY, USA, 12180. \\
\vspace{\baselineskip}
\today\\

\vspace{4\baselineskip}

\noindent{\bf\large Abstract:}

This document describes results for simulating a cylinder dropping in a channel using the Cgins solver
from Overture. Self convergence results are provided. 

\end{flushleft}

% \clearpage
\tableofcontents
% \listoffigures

\clearpage
% ================================================================================================================
\section{Introduction}

This document describes results for simulating a cylinder dropping in a channel using the Cgins solver
from Overture. 

% rising disk in a counter flow
\input tex/diskDrop


% ================================================================================================================
\section{Cylinder in a short channel}

Build grids with variable-width boundary grids:
\begin{lstlisting}
ogen -noplot cylDropGrid.cmd -interp=e -radius=.25 -ya=-1. -yb=2. -cx=1. -cy=1. -prefix=cylGridSmall -factor=2
\end{lstlisting}

Build grids with fixed-width boundary grids:
\begin{lstlisting}
ogen -noplot cylDropGrid.cmd -interp=e -radius=.25 -ya=-1. -yb=2. -cx=1. -cy=1. -rgd=fixed -deltaRadius0=.2 -prefix=cylGridSmall -factor=2
\end{lstlisting}
{
\begin{figure}[hbt]
\newcommand{\figWidth}{7cm}
\newcommand{\trimfig}[2]{\trimh{#1}{#2}{0.2}{.2}{.05}{.1}}
% \newcommand{\figWidtha}{5.5cm}
% \newcommand{\trimfiga}[2]{\trimFigb{#1}{#2}{0.0}{.0}{.0}{.0}}
\begin{center}\small
% ------------------------------------------------------------------------------------------------
\begin{tikzpicture}[scale=1]
  \useasboundingbox (0,0.25) rectangle (16,7.);  % set the bounding box (so we have less surrounding white space)
  \draw ( 0, 0) node[anchor=south west,xshift=-4pt,yshift=-4pt] {\trimfig{fig/cylDropGridSmallFixede2}{\figWidth}};
  \draw ( 5, 0) node[anchor=south west,xshift=-4pt,yshift=-4pt] {\trimfig{fig/cylDropGridSmallFixede4}{\figWidth}};
%   \draw (10.5, 0) node[anchor=south west,xshift=-4pt,yshift=-4pt] {\trimfiga{\backStepDir/fig/backStepGridZoom}{\figWidtha}};
%  \draw[step=1cm,gray] (0,0) grid (16,7.);
\end{tikzpicture}
% ----------------------------------------------------------------------------------------
\caption{
Overlapping grids for the dropping cylinder in a short channel, grids $\Gc^{(2)}$ and $\Gc^{(4)}$
(fixed-width boundary grids).
}
\label{fig:cylDropGrid}
\end{center}
\end{figure}
}


% --------------------------------------------------------------------------------------------------------
\subsection{Non-moving grid: self-convergence results - fixed width grids}

{
\begin{figure}[hbt]
\newcommand{\figWidth}{6.25cm}
\newcommand{\trimfig}[2]{\trimh{#1}{#2}{0.2}{.0}{.0}{.0}}
\begin{center}\small
% ------------------------------------------------------------------------------------------------
\begin{tikzpicture}[scale=1]
  \useasboundingbox (0.5,0.25) rectangle (15.5,6.25);  % set the bounding box (so we have less surrounding white space)
  \draw ( 0, 0) node[anchor=south west,xshift=-4pt,yshift=-4pt] {\trimfig{fig/fallingDropG8NonMovingRampedPressuret2p0pressure}{\figWidth}};
  \draw (5., 0) node[anchor=south west,xshift=-4pt,yshift=-4pt] {\trimfig{fig/fallingDropG8NonMovingRampedPressuret2p0u}{\figWidth}};
  \draw (10, 0) node[anchor=south west,xshift=-4pt,yshift=-4pt] {\trimfig{fig/fallingDropG8NonMovingRampedPressuret2p0v}{\figWidth}};
 % \draw[step=1cm,gray] (0,0) grid (15.5,6.25);
\end{tikzpicture}
% ----------------------------------------------------------------------------------------
\caption{
Non-moving cylinder with ramped pressure inflow.
}
\label{fig:cylDropNonMovingRampedPressure}
\end{center}
\end{figure}
}

Run cases and run {\tt comp} to compute self-convergence rates
\begin{lstlisting}
  runCases.p -option=cylDrop -rgd=fixed -bcOption=rampedPressure -move=0 -dt0=.02 -tf=2 -numResolutions=3
  comp compFallingDrop -logFile=compCylDropNonMovingFixedWidthRampPressure.log
\end{lstlisting}


% -------------- max norm results -------------
\begin{table}[hbt]\tableFont % you should set \tableFont to \footnotesize or other size
% \newcommand{\num}[2]{#1e{#2}} % use this command to set the format of numbers in the table.
% \newcommand{\errFormat}[1]{#1}} % use this command to set the format of the error label.
\begin{center}
\begin{tabular}{|l|c|c|c|c|c|c|} \hline 
   grid              & \errFormat{p} &  r   & \errFormat{u} &  r   & \errFormat{v} &  r  \\ \hline
 fallingDropG2 & \num{1.9}{-3} &      & \num{3.4}{-4} &      & \num{7.8}{-4} &      \\ \hline
 fallingDropG4 & \num{2.7}{-4} &  7.0 & \num{5.3}{-5} &  6.5 & \num{1.4}{-4} &  5.6 \\ \hline
 fallingDropG8 & \num{3.9}{-5} &  7.0 & \num{8.2}{-6} &  6.5 & \num{2.5}{-5} &  5.6 \\ \hline
                      &     2.81      &      &     2.69      &      &     2.48      &     \\ \hline
\end{tabular}
\caption{Max-norm self convergence results, t=2.000000e+00, Sun Nov  6 10:40:41 2016. }
\end{center}
\end{table}

% -------------- l2 norm results -------------
\begin{table}[hbt]\tableFont % you should set \tableFont to \footnotesize or other size
% \newcommand{\num}[2]{#1e{#2}} % use this command to set the format of numbers in the table.
% \newcommand{\errFormat}[1]{#1}} % use this command to set the format of the error label.
\begin{center}
\begin{tabular}{|l|c|c|c|c|c|c|} \hline 
   grid              & \errFormat{p} &  r   & \errFormat{u} &  r   & \errFormat{v} &  r  \\ \hline
 fallingDropG2 & \num{7.6}{-4} &      & \num{1.4}{-4} &      & \num{2.5}{-4} &      \\ \hline
 fallingDropG4 & \num{1.4}{-4} &  5.5 & \num{2.4}{-5} &  5.7 & \num{4.6}{-5} &  5.5 \\ \hline
 fallingDropG8 & \num{2.5}{-5} &  5.5 & \num{4.2}{-6} &  5.7 & \num{8.4}{-6} &  5.5 \\ \hline
                      &     2.46      &      &     2.52      &      &     2.45      &     \\ \hline
\end{tabular}
\caption{L2-norm self convergence results, t=2.000000e+00, Sun Nov  6 10:40:41 2016. }
\end{center}
\end{table}


% --------------------------------------------------------------------------------------------------------
\subsection{Non-moving grid: self-convergence results - variable width grids}

Here are results for {\em variable width boundary grids} where the number of grid points in the normal direction
for the boundary conforming grids are kept fixed. As can be seen, the convergence rates are
not close to the expected second-order accuracy. This behaviour is often found in practice:  the rates may be worse of better
depending on the example. The reason for this behaviour is that the grid around the cylinder changes as the
mesh is refined and thus the interpolation points are located in different parts of the flow as the
mesh is refined. 

\begin{lstlisting}
runCases.p -option=cylDrop -rgd=var -bcOption=rampedPressure -move=0 -dt0=.02 -tf=2 -numResolutions=3
comp compFallingDrop -logFile=compCylDropNonMovingVarWidthRampPressure.log
\end{lstlisting}


% -------------- max norm results -------------
\begin{table}[hbt]\tableFont % you should set \tableFont to \footnotesize or other size
% \newcommand{\num}[2]{#1e{#2}} % use this command to set the format of numbers in the table.
% \newcommand{\errFormat}[1]{#1}} % use this command to set the format of the error label.
\begin{center}
\begin{tabular}{|l|c|c|c|c|c|c|} \hline 
   grid              & \errFormat{p} &  r   & \errFormat{u} &  r   & \errFormat{v} &  r  \\ \hline
 fallingDropG2 & \num{1.4}{-3} &      & \num{2.1}{-4} &      & \num{4.7}{-4} &      \\ \hline
 fallingDropG4 & \num{8.3}{-4} &  1.6 & \num{6.3}{-5} &  3.3 & \num{2.0}{-4} &  2.4 \\ \hline
 fallingDropG8 & \num{5.1}{-4} &  1.6 & \num{1.9}{-5} &  3.3 & \num{8.2}{-5} &  2.4 \\ \hline
                      &     0.71      &      &     1.72      &      &     1.26      &     \\ \hline
\end{tabular}
\caption{Max-norm self convergence results, t=2.000000e+00, Sun Nov  6 12:10:30 2016. }
\end{center}
\end{table}



% --------------------------------------------------------------------------------------------------------
\subsection{Moving rigid body: self-convergence results}

Here are results using the traditional scheme, $\rhob=5$. 


{
\begin{figure}[hbt]
\newcommand{\figWidth}{6.25cm}
\newcommand{\trimfig}[2]{\trimh{#1}{#2}{0.2}{.0}{.0}{.0}}
\begin{center}\small
% ------------------------------------------------------------------------------------------------
\begin{tikzpicture}[scale=1]
  \useasboundingbox (0.5,0.25) rectangle (15.5,6.25);  % set the bounding box (so we have less surrounding white space)
  \draw ( 0, 0) node[anchor=south west,xshift=-4pt,yshift=-4pt] {\trimfig{fig/fallingDropG8MovingRampedPressuret4p0pressure}{\figWidth}};
  \draw (5., 0) node[anchor=south west,xshift=-4pt,yshift=-4pt] {\trimfig{fig/fallingDropG8MovingRampedPressuret4p0u}{\figWidth}};
  \draw (10, 0) node[anchor=south west,xshift=-4pt,yshift=-4pt] {\trimfig{fig/fallingDropG8MovingRampedPressuret4p0v}{\figWidth}};
 % \draw[step=1cm,gray] (0,0) grid (15.5,6.25);
\end{tikzpicture}
% ----------------------------------------------------------------------------------------
\caption{
Moving rigid body with ramped pressure inflow, $\rhos=5$, $\Gc^{(8)}$, $t=4$.
}
\label{fig:cylDropNonMovingRampedPressure}
\end{center}
\end{figure}
}


Run cases:
\begin{lstlisting}
runCases.p -option=cylDrop  -rgd=fixed -bcOption=rampedPressure -move=1 -dt0=.02 -tf=4 -numResolutions=3
\end{lstlisting}

% -------------- max norm results -------------
\begin{table}[hbt]\tableFont % you should set \tableFont to \footnotesize or other size
% \newcommand{\num}[2]{#1e{#2}} % use this command to set the format of numbers in the table.
% \newcommand{\errFormat}[1]{#1}} % use this command to set the format of the error label.
\begin{center}
\begin{tabular}{|l|c|c|c|c|c|c|} \hline 
   grid              & \errFormat{p} &  r   & \errFormat{u} &  r   & \errFormat{v} &  r  \\ \hline
 fallingDropG2 & \num{4.9}{-4} &      & \num{1.5}{-4} &      & \num{2.3}{-4} &      \\ \hline
 fallingDropG4 & \num{9.0}{-5} &  5.5 & \num{3.3}{-5} &  4.6 & \num{4.3}{-5} &  5.4 \\ \hline
 fallingDropG8 & \num{1.6}{-5} &  5.5 & \num{7.3}{-6} &  4.6 & \num{8.0}{-6} &  5.4 \\ \hline
                      &     2.45      &      &     2.19      &      &     2.42      &     \\ \hline
\end{tabular}
\caption{Max-norm self convergence results, t=4.000000e+00, Sun Nov  6 07:58:24 2016. }
\end{center}
\end{table}


\begin{table}[hbt]\tableFont % you should set \tableFont to \footnotesize or other size
% \newcommand{\num}[2]{#1e{#2}} % use this command to set the format of numbers in the table.
% \newcommand{\errFormat}[1]{#1}} % use this command to set the format of the error label.
\begin{center}
\begin{tabular}{|l|c|c|c|c|c|c|} \hline 
   grid              & \errFormat{p} &  r   & \errFormat{u} &  r   & \errFormat{v} &  r  \\ \hline
 fallingDropG2 & \num{1.7}{-4} &      & \num{2.7}{-5} &      & \num{4.1}{-5} &      \\ \hline
 fallingDropG4 & \num{2.2}{-5} &  7.5 & \num{7.1}{-6} &  3.8 & \num{9.2}{-6} &  4.5 \\ \hline
 fallingDropG8 & \num{2.9}{-6} &  7.5 & \num{1.9}{-6} &  3.8 & \num{2.1}{-6} &  4.5 \\ \hline
                      &     2.92      &      &     1.94      &      &     2.16      &     \\ \hline
\end{tabular}
\caption{L1-norm self convergence results, t=4.000000e+00, Sun Nov  6 07:58:24 2016. }
\end{center}
\end{table}

%----------------------------------------------------------------------------------------------------
%----------------------------------------------------------------------------------------------------
% --------------------------------------------------------------------------------------------------------
\clearpage
\subsection{Moving rigid body with AMP, ramped pressure inflow, $\rhob=5$: self-convergence results}

Here are results using the AMP scheme, $\rhob=5$. 

Run cases:
\begin{lstlisting}
 runCases.p -option=cylDrop  -rgd=fixed -bcOption=rampedPressure -move=1 -amp=1 -dt0=.02 -tf=.5 -show=fallingDropAmpRhob5RampedPressure -numResolutions=3

comp compFallingDrop -show=fallingDropAmpRhob5RampedPressure -logFile=compCylDropAmpRhobRampedPressure.log
\end{lstlisting}


% -------------- max norm results -------------
\begin{table}[hbt]\tableFont % you should set \tableFont to \footnotesize or other size
% \newcommand{\num}[2]{#1e{#2}} % use this command to set the format of numbers in the table.
% \newcommand{\errFormat}[1]{#1}} % use this command to set the format of the error label.
\begin{center}
\begin{tabular}{|l|c|c|c|c|c|c|} \hline 
   grid              & \errFormat{p} &  r   & \errFormat{u} &  r   & \errFormat{v} &  r  \\ \hline
 fallingDropAmpRhob5RampedPressureG2 & \num{9.7}{-5} &      & \num{1.1}{-5} &      & \num{2.6}{-5} &      \\ \hline
 fallingDropAmpRhob5RampedPressureG4 & \num{1.8}{-5} &  5.3 & \num{2.6}{-6} &  4.4 & \num{6.0}{-6} &  4.3 \\ \hline
 fallingDropAmpRhob5RampedPressureG8 & \num{3.5}{-6} &  5.3 & \num{5.8}{-7} &  4.4 & \num{1.4}{-6} &  4.3 \\ \hline
                      &     2.41      &      &     2.15      &      &     2.12      &     \\ \hline
\end{tabular}
\caption{Max-norm self convergence results, t=5.000000e-01, Sun Nov  6 14:09:54 2016. }
\end{center}
\end{table}

% -------------- l2 norm results -------------
\begin{table}[hbt]\tableFont % you should set \tableFont to \footnotesize or other size
% \newcommand{\num}[2]{#1e{#2}} % use this command to set the format of numbers in the table.
% \newcommand{\errFormat}[1]{#1}} % use this command to set the format of the error label.
\begin{center}
\begin{tabular}{|l|c|c|c|c|c|c|} \hline 
   grid              & \errFormat{p} &  r   & \errFormat{u} &  r   & \errFormat{v} &  r  \\ \hline
 fallingDropAmpRhob5RampedPressureG2 & \num{5.2}{-5} &      & \num{3.0}{-6} &      & \num{1.1}{-5} &      \\ \hline
 fallingDropAmpRhob5RampedPressureG4 & \num{8.6}{-6} &  6.0 & \num{7.0}{-7} &  4.3 & \num{2.5}{-6} &  4.3 \\ \hline
 fallingDropAmpRhob5RampedPressureG8 & \num{1.4}{-6} &  6.0 & \num{1.6}{-7} &  4.3 & \num{6.0}{-7} &  4.3 \\ \hline
                      &     2.59      &      &     2.09      &      &     2.09      &     \\ \hline
\end{tabular}
\caption{L2-norm self convergence results, t=5.000000e-01, Sun Nov  6 14:09:54 2016. }
\end{center}
\end{table}

%----------------------------------------------------------------------------------------------------
%----------------------------------------------------------------------------------------------------
%----------------------------------------------------------------------------------------------------
\clearpage
\subsection{Moving rigid body with AMP, ramped pressure inflow, $\rhob=2$: self-convergence results}

Here are results using the AMP scheme, $\rhob=2$. 

% -------------- max norm results -------------
\begin{table}[hbt]\tableFont % you should set \tableFont to \footnotesize or other size
% \newcommand{\num}[2]{#1e{#2}} % use this command to set the format of numbers in the table.
% \newcommand{\errFormat}[1]{#1}} % use this command to set the format of the error label.
\begin{center}
\begin{tabular}{|l|c|c|c|c|c|c|} \hline 
   grid              & \errFormat{p} &  r   & \errFormat{u} &  r   & \errFormat{v} &  r  \\ \hline
 fallingDropAmpRhob2RampedPressureG2 & \num{5.8}{-4} &      & \num{1.8}{-4} &      & \num{2.4}{-4} &      \\ \hline
 fallingDropAmpRhob2RampedPressureG4 & \num{1.4}{-4} &  4.1 & \num{4.4}{-5} &  4.1 & \num{5.5}{-5} &  4.4 \\ \hline
 fallingDropAmpRhob2RampedPressureG8 & \num{3.4}{-5} &  4.1 & \num{1.1}{-5} &  4.1 & \num{1.2}{-5} &  4.4 \\ \hline
                      &     2.04      &      &     2.05      &      &     2.14      &     \\ \hline
\end{tabular}
\caption{Max-norm self convergence results, t=4.000000e+00, Sun Nov  6 14:54:48 2016. }
\end{center}
\end{table}
% -------------- l2 norm results -------------
\begin{table}[hbt]\tableFont % you should set \tableFont to \footnotesize or other size
% \newcommand{\num}[2]{#1e{#2}} % use this command to set the format of numbers in the table.
% \newcommand{\errFormat}[1]{#1}} % use this command to set the format of the error label.
\begin{center}
\begin{tabular}{|l|c|c|c|c|c|c|} \hline 
   grid              & \errFormat{p} &  r   & \errFormat{u} &  r   & \errFormat{v} &  r  \\ \hline
 fallingDropAmpRhob2RampedPressureG2 & \num{2.6}{-4} &      & \num{5.3}{-5} &      & \num{6.8}{-5} &      \\ \hline
 fallingDropAmpRhob2RampedPressureG4 & \num{5.7}{-5} &  4.5 & \num{1.2}{-5} &  4.4 & \num{1.5}{-5} &  4.5 \\ \hline
 fallingDropAmpRhob2RampedPressureG8 & \num{1.3}{-5} &  4.5 & \num{2.8}{-6} &  4.4 & \num{3.4}{-6} &  4.5 \\ \hline
                      &     2.17      &      &     2.13      &      &     2.17      &     \\ \hline
\end{tabular}
\caption{L2-norm self convergence results, t=4.000000e+00, Sun Nov  6 14:54:48 2016. }
\end{center}
\end{table}


%----------------------------------------------------------------------------------------------------
%----------------------------------------------------------------------------------------------------
%----------------------------------------------------------------------------------------------------
\clearpage
\subsection{Moving rigid body with AMP, ramped pressure inflow, $\rhob=0.1$: self-convergence results}

Here are results using the AMP scheme, $\rhob=0.1$.

% -------------- max norm results -------------
\begin{table}[hbt]\tableFont % you should set \tableFont to \footnotesize or other size
% \newcommand{\num}[2]{#1e{#2}} % use this command to set the format of numbers in the table.
% \newcommand{\errFormat}[1]{#1}} % use this command to set the format of the error label.
\begin{center}
\begin{tabular}{|l|c|c|c|c|c|c|} \hline 
   show file         & \errFormat{p} &  r   & \errFormat{u} &  r   & \errFormat{v} &  r  \\ \hline
 fallingDropAmpRhob0p1RampedPressureG2 & \num{7.5}{-4} &      & \num{2.1}{-4} &      & \num{2.6}{-4} &      \\ \hline
 fallingDropAmpRhob0p1RampedPressureG4 & \num{2.0}{-4} &  3.8 & \num{5.4}{-5} &  3.9 & \num{6.7}{-5} &  3.9 \\ \hline
 fallingDropAmpRhob0p1RampedPressureG8 & \num{5.2}{-5} &  3.8 & \num{1.4}{-5} &  3.9 & \num{1.7}{-5} &  3.9 \\ \hline
                      &     1.93      &      &     1.97      &      &     1.97      &     \\ \hline
\end{tabular}
\caption{Max-norm self convergence results, t=4.000000e+00, Sun Nov  6 15:37:57 2016. }
\end{center}
\end{table}

% -------------- l2 norm results -------------
\begin{table}[hbt]\tableFont % you should set \tableFont to \footnotesize or other size
% \newcommand{\num}[2]{#1e{#2}} % use this command to set the format of numbers in the table.
% \newcommand{\errFormat}[1]{#1}} % use this command to set the format of the error label.
\begin{center}
\begin{tabular}{|l|c|c|c|c|c|c|} \hline 
   show file         & \errFormat{p} &  r   & \errFormat{u} &  r   & \errFormat{v} &  r  \\ \hline
 fallingDropAmpRhob0p1RampedPressureG2 & \num{3.6}{-4} &      & \num{6.4}{-5} &      & \num{8.1}{-5} &      \\ \hline
 fallingDropAmpRhob0p1RampedPressureG4 & \num{9.2}{-5} &  3.9 & \num{1.6}{-5} &  4.1 & \num{2.1}{-5} &  3.9 \\ \hline
 fallingDropAmpRhob0p1RampedPressureG8 & \num{2.3}{-5} &  3.9 & \num{3.8}{-6} &  4.1 & \num{5.2}{-6} &  3.9 \\ \hline
                      &     1.98      &      &     2.04      &      &     1.97      &     \\ \hline
\end{tabular}
\caption{L2-norm self convergence results, t=4.000000e+00, Sun Nov  6 15:37:57 2016. }
\end{center}
\end{table}

%----------------------------------------------------------------------------------------------------
%----------------------------------------------------------------------------------------------------
%----------------------------------------------------------------------------------------------------
\clearpage
\subsection{Moving rigid body with AMP, ramped pressure inflow, $\rhob=0.01$: self-convergence results}

Here are results using the AMP scheme, $\rhob=0.01$.


% -------------- max norm results -------------
\begin{table}[hbt]\tableFont % you should set \tableFont to \footnotesize or other size
% \newcommand{\num}[2]{#1e{#2}} % use this command to set the format of numbers in the table.
% \newcommand{\errFormat}[1]{#1}} % use this command to set the format of the error label.
\begin{center}
\begin{tabular}{|l|c|c|c|c|c|c|} \hline 
   show file         & \errFormat{p} &  r   & \errFormat{u} &  r   & \errFormat{v} &  r  \\ \hline
 fallingDropAmpRhob0p01RampedPressureG2 & \num{7.5}{-4} &      & \num{2.1}{-4} &      & \num{2.6}{-4} &      \\ \hline
 fallingDropAmpRhob0p01RampedPressureG4 & \num{2.0}{-4} &  3.7 & \num{5.4}{-5} &  4.0 & \num{6.8}{-5} &  3.9 \\ \hline
 fallingDropAmpRhob0p01RampedPressureG8 & \num{5.3}{-5} &  3.7 & \num{1.3}{-5} &  4.0 & \num{1.7}{-5} &  3.9 \\ \hline
                      &     1.90      &      &     1.99      &      &     1.96      &     \\ \hline
\end{tabular}
\caption{Max-norm self convergence results, t=4.000000e+00, Sun Nov  6 16:10:13 2016. }
\end{center}
\end{table}

% -------------- l2 norm results -------------
\begin{table}[hbt]\tableFont % you should set \tableFont to \footnotesize or other size
% \newcommand{\num}[2]{#1e{#2}} % use this command to set the format of numbers in the table.
% \newcommand{\errFormat}[1]{#1}} % use this command to set the format of the error label.
\begin{center}
\begin{tabular}{|l|c|c|c|c|c|c|} \hline 
   show file         & \errFormat{p} &  r   & \errFormat{u} &  r   & \errFormat{v} &  r  \\ \hline
 fallingDropAmpRhob0p01RampedPressureG2 & \num{3.6}{-4} &      & \num{6.5}{-5} &      & \num{8.1}{-5} &      \\ \hline
 fallingDropAmpRhob0p01RampedPressureG4 & \num{9.3}{-5} &  3.8 & \num{1.6}{-5} &  4.1 & \num{2.1}{-5} &  3.9 \\ \hline
 fallingDropAmpRhob0p01RampedPressureG8 & \num{2.4}{-5} &  3.8 & \num{3.9}{-6} &  4.1 & \num{5.4}{-6} &  3.9 \\ \hline
                      &     1.94      &      &     2.03      &      &     1.96      &     \\ \hline
\end{tabular}
\caption{L2-norm self convergence results, t=4.000000e+00, Sun Nov  6 16:10:13 2016. }
\end{center}
\end{table}



%----------------------------------------------------------------------------------------------------
%----------------------------------------------------------------------------------------------------
%----------------------------------------------------------------------------------------------------
\clearpage
\subsection{Moving rigid body with AMP, ramped pressure inflow, $\rhob=0$: self-convergence results}

Here are results using the AMP scheme, $\rhob=0$.
% -------------- max norm results -------------
\begin{table}[hbt]\tableFont % you should set \tableFont to \footnotesize or other size
% \newcommand{\num}[2]{#1e{#2}} % use this command to set the format of numbers in the table.
% \newcommand{\errFormat}[1]{#1}} % use this command to set the format of the error label.
\begin{center}
\begin{tabular}{|l|c|c|c|c|c|c|} \hline 
   show file         & \errFormat{p} &  r   & \errFormat{u} &  r   & \errFormat{v} &  r  \\ \hline
 fallingDropAmpRhob0p0RampedPressureG2 & \num{7.4}{-4} &      & \num{2.1}{-4} &      & \num{2.6}{-4} &      \\ \hline
 fallingDropAmpRhob0p0RampedPressureG4 & \num{2.0}{-4} &  3.8 & \num{5.4}{-5} &  4.0 & \num{6.8}{-5} &  3.9 \\ \hline
 fallingDropAmpRhob0p0RampedPressureG8 & \num{5.2}{-5} &  3.8 & \num{1.3}{-5} &  4.0 & \num{1.7}{-5} &  3.9 \\ \hline
                      &     1.91      &      &     1.99      &      &     1.95      &     \\ \hline
\end{tabular}
\caption{Max-norm self convergence results, t=4.000000e+00, Sun Nov  6 16:50:01 2016. }
\end{center}
\end{table}

% -------------- l2 norm results -------------
\begin{table}[hbt]\tableFont % you should set \tableFont to \footnotesize or other size
% \newcommand{\num}[2]{#1e{#2}} % use this command to set the format of numbers in the table.
% \newcommand{\errFormat}[1]{#1}} % use this command to set the format of the error label.
\begin{center}
\begin{tabular}{|l|c|c|c|c|c|c|} \hline 
   show file         & \errFormat{p} &  r   & \errFormat{u} &  r   & \errFormat{v} &  r  \\ \hline
 fallingDropAmpRhob0p0RampedPressureG2 & \num{3.5}{-4} &      & \num{6.5}{-5} &      & \num{8.1}{-5} &      \\ \hline
 fallingDropAmpRhob0p0RampedPressureG4 & \num{9.1}{-5} &  3.9 & \num{1.6}{-5} &  4.1 & \num{2.1}{-5} &  3.9 \\ \hline
 fallingDropAmpRhob0p0RampedPressureG8 & \num{2.4}{-5} &  3.9 & \num{3.9}{-6} &  4.1 & \num{5.4}{-6} &  3.9 \\ \hline
                      &     1.95      &      &     2.03      &      &     1.96      &     \\ \hline
\end{tabular}
\caption{L2-norm self convergence results, t=4.000000e+00, Sun Nov  6 16:50:01 2016. }
\end{center}
\end{table}


%----------------------------------------------------------------------------------------------------
%----------------------------------------------------------------------------------------------------
%----------------------------------------------------------------------------------------------------
\clearpage
\subsection{Moving rigid body, TP, ramped pressure inflow, ramped gravity, $\rhob=5$: self-convergence results}

Gravity $\gv=[0,0.1,0]$ is up. Body slowly falls. 

% -------------- max norm results -------------
\begin{table}[hbt]\tableFont % you should set \tableFont to \footnotesize or other size
% \newcommand{\num}[2]{#1e{#2}} % use this command to set the format of numbers in the table.
% \newcommand{\errFormat}[1]{#1}} % use this command to set the format of the error label.
\begin{center}
\begin{tabular}{|l|c|c|c|c|c|c|} \hline 
   show file         & \errFormat{p} &  r   & \errFormat{u} &  r   & \errFormat{v} &  r  \\ \hline
 fallingDropRhob5RampedPandGG2 & \num{5.4}{-4} &      & \num{2.7}{-4} &      & \num{5.8}{-4} &      \\ \hline
 fallingDropRhob5RampedPandGG4 & \num{1.2}{-4} &  4.4 & \num{5.0}{-5} &  5.5 & \num{6.6}{-5} &  8.8 \\ \hline
 fallingDropRhob5RampedPandGG8 & \num{2.8}{-5} &  4.4 & \num{9.0}{-6} &  5.5 & \num{7.5}{-6} &  8.8 \\ \hline
                      &     2.13      &      &     2.47      &      &     3.13      &     \\ \hline
\end{tabular}
\caption{Max-norm self convergence results, t=4.000000e+00, Sun Nov  6 17:34:26 2016. }
\end{center}
\end{table}

% -------------- l2 norm results -------------
\begin{table}[hbt]\tableFont % you should set \tableFont to \footnotesize or other size
% \newcommand{\num}[2]{#1e{#2}} % use this command to set the format of numbers in the table.
% \newcommand{\errFormat}[1]{#1}} % use this command to set the format of the error label.
\begin{center}
\begin{tabular}{|l|c|c|c|c|c|c|} \hline 
   show file         & \errFormat{p} &  r   & \errFormat{u} &  r   & \errFormat{v} &  r  \\ \hline
 fallingDropRhob5RampedPandGG2 & \num{1.8}{-4} &      & \num{5.8}{-5} &      & \num{1.1}{-4} &      \\ \hline
 fallingDropRhob5RampedPandGG4 & \num{4.0}{-5} &  4.5 & \num{1.7}{-5} &  3.3 & \num{2.8}{-5} &  3.8 \\ \hline
 fallingDropRhob5RampedPandGG8 & \num{9.0}{-6} &  4.5 & \num{5.2}{-6} &  3.3 & \num{7.3}{-6} &  3.8 \\ \hline
                      &     2.17      &      &     1.74      &      &     1.94      &     \\ \hline
\end{tabular}
\caption{L2-norm self convergence results, t=4.000000e+00, Sun Nov  6 17:34:26 2016. }
\end{center}
\end{table}

%----------------------------------------------------------------------------------------------------
%----------------------------------------------------------------------------------------------------
%----------------------------------------------------------------------------------------------------
\clearpage
\subsection{Moving rigid body, AMP, ramped pressure inflow, ramped gravity, $\rhob=5$: self-convergence results}

Gravity $\gv=[0,0.1,0]$ is up. Body slowly falls. 

% -------------- max norm results -------------
\begin{table}[hbt]\tableFont % you should set \tableFont to \footnotesize or other size
% \newcommand{\num}[2]{#1e{#2}} % use this command to set the format of numbers in the table.
% \newcommand{\errFormat}[1]{#1}} % use this command to set the format of the error label.
\begin{center}
\begin{tabular}{|l|c|c|c|c|c|c|} \hline 
   show file         & \errFormat{p} &  r   & \errFormat{u} &  r   & \errFormat{v} &  r  \\ \hline
 fallingDropAmpRhob5RampedPandGG2 & \num{5.4}{-4} &      & \num{2.7}{-4} &      & \num{5.8}{-4} &      \\ \hline
 fallingDropAmpRhob5RampedPandGG4 & \num{1.2}{-4} &  4.4 & \num{5.0}{-5} &  5.5 & \num{6.6}{-5} &  8.8 \\ \hline
 fallingDropAmpRhob5RampedPandGG8 & \num{2.8}{-5} &  4.4 & \num{9.0}{-6} &  5.5 & \num{7.5}{-6} &  8.8 \\ \hline
                      &     2.13      &      &     2.47      &      &     3.13      &     \\ \hline
\end{tabular}
\caption{Max-norm self convergence results, t=4.000000e+00, Sun Nov  6 18:18:49 2016. }
\end{center}
\end{table}

% -------------- l2 norm results -------------
\begin{table}[hbt]\tableFont % you should set \tableFont to \footnotesize or other size
% \newcommand{\num}[2]{#1e{#2}} % use this command to set the format of numbers in the table.
% \newcommand{\errFormat}[1]{#1}} % use this command to set the format of the error label.
\begin{center}
\begin{tabular}{|l|c|c|c|c|c|c|} \hline 
   show file         & \errFormat{p} &  r   & \errFormat{u} &  r   & \errFormat{v} &  r  \\ \hline
 fallingDropAmpRhob5RampedPandGG2 & \num{1.8}{-4} &      & \num{5.8}{-5} &      & \num{1.1}{-4} &      \\ \hline
 fallingDropAmpRhob5RampedPandGG4 & \num{4.0}{-5} &  4.5 & \num{1.7}{-5} &  3.3 & \num{2.8}{-5} &  3.8 \\ \hline
 fallingDropAmpRhob5RampedPandGG8 & \num{9.0}{-6} &  4.5 & \num{5.2}{-6} &  3.3 & \num{7.3}{-6} &  3.8 \\ \hline
                      &     2.17      &      &     1.74      &      &     1.94      &     \\ \hline
\end{tabular}
\caption{L2-norm self convergence results, t=4.000000e+00, Sun Nov  6 18:18:49 2016. }
\end{center}
\end{table}

%----------------------------------------------------------------------------------------------------
%----------------------------------------------------------------------------------------------------
%----------------------------------------------------------------------------------------------------
\clearpage
\subsection{Moving rigid body, AMP, ramped pressure inflow, ramped gravity, $\rhob=0.1$: self-convergence results}

Gravity $\gv=[0,-1,0]$ is down. Body slowly falls. 

Rates vary a bit from time to time.

May need a longer domain?


% -------------- max norm results -------------
\begin{table}[hbt]\tableFont % you should set \tableFont to \footnotesize or other size
% \newcommand{\num}[2]{#1e{#2}} % use this command to set the format of numbers in the table.
% \newcommand{\errFormat}[1]{#1}} % use this command to set the format of the error label.
\begin{center}
\begin{tabular}{|l|c|c|c|c|c|c|} \hline 
   show file         & \errFormat{p} &  r   & \errFormat{u} &  r   & \errFormat{v} &  r  \\ \hline
 fallingDropAmpRhob0p1RampedPandGG2 & \num{1.1}{-3} &      & \num{7.7}{-4} &      & \num{1.3}{-3} &      \\ \hline
 fallingDropAmpRhob0p1RampedPandGG4 & \num{1.5}{-4} &  7.6 & \num{4.1}{-5} & 18.7 & \num{1.1}{-4} & 11.4 \\ \hline
 fallingDropAmpRhob0p1RampedPandGG8 & \num{2.0}{-5} &  7.6 & \num{2.2}{-6} & 18.7 & \num{9.6}{-6} & 11.4 \\ \hline
                      &     2.92      &      &     4.23      &      &     3.51      &     \\ \hline
\end{tabular}
\caption{Max-norm self convergence results, t=4.000000e+00, Sun Nov  6 18:45:38 2016. }
\end{center}
\end{table}

% -------------- l2 norm results -------------
\begin{table}[hbt]\tableFont % you should set \tableFont to \footnotesize or other size
% \newcommand{\num}[2]{#1e{#2}} % use this command to set the format of numbers in the table.
% \newcommand{\errFormat}[1]{#1}} % use this command to set the format of the error label.
\begin{center}
\begin{tabular}{|l|c|c|c|c|c|c|} \hline 
   show file         & \errFormat{p} &  r   & \errFormat{u} &  r   & \errFormat{v} &  r  \\ \hline
 fallingDropAmpRhob0p1RampedPandGG2 & \num{2.8}{-4} &      & \num{1.1}{-4} &      & \num{1.7}{-4} &      \\ \hline
 fallingDropAmpRhob0p1RampedPandGG4 & \num{6.0}{-5} &  4.8 & \num{1.9}{-5} &  5.4 & \num{4.8}{-5} &  3.5 \\ \hline
 fallingDropAmpRhob0p1RampedPandGG8 & \num{1.2}{-5} &  4.8 & \num{3.6}{-6} &  5.4 & \num{1.4}{-5} &  3.5 \\ \hline
                      &     2.25      &      &     2.43      &      &     1.82      &     \\ \hline
\end{tabular}
\caption{L2-norm self convergence results, t=4.000000e+00, Sun Nov  6 18:45:38 2016. }
\end{center}
\end{table}

% -------------- max norm results -------------
\begin{table}[hbt]\tableFont % you should set \tableFont to \footnotesize or other size
% \newcommand{\num}[2]{#1e{#2}} % use this command to set the format of numbers in the table.
% \newcommand{\errFormat}[1]{#1}} % use this command to set the format of the error label.
\begin{center}
\begin{tabular}{|l|c|c|c|c|c|c|} \hline 
   show file         & \errFormat{p} &  r   & \errFormat{u} &  r   & \errFormat{v} &  r  \\ \hline
 fallingDropAmpRhob0p1RampedPandGG2 & \num{7.3}{-4} &      & \num{4.0}{-4} &      & \num{8.7}{-4} &      \\ \hline
 fallingDropAmpRhob0p1RampedPandGG4 & \num{3.0}{-4} &  2.5 & \num{4.5}{-5} &  8.9 & \num{1.2}{-4} &  7.3 \\ \hline
 fallingDropAmpRhob0p1RampedPandGG8 & \num{1.2}{-4} &  2.5 & \num{5.1}{-6} &  8.9 & \num{1.6}{-5} &  7.3 \\ \hline
                      &     1.30      &      &     3.15      &      &     2.87      &     \\ \hline
\end{tabular}
\caption{Max-norm self convergence results, t=2.000000e+00, Sun Nov  6 19:18:21 2016. }
\end{center}
\end{table}

%----------------------------------------------------------------------------------------------------
%----------------------------------------------------------------------------------------------------
%----------------------------------------------------------------------------------------------------
\clearpage
\subsection{Moving rigid body, AMP, ramped gravity, $\rhob=0.1$: self-convergence results}

Ramped gravity in a closed box. 

Pressure levels may cause troubles with L2-norm -- may need to project out the 
constant. 


% -------------- max norm results -------------
\begin{table}[hbt]\tableFont % you should set \tableFont to \footnotesize or other size
% \newcommand{\num}[2]{#1e{#2}} % use this command to set the format of numbers in the table.
% \newcommand{\errFormat}[1]{#1}} % use this command to set the format of the error label.
\begin{center}
\begin{tabular}{|l|c|c|c|c|c|c|} \hline 
   show file         & \errFormat{p} &  r   & \errFormat{u} &  r   & \errFormat{v} &  r  \\ \hline
 fallingDropAmpRhob5RampedGG2 & \num{2.6}{-3} &      & \num{2.1}{-4} &      & \num{5.6}{-4} &      \\ \hline
 fallingDropAmpRhob5RampedGG4 & \num{7.1}{-4} &  3.7 & \num{2.5}{-5} &  8.5 & \num{8.9}{-5} &  6.3 \\ \hline
 fallingDropAmpRhob5RampedGG8 & \num{1.9}{-4} &  3.7 & \num{3.0}{-6} &  8.5 & \num{1.4}{-5} &  6.3 \\ \hline
                      &     1.89      &      &     3.08      &      &     2.65      &     \\ \hline
\end{tabular}
\caption{Max-norm self convergence results, t=5.000000e-01, Mon Nov  7 08:38:35 2016. }
\end{center}
\end{table}

\begin{table}[hbt]\tableFont % you should set \tableFont to \footnotesize or other size
% \newcommand{\num}[2]{#1e{#2}} % use this command to set the format of numbers in the table.
% \newcommand{\errFormat}[1]{#1}} % use this command to set the format of the error label.
\begin{center}
\begin{tabular}{|l|c|c|c|c|c|c|} \hline 
   show file         & \errFormat{p} &  r   & \errFormat{u} &  r   & \errFormat{v} &  r  \\ \hline
 fallingDropAmpRhob5RampedGG2 & \num{1.5}{-3} &      & \num{6.2}{-5} &      & \num{7.1}{-5} &      \\ \hline
 fallingDropAmpRhob5RampedGG4 & \num{8.7}{-4} &  1.7 & \num{7.1}{-6} &  8.8 & \num{1.0}{-5} &  7.1 \\ \hline
 fallingDropAmpRhob5RampedGG8 & \num{5.2}{-4} &  1.7 & \num{8.0}{-7} &  8.8 & \num{1.4}{-6} &  7.1 \\ \hline
                      &     0.74      &      &     3.14      &      &     2.82      &     \\ \hline
\end{tabular}
\caption{L2-norm self convergence results, t=5.000000e-01, Mon Nov  7 08:38:35 2016. }
\end{center}
\end{table}

%----------------------------------------------------------------------------------------------------
%----------------------------------------------------------------------------------------------------
%----------------------------------------------------------------------------------------------------
\clearpage
\subsection{Fixed body, ramped velocity, $\rhob=5$: self-convergence results}


Run cases:
\begin{lstlisting}
runCases.p -option=cylDrop  -rgd=fixed -bcOption=rampedVelocity -inflowVelocity=-1 -d=1. -move=0 -rampGravity=0 -amp=0 -rhob=5 -dt0=.02 -tf=2 -show=fallingDropRhob5RampedV -numResolutions=3

comp compFallingDrop -show=fallingDropRhob5RampedV
\end{lstlisting}


Inflow velocity profile is a flat profile with parabolic ends. 

Outflow : $p=0$, Neumann for v. 

Note: Trouble with pressure convergence at inflow  at the transition to a flat profile...
make the profile parabolic. 



% -------------- max norm results -------------
\begin{table}[hbt]\tableFont % you should set \tableFont to \footnotesize or other size
% \newcommand{\num}[2]{#1e{#2}} % use this command to set the format of numbers in the table.
% \newcommand{\errFormat}[1]{#1}} % use this command to set the format of the error label.
\begin{center}
\begin{tabular}{|l|c|c|c|c|c|c|} \hline 
   show file         & \errFormat{p} &  r   & \errFormat{u} &  r   & \errFormat{v} &  r  \\ \hline
 fallingDropRhob5RampedVG2 & \num{3.3}{-2} &      & \num{6.4}{-3} &      & \num{1.1}{-2} &      \\ \hline
 fallingDropRhob5RampedVG4 & \num{5.0}{-3} &  6.5 & \num{1.1}{-3} &  5.9 & \num{1.9}{-3} &  6.1 \\ \hline
 fallingDropRhob5RampedVG8 & \num{7.6}{-4} &  6.5 & \num{1.8}{-4} &  5.9 & \num{3.0}{-4} &  6.1 \\ \hline
                      &     2.71      &      &     2.56      &      &     2.61      &     \\ \hline
\end{tabular}
\caption{Max-norm self convergence results, t=2.000000e+00, Mon Nov  7 14:02:18 2016. }
\end{center}
\end{table}

% -------------- l2 norm results -------------
\begin{table}[hbt]\tableFont % you should set \tableFont to \footnotesize or other size
% \newcommand{\num}[2]{#1e{#2}} % use this command to set the format of numbers in the table.
% \newcommand{\errFormat}[1]{#1}} % use this command to set the format of the error label.
\begin{center}
\begin{tabular}{|l|c|c|c|c|c|c|} \hline 
   show file         & \errFormat{p} &  r   & \errFormat{u} &  r   & \errFormat{v} &  r  \\ \hline
 fallingDropRhob5RampedVG2 & \num{4.1}{-3} &      & \num{1.9}{-3} &      & \num{2.4}{-3} &      \\ \hline
 fallingDropRhob5RampedVG4 & \num{9.0}{-4} &  4.5 & \num{3.4}{-4} &  5.5 & \num{4.4}{-4} &  5.5 \\ \hline
 fallingDropRhob5RampedVG8 & \num{2.0}{-4} &  4.5 & \num{6.1}{-5} &  5.5 & \num{7.9}{-5} &  5.5 \\ \hline
                      &     2.18      &      &     2.46      &      &     2.46      &     \\ \hline
\end{tabular}
\caption{L2-norm self convergence results, t=2.000000e+00, Mon Nov  7 14:02:18 2016. }
\end{center}
\end{table}


%----------------------------------------------------------------------------------------------------
%----------------------------------------------------------------------------------------------------
%----------------------------------------------------------------------------------------------------
\clearpage
\subsection{Moving Body, ramped velocity, $\rhob=5$: self-convergence results}

Inflow velocity profile is parabolic.

$v_{\rm inflow}=-.5$


Note -- u-errors are getting larger at outflow (bottom) -- may neeed a longer domain 

$t=1$ results for p are poor.


% -------------- max norm results -------------
\begin{table}[hbt]\tableFont % you should set \tableFont to \footnotesize or other size
% \newcommand{\num}[2]{#1e{#2}} % use this command to set the format of numbers in the table.
% \newcommand{\errFormat}[1]{#1}} % use this command to set the format of the error label.
\begin{center}
\begin{tabular}{|l|c|c|c|c|c|c|} \hline 
   show file         & \errFormat{p} &  r   & \errFormat{u} &  r   & \errFormat{v} &  r  \\ \hline
 fallingDropAmpRhob5RampedVG2 & \num{8.4}{-4} &      & \num{2.7}{-4} &      & \num{4.1}{-4} &      \\ \hline
 fallingDropAmpRhob5RampedVG4 & \num{1.6}{-4} &  5.4 & \num{5.8}{-5} &  4.6 & \num{1.6}{-4} &  2.6 \\ \hline
 fallingDropAmpRhob5RampedVG8 & \num{2.9}{-5} &  5.4 & \num{1.2}{-5} &  4.6 & \num{6.2}{-5} &  2.6 \\ \hline
                      &     2.42      &      &     2.21      &      &     1.37      &     \\ \hline
\end{tabular}
\caption{Max-norm self convergence results, t=2.000000e+00, Mon Nov  7 14:35:48 2016. }
\end{center}
\end{table}

% -------------- l2 norm results -------------
\begin{table}[hbt]\tableFont % you should set \tableFont to \footnotesize or other size
% \newcommand{\num}[2]{#1e{#2}} % use this command to set the format of numbers in the table.
% \newcommand{\errFormat}[1]{#1}} % use this command to set the format of the error label.
\begin{center}
\begin{tabular}{|l|c|c|c|c|c|c|} \hline 
   show file         & \errFormat{p} &  r   & \errFormat{u} &  r   & \errFormat{v} &  r  \\ \hline
 fallingDropAmpRhob5RampedVG2 & \num{3.2}{-4} &      & \num{8.2}{-5} &      & \num{1.2}{-4} &      \\ \hline
 fallingDropAmpRhob5RampedVG4 & \num{3.2}{-5} &  9.8 & \num{2.5}{-5} &  3.3 & \num{3.7}{-5} &  3.2 \\ \hline
 fallingDropAmpRhob5RampedVG8 & \num{3.3}{-6} &  9.8 & \num{7.7}{-6} &  3.3 & \num{1.2}{-5} &  3.2 \\ \hline
                      &     3.29      &      &     1.70      &      &     1.67      &     \\ \hline
\end{tabular}
\caption{L2-norm self convergence results, t=2.000000e+00, Mon Nov  7 14:35:48 2016. }
\end{center}
\end{table}

% -------------- max norm results -------------
\begin{table}[hbt]\tableFont % you should set \tableFont to \footnotesize or other size
% \newcommand{\num}[2]{#1e{#2}} % use this command to set the format of numbers in the table.
% \newcommand{\errFormat}[1]{#1}} % use this command to set the format of the error label.
\begin{center}
\begin{tabular}{|l|c|c|c|c|c|c|} \hline 
   show file         & \errFormat{p} &  r   & \errFormat{u} &  r   & \errFormat{v} &  r  \\ \hline
 fallingDropAmpRhob5RampedVG2 & \num{2.0}{-3} &      & \num{3.5}{-4} &      & \num{1.1}{-3} &      \\ \hline
 fallingDropAmpRhob5RampedVG4 & \num{7.9}{-4} &  2.6 & \num{8.9}{-5} &  3.9 & \num{2.6}{-4} &  4.0 \\ \hline
 fallingDropAmpRhob5RampedVG8 & \num{3.1}{-4} &  2.6 & \num{2.3}{-5} &  3.9 & \num{6.6}{-5} &  4.0 \\ \hline
                      &     1.36      &      &     1.96      &      &     2.01      &     \\ \hline
\end{tabular}
\caption{Max-norm self convergence results, t=1.000000e+00, Mon Nov  7 14:39:04 2016. }
\end{center}
\end{table}

% -------------- l2 norm results -------------
\begin{table}[hbt]\tableFont % you should set \tableFont to \footnotesize or other size
% \newcommand{\num}[2]{#1e{#2}} % use this command to set the format of numbers in the table.
% \newcommand{\errFormat}[1]{#1}} % use this command to set the format of the error label.
\begin{center}
\begin{tabular}{|l|c|c|c|c|c|c|} \hline 
   show file         & \errFormat{p} &  r   & \errFormat{u} &  r   & \errFormat{v} &  r  \\ \hline
 fallingDropAmpRhob5RampedVG2 & \num{9.9}{-4} &      & \num{1.2}{-4} &      & \num{3.6}{-4} &      \\ \hline
 fallingDropAmpRhob5RampedVG4 & \num{3.6}{-4} &  2.8 & \num{3.8}{-5} &  3.2 & \num{7.8}{-5} &  4.6 \\ \hline
 fallingDropAmpRhob5RampedVG8 & \num{1.3}{-4} &  2.8 & \num{1.2}{-5} &  3.2 & \num{1.7}{-5} &  4.6 \\ \hline
                      &     1.48      &      &     1.68      &      &     2.19      &     \\ \hline
\end{tabular}
\caption{L2-norm self convergence results, t=1.000000e+00, Mon Nov  7 14:39:04 2016. }
\end{center}
\end{table}

% ================================================================================================================
% ================================================================================================================
% ================================================================================================================
\clearpage
\section{Offset cylinder in a channel}



% -------------------------------------------------------------------------
\subsection{Falling offset cylinder, $\rhob=5$, TP}

% -------------- max norm results -------------
\begin{table}[hbt]\tableFont % you should set \tableFont to \footnotesize or other size
% \newcommand{\num}[2]{#1e{#2}} % use this command to set the format of numbers in the table.
% \newcommand{\errFormat}[1]{#1}} % use this command to set the format of the error label.
\begin{center}
\begin{tabular}{|l|c|c|c|c|c|c|} \hline 
   grid              & \errFormat{p} &  r   & \errFormat{u} &  r   & \errFormat{v} &  r  \\ \hline
 offsetDropRhob5RampedPG2 & \num{4.4}{-3} &      & \num{2.0}{-3} &      & \num{2.4}{-3} &      \\ \hline
 offsetDropRhob5RampedPG4 & \num{9.3}{-4} &  4.8 & \num{4.3}{-4} &  4.7 & \num{4.6}{-4} &  5.1 \\ \hline
 offsetDropRhob5RampedPG8 & \num{1.9}{-4} &  4.8 & \num{9.0}{-5} &  4.7 & \num{9.1}{-5} &  5.1 \\ \hline
                      &     2.26      &      &     2.24      &      &     2.35      &     \\ \hline
\end{tabular}
\caption{Max-norm self convergence results, Tue Nov  8 11:46:06 2016. }
\end{center}
\end{table}

% -------------- l2 norm results -------------
\begin{table}[hbt]\tableFont % you should set \tableFont to \footnotesize or other size
% \newcommand{\num}[2]{#1e{#2}} % use this command to set the format of numbers in the table.
% \newcommand{\errFormat}[1]{#1}} % use this command to set the format of the error label.
\begin{center}
\begin{tabular}{|l|c|c|c|c|c|c|} \hline 
   grid              & \errFormat{p} &  r   & \errFormat{u} &  r   & \errFormat{v} &  r  \\ \hline
 offsetDropRhob5RampedPG2 & \num{5.8}{-4} &      & \num{3.3}{-4} &      & \num{4.9}{-4} &      \\ \hline
 offsetDropRhob5RampedPG4 & \num{1.2}{-4} &  4.8 & \num{7.1}{-5} &  4.6 & \num{1.1}{-4} &  4.6 \\ \hline
 offsetDropRhob5RampedPG8 & \num{2.5}{-5} &  4.8 & \num{1.6}{-5} &  4.6 & \num{2.3}{-5} &  4.6 \\ \hline
                      &     2.28      &      &     2.20      &      &     2.21      &     \\ \hline
\end{tabular}
\caption{L2-norm self convergence results, Tue Nov  8 11:46:06 2016. }
\end{center}
\end{table}



% -------------------------------------------------------------------------
\clearpage
\subsection{Falling offset cylinder, $\rhob=2$, AMP}

% -------------- max norm results -------------
\begin{table}[hbt]\tableFont % you should set \tableFont to \footnotesize or other size
% \newcommand{\num}[2]{#1e{#2}} % use this command to set the format of numbers in the table.
% \newcommand{\errFormat}[1]{#1}} % use this command to set the format of the error label.
\begin{center}
\begin{tabular}{|l|c|c|c|c|c|c|} \hline 
   grid              & \errFormat{p} &  r   & \errFormat{u} &  r   & \errFormat{v} &  r  \\ \hline
 offsetDropAmpRhob2RampedPG2 & \num{4.1}{-3} &      & \num{1.7}{-3} &      & \num{2.0}{-3} &      \\ \hline
 offsetDropAmpRhob2RampedPG4 & \num{8.1}{-4} &  5.1 & \num{4.0}{-4} &  4.1 & \num{4.5}{-4} &  4.5 \\ \hline
 offsetDropAmpRhob2RampedPG8 & \num{1.6}{-4} &  5.1 & \num{9.9}{-5} &  4.1 & \num{1.0}{-4} &  4.5 \\ \hline
                      &     2.34      &      &     2.03      &      &     2.16      &     \\ \hline
\end{tabular}
\caption{Max-norm self convergence results, Tue Nov  8 11:55:18 2016. }
\end{center}
\end{table}

% -------------- l2 norm results -------------
\begin{table}[hbt]\tableFont % you should set \tableFont to \footnotesize or other size
% \newcommand{\num}[2]{#1e{#2}} % use this command to set the format of numbers in the table.
% \newcommand{\errFormat}[1]{#1}} % use this command to set the format of the error label.
\begin{center}
\begin{tabular}{|l|c|c|c|c|c|c|} \hline 
   grid              & \errFormat{p} &  r   & \errFormat{u} &  r   & \errFormat{v} &  r  \\ \hline
 offsetDropAmpRhob2RampedPG2 & \num{6.4}{-4} &      & \num{3.0}{-4} &      & \num{4.8}{-4} &      \\ \hline
 offsetDropAmpRhob2RampedPG4 & \num{1.2}{-4} &  5.4 & \num{6.9}{-5} &  4.4 & \num{1.1}{-4} &  4.5 \\ \hline
 offsetDropAmpRhob2RampedPG8 & \num{2.2}{-5} &  5.4 & \num{1.6}{-5} &  4.4 & \num{2.4}{-5} &  4.5 \\ \hline
                      &     2.42      &      &     2.14      &      &     2.17      &     \\ \hline
\end{tabular}
\caption{L2-norm self convergence results, Tue Nov  8 11:55:18 2016. }
\end{center}
\end{table}

% -------------------------------------------------------------------------
\clearpage
\subsection{Falling offset cylinder, $\rhob=0.1$, AMP}

% -------------- max norm results -------------
\begin{table}[hbt]\tableFont % you should set \tableFont to \footnotesize or other size
% \newcommand{\num}[2]{#1e{#2}} % use this command to set the format of numbers in the table.
% \newcommand{\errFormat}[1]{#1}} % use this command to set the format of the error label.
\begin{center}
\begin{tabular}{|l|c|c|c|c|c|c|} \hline 
   grid              & \errFormat{p} &  r   & \errFormat{u} &  r   & \errFormat{v} &  r  \\ \hline
 offsetDropAmpRhob0p1RampedPG2 & \num{4.6}{-3} &      & \num{2.5}{-3} &      & \num{1.9}{-3} &      \\ \hline
 offsetDropAmpRhob0p1RampedPG4 & \num{1.2}{-3} &  4.0 & \num{5.1}{-4} &  4.9 & \num{4.9}{-4} &  3.9 \\ \hline
 offsetDropAmpRhob0p1RampedPG8 & \num{2.9}{-4} &  4.0 & \num{1.0}{-4} &  4.9 & \num{1.3}{-4} &  3.9 \\ \hline
                      &     2.00      &      &     2.29      &      &     1.97      &     \\ \hline
\end{tabular}
\caption{Max-norm self convergence results, Tue Nov  8 15:13:16 2016. }
\end{center}
\end{table}

% -------------- l2 norm results -------------
\begin{table}[hbt]\tableFont % you should set \tableFont to \footnotesize or other size
% \newcommand{\num}[2]{#1e{#2}} % use this command to set the format of numbers in the table.
% \newcommand{\errFormat}[1]{#1}} % use this command to set the format of the error label.
\begin{center}
\begin{tabular}{|l|c|c|c|c|c|c|} \hline 
   grid              & \errFormat{p} &  r   & \errFormat{u} &  r   & \errFormat{v} &  r  \\ \hline
 offsetDropAmpRhob0p1RampedPG2 & \num{6.2}{-4} &      & \num{3.0}{-4} &      & \num{4.9}{-4} &      \\ \hline
 offsetDropAmpRhob0p1RampedPG4 & \num{1.4}{-4} &  4.5 & \num{7.1}{-5} &  4.2 & \num{1.1}{-4} &  4.5 \\ \hline
 offsetDropAmpRhob0p1RampedPG8 & \num{3.1}{-5} &  4.5 & \num{1.7}{-5} &  4.2 & \num{2.4}{-5} &  4.5 \\ \hline
                      &     2.17      &      &     2.07      &      &     2.18      &     \\ \hline
\end{tabular}
\caption{L2-norm self convergence results, Tue Nov  8 15:13:16 2016. }
\end{center}
\end{table}


% -------------------------------------------------------------------------
\clearpage
\subsection{Falling offset cylinder, $\rhob=0$, AMP}

**CHECK wiggles in $a_1$ .... these decrease as mesh is refined, probably just due
to grid overlap changing. 



% -------------- max norm results -------------
\begin{table}[hbt]\tableFont % you should set \tableFont to \footnotesize or other size
% \newcommand{\num}[2]{#1e{#2}} % use this command to set the format of numbers in the table.
% \newcommand{\errFormat}[1]{#1}} % use this command to set the format of the error label.
\begin{center}
\begin{tabular}{|l|c|c|c|c|c|c|} \hline 
   grid              & \errFormat{p} &  r   & \errFormat{u} &  r   & \errFormat{v} &  r  \\ \hline
 offsetDropAmpRhob0RampedPG2 & \num{4.8}{-3} &      & \num{2.5}{-3} &      & \num{2.0}{-3} &      \\ \hline
 offsetDropAmpRhob0RampedPG4 & \num{1.2}{-3} &  4.1 & \num{5.1}{-4} &  4.9 & \num{4.9}{-4} &  4.0 \\ \hline
 offsetDropAmpRhob0RampedPG8 & \num{2.9}{-4} &  4.1 & \num{1.0}{-4} &  4.9 & \num{1.2}{-4} &  4.0 \\ \hline
                      &     2.04      &      &     2.30      &      &     2.01      &     \\ \hline
\end{tabular}
\caption{Max-norm self convergence results, Tue Nov  8 12:58:49 2016. }
\end{center}
\end{table}

% -------------- l2 norm results -------------
\begin{table}[hbt]\tableFont % you should set \tableFont to \footnotesize or other size
% \newcommand{\num}[2]{#1e{#2}} % use this command to set the format of numbers in the table.
% \newcommand{\errFormat}[1]{#1}} % use this command to set the format of the error label.
\begin{center}
\begin{tabular}{|l|c|c|c|c|c|c|} \hline 
   grid              & \errFormat{p} &  r   & \errFormat{u} &  r   & \errFormat{v} &  r  \\ \hline
 offsetDropAmpRhob0RampedPG2 & \num{6.1}{-4} &      & \num{3.0}{-4} &      & \num{5.0}{-4} &      \\ \hline
 offsetDropAmpRhob0RampedPG4 & \num{1.4}{-4} &  4.3 & \num{7.1}{-5} &  4.2 & \num{1.1}{-4} &  4.6 \\ \hline
 offsetDropAmpRhob0RampedPG8 & \num{3.4}{-5} &  4.3 & \num{1.7}{-5} &  4.2 & \num{2.4}{-5} &  4.6 \\ \hline
                      &     2.10      &      &     2.07      &      &     2.19      &     \\ \hline
\end{tabular}
\caption{L2-norm self convergence results, Tue Nov  8 12:58:49 2016. }
\end{center}
\end{table}


% -------------- max norm results -------------
\begin{table}[hbt]\tableFont % you should set \tableFont to \footnotesize or other size
% \newcommand{\num}[2]{#1e{#2}} % use this command to set the format of numbers in the table.
% \newcommand{\errFormat}[1]{#1}} % use this command to set the format of the error label.
\begin{center}
\begin{tabular}{|l|c|c|c|c|c|c|} \hline 
   grid              & \errFormat{p} &  r   & \errFormat{u} &  r   & \errFormat{v} &  r  \\ \hline
 offsetDropAmpRhob0RampedPG2 & \num{4.8}{-3} &      & \num{2.2}{-3} &      & \num{1.8}{-3} &      \\ \hline
 offsetDropAmpRhob0RampedPG4 & \num{1.2}{-3} &  4.1 & \num{5.4}{-4} &  4.1 & \num{5.2}{-4} &  3.4 \\ \hline
 offsetDropAmpRhob0RampedPG8 & \num{2.8}{-4} &  4.1 & \num{1.3}{-4} &  4.1 & \num{1.5}{-4} &  3.4 \\ \hline
 offsetDropAmpRhob0RampedPG16 & \num{6.7}{-5} &  4.1 & \num{3.1}{-5} &  4.1 & \num{4.4}{-5} &  3.4 \\ \hline
                      &     2.05      &      &     2.05      &      &     1.78      &     \\ \hline
\end{tabular}
\caption{Max-norm self convergence results, Wed Nov  9 06:48:02 2016. }
\end{center}
\end{table}

% -------------- l2 norm results -------------
\begin{table}[hbt]\tableFont % you should set \tableFont to \footnotesize or other size
% \newcommand{\num}[2]{#1e{#2}} % use this command to set the format of numbers in the table.
% \newcommand{\errFormat}[1]{#1}} % use this command to set the format of the error label.
\begin{center}
\begin{tabular}{|l|c|c|c|c|c|c|} \hline 
   grid              & \errFormat{p} &  r   & \errFormat{u} &  r   & \errFormat{v} &  r  \\ \hline
 offsetDropAmpRhob0RampedPG2 & \num{5.8}{-4} &      & \num{2.8}{-4} &      & \num{4.5}{-4} &      \\ \hline
 offsetDropAmpRhob0RampedPG4 & \num{1.5}{-4} &  3.9 & \num{7.5}{-5} &  3.7 & \num{1.1}{-4} &  3.9 \\ \hline
 offsetDropAmpRhob0RampedPG8 & \num{3.8}{-5} &  3.9 & \num{2.0}{-5} &  3.7 & \num{2.9}{-5} &  3.9 \\ \hline
 offsetDropAmpRhob0RampedPG16 & \num{9.7}{-6} &  3.9 & \num{5.4}{-6} &  3.7 & \num{7.5}{-6} &  3.9 \\ \hline
                      &     1.97      &      &     1.89      &      &     1.97      &     \\ \hline
\end{tabular}
\caption{L2-norm self convergence results, Wed Nov  9 06:48:02 2016. }
\end{center}
\end{table}

% -------------------------------------------------------------------------
\clearpage
\subsection{Falling offset cylinder, $\rhob=0$, AMP, ramped pressure inflow, ramped gravity}

With gravity we can run for a longer time. 


% -------------- max norm results -------------
\begin{table}[hbt]\tableFont % you should set \tableFont to \footnotesize or other size
% \newcommand{\num}[2]{#1e{#2}} % use this command to set the format of numbers in the table.
% \newcommand{\errFormat}[1]{#1}} % use this command to set the format of the error label.
\begin{center}
\begin{tabular}{|l|c|c|c|c|c|c|} \hline 
   grid              & \errFormat{p} &  r   & \errFormat{u} &  r   & \errFormat{v} &  r  \\ \hline
 offsetDropAmpRhob0RampedPGG2 & \num{5.9}{-2} &      & \num{2.6}{-2} &      & \num{3.6}{-2} &      \\ \hline
 offsetDropAmpRhob0RampedPGG4 & \num{6.3}{-3} &  9.4 & \num{4.0}{-3} &  6.6 & \num{3.2}{-3} & 11.2 \\ \hline
 offsetDropAmpRhob0RampedPGG8 & \num{6.7}{-4} &  9.4 & \num{6.0}{-4} &  6.6 & \num{2.9}{-4} & 11.2 \\ \hline
                      &     3.23      &      &     2.73      &      &     3.49      &     \\ \hline
\end{tabular}
\caption{Max-norm self convergence results, Wed Nov  9 07:49:17 2016. }
\end{center}
\end{table}

% -------------- l2 norm results -------------
\begin{table}[hbt]\tableFont % you should set \tableFont to \footnotesize or other size
% \newcommand{\num}[2]{#1e{#2}} % use this command to set the format of numbers in the table.
% \newcommand{\errFormat}[1]{#1}} % use this command to set the format of the error label.
\begin{center}
\begin{tabular}{|l|c|c|c|c|c|c|} \hline 
   grid              & \errFormat{p} &  r   & \errFormat{u} &  r   & \errFormat{v} &  r  \\ \hline
 offsetDropAmpRhob0RampedPGG2 & \num{4.7}{-3} &      & \num{2.9}{-3} &      & \num{3.1}{-3} &      \\ \hline
 offsetDropAmpRhob0RampedPGG4 & \num{5.2}{-4} &  9.1 & \num{3.7}{-4} &  7.8 & \num{3.7}{-4} &  8.4 \\ \hline
 offsetDropAmpRhob0RampedPGG8 & \num{5.7}{-5} &  9.1 & \num{4.8}{-5} &  7.8 & \num{4.4}{-5} &  8.4 \\ \hline
                      &     3.18      &      &     2.96      &      &     3.07      &     \\ \hline
\end{tabular}
\caption{L2-norm self convergence results, Wed Nov  9 07:49:17 2016. }
\end{center}
\end{table}

% -------------- max norm results -------------
\begin{table}[hbt]\tableFont % you should set \tableFont to \footnotesize or other size
% \newcommand{\num}[2]{#1e{#2}} % use this command to set the format of numbers in the table.
% \newcommand{\errFormat}[1]{#1}} % use this command to set the format of the error label.
\begin{center}
\begin{tabular}{|l|c|c|c|c|c|c|} \hline 
   show file         & \errFormat{p} &  r   & \errFormat{u} &  r   & \errFormat{v} &  r  \\ \hline
 offsetDropAmpRhob0RampedPGG2 & \num{8.9}{-2} &      & \num{4.9}{-2} &      & \num{3.8}{-2} &      \\ \hline
 offsetDropAmpRhob0RampedPGG4 & \num{8.7}{-3} & 10.1 & \num{5.9}{-3} &  8.3 & \num{4.6}{-3} &  8.2 \\ \hline
 offsetDropAmpRhob0RampedPGG8 & \num{8.6}{-4} & 10.1 & \num{7.1}{-4} &  8.3 & \num{5.6}{-4} &  8.2 \\ \hline
                      &     3.34      &      &     3.06      &      &     3.04      &     \\ \hline
\end{tabular}
\caption{Max-norm self convergence results, t=6.000000e+00, Wed Nov  9 08:28:32 2016. }
\end{center}
\end{table}

% -------------- l2 norm results -------------
\begin{table}[hbt]\tableFont % you should set \tableFont to \footnotesize or other size
% \newcommand{\num}[2]{#1e{#2}} % use this command to set the format of numbers in the table.
% \newcommand{\errFormat}[1]{#1}} % use this command to set the format of the error label.
\begin{center}
\begin{tabular}{|l|c|c|c|c|c|c|} \hline 
   show file         & \errFormat{p} &  r   & \errFormat{u} &  r   & \errFormat{v} &  r  \\ \hline
 offsetDropAmpRhob0RampedPGG2 & \num{7.5}{-3} &      & \num{4.6}{-3} &      & \num{4.6}{-3} &      \\ \hline
 offsetDropAmpRhob0RampedPGG4 & \num{8.8}{-4} &  8.5 & \num{5.2}{-4} &  8.7 & \num{5.2}{-4} &  9.0 \\ \hline
 offsetDropAmpRhob0RampedPGG8 & \num{1.0}{-4} &  8.5 & \num{6.0}{-5} &  8.7 & \num{5.8}{-5} &  9.0 \\ \hline
                      &     3.09      &      &     3.13      &      &     3.16      &     \\ \hline
\end{tabular}
\caption{L2-norm self convergence results, t=6.000000e+00, Wed Nov  9 08:28:32 2016. }
\end{center}
\end{table}

% \clearpage
% \bibliography{henshaw,henshawPapers}
% \bibliographystyle{siam}



\end{document}
% **************************************************************************************************************
% **************************************************************************************************************
% **************************************************************************************************************
