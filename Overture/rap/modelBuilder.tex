%=======================================================================================================
% ModelBuilder Documentation
%=======================================================================================================
\documentclass[11pt]{article}
% \usepackage[bookmarks=true]{hyperref}  % this changes the page location !
\usepackage[bookmarks=true,colorlinks=true,linkcolor=blue]{hyperref}

% \input documentationPageSize.tex
\hbadness=10000 
\sloppy \hfuzz=30pt

% \voffset=-.25truein
% \hoffset=-1.25truein
% \setlength{\textwidth}{7in}      % page width
% \setlength{\textheight}{9.5in}    % page height

\usepackage{calc}
\usepackage[lmargin=.75in,rmargin=.75in,tmargin=.75in,bmargin=.75in]{geometry}

\input homeHenshaw
\newcommand{\blue}{\color{blue}}
\newcommand{\green}{\color{green}}
\newcommand{\red}{\color{red}}
\newcommand{\black}{\color{black}}


% \usepackage{epsfig}
\usepackage{graphicx}    
\usepackage{moreverb}
\usepackage{amsmath}
% \usepackage{fancybox}  % this destroys the table of contents!
% \usepackage{subfigure}
\usepackage{multicol}

\input wdhDefinitions
% \usepackage{pstricks}
% \input{pstricks}\input{pst-node}
% \input{colours}
\usepackage{tikz}
\usepackage{pgfplots}
\input trimFig.tex

% define the clipFig commands:
% \usepackage{calc}
% \input clipFig.tex


\usepackage{makeidx} % index
\makeindex
\newcommand{\Index}[1]{#1\index{#1}}

\newcommand{\Largebf}{\sffamily\bfseries\Large}
\newcommand{\largebf}{\sffamily\bfseries\large}
\newcommand{\largess}{\sffamily\large}
\newcommand{\Largess}{\sffamily\Large}
\newcommand{\bfss}{\sffamily\bfseries}
\newcommand{\smallss}{\sffamily\small}
\newcommand{\normalss}{\sffamily}

% *** See http://www.eng.cam.ac.uk/help/tpl/textprocessing/squeeze.html
% By default, LaTeX doesn't like to fill more than 0.7 of a text page with tables and graphics, nor does it like too many figures per page. This behaviour can be changed by placing lines like the following before \begin{document}

\renewcommand\floatpagefraction{.9}
\renewcommand\topfraction{.9}
\renewcommand\bottomfraction{.9}
\renewcommand\textfraction{.1}   
\setcounter{totalnumber}{50}
\setcounter{topnumber}{50}
\setcounter{bottomnumber}{50}

\begin{document}



% \end{document}

\def\uvd    {{\bf U}}
\def\ud     {{    U}}
\def\pd     {{    P}}
\def\id     {i}
\def\jd     {j}
\def\kap {\sqrt{s+\omega^2}}

\newcommand{\mapping}{\homeHenshaw/Overture/mapping}
\newcommand{\figures}{\homeHenshaw/OvertureFigures}

\vspace{3\baselineskip}
\begin{flushleft}
  {\Large 
   Model Builder  \\ 
   An Overture Tool for CAD Cleanup and Modification  \\
  }
\vskip 2\baselineskip
{\large Kyle K. Chand,  }             \\
{\large William D. Henshaw,  }             \\
{\large Anders Petersson}             \\
\vskip 1\baselineskip
Centre for Applied Scientific Computing \\
Lawrence Livermore National Laboratory    \\
Livermore, CA, 94551   \\
% henshaw@llnl.gov \\
% http://www.llnl.gov/casc/people/henshaw \\
% http://www.llnl.gov/casc/Overture \\
\vskip 1\baselineskip
\today \\
\vskip 1\baselineskip
% UCRL-MA-???????

\end{flushleft}

\vspace{1\baselineskip}

\begin{abstract}
This document describes the {\tt ModelBuilder} class. This class can be used
to build geometric models, read, edit, repair and modify CAD geometries.
Some of the features are
\begin{itemize}
 \item read CAD geometries from IGES files.
 \item automatic detection of many errors in the CAD representation such as invalid
    trimming curves.
 \item repair CAD geometries through interactive editing of the trimming curves
 \item modify CAD geometries to remove features, ... 
 \item automatically build the topology (connectivity) of the patched model, (closing gaps and removing overlaps),
  and constructing a water-tight representation of the CAD surface using a global triangulation.
\end{itemize}
\end{abstract}

\clearpage
\tableofcontents

\vspace{3\baselineskip}

\section{Introduction}

{\bf NOTE: This documentation is under construction.}




% \begin{figure} \label{fig:ModelBuilder}
%   \begin{center}
%   \includegraphics{modelBuilder.ps}
%   \caption{The ModelBuilder}
%   \end{center}
% \end{figure}


% \section{Sample uses of the ModelBuilder}



% \section{Read an IGES file, repair surfaces, build the topology}

\input retro.tex


\clearpage
% ----------------------------------------------------------------------------------
% \section{Read an IGES file, simplify the geometry, build the topology}

% This section needs to be written.

% ----------------------------------------------------------------------------------
\section{Build a surface of revolution}

This section needs to be written.

% ----------------------------------------------------------------------------------
\clearpage
\input fixingTrimmingCurves


\clearpage
\input nozzleAndCavity

\clearpage 
\input marsCapsule




\clearpage
\section{Model Builder Class Member Functions}\index{ModelBuilder member functions}

\input ModelBuilderInclude.tex



\clearpage
\section{Appendix: Mouse and button commands}
\input mouseAndButtons.tex


\clearpage
\bibliography{/home/henshaw/papers/henshaw}
\bibliographystyle{siam}

\printindex

\end{document}
