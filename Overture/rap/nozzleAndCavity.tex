\newcommand{\vbsize}{\footnotesize}
\section{Grid for a nozzle and cavity}\label{sec:nozzleAndCavity}

In this section we demonstrate how to build an overlapping grid for a geometry of a converging 
diverging nozzle with a cavity at the throat.
The geometry is defined through an IGES file (from CATIA).
This geometry is courtesy of Philippe Lafon's group at EDF.

The ogen command file used for this example is called {\tt nozzleAndCavity.cmd}.
To start, the CAD geometry is read from an IGES file using the commands:
read from an IGES file.
{\vbsize
\begin{verbatim}
create mappings
  read iges file
    /Users/henshaw/res/Overture/ogen/vanne_simplifiee_catia2.igs
    continue
    choose all
\end{verbatim}
}
At this points the CompositeSurface dialog
appears and the CAD surface is plotted as in figure~\ref{fig:nozzleAndCavityCAD} (left).
The CAD surface consists of a set of (trimmed) patches.
The connectivity of the CAD surface is now determined using the topology function,
{\vbsize
\begin{verbatim}
  CSUP:determine topology
    merge tolerance 0.001
    deltaS 0.01
    maximum area 1.e-4
    compute topology
  exit
\end{verbatim}
}
The topology algorithm will merge matching edges from adjacent patches (using the
{\em merge tolerance} to determine when this can be done. Each edge will be 
discretized with a set of points separated by a distance of approximately {\em deltaS}
and a triangluation will be generated with a given {\em maximum area}. The triangulation
is used later in the grid generation process as an aid in projecting points
on to the original CAD patches.

\input nozzleAndCavityCADFig


Upon exiting the CompositeSurface, the {\em builder} option is chosen to open 
the {\em Mapping Builder} dialog.
The target grid spacing for all subsequent grids is set using the perl variable {\em \$ds}.
{\vbsize
\begin{verbatim}
  builder
    target grid spacing $ds $ds (tang,norm)((<0 : use default)
\end{verbatim}
}
After choosing {\em create surface grid...} one enters the hyperbolic grid generator ({\em hype} for short).
An initial curve is selected and a surface grid is grown for the left section of
the nozzle as shown
in figure~\ref{fig:nozzleAndCavityLeftNozzle}.
{\vbsize
\begin{verbatim}
  # left nozzle:
  create surface grid...
    choose boundary curve 1
    done
    $ndist=.46;
    $nx = int($ndist/$ds+1.5);
    lines to march $nx
    generate
    name left_nozzle_surface
  exit
\end{verbatim}
}
\noindent Hype is also used to grow a volume grid,
{\vbsize
\begin{verbatim}
    create volume grid...
      marching options...
      BC: bottom fix x, float y and z
      BC: top fix x, float y and z
      backward
      lines to march $nn
      generate
      name left_nozzle
      Boundary Condition: bottom  3
      Boundary Condition: top     0
      Share Value: back    2
      Share Value: bottom  3
      exit
\end{verbatim}
}

Similiar steps are used to build a surface and volume grid to connect the
cavity with the nozzle as shown in figure~\ref{fig:nozzleAndCavityLeftCorner},
and grids for the cavity itself, figure~\ref{fig:nozzleAndCavityBase}.


\input nozzleAndCavityLeftNozzleFig


The overlapping grid for the geometry, generated by ogen, is shown in figure~\ref{fig:nozzleAndCavityGrid}

\input nozzleAndCavityFig
