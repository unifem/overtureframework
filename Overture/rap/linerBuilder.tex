%=======================================================================================================
%  linerBuilder Documentation
%=======================================================================================================
\documentclass[12pt]{article}

\usepackage{times}  % for embeddable fonts, Also use: dvips -P pdf -G0

\input documentationPageSize.tex

\usepackage{epsfig}
\usepackage{graphicx}    
\usepackage{moreverb}
\usepackage{amsmath}
\usepackage{amssymb}

% \usepackage{fancybox}  % this destroys the table of contents!
% \usepackage{subfigure}
% \usepackage{multicol}

\input wdhDefinitions
\input{pstricks}\input{pst-node}
\input{colours}

\usepackage{makeidx} % index
\makeindex
\newcommand{\Index}[1]{#1\index{#1}}

\newcommand{\Largebf}{\sffamily\bfseries\Large}
\newcommand{\largebf}{\sffamily\bfseries\large}
\newcommand{\largess}{\sffamily\large}
\newcommand{\Largess}{\sffamily\Large}
\newcommand{\bfss}{\sffamily\bfseries}
\newcommand{\smallss}{\sffamily\small}
\newcommand{\normalss}{\sffamily}

\newcommand{\cmd}[1]{``{\tt #1}''}

% \newcommand{\clipfigb}[2]{\psclip{\psframe[linewidth=2pt](1.9,2.9)(16,11.75)}\includegraphics[width=#2]{#1}\endpsclip}
\begin{document}
% This is a test


% \end{document}


\newcommand{\mapping}{/home/henshaw/Overture/mapping}
\newcommand{\figures}{/home/henshaw/Overture/docFigures}

\vspace{3\baselineskip}
\begin{flushleft}
  {\Large 
   Liner Builder:  \\ 
   A Tool for Building Geometric Models of Shaped Charge Liners  \\
  }
\vskip 2\baselineskip
{\large William D. Henshaw  }             \\
\vskip 1\baselineskip
Centre for Applied Scientific Computing \\
Lawrence Livermore National Laboratory    \\
Livermore, CA, 94551   \\
henshaw@llnl.gov \\
\vskip 1\baselineskip
\today \\
\vskip 1\baselineskip
UCRL-SM-208948
\end{flushleft}

\vspace{1\baselineskip}

\begin{abstract}

This document describes the {\em Liner Builder}, an interactive
graphics tool for building a geometric model of the conical liner
that fits in the hollow cavity on the end of a shaped charge. This
tool can also be used to load the volume of the liner with a
distribution of spherical particles.
\end{abstract}


\tableofcontents

\vskip5\baselineskip
\noindent{\bf\large Acknowledgments.}
\vskip\baselineskip

Bill Bateson and Dale Slone have contributed to the design of the {\em Liner Builder}.
Dale has also contributed directly to the development of the {\em Liner Builder}, in particular the
loading of the liner with spheres.
The Liner Builder is built upon components of the Overture framework
(including Rapsodi). Many people have worked on the development of
Overture but Kyle Chand and Anders Petersson, in particular, have made
important contributions to the components used by the {\em Liner
Builder}.



% -----------------------------------------------------------------------------------
\clearpage
\section{Introduction}
% ----------------------------- shaped charge -----------------------------------------------
{
\psset{xunit=1.5cm,yunit=1.5cm,runit=1.5cm}
% figure of a shaped charge
\begin{figure}[ht]
\begin{center}
\begin{pspicture}(-6,0)(4,3)%
%
\pspolygon[fillstyle=solid,fillcolor=turquoise](-4,0)(0,0)(2,2)(-4,2)%
\rput(-2.5,1){\makebox(0,0)[l]{\normalss explosive}}
%
\pspolygon[fillstyle=solid,fillcolor=aquamarine](0,0)(.5,0)(2.5,2)(2,2)%
\rput(2.5,1){\makebox(0,0)[l]{\normalss liner}}
\psline[linewidth=1.pt]{->}(2.5,1.)(1.3,1)
%
\pspolygon[fillstyle=solid,fillcolor=red](-4,0)(-4,.5)(-4.5,.5)(-4.5,0)%
\rput(-4.75,1){\makebox(0,0)[r]{\normalss detonator}}
\psline[linewidth=1.pt]{->}(-5.25,.8)(-4.5,.25)
%
%
\pspolygon[fillstyle=solid,fillcolor=mediumgoldenrod](-4,.5)(-4.,2)(2.5,2)(2.5,2.2)(-4.2,2.2)(-4.2,.5)%
\rput(-3,3){\makebox(0,0)[c]{\normalss container}}
\psline[linewidth=1.pt]{->}(-3,2.8)(-3,2.2)
%
\psline[linewidth=2.pt]{->}(-5,0)(3,0)
\rput(1.,-.25){\makebox(0,0)[l]{\smallss axis of revolution}}
%
% turn on the grid for placement
% \psgrid[subgriddiv=2]%
\end{pspicture}
\end{center}
\caption{Basic geometry of a shaped charge. When revolved about the axis of revolution, the liner will form
   a conical shell.}
\label{fig:shapedCharge}
\end{figure}
}
% ---------------------------------------------------------------------------------


This document describes the {\em Liner Builder}, an interactive
graphics tool for building a geometric model of the conical liner
that fits in the hollow cavity on the end of a shaped charge.

Figure~\ref{fig:shapedCharge}
illustrates the basic geometry of a shaped charge.  The shaped charge
in this figure is assumed to be cylindrically symmetric with the axis of revolution shown.  
The liner is the thin piece of material that is located at the end of the shaped charge
opposite to the detonator. The liner will typcially be of the shape of a conical shell.
When the detonator is set off, it will initiate a detonation
that propagates through the explosive. The detonation wave will strike the
liner material and accelerate it.  The liner will then move and focus
toward the axis of revolution, forming a high speed jet of material. The
tip velocity of the jet can apparently exceed Mach 25 in air, although
the flow velocity of the material remains subsonic. The liner of a
shaped charge can take on many shapes such as hemisphere, cone,
ellipse, tulip, trumpet, pyramid etc.  The liner itself may consist
of a thin layer of metal, plastic, ceramic or other similar
material. Shaped charges have their largest use in the mining and oil
well industries. For further details on shaped charges, see for
example, {\em Fundamentals of Shaped Charges} by Walters and
Zukas\cite{WaltersZukas98}.

The basic steps in designing a liner with the the {\em Liner Builder} are
\begin{enumerate}
  \item Create a two-dimensional curve defining the cross-section of the liner. There are various options 
    available for defining this curve such as using a piece-wise linear curve, a piece-wise
    quadratic curve or a free-form curve.
  \item rotate the cross-section curve to form a surface of revolution
\end{enumerate}
Currently the liner shapes are restricted to be surfaces of revolution although it is envisioned
that general three-dimensional shapes could be supported.

Given a liner there are additional steps that can be taken 
to load the liner with a collection of spheres to be used in a particle simulation of the liner motion.
The spheres represent a model of the liner material. 
\begin{enumerate}
  \item define the properties of the spherical particles such at the distribution of radii; the volume and
    mass fractions in the loading, ...
  \item {\em load} the interior of the liner with a distribution of spherical particles.
  \item define the {\em lighting-curves} that define the initial velocity of the spheres to be used in
     a particle simulation of the liner motion.
\end{enumerate}

The {\em Liner Builder} is part of the more general {\tt rap}
program.  (The name rap is derived from the Rapsodi project -- the name {\tt rap}
name will probably change in the future to something like {\tt
modelBuilder}).  The {\tt rap} program can be used to clean up, fix and
modify CAD geometries as well as build new geometries. 





\section{Getting started}

To start the {\em Liner Builder}, type {\tt rap} to bring up the startup screen for rap. 
The {\tt rap} program is usually found in the
{\tt Overture/bin} directory or in the Dune bin directory(?)
Choose the button \cmd{liner...} to open up the {\em Liner Builder} dialog window.
At this point the screen should look similar to figure~\ref{fig:linerWindows}.
You are now ready to begin the design process.

Note that as commands are entered interactively, the command file {\tt rap.cmd} is
being saved with a text version of each command. This command file can later be given
a new file name, edited with any text editor, and used to rerun the session. 

Before designing a new liner it may be useful to first run through some examples.
See sections~\ref{sec:exampleLinear}, \ref{sec:exampleQuadratic} and \ref{sec:exampleFreeForm} for 
three examples. 

The left mouse botton is use for selecting buttons on the windows and dialogs.
The left mouse button is also used for picking points on the graphics window. Clicking the right
mouse button on any window will bring up any pop-up menu.
Appendix~\ref{app:mouseButtons} is a useful reference for understanding the behaviour of
the buttons, mouse and view characteristics of the the graphics window.



% By default an initial piecewise-linear cross-section curve will appear along with a separate
% dialog window which allows changes to shape of the {\em linear-liner}. 






% To build a surface of revolution, choose the {\tt revolve around axis} option.


% ==========================================================================================================
\clearpage
\section{The Liner Builder Dialog}

{% ----
% .95: \newcommand{\clipfig}[1]{\psclip{\psframe[linewidth=2pt](1.8,2.85)(16.25,11.75)}\epsfig{#1}\endpsclip}
\newcommand{\clipfig}[1]{\psclip{\psframe[linecolor=white](2.2,3.4)(19.6,14.2)}\epsfig{#1}\endpsclip}
\psset{xunit=1.0cm,yunit=1.0cm,runit=1.0cm}
\begin{figure}[htb]
\begin{center}
\begin{pspicture}(0,0)(17,10.75)
% \rput(7, 4.){\clipfig{file=linerWindows.ps,width=.95\linewidth}}
\rput(7.75, 4.75){\clipfig{file=linerWindows.ps,width=1.15\linewidth}}
% \rput(6, 5){\clipfigb{linerWindows.eps}{.95\linewidth}}
% turn on the grid for placement
% \psgrid[subgriddiv=2]
\rput(12,-.3){\psframebox*[fillstyle=solid,fillcolor=mediumgoldenrod]{\smallss Liner Builder dialog}}
\psline[linewidth=1.5pt]{->}(12,0)(12,2.5)
\rput(6.5,-.3){\psframebox*[fillstyle=solid,fillcolor=mediumgoldenrod]{\smallss linear liner dialog}}
\psline[linewidth=1.5pt]{->}(6.5,0)(6.5,1.5)
\rput(3,-.3){\psframebox*[fillstyle=solid,fillcolor=mediumgoldenrod]{\smallss main screen}}
\psline[linewidth=1.5pt]{->}(3,0)(3,2)
%
\rput(2.5,8){\psframebox*[fillstyle=solid,fillcolor=mediumgoldenrod]{\smallss liner cross-section}}
\psline[linewidth=1.5pt]{->}(3.,7.75)(4.7,7.0)
%
\rput(2.,7){\psframebox*[fillstyle=solid,fillcolor=mediumgoldenrod]{\smallss lighting curve}}
\psline[linewidth=1.5pt]{->}(2.5,6.75)(4.7,5)
\end{pspicture}
\end{center}
\caption{Liner Builder windows and dialogs. The main screen plots the results. The {\em Liner Builder} dialog
controls the generation of the shaped charge liner.}
\label{fig:linerWindows}
\end{figure}
  } % ----


The windows associated with the {\em Liner Builder} are show in figure~\ref{fig:linerWindows}.
The main graphics window displays the liner cross-section and the lighting curve. 
The main dialog shown controls the {\em Liner Builder}. A smaller dialog appears for inputing
changes to the linear cross-section (since the method is currently set to \cmd{linear liner}). 

Here are descriptions of the buttons on the Liner Builder dialog
\begin{description}
  \item[method] Choose the type of liner: \cmd{linear liner}, \cmd{quadratic liner} or \cmd{free form liner}.
     Selecting a new method will cause a new dialog to appear, specific to the liner type,
     from which parameters can be set.
  \item[spheres] Choose the technique for colouring the spheres that fill the liner volume. 
  \item[revolve around axis] After the liner cross-section has been created, choose this option
    to build a surface of revolution. At the same time a triangulation covering the surface
    will be created. This triangulation is used for the rapid determination of whether a 
    point is inside or outside the surface.
  \item[topology] this option will recompute the triangulation. Normally this option is
     not directly used unless a poor quality triangulation is generated from the step 
    \cmd{revolve around axis}.
  \item[fill with spheres] After the properties of the spherical particles have been 
    defined, choose this option to fill the liner with spheres.
  \item[lighting curve] edit the lighting curve which determines the direction of propagation
     of the particles, see section~\ref{sec:lighting}.
  \item[lighting time] edit the lighting time function which determines the time delay of the
    motion of the particles, see section~\ref{sec:lighting}.
  \item[lighting speed] edit the lighting speed functio which determines the speed of the 
    particles, see section~\ref{sec:lighting}.
  \item[rotate spheres to the z-axis] rotate the spherical particles so that the axis
    of revolution points along the z-axis. Use this option before outputting the sphere locations for
    use with some discrete simulation codes.
\end{description}

Here are descriptions of the toggle buttons
\begin{description}
  \item[plot curve] turn on or off the plotting the liner cross-section curve.
  \item[plot surface] turn on or off the plotting of the surface of revolution. 
  \item[plot shaded surface] turn on or off the plotting the shaded surface on the surface of revolution. When the
    shaded surface is turned off, only the grid lines on the surface will appear allowing
    one to see inside the surface.
  \item[plot spheres] turn on or off the plotting of the spheres (once they have been computed).
  \item[plot spheres as points] turn on or off the plotting of the spheres as points (it is quicker to draw the
     spheres if they are just shown as points).
  \item[plot lighting curve] turn on or off the plotting of the lighting curve.
  \item[plot velocity arrows] turn on or off the plotting of velocity arrows emanating from each sphere. The velocity
    arrows indicate the initial velocity vector of the spheres.
  \item[plot triangulation] turn on or off the plotting of the underlying triangulation that is associated with 
    the liner surface. 
  \item[fill with evenly spaced spheres] for testing purposes, fill the liner with 
     evenly spaced spheres.
\end{description}

Here are descriptions of the text fields
\begin{description}
  \item[name] Choose the name of the liner.
  \item[radius for spheres] Enter a list of $N$ different sphere radii, $R_1$, $R_2$, \ldots $R_N$. 
  \item[probability for spheres] Enter the $N$ probabilities,  $P_1$, $P_2$, \ldots $P_N$,
                for each of the radii specified in the 
    \cmd{radius for spheres} text box.
  \item[total volume fraction] Enter the total volume fraction (a real number between 0 and 1)
    of the liner volume
    that should be filled with spheres. As the volume fraction is increased it becomes
    more difficult to fill the volume with spheres.
  \item[RNG seed] choose a seed for the random number generator that is used when loading the
    volume with spheres.
\end{description}

\clearpage
\section{Example: a piecewise linear liner}\label{sec:exampleLinear}
{% ----
% .95: \newcommand{\clipfig}[1]{\psclip{\psframe[linewidth=2pt](1.8,2.85)(16.25,11.75)}\epsfig{#1}\endpsclip}
\newcommand{\clipfiga}[1]{\psclip{\psframe[linecolor=white](.25,.2)(8.5,8.75)}\epsfig{#1}\endpsclip}
\newcommand{\clipfig}[1]{\psclip{\psframe[linecolor=white](1.,.2)(8.5,8.75)}\epsfig{#1}\endpsclip}
\newcommand{\figWidth}{.5\linewidth}
\psset{xunit=1.0cm,yunit=1.0cm,runit=1.0cm}
\begin{figure}[htb]
\begin{center}
\begin{pspicture}(0,0)(17,16.9)
% \rput(7, 4.){\clipfig{file=linerWindows.ps,width=.95\linewidth}}
\rput( 4,13){\clipfiga{file=linerLinear-crossSection.ps,width=\figWidth}}
\rput(12,13){\clipfig{file=linerLinear-surface.ps,width=\figWidth}}
\rput( 4,4.5){\clipfig{file=linerLinear-triangulation.ps,width=\figWidth}}
\rput(12,4.5){\clipfig{file=linerLinear-spheres.ps,width=\figWidth}}
% turn on the grid for placement
% \psgrid[subgriddiv=2]
\rput(2,15){\psframebox*[fillstyle=solid,fillcolor=mediumgoldenrod]{\smallss liner cross section}}
\rput(14.5,15){\psframebox*[fillstyle=solid,fillcolor=mediumgoldenrod]{\smallss liner surface}}
\rput(2,7){\psframebox*[fillstyle=solid,fillcolor=mediumgoldenrod]{\smallss surface triangulation}}
\rput(14.5,7){\psframebox*[fillstyle=solid,fillcolor=mediumgoldenrod]{\smallss liner loaded with spheres}}

\end{pspicture}
\end{center}
\caption{Results from running the {\tt linerLinear.cmd} command file. Upper left: the liner cross-section
   is defined. Upper right: the surface of revolution for the liner. Lower left: the underlying triangulation
  of the surface. Lower right: the liner volume loaded with spheres, coloured by start time and with
direction arrows shown.}
\label{fig:linerLinear}
\end{figure}

  } % ----

In this example a linear liner is built and filled with spherical particles, see figure~\ref{fig:linerLinear}.
The command file for generating these results is given here (file {\tt linerLinear.cmd}):
{\footnotesize
\listinginput[1]{1}{/home/henshaw/Overture/rap/linerLinear.cmd}
}
Note that comments in the command file are denoted by a leading asterisk. 
Also note that a blank line will be treated as the end of the command file.
Thus, adding a blank line is one way to stop the program at a particular
point in the command file.

% The command file can be used to regenerate these results. 
This command file can be run using the command \cmd{rap linerLinear.cmd}. 
As the example runs the program will pause at various points (indicated
by the {\tt pause} statements in the command file). Choose {\tt continue} to proceed
with the next commands or choose {\tt break} to break out of the command file.

To run the command file in batch mode with no plotting and no pausing use \cmd{rap noplot nopause linerLinear.cmd}.

% To recreate this example interactively, start up the {\tt rap} program and then


% =======================================================================================================
\clearpage
\section{Example: a piecewise quadratic liner}\label{sec:exampleQuadratic}
{% ----
% .95: \newcommand{\clipfig}[1]{\psclip{\psframe[linewidth=2pt](1.8,2.85)(16.25,11.75)}\epsfig{#1}\endpsclip}
\newcommand{\clipfiga}[1]{\psclip{\psframe[linecolor=white](.25,.2)(8.5,8.75)}\epsfig{#1}\endpsclip}
\newcommand{\clipfig}[1]{\psclip{\psframe[linecolor=white](1.,.2)(8.5,8.75)}\epsfig{#1}\endpsclip}
\newcommand{\figWidth}{.5\linewidth}
\psset{xunit=1.0cm,yunit=1.0cm,runit=1.0cm}
\begin{figure}[htb]
\begin{center}
\begin{pspicture}(0,0)(17,16.9)
% \rput(7, 4.){\clipfig{file=linerWindows.ps,width=.95\linewidth}}
\rput( 4,13){\clipfiga{file=linerQuadratic-crossSection.ps,width=\figWidth}}
\rput(12,13){\clipfig{file=linerQuadratic-surface.ps,width=\figWidth}}
\rput( 4,4.5){\clipfig{file=linerQuadratic-triangulation.ps,width=\figWidth}}
\rput(12,4.5){\clipfig{file=linerQuadratic-spheres.ps,width=\figWidth}}
% turn on the grid for placement
% \psgrid[subgriddiv=2]
\rput(2,15){\psframebox*[fillstyle=solid,fillcolor=mediumgoldenrod]{\smallss liner cross section}}
\rput(14.5,15){\psframebox*[fillstyle=solid,fillcolor=mediumgoldenrod]{\smallss liner surface}}
\rput(2,7){\psframebox*[fillstyle=solid,fillcolor=mediumgoldenrod]{\smallss surface triangulation}}
\rput(14.5,7){\psframebox*[fillstyle=solid,fillcolor=mediumgoldenrod]{\smallss liner loaded with spheres}}

\end{pspicture}
\end{center}
\caption{Results from running the {\tt linerQuadratic.cmd} command file. Upper left: the liner cross-section
   is defined. Upper right: the surface of revolution for the liner. Lower left: the underlying triangulation
  of the surface. Lower right: the liner volume loaded with spheres, coloured by start time and with
direction arrows shown.}
\label{fig:linerQuadratic}
\end{figure}

  } % ----

In this example a piecewise quadratic liner is built and filled with spherical particles,
see figure~\ref{fig:linerQuadratic}.
The command file for generating these results is given here (file {\tt linerQuadratic.cmd}):
{\footnotesize
\listinginput[1]{1}{/home/henshaw/Overture/rap/linerQuadratic.cmd}
}

% The command file can be used to regenerate these results. 
This command file can be run using the command \cmd{rap linerQuadratic.cmd}. 
As the example runs the program will pause at various points (indicated
by the {\tt pause} statements in the command file). Choose {\tt continue} to proceed
with the next commands or choose {\tt break} to break out of the command file.

To run the command file in batch mode with no plotting and no pausing use \cmd{rap noplot nopause linerQuadratic.cmd}.

% =======================================================================================================
\clearpage
\section{Example: a free-form liner}\label{sec:exampleFreeForm}
{% ----
% .95: \newcommand{\clipfig}[1]{\psclip{\psframe[linewidth=2pt](1.8,2.85)(16.25,11.75)}\epsfig{#1}\endpsclip}
\newcommand{\clipfiga}[1]{\psclip{\psframe[linecolor=white](.25,.2)(8.5,8.75)}\epsfig{#1}\endpsclip}
\newcommand{\clipfig}[1]{\psclip{\psframe[linecolor=white](1.,.2)(8.5,8.75)}\epsfig{#1}\endpsclip}
\newcommand{\figWidth}{.5\linewidth}
\psset{xunit=1.0cm,yunit=1.0cm,runit=1.0cm}
\begin{figure}[htb]
\begin{center}
\begin{pspicture}(0,0)(17,16.9)
% \rput(7, 4.){\clipfig{file=linerWindows.ps,width=.95\linewidth}}
\rput( 4,13){\clipfiga{file=linerFreeForm-NurbsCurveEditor.ps,width=\figWidth}}
\rput(12,13){\clipfig{file=linerFreeForm-crossSection,width=\figWidth}}
\rput( 4,4.5){\clipfig{file=linerFreeForm-surface.ps,width=\figWidth}}
\rput(12,4.5){\clipfig{file=linerFreeForm-spheres.ps,width=\figWidth}}
% turn on the grid for placement
% \psgrid[subgriddiv=2]
\rput(2,15){\psframebox*[fillstyle=solid,fillcolor=mediumgoldenrod]{\smallss creating the free-form curve}}
\rput(10.5,15){\psframebox*[fillstyle=solid,fillcolor=mediumgoldenrod]{\smallss liner cross section}}
\rput(2,7){\psframebox*[fillstyle=solid,fillcolor=mediumgoldenrod]{\smallss liner surface}}
\rput(14.5,7){\psframebox*[fillstyle=solid,fillcolor=mediumgoldenrod]{\smallss liner loaded with spheres}}

\end{pspicture}
\end{center}
\caption{Results from running the {\tt linerFreeForm.cmd} command file. Upper left: the liner cross-section
   is created using the Nurbs curve builder. 
Upper right: the completed liner cross-section.
Lower left: the surface of revolution for the liner. 
Lower right: the liner volume loaded with spheres, coloured by start time.}
\label{fig:linerFreeForm}
\end{figure}

  } % ----

In this example a free-form liner is built and filled with spherical particles, see~\ref{fig:linerFreeForm}.
The command file for generating these results is given here (file {\tt linerFreeForm.cmd}):
{\footnotesize
\listinginput[1]{1}{/home/henshaw/Overture/rap/linerFreeForm.cmd}
}

% The command file can be used to regenerate these results. 
This command file can be run using the command \cmd{rap linerFreeForm.cmd}. 
As the example runs the program will pause at various points (indicated
by the {\tt pause} statements in the command file). Choose {\tt continue} to proceed
with the next commands or choose {\tt break} to break out of the command file.

To run the command file with no plotting and no pausing use \cmd{rap noplot nopause linerFreeForm.cmd}.

The free-form cross-section curve in this example is created using the {\em Nurbs Curve Builder},
see section~\ref{sec:freeForm} for further details on building the free form liner and
Appendix~\ref{app:NurbsCurveBuilder} for a description of the {\em Nurbs Curve Builder} commands.

% \clearpage
% \subsection{Defining the cross-section of the liner}


\clearpage
\section{Defining a piecewise-linear liner cross-section}\label{sec:linear}


% figure here of the liner liner defined by 4 pts

%  ------------------ linear liner --------------------
\begin{figure}
\begin{center}
\begin{pspicture}(-1,-1)(6,5)%
%
\pspolygon[linewidth=2pt,fillstyle=solid,fillcolor=mediumaquamarine](0,0)(1,0)(5,4)(4,4)
%\rput(5,2){\makebox(0,0)[l]{\smallss liner}}
%\psline[linewidth=1.pt]{->}(5,2.)(2.6,2)
%
\pscircle[fillstyle=solid,fillcolor=blue](0,0){4pt}
\rput(-.2,.3){\makebox(0,0)[r]{\smallss Point 1}}
\pscircle[fillstyle=solid,fillcolor=blue](4,4){4pt}
\rput(3.8,4){\makebox(0,0)[r]{\smallss Point 2}}
\pscircle[fillstyle=solid,fillcolor=blue](5,4){4pt}
\rput(5.2,4.0){\makebox(0,0)[l]{\smallss Point 3}}
\pscircle[fillstyle=solid,fillcolor=blue](1,0){4pt}
\rput(1.2,.3){\makebox(0,0)[l]{\smallss Point 4}}
%
%\rput(4,4){\qdisk(0,0){4pt}\makebox(0,0)[r]{\smallss Point 2}}
%\rput(5,4){\qdisk(0,0){4pt}\makebox(0,0)[l]{\smallss Point 3}}
%\rput(1,0){\qdisk(0,0){4pt}\makebox(0,0)[l]{\smallss Point 4}}
%
\psline[linewidth=2.pt]{->}(-1,0)(5,0)
\rput(2.,-.5){\makebox(0,0)[l]{\smallss axis of revolution}}
%
% turn on the grid for placement
% \psgrid[subgriddiv=2]%
\end{pspicture}
\end{center}
\caption{The cross-section of the piece-wise linear liner is defined by 4 points.}
\label{fig:linearLiner}
\end{figure}

The linear liner is defined by four points are shown in figure~\ref{fig:linearLiner}.
The coordinates of any of the four points can be altered. Points 1 and 4 should remain 
on the axis of revolution, $y=0$.

\clearpage
\section{Defining a piecewise-quadratic liner cross-section}\label{sec:quadratic}
%  ------------------ quadratic liner --------------------
\begin{figure}
\begin{center}
\begin{pspicture}(-1,-1)(6,5)%
%
\pscustom[linewidth=2pt,fillstyle=solid,fillcolor=mediumaquamarine]{%
  \pscurve(0,0)(2,3)(4,4)
  \pscurve[liftpen=1](5,3.7)(2,2)(1,0)}
% \psbezier[fillstyle=solid,fillcolor=mediumaquamarine](0,0)(2,3)(4,4)(5,4)
% \psbezier[fillstyle=solid,fillcolor=mediumaquamarine](0,0)(2,3)(4,4)(4,4)(5,4)(5,4)(2,1)(1,0)%
% \pscurve[fillstyle=solid,fillcolor=mediumaquamarine](1,0)(2,2)(5,4)
% \pspolygon[fillstyle=solid,fillcolor=mediumaquamarine](0,0)(2.5,3)(4,4)(5,4)(2.5,1)(1,0)
% \pspolygon[fillstyle=solid,fillcolor=mediumaquamarine](0,0)(1,0)(5,4)(4,4)
%\rput(5,2){\makebox(0,0)[l]{\smallss liner}}
%\psline[linewidth=1.pt]{->}(5,2.)(2.6,2)
%
\pscircle[fillstyle=solid,fillcolor=blue](0,0){4pt}
\rput(-.2,.3){\makebox(0,0)[r]{\smallss Point 1}}
%
\pscircle[fillstyle=solid,fillcolor=blue](2,3){4pt}
\rput(1.8,3.){\makebox(0,0)[r]{\smallss Point 2}}
%
\pscircle[fillstyle=solid,fillcolor=blue](4,4){4pt}
\rput(3.8,4){\makebox(0,0)[r]{\smallss Point 3}}
%
\pscircle[fillstyle=solid,fillcolor=blue](5,3.7){4pt}
\rput(5.2,4.0){\makebox(0,0)[l]{\smallss Point 6}}
%
\pscircle[fillstyle=solid,fillcolor=blue](2,2){4pt}
\rput(2.2,2.0){\makebox(0,0)[l]{\smallss Point 5}}
%
\pscircle[fillstyle=solid,fillcolor=blue](1,0){4pt}
\rput(1.2,.3){\makebox(0,0)[l]{\smallss Point 4}}
%
%\rput(4,4){\qdisk(0,0){4pt}\makebox(0,0)[r]{\smallss Point 2}}
%\rput(5,4){\qdisk(0,0){4pt}\makebox(0,0)[l]{\smallss Point 3}}
%\rput(1,0){\qdisk(0,0){4pt}\makebox(0,0)[l]{\smallss Point 4}}
%
\psline[linewidth=2.pt]{->}(-1,0)(5,0)
\rput(2.,-.5){\makebox(0,0)[l]{\smallss axis of revolution}}
%
% turn on the grid for placement
% \psgrid[subgriddiv=2]%
\end{pspicture}
\end{center}
\caption{The cross-section of the piece-wise quadratic liner is defined by 6 points.
A quadratic curve is interpolated through the points 1,2,3. Another quadratic passed
through the points 4,5,6.}
\label{fig:quadraticLiner}
\end{figure}

The piecewise quadratic liner is defined by 6 points as shown in figure~\ref{fig:quadraticLiner}.
A quadratic curve is interpolated through the points 1,2,3. Another quadratic passed 
through the points 4,5,6. Points 1 and 4 should remain on the axis of revolution, $y=0$.


\clearpage
\section{Defining a free-form liner cross-section}\label{sec:freeForm}
%  ------------------ free-form liner --------------------
{%
\begin{figure}
\begin{center}
\begin{pspicture}(-1,-.3)(6,5)%
%
\psset{xunit=1.5cm,yunit=1.5cm,runit=1.5cm}
\pscustom[linewidth=2pt,fillstyle=solid,fillcolor=mediumaquamarine]{%
%  \pscurve[linecolor=red](0,0)(1,2.25)(2,3)(4,3.5)
%  \pscurve[linecolor=green,liftpen=1](4.5,2.75)(2,2)(1,0)
  \pscurve[linecolor=red](0.1,0)(1,2.25)(2,3)(4,3.5)
  \pscurve[linecolor=green,liftpen=1](4.5,2.75)(3.5,2.6)(2.5,2.25)
  \pscurve[liftpen=1](2.5,2.25)(2.5,1.5)(1.5,1.25)
  \pscurve[liftpen=1](1.5,1.25)(1.25,.75)(1.1,0)
}
% \psbezier[fillstyle=solid,fillcolor=mediumaquamarine](0,0)(2,3)(4,4)(5,4)
% \psbezier[fillstyle=solid,fillcolor=mediumaquamarine](0,0)(2,3)(4,4)(4,4)(5,4)(5,4)(2,1)(1,0)%
% \pscurve[fillstyle=solid,fillcolor=mediumaquamarine](1,0)(2,2)(5,4)
% \pspolygon[fillstyle=solid,fillcolor=mediumaquamarine](0,0)(2.5,3)(4,4)(5,4)(2.5,1)(1,0)
% \pspolygon[fillstyle=solid,fillcolor=mediumaquamarine](0,0)(1,0)(5,4)(4,4)
%\rput(5,2){\makebox(0,0)[l]{\smallss liner}}
%\psline[linewidth=1.pt]{->}(5,2.)(2.6,2)
%
\pscircle[fillstyle=solid,fillcolor=blue](0.1,0){4pt}
\rput(-.2,.3){\makebox(0,0)[r]{\smallss Start point}}
%
\pscircle[fillstyle=solid,fillcolor=blue](1.1,0){4pt}
\rput(1.4,.3){\makebox(0,0)[l]{\smallss End point}}
%
%\rput(4,4){\qdisk(0,0){4pt}\makebox(0,0)[r]{\smallss Point 2}}
%\rput(5,4){\qdisk(0,0){4pt}\makebox(0,0)[l]{\smallss Point 3}}
%\rput(1,0){\qdisk(0,0){4pt}\makebox(0,0)[l]{\smallss Point 4}}
%
\psline[linewidth=2.pt]{->}(-1,0)(5,0)
\rput(2.,-.2){\makebox(0,0)[l]{\smallss axis of revolution}}
%
% turn on the grid for placement
% \psgrid[subgriddiv=2]%
\end{pspicture}
\end{center}
\caption{The free-form liner cross-section is defined by a set of sub-curves that are joined
together to form a piecewise smooth curve.}
\label{fig:freeFormLiner}
\end{figure}
}


The free-form liner makes use of a {\em Nurbs Curve Builder} to build a free form curve.
A NURBS is a generalized type of spline (non-uniform rational B-spline).
The free form curve is a built from a set of smooth sub-curves that are joined together.
The global curve is allowed to have corners where the sub-curves meet. 

The basic steps in building a free form curve are (see the example in section~\ref{sec:exampleFreeForm})
\begin{itemize}
  \item interactively choose (or manually input) a collection of points in the two-dimensional plane
        that will be used to form the sub-curves. If you have a lot of points to input you may want
        to manually edit one of the example command files.
  \item select a sub-set of the points to form a smooth sub-curve. The sub-curve will be created
        by interpolating the chosen points. Repeat this process for each sub-curve.
  \item Choose {\em assemble} to join the sub-curves into a single piecewise smooth curve.
        Note: even if only one sub-curve is built it is necessary to choose {\em assemble}.
\end{itemize}

Note that any points created on the axis of revolution should be placed exactly at $y=0$. It may be
necessary to manually enter these points or to edit and change the values in the command file
if the point was created interactively.

For further information on the {\em Nurbs Curve Builder} see the Appendix~\ref{app:NurbsCurveBuilder}.

\clearpage
\section{Building the volume of revolution from the cross-section curve}

The \cmd{revolve around axis} button on the {\em Liner Builder Dialog}
is used to build a surface of revolution from the liner cross-section
curve.  At the same time a triangulation covering the surface will be
created. This triangulation is used for the rapid determination of
whether a point is inside or outside the surface.

Geometric properties of the linear are computed from the triangulation.
These properties are printed to the screen and include the enclosed
volume, center of mass and the moments of inertia matrix (computed
assuming a uniform density throughout the liner). 

The triangulation should be automatically generated although sometimes errors may occur.
Errors in generating the triangulation most commonly occur when the ends of the 
cross-section curve do not exactly touch the axis of revolution.


\section{Loading a liner with spheres}

The current specification for loading the liner volume with spheres is just an initial
attempt. A more sophisticated approach is envisioned for the future.

Currently the basic steps to fill the liner volume with spheres are
\begin{enumerate}
  \item specify a list of $N$ values, $R_i$,  indicating the radii of the spheres. 
  \item specify a list of $N$ values, $P_i$, indicating the probability that a sphere will radius $R_i$
            will appear.
  \item Specify the fraction of the volume that should be occupied by the spheres.
\end{enumerate}

Choosing the \cmd{fill liner spheres} button will cause the sphere loading algorithm to be 
called. The current algorithm is rather primitive and can be slow even for moderate volume
fractions. An improved algorithm is under development.



\clearpage
\section{Lighting curves - specifying the velocity of each particle}\label{sec:lighting}

{
%  ------------------ lighting curves --------------------
\psset{xunit=1.5cm,yunit=1.5cm,runit=1.5cm}
\begin{figure}
\begin{center}
\begin{pspicture}(-1.5,0)(6,5)%
%
\pscustom[linewidth=1.5pt,fillstyle=solid,fillcolor=lightgray]{%
  \pscurve(0,0)(1,2.25)(2,3)(4,3.5)%
  \pscurve[liftpen=1](4.5,2.75)(2,2)(1,0)}%
%
\psline[linewidth=2.5pt,linecolor=blue]{-}(.5,0)(3.5,4.5)%
\pscircle[fillstyle=solid,fillcolor=blue](.5,0){4pt}%
\rput(.5,-.35){\makebox(0,0)[c]{\largess {\bfss C}(0)}}%
\pscircle[fillstyle=solid,fillcolor=blue](3.5,4.5){4pt}%
\rput(3.5,4.85){\makebox(0,0)[c]{\largess {\bfss C}(1)}}%
%
\pscircle[fillstyle=solid,fillcolor=red](3.5,3){7pt}%
% \psline[linewidth=2.pt]{-}(.5,0)(3.5,4.5)%
%
\psline[linewidth=1.5pt,linecolor=green]{-}(2,4)(5,2)%
% \psline[linewidth=1pt,linecolor=black]{-}(2,4)(5,2)%
%
\rput(1.75,3.5){\makebox(0,0)[r]{\largess {\bfss C}(s)}}%
\psline[linewidth=1.5pt]{->}(1.8,3.5)(2.7,3.48)%
\pscircle[fillstyle=solid,fillcolor=black](2.79,3.48){4pt}%
%
\rput(4.6,2.5){\makebox(0,0)[l]{\Largess\red {\bfss v}=U(s) {\bfss n}(s)}}%
\rput(.01,.01){\psline[linewidth=2.5pt,linecolor=red]{->}(3.5,3)(4.4,2.4)}
%
% mark the right angle: (2.75,3.5) --> (2.45,3.7)
\psline[linewidth=1.5pt]{-}(2.45,3.7)(2.65,4.0)
\psline[linewidth=1.5pt]{-}(2.65,4.0)(2.95,3.8)
%
\psline[linewidth=2.pt]{->}(-1,0)(5,0)
% \rput(2.,-.5){\makebox(0,0)[l]{\smallss axis of revolution}}
%
\rput(2.7,.75){\psframebox*[fillstyle=solid,fillcolor=mediumgoldenrod]{\largess lighting curve}}
\psline[linewidth=1.5pt]{->}(1.75,.75)(1,.75)
%
\rput(-.8,2){\psframebox*[fillstyle=solid,fillcolor=mediumgoldenrod]{\largess liner}}
\psline[linewidth=1.5pt]{->}(-.5,2.)(1,1.5)
%
\rput(5.25,4){\psframebox*[fillstyle=solid,fillcolor=mediumgoldenrod]{\largess particle}}
\psline[linewidth=1.5pt]{->}(5.2,3.8)(3.5,3)
%
% turn on the grid for placement
% \psgrid[subgriddiv=2]%
\end{pspicture}
\end{center}
\caption{The lighting curve is $\xv=\Cv(s)$, $s\in[0,1]$. The lighting speed is $U(s)$.
   Any particle which lies along the normal, $\nv(s)$  to the point $\Cv(s)$ is given a 
speed $U(s)$ and direction equal to the normal to $\Cv(s)$.}
\label{fig:lightingCurve}
\end{figure}
% -------------------- end lighting curves ----------------
}


Lighting curves are used to assign velocities to the particles in the liner. 
These curves are used to model the effect of the actual detonation hitting the liner material.
There are three lighting functions, the {\em lighting curve} $\Cv(s)$, the {\em lighting speed} $U(s)$
and the {\em lighting time} $T(s)$. 

Figure~\ref{fig:lightingCurve} shows how the lighting curves are used to assign
a velocity to the particles in the liner volume. Let $\xv=\Cv(s)$, $s\in[0,1]$,
denote the lighting curve. Let $U(s)$ , $s\in[0,1]$, denote the lighting speed.

To determine the velocity of a particle at position $\pv$, first determine
the value of $s$ such that the line normal to the lighting curve at the point $\Cv(s)$
intersects the point $\pv$. 
Denoting the normal to $\Cv(s)$ by $\nv(s)$, then the velocity
for the particle at the point $\pv$ is $\uv=U(s)\nv(s)$. 


An additional {\em lighting time} curve, $T(s)$, defines when each particle begins to move.
For the particle at position $\pv$, the velocity would be zero for $t<T(s)$ and would equal
$U(s)\nv(s)$ for $t>T(s)$.


The lighting curves $\Cv(s)$, $U(s)$ and $T(s)$ are given default values. Each of these
curves can be edited by choosing one of the buttons labeled \cmd{lighting curve}, \cmd{lighting speed},
or \cmd{lighting time}. The curve editor is the 
the {\em Nurbs Curve Builder}, the same that is used to define a free form curve (section~\ref{sec:freeForm}).
The {\em Nurbs Curve Builder} is described in Appendix~\ref{app:NurbsCurveBuilder}.


{\bf NOTE:} The lighting speed, $U(s)$ and the lighting curve $T(s)$
are really scalar valued functions of the parameter s; In other words,
 $U$ is a mapping from the unit interval onto the real numbers, $U : [0,1] \rightarrow
\Real$. However, to make it convenient to graphically edit these
functions, these two lighting functions are represented as
parameterized curves.  Thus $U(s)$ becomes the curve
$\Uv(\xi)=(x_u(\xi),y_u(\xi))$ for $\xi \in [0,1]$, while $T(s)$
becomes $\Tv(\eta)=(x_t(\eta),y_t(\eta))$ for $\eta \in [0,1]$.  Of
course it only makes sense that the parameterized curves $\Uv(\xi)$
and $\Tv(\eta)$ be singled values functions of $\xi$ and $\eta$.
Given a point $\Cv(s)$, in order to compute $U(s)$ is is necessary to relate $s$ to the
parameter value $\xi$ of the curve $\Uv(\xi)=(x_u(\xi),y_u(\xi))$.
This is done by finding $\xi$ such that $x_u(\xi)=s$ and then $U(s)=y_u(\xi)$. In other words,
the ``$x$'' coordinate of the parameterized curves corresponds to $s$.
Written compactly this
becomes $U(s)=y_u(x_u^{-1}(s))$.
Similarly the value of the lighting time equals $T(s) = y_t(x_t^{-1}(s))$.

In summary here is the algorithm for specifying the velocity of a particle that is located at the
point $\pv$:
\begin{enumerate}
 \item Find $s,\alpha$ such that $\Cv(s)+\alpha\nv(s)=\pv$. Here 
$\Cv(s)$ should be the closest point on the curve to $\pv$ while $\nv(s)$ is the normal to
the curve at $\Cv(s)$. 
  \item Compute the speed $U(s) = y_u(x_u^{-1}(s))$ where $\Uv(\xi)=(x_u(\xi),y_u(\xi))$
  is the parameterized lighting speed curve.
  \item Compute the speed $T(s) = y_t(x_t^{-1}(s))$ where $\Tv(\eta)=(x_u(\eta),y_u(\eta))$
  is the parameterized lighting time curve.
\end{enumerate}



% --------------- start the appendix ----------------------------------
\appendix
\clearpage
\section{Appendix: The Nurbs Curve Builder}\label{app:NurbsCurveBuilder}
\input NurbsCurveBuilder.tex




\clearpage
% \section{Appendix: Mouse and button commands}
% \section{Mouse, Button and View Characteristics of the Graphics Window}
\section{Appendix: Features of the Overture Graphical User Interface}
\label{app:mouseButtons}
\input mouseAndButtons.tex


\vfill\eject
\bibliography{/home/henshaw/papers/henshaw}
\bibliographystyle{siam}

\printindex

\end{document}
