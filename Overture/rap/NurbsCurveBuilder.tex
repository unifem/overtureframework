% documentation for the Nurbs curve builder


{% ----
\newcommand{\clipfig}[1]{\psclip{\psframe[linecolor=white](2.2,6.5)(19.2,16.9)}\epsfig{#1}\endpsclip}
\psset{xunit=1.0cm,yunit=1.0cm,runit=1.0cm}
\begin{figure}[ht]
\begin{center}
\begin{pspicture}(0,0)(17,11)
\rput(11,4){\clipfig{file=nurbsCurveBuilder.ps,width=1.5\linewidth}}
% turn on the grid for placement
% \psgrid[subgriddiv=2]
\rput(11,-.3){\psframebox*[fillstyle=solid,fillcolor=mediumgoldenrod]{\normalss Nurbs Curve Builder dialog}}
\psline[linewidth=1.5pt]{->}(11,0)(11,2.)
\rput(3,-.3){\psframebox*[fillstyle=solid,fillcolor=mediumgoldenrod]{\normalss main screen}}
\psline[linewidth=1.5pt]{->}(3,0)(3,2.)
%
\rput(2.,8){\psframebox*[fillstyle=solid,fillcolor=mediumgoldenrod]{\smallss sub-curve}}
\psline[linewidth=1.pt]{->}(2,7.75)(3.5,6.25)
% 
\rput(4.,8.5){\psframebox*[fillstyle=solid,fillcolor=mediumgoldenrod]{\smallss point}}
\psline[linewidth=1.pt]{->}(4,8.25)(5.15,7.65)
%
\end{pspicture}
\end{center}
\caption{Creating sub-curves with the Nurbs Curve Builder. The sub-curves can be joined to form a 
piecewise smooth curve. }
\label{fig:nurbsCurveBuilder}
\end{figure}
  } % ----


The {\em Nurbs Curve Builder} can be used to create and modify two-dimensional curves. The curves are
represented as a NURBS (non-uniform rational B-spline) which is a generalized type of spline.
The {\em Nurbs Curve Builder} dialog is shown in figure~\ref{fig:nurbsCurveBuilder}.
The Curve Builder allows one to create a curve consisting of a set of smooth sub-curves that are joined together.
The assembled curve is allowed to have corners where the sub-curves meet. 
Note that there are two types of curves plotted in the graphics window, the sub-curves and the
assembled curve. Plotting of the assembled curve is controlled by the 
\cmd{Current Curve} toggle button. 

The basic steps to build a curve are to first create a set of points and then choose a 
subset of these points to interpolate a sub-curve. Sub-curves are {\tt assembled} into
a single curve.
\begin{description}
  \item[Build Points] To build points select the \cmd{ Build Point} mouse mode toggle. Points may now
    be chosen interactively by clicking with the mouse. Alternatively, the coordinates of a 
    point can be typed into the text box, \cmd{New point (x,y)}, on the dialog window.
    It may be necessary to first set the {\em plot bounds} (see below) so that the plotting region is the correct size.
  \item[Interpolate Curve] To create a curve from points select the \cmd{Interpolate Curve} mouse
         mode toggle. Points can be selected with the mouse. Choose {\tt done} (from the lower
         left corner of the main window) when a sub-curve is completed.
  \item[Assemble] Choose {\tt Assemble} to join the set of sub-curves (where possible) into a
         single curve. Assemble should be chosen even if there is only one sub-curve.
\end{description}

Additional mouse modes are available including
\begin{description}
  \item[Query Point] With this mode on, selecting a point will cause the point label and coordinates
     to be printed.
  \item[Circular Arc] build a circular arc from two points and a radius of curvature 
      (choose a negative value for the radius of curvature 
      to form the other possible arc that passes through the two points.)
  \item[Hide SubCurve] hide a sub-curve (the hidden sub-curve will not be considered during the
      assemble step).
  \item[Move Curve Endpoint] move the endpoint of one sub-curve to match the endpoint of another
         curve. First select a curve endpoint and then select a point near the end of another curve.
  \item[Join W/Line Segment]  Join the endpoints of two sub-curves with a line segment by selecting
           two points near the ends of the cub-curves.
  \item[Snap to Intersection] If two sub-curves intersect, \cmd{Snap to Intersection} will 
             join the sub-curves at the point of intersection and discard the extra segments. Select each sub-curve
             by clicking a point that lies on the portion of the sub-curve that should be retained.
  \item[Split] split an existing curve into two pieces by selecting a point on the curve.
  \item[Edit SubCurve] Enter the Nurbs editor to edit the properties of a sub-curve.
\end{description}

Here are descriptions of the buttons that appear in the {\em Nurbs Curve Builder} 
dialog,
\begin{description}
  \item[Clear Last Point] Remove the last point that was selected (when selecting points for \cmd{Interpolate Curve},
      for example.)
  \item[Clear All Points] Remove all selected points (when selecting points for \cmd{Interpolate Curve},
      for example).
  \item[Assemble] Join sub-curves (where possible) into an assembled curve (the current curve).
  \item[Show All] Show all sub-curves and the assembled curve.
  \item[Show Used] Only show sub-curves that are used in the assembled curve.
  \item[Hide all] Hide all sub-curves.
  \item[Hide Unused] Hide unused sub-curves.
  \item[Show Last Hidden] Show the last hidden sub-curve.
\end{description}

The toggle buttons are
\begin{description}
  \item[Current Curve] plot the assembled curve if it has been created.
  \item[SubCurves] plot the sub-curves.
\end{description}

The text strings that appear in the {\em Nurbs Curve Builder} are
\begin{description}
  \item[New point (x,y)] build a new point by typing the $x$ and $y$ coordinates.
  \item[plot bounds] Enter new plot bounds as $x_{\rm min}$, $x_{\rm max}$, $y_{\rm min}$, $y_{\rm max}$.
     Points can only be built interactively if the new coordinates lie inside the current plot bounds.
     Thus it may be necessary to increase the plot bounds before building new points.
\end{description}

% 
% Geometry used as the bounding curves of mesh regions are created using
% the interface shown in Figure~\ref{fig:geom}.  Points can be placed
% either by interactively clicking with the mouse (after selecting the
% ``Build Point'' mouse mode toggle) or by inputting the coordinates in
% the text box at the bottom.  Once at least two points are placed, a
% curve can be created by changing the mouse mode to ``Interpolate
% Curve'' and then picking a sequence of points with the mouse.  After
% the last point has been chosen, clicking ``Done'' at the bottom of the
% graphics window creates a spline curve that interpolates the selected
% points.  At this point, the curve could be edited using mouse modes
% such as ``Split Curve''; new curves can be created; or the user can
% exit the geometry creation interface.