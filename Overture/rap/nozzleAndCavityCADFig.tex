{
\begin{figure}[hbt]
\newcommand{\figWidtha}{8cm}
\newcommand{\trimfig}[2]{\trimFigb{#1}{#2}{0}{.0}{.1}{.1}}
\begin{center}
 \begin{tikzpicture}[scale=1]
 \useasboundingbox (0,.75) rectangle (16,7.5);  % set the bounding box (so we have less surrounding white space)
%
 \draw (0,0.0) node[anchor=south west] {\trimfig{nozzleAndCavityCAD}{\figWidtha}};
 \draw (8,0.0) node[anchor=south west] {\trimfig{nozzleAndCavityTopo}{\figWidtha}};
%
%\draw[step=1cm,gray] (0,0) grid (16,7);
%  \draw (current bounding box.south west) rectangle (current bounding box.north east);
 \end{tikzpicture}
\end{center}
\caption{Nozzle and Cavity. Left: CAD geometry represented as a CompositeSurface Mapping. 
  Right: water-tight triangulation generated
  by the Overture topology routine.}
\label{fig:nozzleAndCavityCAD}
\end{figure}
}

%{
%\newcommand{\figWidth}{7.cm}
%\newcommand{\clipfig}[2]{\clipFigb{#1}{#2}{.0}{1.}{.13}{.9}}
%\begin{figure}[hbt]
% \begin{center}
% \begin{pspicture}(0,0)(14,4.25)
%  \rput(3. ,2.0){\clipfig{nozzleAndCavityCAD.ps}{\figWidth}}
%  \rput(11.,2.0){\clipfig{nozzleAndCavityTopo.ps}{\figWidth}}
%% \psgrid[subgriddiv=2]
%\end{pspicture}
%\end{center}
%\caption{Nozzle and Cavity. Left: CAD geometry represented as a CompositeSurface Mapping. 
%  Right: water-tight triangulation generated
%  by the Overture topology routine.}
%\label{fig:nozzleAndCavityCAD}
%\end{figure}
%% 
%}
