\clearpage
\section{Grids for a car rear view mirror (from an IGES file)}\label{sec:retro}

\input retroFig

Figure~\ref{fig:retroCAD} shows a CAD geometry for the rear view mirror of an
automobile. We will use this example to illustrate CAD fixup and grid generation.

The CAD geometry for the rear view mirror has been provided courtesy of PSA Peugeot Citroen.
Also thanks to Mr. Mehdi Bordji who generated some of the grids shown in this section.

\noindent {\bf Getting started:} The first step is to read in the IGES file and look at all the
surfaces:
\begin{verbatim}
 # the next command is needed by ogen but optional for mbuilder
 create mappings
   read iges file
     myFile.igs
   continue
   choose all
\end{verbatim}
Here we assume you are running ogen or mbuilder.

\noindent {\bf Choosing a set of surfaces to work on:} 
It can often be helpful to only look at a sub-set of the surfaces when getting
started or when building a grid for one part of a large geometry. To choose
a subset of surfaces one can
\begin{enumerate}
  \item read in all surfaces (as above), 
  \item hide surfaces that are not wanted (using the mouse to pick individual or collections of surfaces),
  \item delete hidden surfaces, (it is safer to hide first and then delete rather than delete to begin with),
  \item "print parameters" will give a list of the remaining surfaces that can be 
        read in as a list as shown below
\end{enumerate}

\noindent {\bf Reading in a list of surfaces:} A list of surfaces can be read as follows,
\begin{verbatim}
 create mappings
   read iges file
     myFile.igs
   continue
choose a list
  6 7 8 20 22 70
done
\end{verbatim}

\noindent {\bf Fixing broken surfaces:}
The command {\tt show broken surfaces} will print a list of any broken surfaces.
To fix a broken surface, such as surface 5, type {\tt examine a sub-surface 5}.
See section~\ref{sec:fixingTrimCurves} for examples of fixing trimming curves. 

% 
% create mappings
%   read iges file
%     /home/henshaw.0/people/lafon/mehdi/retro_and_vein.igs
%     continue
% #
% # STEP I: choose a subset of surfaces to work on (these commands are commented out below)
% #         1. hide surfaces that are not wanted
% #         2. delete hidden surfaces 
% #         3. "print parameters" will give a list of the remaining surfaces that can be 
% #           read in as a list as shown below



\clearpage
\section{Building grids on triangulated surfaces}\label{sec:retroTriangles}

\input retroTrianglesFig

mBuilder can also be used to build grids directly on a triangulated surface. The triangulated
surface can be defined in a number of ways such as an AVS or STL file.
Figure~\ref{fig:retroGridsAVS} shows a triangulated surface read from an AVS file and
represented as an UnstructuredMapping, 
\begin{verbatim}
create mappings
  unstructured 
    read avs file 
    myFile.avs 
  exit
\end{verbatim}
Given a surface defined by a triangulation,
the next step is to build structured surface grids on the surface. The hyperbolic
grid generator is one way to build such grids. 
Figure~\ref{fig:retroGridsAVS} shows overlapping grids that were built on a surface.
\begin{verbatim}
  builder 
    create surface grid...
\end{verbatim}
To define a surface grid one must create a starting curve. This can be done by
using the mouse to pick a set of points on the surface to define a curve.
After interactively choosing points, the commands in the command file will look like
{\footnotesize
\begin{verbatim}
 initial curve:points on surface
   choose point on surface 0 2.277030e+01 -7.074806e+01 1.534555e+02 9.171783e-02 1.666090e-01
   choose point on surface 0 1.400569e+01 -6.563512e+01 1.329498e+02 6.320195e-01 3.748991e-02
   choose point on surface 0 5.563543e+00 -6.254250e+01 1.132309e+02 4.987929e-02 7.274973e-01
     ...
 done
\end{verbatim}
}
The first three numbers are the $x,y,z$ coordinates of the point. The last two numbers are
related to the parameter space coordinates of the point.
After creating the initial curve one can optionally edit the Mapping that
defines the curve. In the next example we indicate that the initial curve is
periodic:
\begin{verbatim}
 edit initial curve
   periodicity
   2
 exit
\end{verbatim}
