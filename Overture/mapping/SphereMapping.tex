\section{SphereMapping}
\index{sphere mapping}\index{Mapping!SphereMapping}

This mapping defines a spherical shell or spherical surface in three-dimensions, 
\begin{eqnarray*}
\phi &=&  \pi( \phi_0 + r_1( \phi_1-\phi_0) ) \\
\theta &=&  2\pi( \theta_0 + r_2( \theta_1-\theta_0) ) \\
R(r_3) &=& (R_0 + r_3 (R_1-R_0)) \\
\xv(r_1,r_2,r_3) &=& ( R\cos( \theta )\sin(\phi) + x_0 , 
                       R\sin(\theta)\sin(\phi) + y_0 ,
                       R\cos(\phi)+z_0 )
\end{eqnarray*}


This mapping can be inverted analytically with the inverse defined by
\begin{align*}
    r &:= \sqrt{ (x-x_0)^2 + (y-y_0)^2 + (z-z_0)^2 } \\
  r_1 &= [ \cos^{-1}( (z-z_0)/r ) - \pi \phi_0 ] /( \pi(\phi_1-\phi_0)) \\
  r_2 &= [ {\rm atan2}( y_0-y, x_0-x ) + \pi -2\pi\theta_0 ]/( 2\pi(\theta_1-\theta_0)) \\
  r_3 &= (r-R_0)/(R_1-R_0)
\end{align*}

This mapping can have a spherical polar singularity at one or both ends. Either singularity
can be removed by creating an orthographic patch over the pole using the {\tt Reparameterization}
transform. In order to do this we must be able to evaluate the derivatives of the 
{\tt SphereMapping} and its inverse in spherical coordinates.
This means we compute the derivatives of the mapping as
\[
   \left({\partial x_i \over\partial r_1},{1\over\sin(\phi)}{\partial x_i \over\partial r_2},
         {\partial x_i \over\partial r_3} \right)
\]
and the derivatives of the inverse mapping as
\[
   \left({\partial r_1\over\partial x_i},\sin(\phi){\partial r_2 \over\partial x_i},
         {\partial r_3\over\partial x_i} \right).
\]


\subsection{Examples}

\noindent
\begin{minipage}{.4\linewidth}
{\footnotesize
\listinginput[1]{1}{\mapping/sphere.cmd}
}
\end{minipage}\hfill
\begin{minipage}{.6\linewidth}
  \begin{center}
   \includegraphics[width=9cm]{\figures/sphere} \\
   % \epsfig{file=\figures/sphere.ps,height=4.in}  \\
  {A spherical shell built with the {\tt SphereMapping}.}
  \end{center}
\end{minipage}

\noindent
\begin{minipage}{.4\linewidth}
{\footnotesize
\listinginput[1]{1}{\mapping/sphere2.cmd}
}
\end{minipage}\hfill
\begin{minipage}{.6\linewidth}
  \begin{center}
   \includegraphics[width=9cm]{\figures/sphere2} \\
   % \epsfig{file=\figures/sphere2.ps,height=4.in}  \\
  {A partial spherical shell built with the {\tt SphereMapping}.}
  \end{center}
\end{minipage}


%% \input SphereMappingInclude.tex


