\subsection{Constructor}
 
\newlength{\RevolutionMappingIncludeArgIndent}
\begin{flushleft} \textbf{%
\settowidth{\RevolutionMappingIncludeArgIndent}{RevolutionMapping(}% 
RevolutionMapping() 
}\end{flushleft}
\begin{description}
\item[{\bf Purpose:}]  Default Constructor
\end{description}
\subsection{Constructor}
 
\begin{flushleft} \textbf{%
\settowidth{\RevolutionMappingIncludeArgIndent}{RevolutionMapping(}% 
RevolutionMapping(Mapping \& revolutionary\_, \\ 
\hspace{\RevolutionMappingIncludeArgIndent}const real startAngle\_  =0., \\ 
\hspace{\RevolutionMappingIncludeArgIndent}const real endAngle\_  =1.,\\ 
const RealArray \& lineOrigin\_  =Overture::nullRealDistributedArray(),\\ 
const RealArray \& lineTangent\_  =Overture::nullRealDistributedArray()
\hspace{\RevolutionMappingIncludeArgIndent})
}\end{flushleft}
\begin{description}
\item[{\bf Purpose:}]  This constructor takes a mapping to revolve plus option parameters
\item[{\bf revolutionary\_ (input) :}]  mapping to revolve.
\item[{\bf startAngle\_ (input) :}]  starting "angle" (in [0,1]) for the reolution.
\item[{\bf endAngle\_ (input) :}]  ending "angle" (in [0,1]) for the revolution.
\item[{\bf lineOrigin\_ (input) :}]  the point of origin for the line of revolution.
\item[{\bf lineTangent\_ (input) :}]  the tangent to the line of revolution.
\end{description}
\subsection{setRevolutionAngle}
 
\begin{flushleft} \textbf{%
int   \\ 
\settowidth{\RevolutionMappingIncludeArgIndent}{setRevolutionAngle(}%
setRevolutionAngle(const real startAngle\_  =0., \\ 
\hspace{\RevolutionMappingIncludeArgIndent}const real endAngle\_  =1.)
}\end{flushleft}
\begin{description}
\item[{\bf Purpose:}]  Define the angle through which the revolution progresses.
\item[{\bf startAngle\_ (input) :}]  starting "angle" (in [0,1]) for the revolution.
\item[{\bf endAngle\_ (input) :}]  ending "angle" (in [0,1]) for the revolution.
\end{description}
\subsection{getRevolutionAngle}
 
\begin{flushleft} \textbf{%
int   \\ 
\settowidth{\RevolutionMappingIncludeArgIndent}{getRevolutionAngle(}%
getRevolutionAngle( real \& startAngle\_, \\ 
\hspace{\RevolutionMappingIncludeArgIndent}real \& endAngle\_  )
}\end{flushleft}
\begin{description}
\item[{\bf Purpose:}]  Get the bounding angles in the revolution progresses.
\item[{\bf startAngle\_ (input) :}]  starting "angle" for the revolution.
\item[{\bf endAngle\_ (input) :}]  ending "angle" for the revolution.
\end{description}
\subsection{setParameterAxes}
 
\begin{flushleft} \textbf{%
int   \\ 
\settowidth{\RevolutionMappingIncludeArgIndent}{setParameterAxes(}%
setParameterAxes( const int \& revAxis1\_, const int \& revAxis2\_, const int \& revAxis3\_ )
}\end{flushleft}
\begin{description}
\item[{\bf Purpose:}]  Define the parameter axes the mapping. The 2D mapping will
   be evaluated with {\tt (r(I,revAxis1),\-r(I,revAxis2))} while {\tt r(I,revAxis3)}
   will correspond to the angle of revolution $\theta$.
   The choice of these variables is normally only important if the body of revolution
   has a spherical polar singularity at one or both ends and the user wants to remove
   the singularity using the orthographic projection.(reparameterization option). The
   orthographic project expects the mapping to parameterized like a sphere with
   the parameters in the order $(\phi,\theta,r)$. 

  \begin{description}
    \item[revAxis1] The axis corresponding to $\phi$ in a spherical coordinate systems 
         or the axial variable $s$ in cylindrical  coordinates. {\tt revAxis1} will
         normally be $0$ (or $1$) and correspond to the axial like variable in the 2D mapping
         that is being revolved.
    \item[revAxis2] The axis corresponding to $r$ in a spherical coordinate system. Normally
          {revAxis2=2} so the axial variable appears last.
    \item[revAxis3] The axis corresponding to $\theta$ in a spherical coordinate system. Normally
        {revAxis3=1}.
  \end{description}
   
\item[{\bf revAxis1\_,revAxis2\_,revAxis3\_ (input) :}]  A permutation of (0,1,2).
\end{description}
\subsection{setRevolutionary}
 
\begin{flushleft} \textbf{%
int   \\ 
\settowidth{\RevolutionMappingIncludeArgIndent}{setRevolutionary(}%
setRevolutionary(Mapping \& revolutionary\_)
}\end{flushleft}
\begin{description}
\item[{\bf Purpose:}]  Define the mapping that will be revolved.
\item[{\bf revolutionary\_input) :}]  mapping to revolve.
\end{description}
\subsection{setLineOfRevolution}
 
\begin{flushleft} \textbf{%
int   \\ 
\settowidth{\RevolutionMappingIncludeArgIndent}{setLineOfRevolution(}%
setLineOfRevolution(const RealArray \& lineOrigin\_,\\ 
\hspace{\RevolutionMappingIncludeArgIndent}const RealArray \& lineTangent\_ )
}\end{flushleft}
\begin{description}
\item[{\bf Purpose:}]  Define the point of origin and the tangent of the line of revolution. 
 *For now this point and line must
    lie in the x-y plane (lineOrigin\_(2)==0, lineTangent\_(2)==0)
\item[{\bf lineOrigin\_ (input) :}]  the point of origin for the line of revolution. For
    now we require lineOrigin\_(2)==0.
\item[{\bf lineTangent\_ (input) :}]  the tangent to the line of revolution with lineTangent\_(2)==0
\end{description}
