
\section{EllipticGridGenerator}


This class can be used to smooth and adapt an existing grid using elliptic
grid generation. This class is NOT a mapping. It is used by the {\tt EllipticTransform}
and {\tt HyperbolicMapping} classes.



The basic elliptic grid generation equations are
\begin{align*}
  \Delta_{\xv} \rv & = \Qv    \\
      Q_m  & = { g_{mm} \over g } P_m \\
    g_{mm} & = \xv_{r_m}\cdot\xv_{r_m} \\
    g &= | g_{mn} |
\end{align*}
where $\Pv$ are the control functions. These equations are transformed into
\begin{align*}
   \sum_{mn}  g^{mn} \xv_{r_m r_n} + \sum_m g^{mm} P_m \xv_{r_m} &= 0 
\end{align*}
where
\begin{align*}
  g^{mp} & = \rv_{x_m}\cdot\rv_{x_p} \\
         & = {( g_{nq} g_{lr} - g_{nr} g_{lq} ) \over g}  \qquad\mbox{(mnl) and (pqr) cyclic}
\end{align*}
or more exlicitly in two dimensions:
\begin{equation}
 g_{11} (\xv_{r_1 r_1} + P_1 \xv_{r_1}) +g_{22} (\xv_{r_2 r_2} + P_2 \xv_{r_2}) + 2 g_{12} \xv_{r_1 r_2}=0
        \label{eq:ell1}
\end{equation}

\subsection{Surface grid generation}

  There are a number of ways to formulate the problem of generating an elliptic grid on a surface in 3D.
For simplicity (and reuse of code) we treat the surface grid generation problem as a special case
of generation a 3D volume grid, constrained to lie on the original reference surface $\Sv(\xv)=0$,
\begin{align*}
   \sum_{mn}  g^{mn} \xv_{r_m r_n} + \sum_m g^{mm} P_m \xv_{r_m} &= 0  \\
    \Sv(\xv) &=0 
\end{align*}

Here are the basic steps
\begin{enumerate}
  \item First compute the residual to the unconstrained elliptic grid generation equations.
  \item project the residual onto the local tangent plane by removing the component of the
residual in the diection of the local normal to the surface, $\nv$,
\[
      \rv \leftarrow \rv - (\rv\cdot\nv) \nv
\]
   \item solve for the new positions of the grid points, $\xv^n$. These new points may no longer lie
     on the reference surface.
   \item  Project the points $\xv^n$ back onto the reference surface.
\[
         \xv^n \leftarrow \Pv( \xv^n )
\]
\end{enumerate}


\subsection{Adaptation to a weight function}

In an adaptive setting the user will have some criteria for measuring where
the error is large and a smaller grid spacing needed.

Given a positive weight function $w$ that is large where the grid needs to have
a small grid spacing we choose the control function $\Pv$ by
\[
   \Pv = {\grad_\rv w \over w}
\]

The weight function should be strictly positive. As an example, in one dimension one could
define $w$ to be large where the second deriavtive is large
\[
    w = 1 + \alpha ( { | u_{rr} | \over  \| u_{rr} \|_\infty } )
\]
In general we will smooth the weight function before forming $\Pv$.

We motivate this choice of control function from the criteria that 
we choose the grid point positions to satisfy a mean-value condition
so that the position $\xv_\iv$ is a weighted average of its neighbours
\[
    \xv_\iv = { \sum_{\jv\ne\iv} w_\jv \xv_\jv \over \sum_\jv w_\jv }
\]
or
\[
  \sum_{\jv\ne\iv} w_\jv \xv_\jv  - (\sum_\jv w_\jv) \xv_\iv = 0
\]
which looks like an approximation to
\[
  \grad_\rv \cdot( w \xv_\rv) = 0 
\]     
This last formula can be re-written as
\begin{equation}
    \sum_m \xv_{r_m r_m}  + {w_{r_m}\over w } \xv_{r_m} = 0  \label{eq:ell2}
\end{equation}
Comparing this last equation to (\ref{eq:ell1}) suggests that we take $P_m = {w_{r_m} / w }$.


\input EllipticGridGeneratorInclude.tex
 



