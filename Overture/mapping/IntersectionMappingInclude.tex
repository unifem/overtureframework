\subsection{Constructor}
 
\newlength{\IntersectionMappingIncludeArgIndent}
\begin{flushleft} \textbf{%
\settowidth{\IntersectionMappingIncludeArgIndent}{IntersectionMapping(}% 
IntersectionMapping() 
}\end{flushleft}
\begin{description}
\item[{\bf Purpose:}]  Default Constructor
\item[{\bf Author:}]  WDH
\end{description}
\subsection{Constructor}
 
\begin{flushleft} \textbf{%
\settowidth{\IntersectionMappingIncludeArgIndent}{IntersectionMapping(}% 
IntersectionMapping(Mapping \& map1\_,\\ 
\hspace{\IntersectionMappingIncludeArgIndent}Mapping \& map2\_ )
}\end{flushleft}
\begin{description}
\item[{\bf Purpose:}] 
   Define a mapping for the intersection of map1\_ and map2\_
\item[{\bf map1\_, map2\_ :}]  two surfaces in 3D
\end{description}
\subsection{intersect}
 
\begin{flushleft} \textbf{%
int  \\ 
\settowidth{\IntersectionMappingIncludeArgIndent}{intersect(}%
intersect(Mapping \& map1\_, Mapping \& map2\_,\\ 
\hspace{\IntersectionMappingIncludeArgIndent}GenericGraphicsInterface *gi  =NULL,\\ 
\hspace{\IntersectionMappingIncludeArgIndent}GraphicsParameters \& params  =nullGraphicsParameters)
}\end{flushleft}
\begin{description}
\item[{\bf Description:}] 
   Determine the intersection between two mappings, optionally supply graphic parameters
    so the intersection curves can be plotted, (for debugging purposes).
   NEW FEATURE: If the intersection curve has disjoint segments, these segments will be 
   stored as sub curves in the NURBS for the physical and parameter curves on each surface.
\item[{\bf map1\_, map2\_ (input) :}]  These two mappings will be intersected.
\item[{\bf gi, paramas (input) :}]  Optional parameters for graphics.
\item[{\bf Return value:}]  0 for success
\end{description}
\subsection{intersect}
 
\begin{flushleft} \textbf{%
int  \\ 
\settowidth{\IntersectionMappingIncludeArgIndent}{intersectWithCompositeSurface(}%
intersectWithCompositeSurface(Mapping \& map1\_, CompositeSurface \& cs,\\ 
\hspace{\IntersectionMappingIncludeArgIndent}GenericGraphicsInterface *gi  =NULL,\\ 
\hspace{\IntersectionMappingIncludeArgIndent}GraphicsParameters \& params  =nullGraphicsParameters)
}\end{flushleft}
\begin{description}
\item[{\bf Description:}] 
    A Protected routine that computes the intersection between a Mapping
 and a CompositeSurface. 

\item[{\bf map1\_, map2\_ (input) :}]  These two mappings will be intersected.
\item[{\bf gi, paramas (input) :}]  Optional parameters for graphics.
\item[{\bf Return value:}]  0 for success

\item[{\bf Output:}]  The output intersection curve is a NurbsMapping. The number of subcurves of this
 mapping defines the number of disconnected components of the intersection.

\end{description}
\subsection{newtonIntersection}
 
\begin{flushleft} \textbf{%
int  \\ 
\settowidth{\IntersectionMappingIncludeArgIndent}{newtonIntersection(}%
newtonIntersection(realArray \& x, realArray \& r1, realArray \& r2, const realArray \& n )
}\end{flushleft}
\begin{description}
\item[{\bf Description:}]  
   This is a protected routine to determine the exact intersection point on two surfaces using Newton's
  method.

 Solve for (x,r1,r2 ) such that 
 \begin{verbatim}

    map1(r1) - x = 0
    map2(r2) - x = 0
    n.x = c
 \end{verbatim}

\item[{\bf x(.,3) (input/output) :}]  initial guess to the intersection point (in the Range space)
\item[{\bf r1(.,2) (input/output):}]  initial guess to the intersection point (in the domain space of map1)
\item[{\bf r2(.,2) (input/output):}]  initial guess to the intersection point (in the domain space of map2)
\item[{\bf n(.,3) :}]  a normal vector to a plane that crosses the intersection curve, often choosen
         to be  n(i,.) = x(i+1,.) - x(i-1,.) if we are computing x(i,.)
\item[{\bf Return values:}]  0 for success. 1 if the newton iteration did not converge, 2 if there is a
 zero normal vector.
\end{description}
\subsection{project}
 
\begin{flushleft} \textbf{%
int  \\ 
\settowidth{\IntersectionMappingIncludeArgIndent}{project(}%
project( realArray \& x,\\ 
\hspace{\IntersectionMappingIncludeArgIndent}int \& iStart, \\ 
\hspace{\IntersectionMappingIncludeArgIndent}int \& iEnd,\\ 
\hspace{\IntersectionMappingIncludeArgIndent}periodicType periodic)
}\end{flushleft}
\begin{description}
\item[{\bf Description:}]  
     Project the points x(iStart:iEnd,0:6) onto the intersection
  NOTE: When the points are projected onto the curves it is possible that points
      fold back on themselves if they get out of order. This routine will try and
      detect this situation and it may remove some points to fix the problem.
 Return values: 0 for success, otherwise failure.
\end{description}
\subsection{determineIntersection}
 
\begin{flushleft} \textbf{%
int  \\ 
\settowidth{\IntersectionMappingIncludeArgIndent}{determineIntersection(}%
determineIntersection(GenericGraphicsInterface *gi  =NULL,\\ 
\hspace{\IntersectionMappingIncludeArgIndent}GraphicsParameters \& params  =nullGraphicsParameters)
}\end{flushleft}
\begin{description}
\item[{\bf Description:}]  
   This is a protected routine to determine the intersection curve(s) between two surfaces.

\item[{\bf Notes:}] 

  (1) First obtain an initial guess to the intersection: Using the bounding boxes that cover
      the surface to determine a list of pairs of (leaf) bounding boxes that intersect. Triangulate
      the surface quadrilaterals that are found in this "collision" list and find all line segments
      that are formed when two triangles intersect.

  (2) Join the line segments found in step 1 into a continuous curve(s). There will be three 
    different intersection curves -- a curve in the Range space (x) and a curve in each of the
    domain spaces (r). Since the domain spaces may be periodic it may be necessary to shift 
    parts of the domain-space curves by +1 or -1 so that the curves are continuous. Note that
    the domain curves will sometimes have to be outside the unit square. It is up to ?? to map
    these values back to [0,1] if they are used.
    
  (3) Now fit a NURBS curve to all of the intersection curves, using chord-length of the space-curve
      to parameterize the three curves.
  (4) Re-evaluate the points on the curve using Newton's method to obtain the points that are exactly
      on on the intersection of the surfaces. Refit the NURBS curves using these new points. 

\item[{\bf Return values:}]  0 for success, otherwise failure.
\end{description}
\subsection{map}
 
\begin{flushleft} \textbf{%
int  \\ 
\settowidth{\IntersectionMappingIncludeArgIndent}{reparameterize(}%
reparameterize(const real \& arcLengthWeight  =1., \\ 
\hspace{\IntersectionMappingIncludeArgIndent}const real \& curvatureWeight  =.2)
}\end{flushleft}
\begin{description}
\item[{\bf Purpose:}]  
    Redistribute points on the intersection curve to place more points where the
  curvature is large. 
\item[{\bf Description:}]  The default distribution of points in the intersection curve
  is equally spaced in arc length (really chord length). To cluster more points
  near sharp corners, call this routine with a non-zero value for {\tt curvatureWeight}.
  In this case the points will be placed to equidistribute the weight function
  \begin{verbatim}
      w(r) = 1 + arcLength(r)*arcLengthWeight + curvature(r)*curvatureWeight
   where
      arcLength(r) =  | x_r |
      curvature(r) =  | x_rr |    (*** this is not really the curvature, but close ***)
  \end{verbatim}
 Note that the point distribution only depends on the ratio of arcLengthWeight to curvatureWeight 
 and not on their absolute vaules. The weight function must be positive everywhere.
 Also note that for the unit circle, $| x_r |=2\pi$ and $| x_{rr}|= (2\pi)^2$ so that the curvature
 is naturally $2\pi$ times larger in the weight function.

\item[{\bf arcLengthWeight (input) :}]  weight for the arc length, should be positive.
\item[{\bf curvatureWeight (input) :}]  weight for the curvature, should normally be non-negative. 
\end{description}
\subsection{intersectCurves}
 
\begin{flushleft} \textbf{%
int  \\ 
\settowidth{\IntersectionMappingIncludeArgIndent}{intersectCurves(}%
intersectCurves(Mapping \& curve1, \\ 
\hspace{\IntersectionMappingIncludeArgIndent}Mapping \& curve2, \\ 
\hspace{\IntersectionMappingIncludeArgIndent}int \& numberOfIntersectionPoints, \\ 
\hspace{\IntersectionMappingIncludeArgIndent}realArray \& r1, \\ 
\hspace{\IntersectionMappingIncludeArgIndent}realArray \& r2,\\ 
\hspace{\IntersectionMappingIncludeArgIndent}realArray \& x )
}\end{flushleft}
\begin{description}
\item[{\bf Description:}]  
   Determine the intersection between two 2D curves.

\item[{\bf curve1, curve2 (input) :}]  intersect these curves
\item[{\bf numberOfIntersectionPoints (output):}]  the number of intersection points found.
\item[{\bf r1,r2,x (output) :}]  r1(i),r2(i),x(0:1,i) the intersection point(s) for $i=0,\ldots,numberOfIntersectionPoints-1$
    are $curve1(r1(i))=curve2(r2(i))=x(i)$
     

\end{description}
\subsection{map}
 
\begin{flushleft} \textbf{%
void  \\ 
\settowidth{\IntersectionMappingIncludeArgIndent}{map(}%
map( const realArray \& r, realArray \& x, realArray \& xr, MappingParameters \& params )
}\end{flushleft}
\begin{description}
\item[{\bf Purpose:}]  Evaluate the intersection curve.
\end{description}
\subsection{get}
 
\begin{flushleft} \textbf{%
int  \\ 
\settowidth{\IntersectionMappingIncludeArgIndent}{get(}%
get( const GenericDataBase \& dir, const aString \& name)
}\end{flushleft}
\begin{description}
\item[{\bf Purpose:}]  get a mapping from the database.
\end{description}
\subsection{put}
 
\begin{flushleft} \textbf{%
int  \\ 
\settowidth{\IntersectionMappingIncludeArgIndent}{put(}%
put( GenericDataBase \& dir, const aString \& name) const
}\end{flushleft}
\begin{description}
\item[{\bf Purpose:}]  put the mapping to the database.
\end{description}
\subsection{update}
 
\begin{flushleft} \textbf{%
int  \\ 
\settowidth{\IntersectionMappingIncludeArgIndent}{update(}%
update( MappingInformation \& mapInfo ) 
}\end{flushleft}
\begin{description}
\item[{\bf Purpose:}]  Interactively create and/or change the mapping.
\item[{\bf mapInfo (input):}]  Holds a graphics interface to use.
\end{description}
