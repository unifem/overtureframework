\section{HyperbolicSurfaceMapping}

{\bf This mapping currently uses HYPGEN from NASA which we cannot distribute and thus
   will not work for users outside of Los Alamos}.

This mapping class can compute a grid on a 3D surface, starting from a space
curve that is close to the surface.

\input HyperbolicSurfaceMappingInclude.tex

\subsection{examples}

\subsubsection{Surface grid on a sphere}
In the left column is the command file that was used to generate the grid on the right.

\noindent
\begin{minipage}{.4\linewidth}
{\footnotesize
\listinginput[1]{1}{\mapping /surgrdSphere.cmd}
}
\end{minipage}\hfill
\begin{minipage}{.6\linewidth}
  \begin{center}
   \epsfig{file=\figures/surgrdSphere.ps,height=4.in}  \\
  {A surface grid (red) created on a sphere using hyperbolic grid generation. The surface grid
    was grown starting from a circle (shown in green). The surface
    grid nicely covers the polar singularity in the original sphere.}
  \end{center}
\end{minipage}

% 
% \noindent
% \begin{minipage}{.475\linewidth}
%   \begin{center}
%    \epsfig{file=\mapping /hyper1.ps,width=\linewidth}  \\
%   {Hyperbolic mapping}
%   \end{center}
% \end{minipage}\hfill
% \begin{minipage}{.475\linewidth}
%   \begin{center}
%    \epsfig{file=\mapping /hyper2.ps,width=\linewidth}  \\
%   {Hyperbolic mapping}
%   \end{center}
% \end{minipage}





