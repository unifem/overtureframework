\subsection{Constructor}
 
\newlength{\NurbsMappingIncludeArgIndent}
\begin{flushleft} \textbf{%
\settowidth{\NurbsMappingIncludeArgIndent}{NurbsMapping(}% 
NurbsMapping() 
}\end{flushleft}
\begin{description}
\item[{\bf Purpose:}]  Default Constructor, make a null NURBS. 
\item[{\bf Remarks:}] 
    The implementation here is based on the reference, {\sl The NURBS Book}
 by Les Piegl and Wayne Tiller, Springer, 1997. The notation here is:
 \begin{itemize}
   \item degree = p  (variables p1,p2 for one and 2D)
   \item  number of control points is n+1 (variables n1,n2)
   \item  number of knots is m+1 (m=n+p+1) (variables m1,m2)
   \item cPoint(0:n,0:r) :  holds the control points and weights. r=rangeDimension.
   \item uKnot(0:m) : holds knots along axis1. These are normally scaled to [0,1] (see notes below).
   \item vKnot(0:m) : holds knots along axis2 (if domainDimension==2)  
   \item note : Knots are scaled to [0,1]
 \end{itemize}
 
\item[{\bf NOTES:}]  for those wanting to make changes to this class

   {\bf uMin,uMax,vMin,vMax} : A typical NURBS will have knots that span an arbitrary interval.
     For example the knots may go from $[.5,1.25]$. This mapping however, is parameterized on [0,1].
     To fix this we first save the actual min and max values for uKnot in [uMin,uMax] and similarly
     for [vMin,vMax]. We then rescale uKnot and vKnot to lie on the interval [0,1]. Note that
     the {\tt reparameterize} function may subsequently rescale the knots to a larger interval
     in which case the NURBS will only represent a part of the initial surface. If we do this then
     we also rescale uMin,uMax,vMin,vMax. The {\tt parametricCurve} function is used to indicate
     that this NURBS is actually a parametric curve on another NURBS, nurbs2. By default the values of 
     uMin,uMax,vMin,vMax from nurbs2 are used to scale this NURBS in order to make it compatible with
     the rescaled nurbs2.
     
\end{description}
\subsection{Constructor}
 
\begin{flushleft} \textbf{%
\settowidth{\NurbsMappingIncludeArgIndent}{NurbsMapping(}% 
NurbsMapping(const int \& domainDimension\_ , const int \& rangeDimension\_ ) 
}\end{flushleft}
\begin{description}
\item[{\bf Purpose:}]  Constructor, make a default NURBS of the give domain dimension (1,2)
\end{description}
\subsection{intersect3DLines}
 
\begin{flushleft} \textbf{%
int  \\ 
\settowidth{\NurbsMappingIncludeArgIndent}{intersect3DLines(}%
intersect3DLines( const RealArray \& pt0, const RealArray \&  t0, \\ 
\hspace{\NurbsMappingIncludeArgIndent}const RealArray \& pt1, const RealArray \& t1,\\ 
\hspace{\NurbsMappingIncludeArgIndent}real \& alpha0, real \& alpha1,\\ 
\hspace{\NurbsMappingIncludeArgIndent}RealArray \& pt2) const
}\end{flushleft}
\begin{description}
\item[{\bf Description:}] 
    Intersect two lines in 3D:
 x0(s) = pt0 + s * t0
 x1(t) = pt1 + t * t1

\item[{\bf alpha0,alpha1 :}]  values of s and t at the intersection.
\item[{\bf pt2 :}]  point of intrsection, x0(alpha0)=pt2=x1(alpha1)

\item[{\bf Return values:}]  1 if the line are parallel, 0 otherwise.
\end{description}
\subsection{buildCurveOnSurface}
 
\begin{flushleft} \textbf{%
int  \\ 
\settowidth{\NurbsMappingIncludeArgIndent}{buildCurveOnSurface(}%
buildCurveOnSurface(NurbsMapping \& curve,\\ 
\hspace{\NurbsMappingIncludeArgIndent}real r0, \\ 
\hspace{\NurbsMappingIncludeArgIndent}real r1  =-1.)
}\end{flushleft}
\begin{description}
\item[{\bf Description:}] 
   Build a new Nurbs curve  that matches a coordinate line on the surface.
\item[{\bf curve (output) :}]  on output a curve that matches a coordinate line on the surface.
\item[{\bf r0,r1 (input) :}]  if r1==-1 make a curve ${\mathbf c}(r) = {\mathbf s}(r0,r)$ where ${\mathbf s}(r_0,r_1)$
   is the NURBS surface defined by this mapping. If r0==-1   
     make the curve ${\mathbf c}(r) = {\mathbf s}(r,r1)$ 
  the arc, measured starting from x.

\end{description}
\subsection{buildComponentCurve}
 
\begin{flushleft} \textbf{%
int  \\ 
\settowidth{\NurbsMappingIncludeArgIndent}{buildComponentCurve(}%
buildComponentCurve(NurbsMapping \& curve,\\ 
\hspace{\NurbsMappingIncludeArgIndent}int component  = 0)
}\end{flushleft}
\begin{description}
\item[{\bf Description:}] 
     Given a curve, (x(s),y(s)[,z(s)]),  in 2D or 3D, build a new curve in 1D that 
  represents the a given component such as x(s) or y(s) or z(s).
\item[{\bf curve (output) :}]  on output a component curve.
\item[{\bf component (input) :}]  Choose component=0,1, or 2 to form the curve for x,y, or z

\end{description}
\subsection{circle}
 
\begin{flushleft} \textbf{%
int  \\ 
\settowidth{\NurbsMappingIncludeArgIndent}{circle(}%
circle(RealArray \& o,\\ 
\hspace{\NurbsMappingIncludeArgIndent}RealArray \& x, \\ 
\hspace{\NurbsMappingIncludeArgIndent}RealArray \& y, \\ 
\hspace{\NurbsMappingIncludeArgIndent}real r,\\ 
\hspace{\NurbsMappingIncludeArgIndent}real startAngle  =0.,\\ 
\hspace{\NurbsMappingIncludeArgIndent}real endAngle  =1.)
}\end{flushleft}
\begin{description}
\item[{\bf Description:}] 
   Build a circular arc. Reference the NURBS book Algorithm A7.1
\item[{\bf o (input):}]  center of the circle.
\item[{\bf x,y (input):}]  orthogonal unit vectors in the plane of the circle.
\item[{\bf startAngle,endAngle :}]  normalized angles [0,1] for the start and end of
  the arc, measured starting from x.

\end{description}
\subsection{conic(a,b,c,...)}
 
\begin{flushleft} \textbf{%
int  \\ 
\settowidth{\NurbsMappingIncludeArgIndent}{conic(}%
conic( const real a, const real b, const real c, const real d, const real e, const real f, \\ 
\hspace{\NurbsMappingIncludeArgIndent}const real z, const real x1, const real y1, const real x2, const real y2 )
}\end{flushleft}
\begin{description}
\item[{\bf Description:}] 
 Build a NURBS for a conic defined by an implicit formula and two end points
 \begin{verbatim}
      a*x^2 + b*x*y + c*y^2 + d*x + e*y + f = 0
 \end{verbatim}

\item[{\bf a,b,c,d,e,f :}]  implicit formula for a conic
\item[{\bf z,x1,y1,x2,y2:}]  the end points of the conic are (x1,y1,z) and (x2,y2,z)

\item[{\bf return values:}]  0=success, 1=error

\end{description}
\subsection{conic}
 
\begin{flushleft} \textbf{%
int  \\ 
\settowidth{\NurbsMappingIncludeArgIndent}{conic(}%
conic( const RealArray \&pt0, const RealArray \&t0, const RealArray \&pt2, const RealArray \&t2, const RealArray \&p )
}\end{flushleft}
\begin{description}
\item[{\bf Description:}] 
 Build a NURBS for a conic defined by end points, two tangents, and an additional point
\item[{\bf pt0(0,0:}] 2),pt2(0,0:2) : end points
\item[{\bf t0(0,0:}] 2),t2(0,0:2) : tangent directions at end points
\item[{\bf p(0,0:}] 2) : another point on the conic

\item[{\bf NOTES:}]  Construct open conic arc in 3D (c.f. makeOpenConic, NURBS book p. 317)

\item[{\bf NOTE:}]  ****TODO: case of full ellipse, see page 319

\item[{\bf return values:}]  0=success, 1=error

\end{description}
\subsection{generalCylinder(curve,shift)}
 
\begin{flushleft} \textbf{%
int  \\ 
\settowidth{\NurbsMappingIncludeArgIndent}{generalCylinder(}%
generalCylinder( const Mapping \& curve, real d[3] )
}\end{flushleft}
\begin{description}
\item[{\bf Description:}] 
    Build a general cylinder (tabulated cylnder) by extruding a curve along a direction vector d.
    The general cylinder is defined as
 \begin{verbatim}
       C(u,v) = curve(u) + v*d
 \end{verbatim}

\item[{\bf curve (input) :}]  curve to extrude. If this curve is not a NurbsMapping then it is converted into
   a NurbsMapping by interpolating the grid points from curve. Increase the number of grid points
      on curve if you want a more accurate result.
\item[{\bf d[3] input :}]  extrude the curve along this direction vector 
\end{description}
\subsection{generalCylinder(curve1,curbe2)}
 
\begin{flushleft} \textbf{%
int  \\ 
\settowidth{\NurbsMappingIncludeArgIndent}{generalCylinder(}%
generalCylinder( const Mapping \& curve1, const Mapping \& curve2 )
}\end{flushleft}
\begin{description}
\item[{\bf Description:}] 
    Build a general cylinder (tabulated cylnder) by interpolating between two curves
    The general cylinder is defined as
 \begin{verbatim}
       C(u,v) = (1-v)*curve1(u) + v*curve2
 \end{verbatim}

\item[{\bf curve1, curve2 (input) :}]  curves to interpolate. If either curve is not a NurbsMapping then it is converted into
   a NurbsMapping by interpolating the grid points from curve. Increase the number of grid points
      on curve if you want a more accurate result.

\end{description}
\subsection{plane}
 
\begin{flushleft} \textbf{%
int  \\ 
\settowidth{\NurbsMappingIncludeArgIndent}{plane(}%
plane( real pt1[3], real pt2[3], real pt3[3] )
}\end{flushleft}
\begin{description}
\item[{\bf Description:}] 
    Build a plane that passes through three points
              
 \begin{verbatim}
           pt3
            * 
            |
            |
            *------* pt2
           pt1
 \end{verbatim}
 
    The plane is defined as 
 \begin{verbatim}
       C(u,v) = pt1 +u*(pt2-pt1) + v*(pt3-pt1)
 \end{verbatim}

\item[{\bf pt1[3], pt2[3], pt3[3] (input) :}]  3 points on the plane

\end{description}
\subsection{getKnots}
 
\begin{flushleft} \textbf{%
const RealArray \&  \\ 
\settowidth{\NurbsMappingIncludeArgIndent}{getKnots(}%
getKnots( int direction  =0) const
}\end{flushleft}
\begin{description}
\item[{\bf Purpose:}]  get uKnot or vKnot, the knots in the first or second direction.
\item[{\bf direction:}]  0=return uKnot, 1= return vKnot.
\end{description}
\subsection{getControlPoints}
 
\begin{flushleft} \textbf{%
const RealArray \&  \\ 
\settowidth{\NurbsMappingIncludeArgIndent}{getControlPoints(}%
getControlPoints() const
}\end{flushleft}
\begin{description}
\item[{\bf Purpose:}]  
    Return the control points, scaled by the weight.
\end{description}
\subsection{insertKnot}
 
\begin{flushleft} \textbf{%
int  \\ 
\settowidth{\NurbsMappingIncludeArgIndent}{insertKnot(}%
insertKnot(const real \& uBar, \\ 
\hspace{\NurbsMappingIncludeArgIndent}const int \& numberOfTimesToInsert\_  =1)
}\end{flushleft}
\begin{description}
\item[{\bf Purpose:}]  Insert a knot
\item[{\bf uBar (input):}]  Insert this knot value.
\item[{\bf numberOfTimesToInsert\_ (input):}]  insert the knot this many times. The multiplicity
    of the knot will not be allowed to exceed p1.
\end{description}
\subsection{insertKnot}
 
\begin{flushleft} \textbf{%
int  \\ 
\settowidth{\NurbsMappingIncludeArgIndent}{normalizeKnots(}%
normalizeKnots() 
}\end{flushleft}
\begin{description}
\item[{\bf Access:}]  Protected routine.
\item[{\bf Purpose:}]  
    Normalize the knots, uKnot (and vKnot if domainDimension==2) to
 lie from 0 to 1. This routine will NOT change the values of uMin,uMax, vMin,vMax since
 these values indicate the original bounds on uKnot and vKnot.
\end{description}
\subsection{readFromIgesFile}
 
\begin{flushleft} \textbf{%
int  \\ 
\settowidth{\NurbsMappingIncludeArgIndent}{readFromIgesFile(}%
readFromIgesFile( IgesReader \& iges, const int \& item, bool normKnots /*=true*/  )
}\end{flushleft}
\begin{description}
\item[{\bf Purpose:}]  Read a NURBS from an IGES file. An IGES file is a data file
 containing geometrical objects, usually generated by a CAD program.
\item[{\bf iges (input) :}]  Use this object to read the IGES file.
\item[{\bf item (input) :}]  read this item from the IGES file.
\end{description}
\subsection{parametricCurve}
 
\begin{flushleft} \textbf{%
int   \\ 
\settowidth{\NurbsMappingIncludeArgIndent}{parametricCurve(}%
parametricCurve(const NurbsMapping \& nurbs,\\ 
\hspace{\NurbsMappingIncludeArgIndent}const bool \& scaleParameterSpace  = true)
}\end{flushleft}
\begin{description}
\item[{\bf Purpose:}] 
   Indicate that this nurb is actually a parametric curve on another nurb surface.
\item[{\bf nurbs (input) :}]  Here is the NURBS surface for which this NURBS is a parametric surface.
\item[{\bf scaleParameterSpace (input) :}]  if true, scale the range space of this nurb
   to be on the unit interval. This is usually required since the NurbsMapping scales
   the knots to lie on [0,1] (normally) and so we then need to scale this Mapping to
   be consistent.
\end{description}
\subsection{shift}
 
\begin{flushleft} \textbf{%
int  \\ 
\settowidth{\NurbsMappingIncludeArgIndent}{shift(}%
shift(const real \& shiftx  =0., \\ 
\hspace{\NurbsMappingIncludeArgIndent}const real \& shifty  =0., \\ 
\hspace{\NurbsMappingIncludeArgIndent}const real \& shiftz /* =0.*/ )
}\end{flushleft}
\begin{description}
\item[{\bf Purpose:}]  Shift the NURBS in space.
\end{description}
\subsection{scale}
 
\begin{flushleft} \textbf{%
int  \\ 
\settowidth{\NurbsMappingIncludeArgIndent}{scale(}%
scale(const real \& scalex  =0., \\ 
\hspace{\NurbsMappingIncludeArgIndent}const real \& scaley  =0., \\ 
\hspace{\NurbsMappingIncludeArgIndent}const real \& scalez /* =0.*/ )
}\end{flushleft}
\begin{description}
\item[{\bf Purpose:}]  Scale the NURBS in space.
\end{description}
\subsection{rotate}
 
\begin{flushleft} \textbf{%
int  \\ 
\settowidth{\NurbsMappingIncludeArgIndent}{rotate(}%
rotate( const int \& axis, const real \& theta )
}\end{flushleft}
\begin{description}
\item[{\bf Purpose:}]  Perform a rotation about a given axis. This rotation is applied
   after any existing transformations. 
\item[{\bf axis (input) :}]  axis to rotate about (0,1,2)
\item[{\bf theta (input) :}]  angle in radians to rotate by.
\end{description}
\subsection{rotate}
 
\begin{flushleft} \textbf{%
int  \\ 
\settowidth{\NurbsMappingIncludeArgIndent}{matrixTransform(}%
matrixTransform( const RealArray \& m )
}\end{flushleft}
\begin{description}
\item[{\bf Purpose:}]  
    Perform a general matrix transform using a 2x2 or 3x3 matrix.
  Convert the NURBS to 2D or 3D if the transformation so specifies -- i.e.
 if you transform a NURBS with rangeDimension==2 with a 3x3 matrix then the
 result will be a NURBS with rangeDimension==3.
\item[{\bf m (input) :}]  m(0:2,0:2) matrix to transform with
\end{description}
\subsection{specify knots and control points}
 
\begin{flushleft} \textbf{%
int  \\ 
\settowidth{\NurbsMappingIncludeArgIndent}{specify(}%
specify(const int \&  m,\\ 
\hspace{\NurbsMappingIncludeArgIndent}const int \& n,\\ 
\hspace{\NurbsMappingIncludeArgIndent}const int \& p,\\ 
\hspace{\NurbsMappingIncludeArgIndent}const RealArray \& knot,\\ 
\hspace{\NurbsMappingIncludeArgIndent}const RealArray \& controlPoint,\\ 
\hspace{\NurbsMappingIncludeArgIndent}const int \& rangeDimension\_  =3,\\ 
\hspace{\NurbsMappingIncludeArgIndent}bool normalizeTheKnots /* =true*/  )
}\end{flushleft}
\begin{description}
\item[{\bf Purpose:}]  Specify a {\bf curve} in 2D or 3D using knots and control points
\item[{\bf m (input) :}]  The number of knots is m+1
\item[{\bf n (input) :}]  the number of control points is n+1
\item[{\bf p (input) :}]  order of the B-spline
\item[{\bf controlPoint(0:}] n,0:rangeDimension) (input) : control points and weights
\item[{\bf normalizeTheKnots (input) :}]  by default, normalize the knots to [0,1]. Set to false
    if you do not want the knots normalized. 
\end{description}
\subsection{specify knots and control points}
 
\begin{flushleft} \textbf{%
int  \\ 
\settowidth{\NurbsMappingIncludeArgIndent}{specify(}%
specify(const int \& n1\_, \\ 
\hspace{\NurbsMappingIncludeArgIndent}const int \& n2\_,\\ 
\hspace{\NurbsMappingIncludeArgIndent}const int \& p1\_, \\ 
\hspace{\NurbsMappingIncludeArgIndent}const int \& p2\_, \\ 
\hspace{\NurbsMappingIncludeArgIndent}const RealArray \& uKnot\_, \\ 
\hspace{\NurbsMappingIncludeArgIndent}const RealArray \& vKnot\_,\\ 
\hspace{\NurbsMappingIncludeArgIndent}const RealArray \& controlPoint,\\ 
\hspace{\NurbsMappingIncludeArgIndent}const int \& rangeDimension\_  =3,\\ 
\hspace{\NurbsMappingIncludeArgIndent}bool normalizeTheKnots /* =true*/  )
}\end{flushleft}
\begin{description}
\item[{\bf Purpose:}]  Specify a NURBS with domainDimension==2 using knots and control points
\item[{\bf n1\_,n2\_ (input) :}]  the number of control points is n1+1 by n2+1
\item[{\bf p1\_,p2\_ (input) :}]  order of the B-spline in each direction.
\item[{\bf uKnot\_,vKnot\_ (input) :}]  knots.
\item[{\bf controlPoint(0:}] n1,0:n2,0:rangeDimenion) (input) : control points and weights
\item[{\bf normalizeTheKnots (input) :}]  by default, normailize the knots to [0,1]. Set to false
    if you do not want the knots normalized. 
\end{description}
\subsubsection{setDomainInterval}
 
\begin{flushleft} \textbf{%
int   \\ 
\settowidth{\NurbsMappingIncludeArgIndent}{setDomainInterval(}%
setDomainInterval(const real \& r1Start  =0., \\ 
\hspace{\NurbsMappingIncludeArgIndent}const real \& r1End  =1.,\\ 
\hspace{\NurbsMappingIncludeArgIndent}const real \& r2Start  =0., \\ 
\hspace{\NurbsMappingIncludeArgIndent}const real \& r2End  =1.,\\ 
\hspace{\NurbsMappingIncludeArgIndent}const real \& r3Start  =0., \\ 
\hspace{\NurbsMappingIncludeArgIndent}const real \& r3End  =1.)
}\end{flushleft}
\begin{description}
\item[{\bf Description:}] 
 Restrict the domain of the nurbs.
 By default the nurbs is parameterized on the interval [0,1] (1D) or [0,1]x[0,1] in 2D etc.
 You may choose a sub-section of the nurbs by choosing a new interval [rStart,rEnd].
 For periodic nurbss the interval may lie in [-1,2] so the sub-section can cross the branch cut.
 You may even choose rEnd<rStart to reverse the order of the parameterization.
\item[{\bf rStart1,rEnd1,rStart2,rEnd2,rStart3,rEnd3 (input) :}]  define the new interval.
\end{description}
\subsection{initialize()}
 
\begin{flushleft} \textbf{%
void  \\ 
\settowidth{\NurbsMappingIncludeArgIndent}{initialize(}%
initialize( bool setMappingHasChanged  =true)
}\end{flushleft}
\begin{description}
\item[{\bf Purpose:}]  Initialize the NURBS. This is a protected routine.
 Determine if the weights are constant so
 that we can use more efficient routines. Set bounds for the Mapping.

\item[{\bf setMappingHasChanged (input) :}]  if true (by default) we call mappingHasChanged so that the grid etc. will be marked
                  out of date. 
\item[{\bf NOTES:}]  Normally we multiply the control points by the weights.
     BUT,  if the weights are constant we divide everything by this constant value so
 we can avoid dividing by the weight term when we evaluate. When the weights are
 constant {\tt nonUniformWeights==false};
\end{description}
\subsection{setBounds}
 
\begin{flushleft} \textbf{%
void  \\ 
\settowidth{\NurbsMappingIncludeArgIndent}{setBounds(}%
setBounds()
}\end{flushleft}
\begin{description}
\item[{\bf Purpose:}]  protected routine. Set the approximate bounds on the mapping, used by plotting routines etc.
 Use the control points as an approximation
 *** note only apply this to the normalized control-points ***
\end{description}
\subsection{removeKnot}
 
\begin{flushleft} \textbf{%
int  \\ 
\settowidth{\NurbsMappingIncludeArgIndent}{removeKnot(}%
removeKnot(const int \& index, \\ 
\hspace{\NurbsMappingIncludeArgIndent}const int \& numberOfTimesToRemove why is this int passed by reference?,\\ 
\hspace{\NurbsMappingIncludeArgIndent}int \&  numberRemoved, const real \& tol  ) why is tol passed by reference?\\ 
}\end{flushleft}
\begin{description}
\item[{\bf Purpose:}]  Remove a knot (if possible) so that the Nurbs remains unchanged
\item[{\bf index (input) :}]  try to remove the knot at this index. AP: NOTE: uKnot(index) != uKnot(index+1)
\item[{\bf numberOfTimesToRemove (input) :}]  the number of times to try and remove the knot. 
\item[{\bf numberRemoved (output):}]   the actual number of times the knot was removed
\end{description}
\subsection{getParameterBounds}
 
\begin{flushleft} \textbf{%
int   \\ 
\settowidth{\NurbsMappingIncludeArgIndent}{getParameterBounds(}%
getParameterBounds( int axis, real \& rStart\_, real \& rEnd\_ ) const
}\end{flushleft}
\begin{description}
\item[{\bf Purpose:}]  
   Return current values for the parameter bounds.
\item[{\bf axis (input) :}]  return bounds for this axis.
\item[{\bf rStart\_, rEnd\_:}]  bounds.
\end{description}
\subsection{reparameterize}
 
\begin{flushleft} \textbf{%
int   \\ 
\settowidth{\NurbsMappingIncludeArgIndent}{reparameterize(}%
reparameterize(const real \& uMin\_, \\ 
\hspace{\NurbsMappingIncludeArgIndent}const real \& uMax\_,\\ 
\hspace{\NurbsMappingIncludeArgIndent}const real \& vMin\_  =0., \\ 
\hspace{\NurbsMappingIncludeArgIndent}const real \& vMax\_  =1.)
}\end{flushleft}
\begin{description}
\item[{\bf Purpose:}]  Reparameterize the nurb to only use a sub-rectangle of the parameter space.
    This function can also be used to reverse the direction of the parameterization by choosing
   $\mbox{uMin} > \mbox{uMax}$ and/or $\mbox{vMin} > \mbox{vMax}$.
\item[{\bf uMin,uMax (input):}]   subrange of u values to use, normally $0 \le \mbox{uMin} \ne \mbox{uMax} \le 1$
\item[{\bf vMin,vMax (input):}]  subrange of v values to use, normally  $0 \le \mbox{vMin} \ne \mbox{vMax} \le 1$ 
     (for domainDimension==2)

\item[{\bf Notes:}]  this routine just scales the knots to be on a larger interval than [0,1]. Thus when
    the Mapping is evaluated on [0,1] the result will only be a portion of the original surface.
\item[{\bf Return values:}]  0 : success,  1 : failure
\end{description}
\subsection{transformKnots}
 
\begin{flushleft} \textbf{%
int   \\ 
\settowidth{\NurbsMappingIncludeArgIndent}{transformKnots(}%
transformKnots(const real \& uScale, \\ 
\hspace{\NurbsMappingIncludeArgIndent}const real \& uShift,\\ 
\hspace{\NurbsMappingIncludeArgIndent}const real \& vScale  =1., \\ 
\hspace{\NurbsMappingIncludeArgIndent}const real \& vShift  =0.)
}\end{flushleft}
\begin{description}
\item[{\bf Purpose:}]  
    Apply a scaling and shift to the to the knots: uScale*uKnots+uShift.
 The scale factors should be positive.

\item[{\bf uScale,uShift (input):}]   scaling and shift for the knots in the u direction.
\item[{\bf vScale,vShift (input):}]  scaling and shift for the knots in the v direction.
     (for domainDimension==2)

\end{description}
\subsection{elevateDegree}
 
\begin{flushleft} \textbf{%
int  \\ 
\settowidth{\NurbsMappingIncludeArgIndent}{elevateDegree(}%
elevateDegree(const int increment)
}\end{flushleft}
\begin{description}
\item[{\bf Purpose:}]  
    Elevate the degree of the nurbs.
\item[{\bf increment (input):}]  increase the degree of the nurb by this amount >=0 
\item[{\bf Return values:}]  0 : success, 1 : failure
\end{description}
\subsection{merge}
 
\begin{flushleft} \textbf{%
int  \\ 
\settowidth{\NurbsMappingIncludeArgIndent}{merge(}%
merge(NurbsMapping \& nurbs, bool keepFailed  = true, real eps /*=-1*/, bool attemptPeriodic /*=true*/ )
}\end{flushleft}
\begin{description}
\item[{\bf Purpose:}]  Try to merge "this" nurbs with the input nurbs. This routine will merge
  the two NURBS's into one if the endpoint of one matches the end point of the second.
\item[{\bf nurbs (input):}]  nurbs to merge with
\item[{\bf Return values:}]  0 : success, 1 : failure
\end{description}
\subsection{forcedMerge}
 
\begin{flushleft} \textbf{%
int  \\ 
\settowidth{\NurbsMappingIncludeArgIndent}{forcedMerge(}%
forcedMerge(NurbsMapping \& nurbs  )
}\end{flushleft}
\begin{description}
\item[{\bf Purpose:}]  Force a merge of "this" nurbs with the input nurbs. This routine will merge
  the two NURBS's into one if the endpoint of one matches the end point of the second. If
  the endpoints do not match, a straight line section is added between the closest end
  points.
\item[{\bf nurbs (input):}]  nurbs to merge with
\item[{\bf Return values:}]  0 : success, 1 : failure
\end{description}
\subsection{forcedPeriodic}
 
\begin{flushleft} \textbf{%
int  \\ 
\settowidth{\NurbsMappingIncludeArgIndent}{forcePeriodic(}%
forcePeriodic()
}\end{flushleft}
\begin{description}
\item[{\bf Purpose:}]  force this mapping to be periodic by making the last control points the
 same as the first ( if the knots are "clamped", eg the knots are 0 0 0 0 ... 1 1 1 1 )
\item[{\bf Return values:}]  0 : success, 1 : failure
\end{description}
\subsection{split}
 
\begin{flushleft} \textbf{%
int  \\ 
\settowidth{\NurbsMappingIncludeArgIndent}{split(}%
split(real uSplit, NurbsMapping \&c1, NurbsMapping\&c2, bool normalizePieces  =true)
}\end{flushleft}
\begin{description}
\item[{\bf Description:}] 
  Split a nurb curve into two pieces.
\item[{\bf uSplit (input) :}]  parameter value to split the curve at
\item[{\bf c1 (output) :}]  curve on the "left", parameter bounds [0,uSplit]
\item[{\bf c2 (output) :}]  curve on the "right", parameter bounds [uSplit,1]
\item[{\bf normalizePieces (input) :}]  in true then normalize each piece
\item[{\bf Returns :}]  0 on success, 1 on failure ( uSplit<0 or uSplit>1 )
\end{description}
\subsection{moveEndpoint}
 
\begin{flushleft} \textbf{%
int  \\ 
\settowidth{\NurbsMappingIncludeArgIndent}{moveEndpoint(}%
moveEndpoint( int end, const RealArray \&endPoint, real tol /*=-1*/ )
}\end{flushleft}
\begin{description}
\item[{\bf Description:}] 
  Move either the beginning or the end of the curve to endPoint.
\end{description}
\subsection{numberOfSubCurves}
 
\begin{flushleft} \textbf{%
int  \\ 
\settowidth{\NurbsMappingIncludeArgIndent}{numberOfSubCurves(}%
numberOfSubCurves() const
}\end{flushleft}
\begin{description}
\item[{\bf Description:}] 
  If the Nurb is formed by merging a sequence of Nurbs then function will return that number.
  By default the numberOfSubCurves would be 1 if no Nurbs were merged.
\end{description}
\subsection{numberOfSubCurvesInList}
 
\begin{flushleft} \textbf{%
int  \\ 
\settowidth{\NurbsMappingIncludeArgIndent}{numberOfSubCurvesInList(}%
numberOfSubCurvesInList() const
}\end{flushleft}
\begin{description}
\item[{\bf Description:}] 
  Return the number of subcurves used to build the Nurb plus the number of hidden curves
  By default the numberOfSubCurvesInList would be 1 if no Nurbs were merged.
\end{description}
\subsection{subCurve}
 
\begin{flushleft} \textbf{%
NurbsMapping\&  \\ 
\settowidth{\NurbsMappingIncludeArgIndent}{subCurve(}%
subCurve(int subCurveNumber)
}\end{flushleft}
\begin{description}
\item[{\bf Description:}] 
  If the Nurb is formed by merging a sequence of Nurbs then function will return that Nurbs.
 If the numberOfSubCurves is 1 then the current (full) Nurbs is returned.
  
\end{description}
\subsection{subCurveFromList}
 
\begin{flushleft} \textbf{%
NurbsMapping \&  \\ 
\settowidth{\NurbsMappingIncludeArgIndent}{subCurveFromList(}%
subCurveFromList(int subCurveNumber)
}\end{flushleft}
\begin{description}
\item[{\bf Description:}] 
  Return a nurb curve directly from the list of subcurves.  This can be a curve used to generate
 the nurb itself or one of the "hidden" curves.
 If the numberOfSubCurves is 1 then the current (full) Nurbs is returned.
  
\end{description}
\subsection{interpolate}
 
\begin{flushleft} \textbf{%
void  \\ 
\settowidth{\NurbsMappingIncludeArgIndent}{interpolate(}%
interpolate(const RealArray \& x, \\ 
\hspace{\NurbsMappingIncludeArgIndent}const int \& option     = 0,\\ 
RealArray \& parameterization  =Overture::nullRealArray(),\\ 
\hspace{\NurbsMappingIncludeArgIndent}int degree  = 3,\\ 
\hspace{\NurbsMappingIncludeArgIndent}ParameterizationTypeEnum parameterizationType  =parameterizeByChordLength,\\ 
\hspace{\NurbsMappingIncludeArgIndent}int numberOfGhostPoints  =0)
}\end{flushleft}
\begin{description}
\item[{\bf Purpose:}]  
    Define a new NURBS curve that interpolates the points x(0:n1,0:r-1) OR
 define a new NURBS surface that interpolates the points x(0:n1,0:n2,0:r-1) (NEW feature). 
 An even more recent (051031) addition allows the interpolation of a volume defined
 by the points x(0:n1,0:n2,0:n3,0:r-1);
  By default the NURBS curve will be parameterized by a the chord length.
 
\item[{\bf option (input) :}]  if option==0 then use the array parameterization.
     if option==1 then return the parameterization used in the array parameterization.
\item[{\bf parameterization\_(0:}] n1) (input) : optionally specify the parameterization. These values
 should start from 0, end at 1 and be increasing. If this argument is not given then
   the parameterization will be based on chord length. If option==1 then the 
   actual parameterization used will be returned in this array.
\item[{\bf degree (input) :}]  degree of approximation. Normally a value such as 1,2,3.
\item[{\bf parameterizationType (input) :}]  the default parameterization (if not user defined) is by chord-length.
       One can also specify to parameterizeByIndex which means use a uniform parameterization.
\item[{\bf numberOfGhostPoints (input) :}]  The array x includes this many ghost points on all sides.

\end{description}
\subsection{interpolate}
 
\begin{flushleft} \textbf{%
void  \\ 
\settowidth{\NurbsMappingIncludeArgIndent}{interpolate(}%
interpolate(Mapping \& map, int degree  =3, \\ 
\hspace{\NurbsMappingIncludeArgIndent}ParameterizationTypeEnum parameterizationType  =parameterizeByChordLength,\\ 
\hspace{\NurbsMappingIncludeArgIndent}int numberOfGhostPoints  =0,\\ 
\hspace{\NurbsMappingIncludeArgIndent}int *numPointsToInterpolate  =NULL)
}\end{flushleft}
\begin{description}
\item[{\bf Description:}] 
   Construct a NURBS by interpolating another mapping.
\item[{\bf map (input) :}]  interpolate this Mapping.
\item[{\bf degree (input) :}]  degree of NURBS
\item[{\bf parameterization (input) :}]  
\item[{\bf numberOfGhostPoints (input) :}]  include this many ghost points
\item[{\bf numPointsToInterpolate (input) :}]  numPointsToInterpolate[dir], dir=0,1,...,domainDimension-1, 
       optionally specify how many points to interpolate in each direction. By default use the number 
       of grid points from the Mapping "map". 
\end{description}
\subsection{put(fileName)}
 
\begin{flushleft} \textbf{%
int  \\ 
\settowidth{\NurbsMappingIncludeArgIndent}{put(}%
put( const aString \& fileName, const FileFormat \& fileFormat  = xxww) 
}\end{flushleft}
\begin{description}
\item[{\bf Description:}] 
   put NURBS data into an ascii readable file.
\item[{\bf fileName (input) :}]  name of the file.
\item[{\bf fileFormat (input) :}]  specify the file format. (see the comments with the get(const aString\&,...) function).
\end{description}
\subsection{put(FILE*)}
 
\begin{flushleft} \textbf{%
int  \\ 
\settowidth{\NurbsMappingIncludeArgIndent}{put(}%
put( FILE *file, const FileFormat \& fileFormat  = xxww)
}\end{flushleft}
\begin{description}
\item[{\bf Description:}] 
   Save the NURBS data to an ascii readable file.
\item[{\bf fileFormat (input) :}]  specify the file format. (see the comments with the get(const aString\&,...) function).
\end{description}
\subsection{get(fileName)}
 
\begin{flushleft} \textbf{%
int  \\ 
\settowidth{\NurbsMappingIncludeArgIndent}{get(}%
get( const aString \& fileName, const FileFormat \& fileFormat  = xxww)
}\end{flushleft}
\begin{description}
\item[{\bf Description:}] 
   read NURBS data from an ascii readable file.
\item[{\bf fileName (input) :}]  get from this file.
\item[{\bf fileFormat (input) :}]  specify the file format.

 \noindent Here is the file format for {\tt fileFormat=xxww} for a surface in 3D
 \begin{verbatim}
  domainDimension rangeDimension p1 n1 p2 n2  
  uKnot(0) uKnot(1) ... uKnot(m1)  --- on possibly multiple lines, at most 10 values per line
  vKnot(0) vKnot(1) ... vKnot(m2)
  x0 x1 x2 ...            --- x coords of control pts. on multiple lines, at most 10 per line
  y0 y1 y2 ...            --- y coords of control pts.                                  
  z0 z1 z2 ...            --- z coords of control pts.                                  
  w0 w1 w2 ...            --- weights of control pts.                                  
 \end{verbatim}
 If the domainDimension==1 then leave off p2 and n2. If the rangeDimension is 2 then leave 
  off the z values. Here m1=n1+p1+1 and m2=n2+p2+1.

 \noindent Here is the file format for {\tt fileFormat=xwxw} for a surface in 3D
 \begin{verbatim}
  domainDimension rangeDimension p1 n1 p2 n2
  uKnot(0) uKnot(1) ... uKnot(m1)  --- on possibly multiple lines, at most 10 values per line
  vKnot(0) vKnot(1) ... vKnot(m2)
  x0 y0 z0 w0                   --- control point 0
  x1 y1 z1 w1                   --- control point 1
  x1 y1 z1 w1                   --- control point 2
  ... 
 \end{verbatim}
 If the domainDimension==1 then leave off p2 and n2. If the rangeDimension is 2 then leave 
  off the z values.
\end{description}
\subsection{put(FILE *)}
 
\begin{flushleft} \textbf{%
int  \\ 
\settowidth{\NurbsMappingIncludeArgIndent}{get(}%
get( FILE *file, const FileFormat \& fileFormat  = xxww)
}\end{flushleft}
\begin{description}
\item[{\bf Description:}] 
   read NURBS data from an ascii readable file.
\item[{\bf file (input) :}]  get from this file.
\item[{\bf fileFormat (input) :}]  specify the file format. (see the comments with the get(const aString\&,...) function).
\end{description}
\subsection{getOrder}
 
\begin{flushleft} \textbf{%
int  \\ 
\settowidth{\NurbsMappingIncludeArgIndent}{getOrder(}%
getOrder( int axis  =0) const
}\end{flushleft}
\begin{description}
\item[{\bf Purpose:}]  
   Return the order, p.
\end{description}
\subsection{getNumberOfKnots}
 
\begin{flushleft} \textbf{%
int  \\ 
\settowidth{\NurbsMappingIncludeArgIndent}{getNumberOfKnots(}%
getNumberOfKnots( int axis  =0) const
}\end{flushleft}
\begin{description}
\item[{\bf Purpose:}]  
  Return the number of knots, m+1.
\end{description}
\subsection{getNumberOfControlPoints}
 
\begin{flushleft} \textbf{%
int  \\ 
\settowidth{\NurbsMappingIncludeArgIndent}{getNumberOfControlPoints(}%
getNumberOfControlPoints( int axis  =0) const
}\end{flushleft}
\begin{description}
\item[{\bf Purpose:}]  
   Return the number of control points, n+1.
\end{description}
\subsection{buildSubCurves}
 
\begin{flushleft} \textbf{%
int  \\ 
\settowidth{\NurbsMappingIncludeArgIndent}{buildSubCurves(}%
buildSubCurves( real angle  =60.)
}\end{flushleft}
\begin{description}
\item[{\bf Purpose:}] 
   Split a NURBS curve at corners into sub-curves. Currently this only applies if the
  order of the NURBS is 1 (piece-wise linear).
\item[{\bf angle (input) :}]  divide the curve at points where the tangent changes by more than
    this angle (degrees)
\end{description}
\subsection{truncateToDomainBounds}
 
\begin{flushleft} \textbf{%
int  \\ 
\settowidth{\NurbsMappingIncludeArgIndent}{truncateToDomainBounds(}%
truncateToDomainBounds()
}\end{flushleft}
\begin{description}
\item[{\bf Purpose:}]  clip the knots and control polygon to the bounds set in rstart and rend
\end{description}
\subsection{toggleSubCurveVisibility}
 
\begin{flushleft} \textbf{%
int  \\ 
\settowidth{\NurbsMappingIncludeArgIndent}{toggleSubCurveVisibility(}%
toggleSubCurveVisibility( int sc )
}\end{flushleft}
\begin{description}
\item[{\bf Description:}] 
   Toggle a subcurve's "visibility", 
      a visible subcurve is accessible through NurbsMapping::subCurve(..) method
    an invisible subcurve is only accessible through NurbsMapping::subCurveFromList()
\item[{\bf sc (input) :}]  the subcurve to toggle
 Returns : the new subcurve number
 NOTES : this will reorder the subcurves in the subCurves array
\end{description}
\subsection{isSubCurveHidden}
 
\begin{flushleft} \textbf{%
bool  \\ 
\settowidth{\NurbsMappingIncludeArgIndent}{isSubCurveHidden(}%
isSubCurveHidden( int sc )
}\end{flushleft}
\begin{description}
\item[{\bf Description:}] 
   find out if a subcurve is hidden or not, returns true if hidden, false if visible
\item[{\bf sc (input) :}]  the subcurve to querry
\end{description}
\subsection{isSubCurveOriginal}
 
\begin{flushleft} \textbf{%
bool  \\ 
\settowidth{\NurbsMappingIncludeArgIndent}{isSubCurveOriginal(}%
isSubCurveOriginal( int sc )
}\end{flushleft}
\begin{description}
\item[{\bf Description:}] 
   find out if a subcurve is marked as "original"
\item[{\bf sc (input) :}]  the subcurve to querry
\end{description}
\subsection{toggleSubCurveOriginal}
 
\begin{flushleft} \textbf{%
void  \\ 
\settowidth{\NurbsMappingIncludeArgIndent}{toggleSubCurveOriginal(}%
toggleSubCurveOriginal( int sc )
}\end{flushleft}
\begin{description}
\item[{\bf Description:}] 
   toggle the "original" status on a subcurve, "original" is just a marker
   used to distingish the original subcurves used to build this nurb from 
   subsequent modifications.
\item[{\bf sc (input) :}]  the subcurve to alter
\end{description}
\subsection{addSubCurve}
 
\begin{flushleft} \textbf{%
int  \\ 
\settowidth{\NurbsMappingIncludeArgIndent}{addSubCurve(}%
addSubCurve(NurbsMapping \&nurbs)
}\end{flushleft}
\begin{description}
\item[{\bf Description:}] 
   Add a subcurve to this mapping.  Note that the nurb is copied and is 
  set to visible.  The "original" marker is set to false;
\item[{\bf Returns :}]  the index of the new curve in the list of visible curves
\end{description}
\subsection{deleteSubCurve}
 
\begin{flushleft} \textbf{%
int  \\ 
\settowidth{\NurbsMappingIncludeArgIndent}{deleteSubCurve(}%
deleteSubCurve(int sc)
}\end{flushleft}
\begin{description}
\item[{\bf Description:}] 
   Delete a subcurve from the list of curves.  Note this shifts the subcurve list
    making previous indices invalid
\item[{\bf sc (input):}]  the curve to delete
\item[{\bf Returns :}]  0 on success
\end{description}
\subsection{update}
 
\begin{flushleft} \textbf{%
int  \\ 
\settowidth{\NurbsMappingIncludeArgIndent}{update(}%
update( MappingInformation \& mapInfo ) 
}\end{flushleft}
\begin{description}
\item[{\bf Purpose:}]  Interactively create and/or change the nurbs mapping.
\item[{\bf mapInfo (input):}]  Holds a graphics interface to use.
\end{description}
\subsection{interpolatLoftedSurface}
 
\begin{flushleft} \textbf{%
int  \\ 
\settowidth{\NurbsMappingIncludeArgIndent}{interpolateLoftedSurface(}%
interpolateLoftedSurface(vector$<$Mapping*$>$ \&curves, int degree1 /*=3*/, \\ 
\hspace{\NurbsMappingIncludeArgIndent}int degree2 /*=3*/ ,\\ 
\hspace{\NurbsMappingIncludeArgIndent}ParameterizationTypeEnum  parameterizationType  =parameterizeByChordLength,\\ 
\hspace{\NurbsMappingIncludeArgIndent}int numberOfGhostPoints  =0)
}\end{flushleft}
\begin{description}
\item[{\bf Description:}] 
    Interpolate a list of curves to make a lofted surface.  Note this actually
    interpolates each curve evaluated at a number of points; an actual Nurbs lofted
    surface could be created using a list of Nurbs curves but we don't do that 
    right now.

\item[{\bf curves (input) :}]  a list of mappings to interpolate
\end{description}
