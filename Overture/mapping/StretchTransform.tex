%--------------------------------------------------------------
\section{StretchTransform: stretch grid lines of an existing mapping}
\index{stretch-transform mapping}\index{Mapping!StretchTransformMapping}
%-------------------------------------------------------------

\subsection{Description}

This mapping can be used to reparameterize another mapping by stretcing
the grid lines in the parameter directions. It does this by composing
the {\ff StretchedSquare} mapping with the given mapping. The {\ff StretchedSquare}
mapping in turn uses the {\ff StretchMapping} to create stretching functions.



% \subsection{Constructors}
% 
% 
% \begin{tabbing}
% {\ff StretchTransform()xxxx}\= \kill
% {\ff StretchTransform()}    \> Default constructor\\
% \end{tabbing}
% 
% 
% \subsection{Data Members}
% 
% \subsection{Member Functions}
% 
% \begin{tabbing}
% 0123456789012345678901234567689012345678901234567890123 \= \kill
% {\ff void map( ... ) }     \> evaluate the mapping and derivative  \\
% {\ff void inverseMap( ... ) }  \> evaluate the inverse mapping and derivative  \\
% {\ff void get( const Dir \& dir, const String \& name)} \> get from a database file \\
% {\ff void put( const Dir \& dir, const String \& name)} \> put to a database file \\
% \end{tabbing}
