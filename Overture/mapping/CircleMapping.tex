\section{CircleMapping (ellipse too)}\index{circle mapping}\index{Mapping!CircleMapping}\index{ellipse}

This mapping defines a circle or ellipse in two or three dimensions:
\begin{eqnarray*}
    \xv(r) &=& ( a \cos(2\pi r)+x_0, b\sin(2\pi r) + y_0 ) \\
    \xv(r) &=& ( a \cos(2\pi r)+x_0, b\sin(2\pi r) + y_0 , z_0 )
\end{eqnarray*}
on a constant $z-plane$.
A partial arc can also be defined (see the figure with AnnulusMapping).

%% \input CircleMappingInclude.tex
%% \vfill\eject

