\section{IntersectionMapping}\index{intersection mapping}\index{Mapping!IntersectionMapping}

This mapping class can compute the intersection between two other mappings, such as the
curve of intersection between two surfaces. See the comments with the {\tt determineIntersection}
function for a description of the fairly robust way in which we find the intersection.


\begin{figure}[hbt]
\newcommand{\figWidth}{8cm}
\newcommand{\trimfig}[2]{\trimFig{#1}{#2}{0.0}{.0}{.1}{.1}}
\newcommand{\figWidtha}{6cm}
\newcommand{\trimfiga}[2]{\trimFig{#1}{#2}{0.075}{.1}{.325}{.35}}
\begin{center}\small
% ------------------------------------------------------------------------------------------------
\begin{tikzpicture}
  \useasboundingbox (0,0.25) rectangle (12,12);  % set the bounding box (so we have less surrounding white space)
% 
  \draw (2,4.5) node[anchor=south west,xshift=-4pt,yshift=-4pt] {\trimfig{\figures/ss_intersect}{\figWidth}};
  \draw (0, 0) node[anchor=south west,xshift=-4pt,yshift=-4pt] {\trimfiga{\figures/ss1_intersect}{\figWidtha}};
  \draw (6, 0) node[anchor=south west,xshift=-4pt,yshift=-4pt] {\trimfiga{\figures/ss2_intersect}{\figWidtha}};
% grid:
%  \draw[step=1cm,gray] (0,0) grid (12,12);
\end{tikzpicture}
% ----------------------------------------------------------------------------------------
 \caption{The sphere-sphere intersection curve in the range space and the domain spaces (unit squares).}
\end{center}
\end{figure}



%% \input IntersectionMappingInclude.tex

\begin{figure}[hbt]
\newcommand{\figWidth}{8cm}
\newcommand{\trimfig}[2]{\trimFig{#1}{#2}{0.0}{.0}{.1}{.1}}
\newcommand{\figWidtha}{6cm}
\newcommand{\trimfiga}[2]{\trimFig{#1}{#2}{0.0}{.1}{.1}{.125}}
\begin{center}\small
% ------------------------------------------------------------------------------------------------
\begin{tikzpicture}
  \useasboundingbox (0,0.25) rectangle (12,12);  % set the bounding box (so we have less surrounding white space)
% 
  \draw (2,5.5) node[anchor=south west,xshift=-4pt,yshift=-4pt] {\trimfig{\figures/cc_intersect}{\figWidth}};
  \draw (0, 0) node[anchor=south west,xshift=-4pt,yshift=-4pt] {\trimfiga{\figures/cc1_intersect}{\figWidtha}};
  \draw (6, 0) node[anchor=south west,xshift=-4pt,yshift=-4pt] {\trimfiga{\figures/cc2_intersect}{\figWidtha}};
% grid:
% \draw[step=1cm,gray] (0,0) grid (12,13);
\end{tikzpicture}
% ----------------------------------------------------------------------------------------
 \caption{The cylinder-cylinder intersection curves in the range space and the domain spaces (unit squares).
     This is a reasonably hard case since the cylinders have the same radius and thus the surfaces
     are tangent at two points.}
\end{center}
\end{figure}

\begin{figure}[hbt]
\newcommand{\figWidth}{8cm}
\newcommand{\trimfig}[2]{\trimFig{#1}{#2}{0.0}{.0}{.1}{.1}}
\newcommand{\figWidtha}{6cm}
\newcommand{\trimfiga}[2]{\trimFig{#1}{#2}{0.075}{.1}{.325}{.35}}
\begin{center}\small
% ------------------------------------------------------------------------------------------------
\begin{tikzpicture}
  \useasboundingbox (0,0.25) rectangle (12,11);  % set the bounding box (so we have less surrounding white space)
% 
  \draw (2,4.5) node[anchor=south west,xshift=-4pt,yshift=-4pt] {\trimfig{\figures/wing_body}{\figWidth}};
  \draw (0, 0) node[anchor=south west,xshift=-4pt,yshift=-4pt] {\trimfiga{\figures/wing_body1}{\figWidtha}};
  \draw (6, 0) node[anchor=south west,xshift=-4pt,yshift=-4pt] {\trimfiga{\figures/wing_body2}{\figWidtha}};
% grid:
%   \draw[step=1cm,gray] (0,0) grid (12,12);
\end{tikzpicture}
% ----------------------------------------------------------------------------------------
 \caption{Intersection curve for a wing-body configuration. The curve was reparameterized weighting arclength
       and curvature in order to redistribute more points to the high curvature regions.}
\end{center}
\end{figure}


