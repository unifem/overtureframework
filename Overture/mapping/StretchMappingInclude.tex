\subsubsection{Constructor}
 
\newlength{\StretchMappingIncludeArgIndent}
\begin{flushleft} \textbf{%
\settowidth{\StretchMappingIncludeArgIndent}{StretchMapping(}% 
StretchMapping( const StretchingType \& stretchingType\_  = noStretching)
}\end{flushleft}
\begin{description}
\item[{\bf Purpose:}]  Construct a function with the given stretching type, one of
  \begin{description}
   \item[inverseHyperbolicTangent] : the most commonly used stretching function
      defined in an inverse way as a combination of hyperbolic tangents and
      logarithms of hyperbolic cosines.
   \item[hyperbolicTangent] : hyperbolic tangent stretching.
   \item[exponential] : exponential stretching.
   \item[exponentialBlend] : a $C^\infty$ blending function that is exactly 0 for $r<{1\over4}$
        and exactly 1 for $r>{3\over4}$.
   \item[exponentialToLinear] exponential stretching that transitions to linear stretching
  \end{description}
\item[{\bf stretchingType\_ (input):}] 
\end{description}
\subsubsection{Constructor}
 
\begin{flushleft} \textbf{%
\settowidth{\StretchMappingIncludeArgIndent}{StretchMapping(}% 
StretchMapping( const int numberOfLayers\_, \\ 
\hspace{\StretchMappingIncludeArgIndent}const int numberOfIntervals\_  = 0) 
}\end{flushleft}
\begin{description}
\item[{\bf Purpose:}]  Construct an {\tt inverseHyperbolicTangent} stretching function.
\item[{\bf numberOfLayers\_ (input):}]  number of layers.
\item[{\bf numberOfIntervals\_ (input):}]  number of intervals.
\end{description}
\subsubsection{setStretchingType}
 
\begin{flushleft} \textbf{%
int   \\ 
\settowidth{\StretchMappingIncludeArgIndent}{setStretchingType(}%
setStretchingType(  const StretchingType \& stretchingType\_ )
}\end{flushleft}
\begin{description}
\item[{\bf Description:}]  Set the stretching type, one of
  \begin{description}
   \item[inverseHyperbolicTangent] : the most commonly used stretching function
      defined in an inverse way as a combination of hyperbolic tangents and
      logarithms of hyperbolic cosines.
   \item[hyperbolicTangent] : hyperbolic tangent stretching.
   \item[exponential] : exponential stretching.
   \item[exponentialBlend] : a $C^\infty$ blending function that is exactly 0 for $r<{1\over4}$
        and exactly 1 for $r>{3\over4}$.
   \item[exponentialToLinear] exponential stretching that transitions to linear stretching
  \end{description}
\item[{\bf stretchingType\_ (input):}] 
\end{description}
\subsubsection{setNumberOfLayers}
 
\begin{flushleft} \textbf{%
int  \\ 
\settowidth{\StretchMappingIncludeArgIndent}{setNumberOfLayers(}%
setNumberOfLayers( const int numberOfLayers\_ ) 
}\end{flushleft}
\begin{description}
\item[{\bf Description:}]  
    Set the number of layer (tanh) functions in the {\tt inverseHyperbolicTangent} stretching
   function.
\item[{\bf numberOfLayers\_ (input):}] 
\item[{\bf Return value:}]  0 on success, 1 if the stretching type has not been set to {\tt inverseHyperbolicTangent}
   in which case no changes are made.  
\end{description}
\subsubsection{setNumberOIntervals}
 
\begin{flushleft} \textbf{%
int  \\ 
\settowidth{\StretchMappingIncludeArgIndent}{setNumberOfIntervals(}%
setNumberOfIntervals( const int numberOfIntervals\_ )
}\end{flushleft}
\begin{description}
\item[{\bf Description:}]  
    Set the number of interval (log(cosh)) functions in the {\tt inverseHyperbolicTangent} stretching
   function.
\item[{\bf numberOfIntervals\_ (input):}] 
\item[{\bf Return value:}]  0 on success, 1 if the stretching type has not been set to {\tt inverseHyperbolicTangent}
   in which case no changes are made.  
\end{description}
\subsubsection{setNumberOfSplinePoints}
 
\begin{flushleft} \textbf{%
int  \\ 
\settowidth{\StretchMappingIncludeArgIndent}{setNumberOfSplinePoints(}%
setNumberOfSplinePoints( const int numberOfSplinePoints0 )
}\end{flushleft}
\begin{description}
\item[{\bf Description:}]  
    Set the number of interval (log(cosh)) functions in the {\tt inverseHyperbolicTangent} stretching
   function.
\item[{\bf numberOfIntervals\_ (input):}] 
\item[{\bf Return value:}]  0 on success, 1 if the stretching type has not been set to {\tt inverseHyperbolicTangent}
   in which case no changes are made.  
\end{description}
\subsubsection{setLayerParameters}
 
\begin{flushleft} \textbf{%
int  \\ 
\settowidth{\StretchMappingIncludeArgIndent}{setLayerParameters(}%
setLayerParameters( const int index, const real a, const real b, const real c )
}\end{flushleft}
\begin{description}
\item[{\bf Description:}]  
    Set parameters for the interval (log(cosh)) function numbered {\tt index}.
\item[{\bf a,b,c (input):}] 
\item[{\bf Return value:}]  0 on success, 1 if the stretching type has not been set to {\tt inverseHyperbolicTangent}
   in which case no changes are made.  
\end{description}
\subsubsection{setIntervalParameters}
 
\begin{flushleft} \textbf{%
int  \\ 
\settowidth{\StretchMappingIncludeArgIndent}{setIntervalParameters(}%
setIntervalParameters( const int index, const real d, const real e,  const real f )
}\end{flushleft}
\begin{description}
\item[{\bf Description:}]  
    Set parameters for the interval (log(cosh)) function numbered {\tt index}.
\item[{\bf d,e,f (input):}] 
\item[{\bf Return value:}]  0 on success, 1 if the stretching type has not been set to {\tt inverseHyperbolicTangent}
   in which case no changes are made.  
\end{description}
\subsubsection{setEndPoints}
 
\begin{flushleft} \textbf{%
int  \\ 
\settowidth{\StretchMappingIncludeArgIndent}{setEndPoints(}%
setEndPoints( const real rmin, const real rmax )
}\end{flushleft}
\begin{description}
\item[{\bf Description:}]  
    Set the end points for the {\tt inverseHyperbolicTangent} stretching function.
\item[{\bf rmin,rmax (input):}] 
\item[{\bf Return value:}]  0 on success, 1 if the stretching type has not been set to {\tt inverseHyperbolicTangent}
   in which case no changes are made.  
\end{description}
\subsubsection{setIsNormalized}
 
\begin{flushleft} \textbf{%
int  \\ 
\settowidth{\StretchMappingIncludeArgIndent}{setIsNormalized(}%
setIsNormalized( const bool \& trueOrFalse  =TRUE)
}\end{flushleft}
\begin{description}
\item[{\bf Description:}]  
    Indicate whether the stretching function should be normalized to go from 0 to 1.
\item[{\bf trueOrFalse (input):}]  if TRUE the function is normalized.
\end{description}
\subsubsection{setScaleParameters}
 
\begin{flushleft} \textbf{%
int  \\ 
\settowidth{\StretchMappingIncludeArgIndent}{setScaleParameters(}%
setScaleParameters( const real origin\_, const real scale\_ )
}\end{flushleft}
\begin{description}
\item[{\bf Description:}]  
    Set the origin and scale parameters for the {\tt inverseHyperbolicTangent} stretching
   function.
\item[{\bf origin\_, scale\_ (input):}] 
\item[{\bf Return value:}]  0 on success, 1 if the stretching type has not been set to {\tt inverseHyperbolicTangent}
   in which case no changes are made.  
\end{description}
\subsubsection{setIsPeriodic}
 
\begin{flushleft} \textbf{%
int  \\ 
\settowidth{\StretchMappingIncludeArgIndent}{setIsPeriodic(}%
setIsPeriodic( const int trueOrFalse )
}\end{flushleft}
\begin{description}
\item[{\bf Description:}]  
   Define the periodicity of the function, only applies to the {\tt inverseHyperbolicTangent} stretching
   function.
\item[{\bf trueOrFalse (input):}]  TRUE or FALSE.
\item[{\bf Return value:}]  0 on success, 1 if the stretching type has not been set to {\tt inverseHyperbolicTangent}
   in which case no changes are made.  
\end{description}
\subsubsection{setHyperbolicTangentParameters}
 
\begin{flushleft} \textbf{%
int  \\ 
\settowidth{\StretchMappingIncludeArgIndent}{setHyperbolicTangentParameters(}%
setHyperbolicTangentParameters(const real \& a0\_,\\ 
\hspace{\StretchMappingIncludeArgIndent}const real \& ar\_, \\ 
\hspace{\StretchMappingIncludeArgIndent}const real \& a1\_, \\ 
\hspace{\StretchMappingIncludeArgIndent}const real \& b1\_, \\ 
\hspace{\StretchMappingIncludeArgIndent}const real \& c1\_)
}\end{flushleft}
\begin{description}
\item[{\bf Description:}]  
    Set the parameters for the {\tt hyperbolicTangent} stretching
   function.
\item[{\bf a0\_,ar\_,a1\_,b1\_,c1\_, (input):}] 
\item[{\bf Return value:}]  0 on success, 1 if the stretching type has not been set to {\tt hyperbolicTangent}
   in which case no changes are made.  
\end{description}
\subsubsection{setExponentialParameters}
 
\begin{flushleft} \textbf{%
int  \\ 
\settowidth{\StretchMappingIncludeArgIndent}{setExponentialParameters(}%
setExponentialParameters(const real \& a0\_, \\ 
\hspace{\StretchMappingIncludeArgIndent}const real \& ar\_, \\ 
\hspace{\StretchMappingIncludeArgIndent}const real \& a1\_, \\ 
\hspace{\StretchMappingIncludeArgIndent}const real \& b1\_, \\ 
\hspace{\StretchMappingIncludeArgIndent}const real \& c1\_)
}\end{flushleft}
\begin{description}
\item[{\bf Description:}]  
    Set the parameters for the {\tt exponential} stretching
   function.
\item[{\bf a0\_,a1\_,b1\_,c1\_, (input):}] 
\item[{\bf Return value:}]  0 on success, 1 if the stretching type has not been set to {\tt exponential}
   in which case no changes are made.  
\end{description}
\subsubsection{setLinearSpacingParameters}
 
\begin{flushleft} \textbf{%
int  \\ 
\settowidth{\StretchMappingIncludeArgIndent}{setLinearSpacingParameters(}%
setLinearSpacingParameters(const real \& a0\_, const real \& a1\_)
}\end{flushleft}
\begin{description}
\item[{\bf Description:}]  
    Set the parameters for the {\tt linear spacing} stretching
   function. 
    The grid spacing for this stretching will exponentially increase from a0 to a1. (wdh)
\item[{\bf a0, a1 (input):}]  specify the grid spacings at r=0 and r=1
\item[{\bf Return value:}]  0 on success, 1 if the stretching type has not been set to {\tt linearSpacing}
   in which case no changes are made.  
\end{description}
\subsubsection{setExponentialToLinearParameters}
 
\begin{flushleft} \textbf{%
int  \\ 
\settowidth{\StretchMappingIncludeArgIndent}{setExponentialToLinearParameters(}%
setExponentialToLinearParameters(const real \& a, \\ 
\hspace{\StretchMappingIncludeArgIndent}const real \& b,\\ 
\hspace{\StretchMappingIncludeArgIndent}const real \& c )
}\end{flushleft}
\begin{description}
\item[{\bf Description:}]  
    Set the parameters for the {\tt exponentialToLinear} stretching
   function.
\item[{\bf a,b,c (input):}]  see documentation for the formula.
\item[{\bf Return value:}]  0 on success, 1 if the stretching type has not been set to {\tt exponential}
   in which case no changes are made.  
\end{description}
