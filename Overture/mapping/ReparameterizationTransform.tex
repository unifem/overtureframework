%--------------------------------------------------------------
\section{ReparameterizationTransform: reparameterize an existing mapping 
(e.g. remove a polar singularity)}
\index{reparameterization mapping}\index{Mapping!ReparameterizationTransform}\index{polar singularity!remove}
\index{restrict a mapping to a sub-rectangle}\index{orthographic}\index{reorder domain coordinates}
%-------------------------------------------------------------

\subsection{Description}

The {\tt ReparameterizationTransform} can reparameterize a given Mapping
in one of the following ways:
\begin{description}
   \item[Orthographic:] remove a polar singularity by using a orthographic projection
      to define a new patch over the singularity.
   \item[Restriction:] restrict the parameter space to a sub-rectangle of the
       original parameter space. Use this, for example, to define a refined patch in an
       adapative grid.
   \item[Reorder domain coordinates:] uses the ReorientMapping to reorder the domain (i.e. parameter space)
     coordinates of a mapping. This can be useful for debugging purposes in order to see that
     the results of some application are independent of the order of $(r_1,r_2,r_3)$.
\end{description}   


\subsection{Reparameterizing a spherical-polar or cylindrical-polar singularity}

  The {\tt orthographic} reparameterization can be used to remove a spherical polar
singluarity or cylindrical polar singularity by defining a new patch over the singularity. 

  In order for the {\tt orthographic} reparameterization to be applicable the Mapping
to be reparmeterized must have the following properties:
\begin{description}
  \item[a polar singularity] : The mapping must have a polar singularity at $r_1=0$ or
$r_1=1$ and the coordinate direction $r_2$ must be the angular ($\theta$) variable. ($r_3$ would
be the radial direction). If the mapping has such a singularity then one should indicate this
property with the call of the form
\begin{verbatim}
	setTypeOfCoordinateSingularity( side,axis,polarSingularity ); 
\end{verbatim}
  \item[can be evaluated in spherical (or cylindrical) coordinates]: this property should be set with a call
\begin{verbatim}
	setCoordinateEvaluationType( spherical,TRUE ); 
\end{verbatim}
or
\begin{verbatim}
	setCoordinateEvaluationType( cylindrical,TRUE ); 
\end{verbatim}
   A Mapping that can be evaluated in spherical or cylindrical coordinates must define the {\tt map} and
   {\tt basicInverse} functions to optionally return the derivatives
   in a special form. For spherical coordinates the derivatives of the mapping are computed as
\[
   \left({\partial x_i \over\partial r_1},{1\over\sin(\phi)}{\partial x_i \over\partial r_2},
         {\partial x_i \over\partial r_3} \right)
\]
and the derivatives of the inverse mapping as
\[
   \left({\partial r_1\over\partial x_i},\sin(\phi){\partial r_2 \over\partial x_i},
         {\partial r_3\over\partial x_i} \right).
\]
Here $\phi=\pi r_0$ is the parameter (latitude) for which the spherical singularities occur
at $\phi=0,\pi$.
See the implementation of the {\tt SphereMapping} or the {\tt RevolutionMapping} for two examples.

For cylindrical coordinates the derivatives of the mapping are computed as
\[
   \left(-\rho {\partial x_i \over\partial r_1},{1\over\rho}{\partial x_i \over\partial r_2},
         {\partial x_i \over\partial r_3} \right)
\]
and the derivatives of the inverse mapping as
\[
   \left({-1\over\rho}{\partial r_1\over\partial x_i},\rho{\partial r_2 \over\partial x_i},
         {\partial r_3\over\partial x_i} \right).
\]
Here the variable $\rho$, defined by
\begin{align*}
 \zeta &= 2 r_0 -1. \\
  \rho &= \sqrt{ 1- r_0^2 } 
\end{align*}
goes to zero at the singularity.
See the implementation of the ellipse in {\tt CrossSectionMapping.C} for an example of cylindrical
coordinates.
\end{description}



%% \input ReparameterizationTransformInclude.tex

