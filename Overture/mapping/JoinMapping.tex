\section{JoinMapping}\index{join mapping}\index{Mapping!JoinMapping}\index{intersecting surfaces}

\newcommand{\sourceMapping}{{\sl source mapping}}
\newcommand{\clipSurface}{{\sl clip surface}}

This mapping can be used to join together two Mappings that intersect. 
This is an alternative way to the {\tt FilletMapping} to connect two intersecting surfaces.

The protypical example of the use of a {\tt JoinMapping} is the intersection of
a wing (the {\sl source} mapping, i.e. the mapping that will be changed) 
with a fuselage (the {\sl clip-surface} mapping).
If the end of the wing does not match exactly to the fuselage, there will be a part of
of the wing that extends inside the fuselage. The {\tt JoinMapping} can be used to 
remove the part of the wing that is inside the fuselage and reparameterize the rest
of the wing so that the new wing matches exactly to the fuselage. 


\subsection{A 2D example}

In this first example we consider an annulus (source mapping) that intersects a
circle (the clip surface).  We generate a new mapping that consists of the
portion of the annulus that lies outside the circle. (We could also choose the
part of the annulus that lies inside the circle).

\noindent
\begin{minipage}{.4\linewidth}
{\footnotesize
\listinginput[1]{1}{\mapping/joinAnnulusToCircle.cmd}
}
\end{minipage}\hfill
\begin{minipage}{.6\linewidth}
  \begin{center}
  \includegraphics[width=12cm]{\figures/joinAnnulusToCircle1}\\
  \includegraphics[width=8cm]{\figures/joinAnnulusToCircle2}\\
  {A new mapping (bottom) is generated that replaces the annulus by a partial annulus that
   exactly matches to the circle.}
  \end{center}
\end{minipage}


\subsection{Intersecting surfaces}

Consider the case where the {\sl source mapping} and the {\sl clip-surface}
Mappings are both surfaces in 3D.
The {\tt JoinMapping} will first compute the curve of intersection between the {\sl source}
and the {\sl clip-surface} mappings.  
The curve of intersection, as generated by the {\tt IntersectionMapping},
will have three representations:
\begin{itemize}
   \item A space curve $\xv_i(s)$, $s\in[0,1]$. matching the curve of intersection in physical space.
   \item A curve in parameter space of the source-mapping, $\rv_i(s)$, $s\in[0,1]$, that is
       the pre-image of the space curve $\xv_i(s)$. Thus if $\xv=\Cv_i(\rv)$ denotes the source 
       mapping,  then $\xv_i(s)=\Cv_i(\rv_i(s))$.
   \item There is also a parametric curve for the clip-surface mapping, $\rv_e(s)$, with $\xv_i(s)=\Cv_e(\rv_i(s))$,
      where $\Cv_e(\rv)$ is the clip-surface mapping.
\end{itemize}
To reparmeterize the source mapping we first define a new mapping in the parameter space of the
source-mapping that is bounded on one side by the parametric intersection curve $\rv_i$. In the typical case this
new mapping can be defined by trans-finite interpolation ({\tt TFIMapping}), such as
\[
     \Pv(\rv) = (1-r_2) \rv_i(r_1) + r_2 (r_1,1)  
\]
In this case the curve $\rv_i$ is assumed to be mainly in the $r_1$ direction and we have chosen to extend the
patch to $r_2=1$ (we could also choose to extend to $r_2=0$ or some other value of $r_2$ (see the {\tt setEndOfJoin}
function).

The {\tt JoinMapping} is now defined by the composite Mapping,
\[
  \Jv(\rv)=\Cv_i(\Pv(\rv))
\]
Through the definitions we see that $\Jv(\rv)$ will exactly match the curve of intersection at $r_2=0$,
\[
   \Jv(r_1,0)=\Cv_i(\Pv(r_1,0)) = \Cv_i(\rv_i(r_1)) = \xv_i(r_1)
\]

\noindent
\begin{minipage}{.4\linewidth}
{\footnotesize
\listinginput[1]{1}{\mapping/joinTwoCyl.cmd}
}
\end{minipage}\hfill
\begin{minipage}{.6\linewidth}
  \begin{center}
    \includegraphics[width=9cm]{\figures/joinTwoCyl1}\\
    \includegraphics[width=9cm]{\figures/joinTwoCyl2}\\
  {A new surface mapping is generated (the upper `cylinder' in the bottom figure)
   that lies on the vertical cylinder (source-mapping) 
    and exactly matches to the horizontal cylinder (clip-surface).}
  \end{center}
\end{minipage}

\subsection{Intersecting a volume intersector mapping with a surface intersectee mapping.}

  Suppose now that source-mapping defines a volume mapping, $R^3\rightarrow R^3$.
The JoinMapping can be used to build a new volume Mapping that will match exactly to the
clip-surface. 

   In this case we require that the two faces of the source-mapping intersect the clip-surface,
say, $\Cv_i(r_1,r_2,0)$ and $\Cv_i(r_1,r_2,1)$. We proceed as before to generate a {\tt JoinMapping}
for each of these intersecting surfaces, $\Jv_m(\rv)$, $m=1,2$. We also generate a third...


\noindent
\begin{minipage}{.4\linewidth}
{\footnotesize
\listinginput[1]{1}{\mapping/joinTwoCyl.cmd}
}
\end{minipage}\hfill
\begin{minipage}{.6\linewidth}
  \begin{center}
  \includegraphics[width=9cm]{\figures/joinCyl1}\\
  \includegraphics[width=9cm]{\figures/joinCyl2}\\
  {A new volume mapping is generated that consists of a portion of the vertical cylinder (source-mapping)
   and exactly matches to the horizontal cylinder (clip-surface).}
  \end{center}
\end{minipage}

%% \input JoinMappingInclude.tex




