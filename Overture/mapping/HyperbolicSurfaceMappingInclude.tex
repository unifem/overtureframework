\subsection{Constructor}
 
\newlength{\HyperbolicSurfaceMappingIncludeArgIndent}
\begin{flushleft} \textbf{%
\settowidth{\HyperbolicSurfaceMappingIncludeArgIndent}{HyperbolicSurfaceMapping(}% 
HyperbolicSurfaceMapping() 
}\end{flushleft}
\begin{description}
\item[{\bf Purpose:}]  
    Create a mapping that can be used to generate a hyperbolic volume grid.
 
\end{description}
\subsection{Constructor}
 
\begin{flushleft} \textbf{%
\settowidth{\HyperbolicSurfaceMappingIncludeArgIndent}{HyperbolicSurfaceMapping(}% 
HyperbolicSurfaceMapping(Mapping \& surface\_, Mapping \& startingCurve) 
}\end{flushleft}
\begin{description}
\item[{\bf Purpose:}]  
    Create a mapping that can be used to generate a hyperbolic surface grid.
\item[{\bf surface\_ (input):}]  Generate the grid on this reference surface.
\item[{\bf startingCurve (input):}]  Grow the surface grid starting from this curve on the
    surface (the curve is defined in physical space, not in the parameter space).
 surface (3D)
\end{description}
\subsection{generate}
 
\begin{flushleft} \textbf{%
int  \\ 
\settowidth{\HyperbolicSurfaceMappingIncludeArgIndent}{generate(}%
generate( )
}\end{flushleft}
\begin{description}
\item[{\bf Purpose:}]  
    Generate the surface grid.
\end{description}
\subsection{setSurfaceAndCurve}
 
\begin{flushleft} \textbf{%
int  \\ 
\settowidth{\HyperbolicSurfaceMappingIncludeArgIndent}{setSurfaceAndCurve(}%
setSurfaceAndCurve(Mapping \& surface\_, Mapping \& startingCurve)
}\end{flushleft}
\begin{description}
\item[{\bf Purpose:}]  
    Supply the curve/surface from which the grid will be generated.
\item[{\bf surface\_ (input):}]  Generate the grid starting from this curve (2D) or
 surface (3D)
\end{description}
\subsection{setParameters}
 
\begin{flushleft} \textbf{%
int  \\ 
\settowidth{\HyperbolicSurfaceMappingIncludeArgIndent}{setParameters(}%
setParameters(const HyperbolicParameter \& par, \\ 
const IntegerArray \& ipar  = Overture::nullIntArray(), \\ 
const RealArray \& rpar  = Overture::nullRealArray(),\\ 
\hspace{\HyperbolicSurfaceMappingIncludeArgIndent}const Direction \& direction  = bothDirections)
}\end{flushleft}
\begin{description}
\item[{\bf Purpose:}]  
    Define a parameter for the hyperbolic grid generator.
\item[{\bf par (input):}]  The possible value come from the enum {\tt  HyperbolicParameter}:
  \begin{description}
   \item[linesInTheNormalDirecion] : specify the number of lines to use in the normal direction.
   \item[boundaryConditions] : ipar(0) = ibcja, ipar(1)=ibcjb
   \item[dissipation] : rpar(0) = explicit smoothing (smu), rpar(1)=implicit smoothing (tim)
   \item[distanceToMarch] : rpar(0) = distance
   \item[spacing] : rpar(0) = deta, rpar(1)=dfar
   \item[volumeParameters] : ipar(0) = itsvol
   \item[growInTheReverseDirection] : grow the grid in the reverse direction (this will
      result in a left handed coordinate system.
   \item[growInBothDirections] : grow the grid in both directions.
  \end{description}
\item[{\bf value (input):}] 
\item[{\bf direction (input) :}]  The hyperbolic surface can be grown in two possible directions
  (or both directions). {\tt direction} indicates which direction the new parameter 
   values should apply to: (enum Direction)
  \begin{description}
    \item[direction=bothDirections]  : parameters apply to both the forward and reverse directions.
    \item[direction=forwardDirection] : parameters apply to the forward direction.
    \item[direction=reverseDirection] : parameters apply to the reverse direction.
  \end{description}
\end{description}
\subsection{surfaceGridGenerator}
 
\begin{flushleft} \textbf{%
int  \\ 
\settowidth{\HyperbolicSurfaceMappingIncludeArgIndent}{surfaceGridGenerator(}%
surfaceGridGenerator()
}\end{flushleft}
\begin{description}
\item[{\bf Purpose:}]  
    Generate the surface grid.
\end{description}
