\section{AnnulusMapping}\index{annulus mapping}\index{Mapping!AnnulusMapping}


This Mapping defines a circular or elliptical annulus in two or three space dimensions
\begin{eqnarray*}
\theta &=&  2\pi( \theta_0 + r_1( \theta_1-\theta_0) ) \\
\xv(r_1,r_2) &=& (R_0 + r_2 (R_1-R_0)) ( \alpha \cos( \theta ) + x_0 , \sin(\theta) + y_0, z_0 )
\end{eqnarray*}
By default the annulus is parameterized with a left-handed coordinate system. You can 
make the system right handed by choosing the {\tt outerRadius} to be less than the
{\tt innerRadius}.

By default the annulus is two dimensional. To make a three dimensional annulus use the
{\tt setRangeDimension()} function or use the {\tt setOrigin(x0,y0,z0)} function with a
non-zero value of {\tt z0}.



%- \begin{figure}[ht]
%-   \begin{center}
%-   \includegraphics[width=10cm]{\figures/AnnulusMapping_idraw}
%-   % \epsfig{file=\figures/AnnulusMapping.idraw.ps,width=10cm}
%-   \caption{The AnnulusMapping defines an annulus}
%-   \end{center}
%- \label{fig:AnnulusMapping}
%- \end{figure}
%- 

%% \input AnnulusMappingInclude.tex

%- \begin{figure}[ht]
%-   \begin{center}
%-   \includegraphics[width=10cm]{\figures/annulus}
%-   % \epsfig{file=\figures/annulus.ps,width=.5\linewidth}
%-   \caption{A mapping for a partial annulus.}
%-   \end{center}
%- \end{figure}


\begin{figure}[hbt]
\newcommand{\figWidth}{10cm}
\newcommand{\trimfig}[2]{\trimFig{#1}{#2}{0.075}{.1}{.1}{.1}}
\begin{center}\small
% ------------------------------------------------------------------------------------------------
\begin{tikzpicture}
  \useasboundingbox (0,0.5) rectangle (10,10);  % set the bounding box (so we have less surrounding white space)
% 
  \draw (0, 0) node[anchor=south west,xshift=-4pt,yshift=-4pt] {\trimfig{\figures/annulus}{\figWidth}};
% grid:
%  \draw[step=1cm,gray] (0,0) grid (10,10);
\end{tikzpicture}
% ----------------------------------------------------------------------------------------
 \caption{A mapping for a partial annulus.}
\label{fig:AnnulusMapping}
\end{center}
\end{figure}
