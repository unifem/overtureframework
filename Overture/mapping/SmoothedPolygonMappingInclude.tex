\subsection{setPolygon}
 
\newlength{\SmoothedPolygonMappingIncludeArgIndent}
\begin{flushleft} \textbf{%
int  \\ 
\settowidth{\SmoothedPolygonMappingIncludeArgIndent}{setPolygon(}%
setPolygon( const RealArray \& xv,\\ 
const RealArray \& sharpness  = Overture::nullRealArray(),\\ 
\hspace{\SmoothedPolygonMappingIncludeArgIndent}const real normalDist  = 0.,\\ 
const RealArray \& variableNormalDist  = Overture::nullRealArray(),\\ 
const RealArray \& tStretch  = Overture::nullRealArray(),\\ 
const RealArray \& rStretch  = Overture::nullRealArray(),\\ 
\hspace{\SmoothedPolygonMappingIncludeArgIndent}const bool correctForCorners  = false)
}\end{flushleft}
 
\begin{description}
\item[{\bf Description:}] 
     Set the verticies for the smooth polygon and optionally set other properties.
 
   **finish me for setting other parameters***  
 
\item[{\bf xv (input) :}]  array of vertices, xv(0:numberOfVertices-1,0:1)
\item[{\bf sharpness (input) :}]  sharpness(0:numberOfVertices-1), if specified, provide the sharpness of each corner.
\item[{\bf normalDist (input) :}]  if non-zero then use this as the fixed normal distance (sign is important).
\item[{\bf variableNormalDist (input) :}]  variableNormalDistance(0:2,0:numberOfVertices-1) if provided this is
     the variable normal distance and transition exponent.
\item[{\bf tStretch (input) :}]  tStretch(0:1,0:numberOfVertices-1) tangential stretching weight and exponent
\item[{\bf rStretch (input) :}]  rStretch(0:2) radial stretching weight, exponent and location.

\end{description}
