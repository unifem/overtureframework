\newcommand{\shat}{\hat{s}}

%--------------------------------------------------------------
\section{LoftedSurfaceMapping: build a lofted surface}\index{lofted surface}\index{Mapping!LoftedSurfaceMapping}
%-------------------------------------------------------------


The LoftedSurfaceMapping can be used to define a lofted surface such as the
surface of a wing with a tip.  A lofted surface is defined using {\em sections}
(formulas that define the basic cross sections) and {\em profile} curves (that
can be used to deform the sections, such as to form a wing tip).
Figures~\ref{fig:JoukowskyLoftedSurface},
\ref{fig:WindTurbineBladeLoftedSurface} and~\ref{fig:BoxLoftedSurface} show examples of some lofted
surfaces.  

NOTE: The current intention is that users should edit the {\tt
LoftedSurfaceMapping.C} file to define a new lofted surface by changing the
formulas for the sections and profile curve.

{
\newcommand{\figWidth}{8cm}
\newcommand{\trimfig}[2]{\trimPlot{#1}{#2}{.0}{.0}{.25}{.25}}
\newcommand{\figWidtha}{12cm}
\newcommand{\trimfiga}[2]{\trimPlot{#1}{#2}{.0}{.0}{.35}{.35}}
\begin{figure}[hbt]
\begin{center}
\begin{tikzpicture}[scale=1]
  \useasboundingbox (0,.5) rectangle (16.5,8.5);  % set the bounding box (so we have less surrounding white space)
%
 \begin{scope}[yshift=4cm]
    \draw ( 0.0,0.) node[anchor=south west,xshift=-4pt,yshift=+0pt] {\trimfig{\figures/roundTipProfile}{\figWidth}};
    \draw(1,1.35) node[anchor=west] {\small $s=1$};
    \draw(1,2.75) node[anchor=west] {\small $s=0$};
    \draw(1,3.4) node[anchor=west] {\small $\shat=0$};
    \draw(7.05,2.1) node[anchor=west] {\small $\shat=1$};
    %
    \draw(3.25,2.1) node[anchor=east] {\small $w(\shat)$};
    \draw[<->,thick] (3.25,1.2) -- (3.25,3.10); 
    %
    \draw(5.7,2.1) node[anchor=west] {\small $s=s_T$};
    %
    \draw(4.0,1.4) node[anchor=west] {\small $\fv^b(s)$};
    \draw(4.0,2.75) node[anchor=west] {\small $\fv^t(s)$};
 \end{scope}
  \draw ( 8.2,4.) node[anchor=south west,xshift=-4pt,yshift=+0pt] {\trimfig{\figures/flatTipProfile}{\figWidth}};
  \draw ( 2.1,0.) node[anchor=south west,xshift=-4pt,yshift=+0pt] {\trimfiga{\figures/windTurbineProfile}{\figWidtha}};
% grid:
% \draw[step=1cm,gray] (0,0) grid (16,8);
\end{tikzpicture}
\end{center}
\caption{Some lofted surface profile curves. Top left: rounded-tip-profile. Top right: flat-tip-profile.
      Bottom: wind-turbine-profile.}
\label{fig:LoftedSurfaceProfileCruves}
\end{figure}
}

% {
% \newcommand{\figWidth}{9cm}
% \newcommand{\figWidthb}{13cm}
% \newcommand{\clipfig}[2]{\clipFig{#1}{#2}{.04}{.9}{.3}{.67}}
% \newcommand{\clipfigb}[2]{\clipFig{#1}{#2}{.02}{.9}{.375}{.60}}
% \begin{figure}[hbt]
% \begin{center}
% \begin{pspicture}(0,.0)(16.5,5.6)
%  \rput(4.  ,4.0){\clipfig{\figures/roundTipProfile.ps}{\figWidth}}
% %
%  \rput[l](1,5.4){$\shat=0$}\rput[l](7.8,4){$\shat=1$}
%  \rput[l](1,4.8){$s=0$}
%  \rput(5,4.8){$\fv^t(s)$}
%  \rput[r](6.7,4){$s=s_T$~~\psline[linewidth=1.pt]{->}(0,0)(.8,0)}
%  \rput(5,3.2){$\fv^b(s)$}
%  \rput[l](1,3.2){$s=1$}
%  \rput(3.,4){$w(\shat)$}\rput(3.5,4){\psline[linewidth=1.pt]{<->}(0,-1.1)(0,1.1)}
% %
%  \rput(12.75,4.0){\clipfig{\figures/flatTipProfile.ps}{\figWidth}}
%  \rput(8   ,.9){\clipfigb{\figures/windTurbineProfile.ps}{\figWidthb}}
% % 
% %\psgrid[subgriddiv=2]
% \end{pspicture}
% \end{center}
% \caption{Some lofted surface profile curves. Top left: rounded-tip-profile. Top right: flat-tip-profile.
%       Bottom: wind-turbine-profile.}
% \label{fig:LoftedSurfaceProfileCruves}
% \end{figure}
% }

The generic lofted surface is a three-dimensional surface, $\xv=\Lv(s,\theta)$, $\Lv : \Real^2\rightarrow \Real^3$,
that is a function of an axial variable $s$ and a tangential variable $\theta$. In the examples
given here, the surface is thought to extend primarily in the $z$ direction, and be of the form
\begin{align}
  \Lv(s,\theta) &= (x(s,\theta),y(s,\theta),z(s,\theta)) = ( \Sv(s,\theta), Z_L(s) ),  \label{eq:loftedSurface}
\end{align}
where the function $\Sv(s,\theta)$ defines the $(x,y)$ coordinates of the cross-sections and is usually of the form
\begin{align}
\Sv(s,\theta) &=   R(s,\theta) \big( \cv(s,\theta)~w(s) \big) + \gv(s) ,  \label{eq:loftedSectionCurve}
\end{align}
where 
\begin{align*}
  \cv(s,\theta) & = \text{basic unscaled cross-section curves}, \\
  w(s) &= \text{profile width function (that defines the shape of the tip)}, \\
  R(s,\theta) &= \text{twist and scale matrix operator}, \\
  \gv(s) &= \text{flex operator that bends the surface}.
\end{align*}
{\em Profile curves} are used to define $w(s)$ and $Z(s)$. 
Examples of profile curves are given in Figure~\ref{fig:LoftedSurfaceProfileCruves}.
In order to have a nicely defined
tip, we first define a single profile function $(x,y)= \fv^p(s)$, that is subsequently split into a top and bottom portion, 
 \begin{align*}
  \fv^p(s) &= \begin{cases}
              \fv^t(s) & \text{for $0<s<s_T$}, \\
              \fv^b(s) & \text{for $s_T<s<1$}, 
              \end{cases}
\end{align*}
where $s_T$ denotes the position of the ``tip''. 
The profile curves have $x$ and $y$ components denoted by $\fv^t(s)=(f_x^t(s),f_y^t(s))$ and $\fv^b(s)=(f_x^b(s),f_y^b(s))$.
From the profile curves we define the {\em profile width} as the (vertical) $y$-distance between the top
and bottom profile curves,
 \begin{align*}
  w(\shat) &= f^t_y(\shat\, s_T) - f^b_y(1- (1-s_T)\shat), \qquad \text{for $0 \le \shat \le 1$, ($\shat=s/s_T$) },
\end{align*}
while the axial distance function $z=Z_L(s)$ in~\eqref{eq:loftedSurface} is defined as the $x$-component of the top profile,
 \begin{align*}
    Z_L(\shat) &= f^t_x(\shat\, s_T), \qquad \text{for $0 \le \shat \le 1$}.
\end{align*}
Note that in the examples shown, $w(1)=0$, and thus the surface will close to a point at the tip. This point on the
surface is a coordinate singularity and later in this section we discuss how to deal with this
singularity when constructing an overlapping grid.


{
\newcommand{\figWidth}{8cm}
\newcommand{\trimfig}[2]{\trimPlot{#1}{#2}{.0}{.0}{.25}{.25}}
\begin{figure}[hbt]
\begin{center}
\begin{tikzpicture}[scale=1]
  \useasboundingbox (0,.5) rectangle (16.5,11.75);  % set the bounding box (so we have less surrounding white space)
%
  \draw ( 0.0,7.5) node[anchor=south west,xshift=-4pt,yshift=+0pt] {\trimfig{\figures/roundTipProfile}{\figWidth}};
  \draw ( 0.0,4.) node[anchor=south west,xshift=-4pt,yshift=+0pt] {\trimfig{\figures/twistedJoukowskyRoundTipTopView}{\figWidth}};
  \draw ( 0.0,0.) node[anchor=south west,xshift=-4pt,yshift=+0pt] {\trimfig{\figures/twistedJoukowskyRoundedTip}{\figWidth}};
% 
 \draw ( 8.0,7.5) node[anchor=south west,xshift=-4pt,yshift=+0pt] {\trimfig{\figures/flatTipProfile}{\figWidth}};
  \draw ( 8.0,4.) node[anchor=south west,xshift=-4pt,yshift=+0pt] {\trimfig{\figures/twistedJoukowskyFlatTipTopView}{\figWidth}};
  \draw ( 8.0,0.) node[anchor=south west,xshift=-4pt,yshift=+0pt] {\trimfig{\figures/twistedJoukowskyFlatTip}{\figWidth}};
% grid:
% \draw[step=1cm,gray] (0,0) grid (16,12);
\end{tikzpicture}
\end{center}
\caption{Lofted surface : generic wing with Joukowsky sections, variable chord, twist and flex. 
         Left: rounded tip wing showing the profile function (top) and two views of the surface. Right: flat tip wing showing the profile function (top) and two views of the surface. The profile functions act primarily to form the tip. A scale function
causes the chord to decrease through the main portion of the surface.}
\label{fig:JoukowskyLoftedSurface}
\end{figure}
}

% {
% \newcommand{\figWidth}{9cm}
% \newcommand{\clipfiga}[2]{\clipFig{#1}{#2}{.0}{1.}{.3}{.67}}
% \newcommand{\clipfig}[2]{\clipFig{#1}{#2}{.025}{1.}{.3}{.70}}
% \newcommand{\clipfigb}[2]{\clipFig{#1}{#2}{.05}{1.}{.25}{.70}}
% \begin{figure}[hbt]
% \begin{center}
% \begin{pspicture}(0,-1.)(16.5,8.70)
%  \rput(4.,7.5){\clipfiga{\figures/roundTipProfile.ps}{\figWidth}}
%  \rput(4.,4.0){\clipfig{\figures/twistedJoukowskyRoundTipTopView.ps}{\figWidth}}
%  \rput(4.,0.5){\clipfigb{\figures/twistedJoukowskyRoundedTip.ps}{\figWidth}}
% %
%  \rput(12.5,7.5){\clipfiga{\figures/flatTipProfile.ps}{\figWidth}}
%  \rput(12.5,4.0){\clipfig{\figures/twistedJoukowskyFlatTipTopView.ps}{\figWidth}}
%  \rput(12.5,0.5){\clipfigb{\figures/twistedJoukowskyFlatTip.ps}{\figWidth}}
% % 
% % \psgrid[subgriddiv=2]
% \end{pspicture}
% \end{center}
% \caption{Lofted surface : generic wing with Joukowsky sections, variable chord, twist and flex. 
%          Left: rounded tip wing showing the profile function (top) and two views of the surface. Right: flat tip wing showing the profile function (top) and two views of the surface. The profile functions act primarily to form the tip. A scale function
% causes the chord to decrease through the main portion of the surface.}
% \label{fig:JoukowskyLoftedSurface}
% \end{figure}
% }



Two lofted surfaces for a generic wing with twist and flex are in shown in Figure~\ref{fig:JoukowskyLoftedSurface}.
The basic section curve for this wing is a Joukowsky airfoil (defined in Section~\ref{sec:AirfoilMapping}), 
 \begin{align*}
  \cv(s,\theta) &= (x_J(\theta),y_J(\theta) =  \Jv(\theta) = \text{Joukowsky airfoil}.
\end{align*}
The {\em twist and scale matrix} $R$ in~\eqref{eq:loftedSectionCurve} is the product of
a rotation matrix and a scaling factor,
 \begin{align*}
  R(\zeta,\theta) &= \begin{bmatrix}
                   \cos(\phi) & -\sin(\phi) \\
                   \sin(\phi) & \cos(\phi) \\
                 \end{bmatrix} ~~ S(\zeta) 
\end{align*}
where the rotation angle $\phi=\phi(\zeta)$ and scale function $S(\zeta)$ are some function of the
scaled $z$ coordinate, $\zeta$, defined by 
\begin{align*}
   \zeta &= Z(s)/L, \\
    L &= Z(s_T) . \qquad \text{(length of the wing)}
\end{align*}
Normally on would use $\zeta$ as a variable rather than $s$ when defining properties of the surface in physical coordinates. 
For example, with the flat-tip-profile in Figure~\ref{fig:LoftedSurfaceProfileCruves}, $s$ continues to vary along
the flat face even though $\zeta$ does not.
In the example shown in Figure~\ref{fig:JoukowskyLoftedSurface}
we choose $\phi$ to vary between $\phi_a=0$ and $\phi_b=-30^o$ by the formula
\begin{align*}
  \phi(\zeta) &= \phi_a (1-\zeta^2) + \phi_b \zeta^2, 
\end{align*}
while the scale function varies between $S_a=1$ and $S_b=.7$ according to 
\begin{align*}
  S(\zeta) &= S_a (1-\zeta^3) + S_b \zeta^3 . 
\end{align*}
The scale function acts to shorten the chord and make the surface thinner.
The flex function is chosen to bend the surface in the $y$-direction and is given by
\begin{align*}
  \gv(\zeta) &= (0,g_y(\zeta)), \\
  g_y(\zeta) &= g_a (1-\zeta^3) + g_b \zeta^3, \\
  g_a&=0, ~~ g_b=1/4~ . 
\end{align*}
The powers of $\zeta$ used in these formulae are arbitrary.


% \begin{align*}
%   \Jv(\theta) & = \text{Joukowsky airfoil} \\
%   W(s) &= P^t_y(s) - P^b_y(s),  \\
%   \Sv_0(s,\theta) &= \Jv(\theta)~W(s) , \qquad \text{multiply section by profile width,} \\
%   z &= P_x(s), \qquad \zeta  = z/L, \\
%   \Sv(s,\theta) &= R(\zeta) \Sv_0(s,\theta) + \fv(\zeta), \qquad \text{apply twist and flex}.
% \end{align*}
% 
% The {\em rounded tip profile} curve is used to form a rounded tip with an
% elliptical end. The {\em flat tip profile} curve is used to form a (nearly) flat wing tip. 
% 
% 
% 
% A lofted surface for a wind-turbine blade in shown in figure~\ref{fig:wingTurbineLoftedSurface} together
% with the profile curve.
% The formula for the lofted surface is 
% % \begin{align*}
% % \end{align*}


{
\newcommand{\figWidth}{8cm}
\newcommand{\trimfig}[2]{\trimPlot{#1}{#2}{.0}{.0}{.30}{.20}}
\newcommand{\figWidtha}{8cm}
\newcommand{\trimfiga}[2]{\trimPlot{#1}{#2}{.0}{.0}{.20}{.25}}
\begin{figure}[hbt]
\begin{center}
\begin{tikzpicture}[scale=1]
  \useasboundingbox (0,.75) rectangle (16.5,8);  % set the bounding box (so we have less surrounding white space)
%
  \draw ( 0.0,4.) node[anchor=south west,xshift=-4pt,yshift=+0pt] {\trimfig{\figures/windTurbineProfile}{\figWidth}};
  \draw ( 8.0,4.) node[anchor=south west,xshift=-4pt,yshift=+0pt] {\trimfig{\figures/cylinderJoukowskyWindTurbineTopView}{\figWidth}};
  \draw ( 4.0,0.) node[anchor=south west,xshift=-4pt,yshift=+0pt] {\trimfiga{\figures/cylinderJoukowskyWindTurbine}{\figWidtha}};
% grid:
%\draw[step=1cm,gray] (0,0) grid (16,8);
\end{tikzpicture}
\end{center}
\caption{Lofted surface : wind-turbine blade showing the profile (top left) and two views of the wing. The
profile function determines the overall shape and the tip.}
\label{fig:WindTurbineBladeLoftedSurface}
\end{figure}
}



The wind-turbine surface in Figure~\ref{fig:WindTurbineBladeLoftedSurface} uses the profile function 
to shape the entire surface. The unscaled cross-section function $\cv(s,\theta)$ starts as a circle
and transitions to a Joukowsky airfoil. 

{\bf Notes:} 
\begin{enumerate}
  \item In the examples represented here the surface converges to a point at the tip. This is a result
        of the profile width function $w(s)$ going to zero as $s$ goes to 1. In order that the 
       surface near the tip be single valued, it is important that the point $(x,y)=(0,0)$ be inside the
       the cross-section curves $\cv(s,\theta)$ as $s$ approaches 1. To achieve this property we 
       shift the basic cross-section (such as the Joukowsky profile) so that this is true. We then 
       rely on the rotation, scaling  and flex functions to define the actual shape (note that these
      functions as applied after we multiply by $w(s)$).
\end{enumerate}


Figure~\ref{fig:BoxLoftedSurface} shows a surface for a box with rounded corners generated
with the lofted surface mapping. The profile is the {\em flat double tip profile}. The sections
are defined from a smoothed polygon mapping that blends into a circle at the tips. 
This mapping can be used to create a grid around the exterior of a box. 
{
\newcommand{\figWidth}{8cm}
\newcommand{\trimfig}[2]{\trimPlot{#1}{#2}{.0}{.0}{.25}{.25}}
\newcommand{\figWidtha}{8cm}
\newcommand{\trimfiga}[2]{\trimPlot{#1}{#2}{.0}{.0}{.0}{.0}}
\begin{figure}[hbt]
\begin{center}
\begin{tikzpicture}[scale=1]
  \useasboundingbox (0,.75) rectangle (16.5,8);  % set the bounding box (so we have less surrounding white space)
%
  \draw ( 0.0,2.) node[anchor=south west,xshift=-4pt,yshift=+0pt] {\trimfig{\figures/loftedSurfaceFlatDoubleTipProfile}{\figWidth}};
  \draw ( 8.0,0.) node[anchor=south west,xshift=-4pt,yshift=+0pt] {\trimfiga{\figures/loftedSurfaceSmoothPolygonFlatDoubleTip}{\figWidtha}};
% grid:
%  \draw[step=1cm,gray] (0,0) grid (16,8);
\end{tikzpicture}
\end{center}
\caption{Lofted surface : box with rounded corners constructed from the {\em smooth polygon sections} and the {\em flat double tip profile}. Left: profile. Right: lofted surface for the box. Note that the cross-sections transition to a circle at the singular poles.}
\label{fig:BoxLoftedSurface}
\end{figure}
}


% --------------------------------------------------------------------------------------------------------
\subsection{Lofted surface for a Wigely ship hull}\index{ship hull}

{
\newcommand{\figWidth}{8cm}
\newcommand{\trimfig}[2]{\trimPlot{#1}{#2}{.0}{.0}{.15}{.15}}
\begin{figure}[hbt]
\begin{center}
\begin{tikzpicture}[scale=1]
  \useasboundingbox (0,.75) rectangle (16.5,6);  % set the bounding box (so we have less surrounding white space)
%
  \draw ( 0.0,0.) node[anchor=south west,xshift=-4pt,yshift=+0pt] {\trimfig{\figures/loftedShipHullFull}{\figWidth}};
  \draw ( 8.0,0.) node[anchor=south west,xshift=-4pt,yshift=+0pt] {\trimfig{\figures/loftedShipHull}{\figWidth}};
% grid:
%  \draw[step=1cm,gray] (0,0) grid (16,6);
\end{tikzpicture}
\end{center}
\caption{Lofted surface for a Wigley ship hull is constructed from the {\em ship hull sections} and the {\em flat double tip profile}. 
    Full revolution (left) and half revolution (right), using $L=2$, $B=L/10$ and $D=L/16$.}
\label{fig:loftedShipHull}
\end{figure}
}


A lofted surface for a ship hull is shown in Figure~\ref{fig:loftedShipHull}.
The ship hull is of the form of a Wigley hull. Wigley hull's form a class of
simple hulls that are used in validation.
The cross-section we use is defined as
 \begin{align*}
  \cv(s,\theta) &= (x_S(s,\theta),y_S(s,\theta)), \\
  x_S(s,\theta) &= b(s) \cos(\theta), \\
  y_S(s,\theta) &= d(s) \sin(\theta),  \\
  b(s) &=   2 B \hat{z} (1- \hat{z}) , \\
  \hat{z} &= (1-2\delta) (z/ L) + \delta, \\
  d(s) &= D ,
\end{align*}
where $L$ is the length, $B$ is the breadth (width), and $D$ is the depth (draft).


The breadth of the hull is parabolic in shape 
as a function of $z$ (this is the basic characteristic of Wigley hulls), given approximately by $ 2 B (z/L)(1-z/L)$.
To avoid the sharp bow/stern at $z=0$ and $z=L$ we round off the hull a bit. This is done in a simple way by making
the breadth finite at the ends as determined by the small parameter $\delta$ (e.g. $\delta=.02$). 
The {\em flat double tip profile} is used to close off the bow and stern as a smooth surface.

The cross-section shape is taken as an ellipse of constant depth and variable breadth. Different Wigley
hulls can be defined with different cross sections (e.g. square, rectangular).
Note that in order to make a nice symmetric surface we first construct the hull as a full revolution in $\theta$. 
The Reparameterization Mapping can then be used to create just half the surface.

An overlapping grid for the ship hull is constructed with the Ogen script {\tt loftedShipHullGrid.cmd}, 
see the Ogen documentation~\cite{OGEN} for further details.
