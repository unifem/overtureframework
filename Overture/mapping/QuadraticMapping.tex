%--------------------------------------------------------------
\section{QuadraticMapping: define a quadratic curve or surface.}
\index{quadratic mapping}\index{Mapping!QuadraticMapping}
\index{parabola}\index{paraboloid}
\index{hyperbola}\index{hyperboloid}
%-------------------------------------------------------------


Use this mapping to define a quadratic curve or surface.

A parabola (curve in 2D) is defined by
\begin{align*}
   x_0 &= c_{0x} + c_{1x} r_0 \\
   x_1 &= a_{00} + a_{10} x_0 + a_{20} x_0^2
\end{align*}
A 3d paraboloid (surface) is defined by
\begin{align*}
   x_0 &= c_{0x} + c_{1x} r_0 \\
   x_1 &= c_{0y} + c_{1y} r_1 \\
   x_2 &= a_{00} + a_{10} x_0 + a_{01} x_1 + a_{20} x_0^2 + a_{11}x_0 x_1 + a_{02} x_1^2
\end{align*}

A hyperbola (2d curve) is defined by
\begin{align*}
   x_0 &= c_{0x} + c_{1x} r_0 \\
   x_1 &= \pm (a_{00} + a_{10} x_0 + a_{20} x_0^2 )^{1/2}
\end{align*}

A 3d hyperboloid (surface) is defined by
\begin{align*}
   x_0 &= c_{0x} + c_{1x} r_0 \\
   x_1 &= c_{0y} + c_{1y} r_1 \\
   x_2 &= \pm (a_{00} + a_{10} x_0 + a_{01} x_1 + a_{20} x_0^2 + a_{11}x_0 x_1 + a_{02} x_1^2)^{1/2}
\end{align*}



\subsection{Examples}
  \begin{center}
  \includegraphics[width=9cm]{\figures/parabola} \\
  % \epsfig{file=\figures/parabola.ps,height=.6\linewidth}  \\
  {A 2D parabola.} \\
  \includegraphics[width=9cm]{\figures/paraboloid} \\
  % \epsfig{file=\figures/paraboloid.ps,height=.6\linewidth}  \\
  {A 3D parabolic surface.}
  \end{center}

%% \input QuadraticMappingInclude.tex

