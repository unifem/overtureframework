%--------------------------------------------------------------
\section{AirfoilMapping: create some airfoil related grids or curves}\label{sec:AirfoilMapping}
%-------------------------------------------------------------

\subsection{NACA airfoils}\index{airfoil mapping}\index{Mapping!AirfoilMapping}\index{NACA}

The NACA 4 digit series airfoils (such as the NACA0012) are defined by 
\begin{alignat*}{2}
  x_u(r) &= (r - y_t(r) \sin(\theta) ) c   &\qquad &\mbox{upper surface}\\
  y_u(r) &= (y_c(r) + y_t(r) \cos(\theta) ) c &\qquad &\mbox{upper surface}\\
  x_l(r) &= (r + y_t(r) \sin(\theta)  ) c   &\qquad &\mbox{lower surface}\\
  y_l(r) &= (y_c(r) - y_t(r) \cos(\theta) ) c &\qquad &\mbox{lower surface}
\end{alignat*}
where $c$ is the chord length. The camber line, $y_c$, is defined by
\begin{align*}
   y_c(r) & = c_{\rm max} {1\over x_1^2} ( 2 x_1 r - r^2 ) \qquad\mbox{for $0\le r \le x_1$} \\
   y_c(r) & = c_{\rm max} {1\over (1-x_1)^2} ( (1-2 x_1) + 2 x_1 r - r^2 ) \qquad\mbox{for $x_1\le r \le 1$} \\
   x_1  &= \mbox{ position of the maximum camber}
\end{align*}
and the thickness is defined by
\[
   y_t(r) = 5 \delta( 0.29690\sqrt{r} - 0.12600 r -0.35160 r^2 + .28430 r^3 - 0.10150 r^4 )
\]
where
\begin{align*}
   \delta &= \mbox{thickness/chord}
\end{align*}

The NACA[c][p][tc] airfoil is defined by:
\begin{description}
 \item[c]  maximum camber/chord $\times 100$  ($c_{\rm max}\times 100$).
 \item[p]  position of maximum camber/chord $\times 10$ ($x_1\times 10$).
 \item[tc]  thickness/Chord $\times 100$ ($\delta\times100$).
\end{description}
Thus the NACA0012 has $c_{\rm max}=0$, $x_1=0$ and $\delta=.12$.

\subsection{Joukowsky Airfoil}

    The Joukowsky airfoil is defined by
\begin{align*}
   z &= x + i y = w + {1\over w} \qquad\mbox{z and w are complex numbers}\\
   w &= a e^{i\theta} + i d e^{i\delta} \\
   \theta = 2\pi r_0
\end{align*}
The parameters $a,d,\delta$ in the definition have the following approximate properties,
\begin{description}
  \item[a] : $<=1$, closer to 1 implies sharper trailing edge.
  \item[d] : bigger d implies larger camber.
  \item[$\delta$] : bigger $\delta$ implies greater asymmetry between leading and trailing edges.
\end{description}
 

%% \subsection{Member function descriptions}
%% \subsubsection{Constructor}
 
\newlength{\AirfoilMappingIncludeArgIndent}
\begin{flushleft} \textbf{%
\settowidth{\AirfoilMappingIncludeArgIndent}{AirfoilMapping(}% 
AirfoilMapping(const AirfoilTypes \& airfoilType\_, \\ 
\hspace{\AirfoilMappingIncludeArgIndent}const real xa  = -1.5, \\ 
\hspace{\AirfoilMappingIncludeArgIndent}const real xb  = 1.5, \\ 
\hspace{\AirfoilMappingIncludeArgIndent}const real ya  = 0., \\ 
\hspace{\AirfoilMappingIncludeArgIndent}const real yb  = 2.) 
}\end{flushleft}
\begin{description}
\item[{\bf Description:}] 
    Create a mapping for an airfoil.
\item[{\bf Notes:}]  An airfoil mapping can be made from oneof the following (enum AirfoilTypes)
  \begin{description}
   \item[arc] : grid with a bump on the bottom that is an arc of a circle.
   \item[sinusoid] : grid with a bump on the bottom that is an sinusoid.
   \item[diamond] : grid with a bump on the bottom that is a diamond.
   \item[naca] : a curve that is one of the NACA 4 digit airfoils.
   \item[joukowsky] : a curve defining a Joukowsky airfoil.
  \end{description}
\item[{\bf airfoilType\_ (input):}]  an airfoil type from the above choices.
\item[{\bf xa,xb,ya,yb (input) :}]  boundaries of the bounding box (not used for naca airfoils).
\end{description}
\subsubsection{setBoxBounds}
 
\begin{flushleft} \textbf{%
int  \\ 
\settowidth{\AirfoilMappingIncludeArgIndent}{setBoxBounds(}%
setBoxBounds(const real xa  =-1.5, \\ 
\hspace{\AirfoilMappingIncludeArgIndent}const real xb  =1.5, \\ 
\hspace{\AirfoilMappingIncludeArgIndent}const real ya  =0., \\ 
\hspace{\AirfoilMappingIncludeArgIndent}const real yb  =2.)
}\end{flushleft}
\begin{description}
\item[{\bf Description:}] 
 set bounds on the rectangle that the airfoil sits in
\item[{\bf xa,xb,ya,yb (input) :}]  boundaries of the bounding box (not used for naca airfoils).
\end{description}
\subsubsection{setParameters}
 
\begin{flushleft} \textbf{%
int   \\ 
\settowidth{\AirfoilMappingIncludeArgIndent}{setParameters(}%
setParameters(const AirfoilTypes \& airfoilType\_,\\ 
\hspace{\AirfoilMappingIncludeArgIndent}const real \& chord\_  =1., \\ 
\hspace{\AirfoilMappingIncludeArgIndent}const real \& thicknessToChordRatio\_  =.1,\\ 
\hspace{\AirfoilMappingIncludeArgIndent}const real \& maximumCamber\_  =0.,\\ 
\hspace{\AirfoilMappingIncludeArgIndent}const real \& positionOfMaximumCamber\_  =0.,\\ 
\hspace{\AirfoilMappingIncludeArgIndent}const real \& trailingEdgeEpsilon\_   =.02,\\ 
\hspace{\AirfoilMappingIncludeArgIndent}const real \& sinusoidPower\_  = 1.)
}\end{flushleft}
\begin{description}
\item[{\bf Description:}] 
    Create a mapping for an airfoil.
\item[{\bf Notes:}]  An airfoil mapping can be made from oneof the following (enum AirfoilTypes)
  \begin{description}
   \item[arc] : grid with a bump on the bottom that is an arc of a circle.
   \item[sinusoid] : grid with a bump on the bottom that is an sinusoid (or power of a sinusoid).
   \item[diamond] : grid with a bump on the bottom that is a diamond.
   \item[naca] : a curve that is one of the NACA 4 digit airfoils.
   \item[joukowsky] : Joukowsky airfoil. The other parameters in the argument list
      do not apply in this case. Use the {\tt setJoukowskyParameters} function instead.
  \end{description}
\item[{\bf airfoilType\_ (input):}]  an airfoil type from the above choices.
\item[{\bf chord\_ (input):}]  length of the chord.
\item[{\bf thicknessToChordRatio\_ (input):}]  thickness to chord ratio.
\item[{\bf maximumCamber\_ (input):}]  maximum camber
\item[{\bf positionOfMaximumCamber\_ (input):}]  position of maximum camber
\item[{\bf trailingEdgeEpsilon\_ (input) :}]  parameter for rounding the trailing edge.
\end{description}
\subsubsection{setJoukowskyParameters}
 
\begin{flushleft} \textbf{%
int  \\ 
\settowidth{\AirfoilMappingIncludeArgIndent}{setJoukowskyParameters(}%
setJoukowskyParameters( const real \& a, const real \& d, const real \& delta )
}\end{flushleft}
\begin{description}
\item[{\bf Description:}] 
    Set parameters for the Joukowsky airfoil.
\item[{\bf a,d,delta :}]  see the documentation for a desciption of these.
\end{description}


\subsection{Examples}

\noindent
\begin{minipage}{.475\linewidth}
  \begin{center}
   \includegraphics[width=9cm]{\figures/airfoilArc}
   % \epsfig{file=\figures/airfoilArc.ps,width=\linewidth}  \\
  {Airfoil grid created with {\tt airfoilType=arc}}
  \end{center}
\end{minipage}\hfill
\begin{minipage}{.475\linewidth}
  \begin{center}
   \includegraphics[width=9cm]{\figures/airfoilDiamond}
   % \epsfig{file=\figures/airfoilDiamond.ps,width=\linewidth}  \\
  {Airfoil grid created with {\tt airfoilType=diamond}}
  \end{center}
\end{minipage}
\noindent
\begin{minipage}{.475\linewidth}
  \begin{center}
   \includegraphics[width=9cm]{\figures/airfoilSinusoid}
   % \epsfig{file=\figures/airfoilSinusoid.ps,width=\linewidth}  \\
  {Airfoil grid created with {\tt airfoilType=sinusoid}}
  \end{center}
\end{minipage}\hfill
\begin{minipage}{.475\linewidth}
  \begin{center}
   \includegraphics[width=9cm]{\figures/airfoilNACA0012}
   % \epsfig{file=\figures/airfoilNACA0012.ps,width=\linewidth}  \\
  {NACA0012 airfoil created with {\tt airfoilType=naca}}
  \end{center}
\end{minipage}


