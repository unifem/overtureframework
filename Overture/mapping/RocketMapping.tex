\section{RocketMapping: create rocket geometry curves} \label{sec:RocketMapping}
\index{Rocket mapping}\index{Mapping!RocketMapping}

  The {\tt RocketMapping} defines a variety of curves related to rockets. The curves defined
in this class were originally written by Nathan Crane (as three separate Mapping's), and 
then subsequently reorganized into a single class by WDH. There are currently 3 cross-section
shapes supported, the {\bf slot}, {\bf star} and {\bf circular} shapes. The slot and star 
shapes are illustrated in the next figure.

\begin{minipage}{.5\linewidth}
  \begin{center}
   \includegraphics[width=9cm]{\figures/slot_mapping_pic}\\
   % \epsfig{file=\figures/slot_mapping_pic.ps,width=.95\linewidth}  \\
  {The slot shape.}
  \end{center}
\end{minipage}
\begin{minipage}{.5\linewidth}
  \begin{center}
   \includegraphics[width=9cm]{\figures/star_mapping_pic}\\
   % \epsfig{file=\figures/star_mapping_pic.ps,width=.95\linewidth}  \\
  {The star cross-section.}
  \end{center}
\end{minipage}


\subsection{Slot}

{\bf Overview} : The slot option creates by default a sloted grain  shaped spline in the z=0 plane.
           The mapping should be usable in every way as a standard spline.  The slot spline mapping
           will always be periodic, the number and location of the spline knot points are
           generated automatically according to the slotted grain input parameters.  A graphical
           description of the various parameters can found in the above figure.

Options:
\begin{description}
  \item[set range dimension] : Toggle between a 2D and 3D spline (spline will always lie in a plane,
   but in 3D that plane can be rotated or shifted to a arbitrary positon.)
  \item[shape preserving (toggle)] : toggle between shape preserving and tension spline (see standard
   spline mapping documentation for more info)
  \item[set bounding radii] : Set the inner and outer bounding radii for the slot grain.
   slots will just touch a circle of radius outer bounding radius.  The slots will intersect a circle
   of radius inner bounding radius.
  \item[set slot width] : Set the width of each slot.  The slots wil be rounded on the ends by a
    circle of diameter slot width.
  \item[set z value] : by default the spline lies in the z=0 plane.  Changing the z value moves the
   spline to some other constant z value plane.  This command is a shortcut for the shift operator.
   This command cannot be used with a 2D sloted grain mapping.
  \item[set element size] : Set the size of the elements along the spline.  The total length of the
   spline is computed, and the number of lines of grid points is taken as total\_length/el\_size. the
   number of spline knot points is taken as the same as the number of lines.  Element size and
   number of lines are mutally exclusive commands.
  \item[set number of vertex] : Set the number of vertices (number of slots) of the mapping valid
     values are 2 vertices and up.
\end{description}

% \begin{figure}[H]
%   \begin{center}
%    \epsfig{file=\figures/slot_mapping_pic.ps,width=.4\linewidth} % .7
%   \caption{The slot cross-section}  \label{fig:slot}
%   \end{center}
% \end{figure}

\subsection{Star}

{\bf Overview}: The star option creates by default a star shaped spline in the z=0 plane.  The
           mapping should be usable in every way as a standard spline.  The star spline mapping will
           always be periodic, the the number and location of the spline knot points are generated 
           automatically according to the star input parameters.  A graphical description of the
           various parameters can found in the above figure.

Options:
\begin{description}
  \item[set range dimension] : Toggle between a 2D and 3D spline (spline will always lie in a plane,
    but in 3D that plane can be rotated or shifted to a arbitrary positon.)
  \item[shape preserving (toggle)] : toggle between shape preserving and tension spline (see standard
    spline mapping documentation for more info)
  \item[set bounding radii] : set the inner and outer bounding radii for the star.  The inside points
    the star will be circimscribed between the inner and outer radi.  The outer points of the star will
    just touch a circle of radius outer bounding radius.  The inner points of the star will just touch
    a circle of radius inner bounding radius.
  \item[set fillet radii] : set the inner and outer fillet radii for the star.  The fillet radi
    determine the sharpness of the points of the star.  Large fillet radii create a more blun star, while
    smaller fillet radi create a sharper star.  Note that when createing in 3D volume, a sharper pointed
    star will require a thiner boundry mesh, and thus more elements to mesh than a blunted star.
  \item[set z value] : by default the star lies in the z=0 plane.  Changing the z value moves the  
    star to some other constant z value plane.  This command is a shortcut for the shift operator.  This
    command cannot be used with a 2D star.
  \item[set element size] : Set the size of the elements along the star.  The total length of the
    star spline is computed, and the number of lines of grid points is taken as total\_length/el\_size.
    the number of spline knot points is taken as the same as the number of lines.  Element size and
    number of lines are mutally exclusive commands.
 \item[set number of vertex] : Set the number of vertices (number of arms) of the star valid values
   are 2 vertices and up.  For instance a space shuttle fuel grain is described by a 11 vertex star.
 \item[set number of points] : explicitly sets the number of knot points for the spline.  This
  command overrides the set element size command.
\end{description}

% \begin{figure}[H]
%   \begin{center}
%    \epsfig{file=\figures/star_mapping_pic.ps,width=.4\linewidth} % .7
%   \caption{The star cross-section}  \label{fig:star}
%   \end{center}
% \end{figure}


\subsection{circle}

{\bf Overview} : The CircSplineMapping creates by default a circular spline in the z=0 plane.  The
           mapping should be usable in every way as a standard spline.  The circular spline mapping
           will always be periodic, the the number and location of the spline knot points are
           generated automatically according to the circle parameters.  The circualr spline mapping
           is need when creating rocket cross sections to correctly parameterize the star, anular,
           and sloted grain portions in a compatible way.

%% \subsection{Member functions}
%% \input RocketMappingInclude.tex

