\documentclass[letterpaper,12pt]{article}
\usepackage[bookmarks=true]{hyperref}

\usepackage{epsfig}
\usepackage{subfigure}

\input{pstricks}
%\input{pst-node}
\input{colours}
\usepackage{chngpage}
\input{llnlCoverPage}
\usepackage{calc}
\usepackage{program}

\usepackage{epsfig}

%\voffset=-1.25truein
\hoffset=-1.truein
\setlength{\textwidth}{7in}      % page width
%\setlength{\textheight}{9.5in}    % page height for xdvi

\begin{document}


\def\uvd    {{\bf U}}
\def\ud     {{    U}}
\def\pd     {{    P}}
\def\id     {i}
\def\jd     {j}
\def\kap {\sqrt{s+\omega^2}}

\newcommand{\figures}{/home/chand/figures/smesh}

\newcommand{\umap}{{\tt UnstructuredMapping} }
\newcommand{\umapI}{{\tt UnstructuredMappingIterator} }
\newcommand{\umapAI}{{\tt UnstructuredMappingAdjacencyIterator} }
\newcommand{\forceb}{\linebreak[4]}
\newcommand{\smesh}{{\tt smesh}~}
\newcommand{\overt}{{\tt Overture}~}
\newcommand{\bib}{/home/chand/papers/bib}

%\makeLLNLCover{UCRL-MA-153081}{\smesh User's Guide}{K.K.~Chand}{}{}{0in}{0in}

\vspace{3\baselineskip}
\begin{flushleft}
  {\Large 
    \smesh User's Guide\\
  }
  A 2D unstructured mesh generator for \overt\\
\vspace{2\baselineskip}
Kyle K. Chand \\
Centre for Applied Scientific Computing \\
Lawrence Livermore National Laboratory    \\
Livermore, CA, 94551   \\
chand1@llnl.gov \\
http://www.llnl.gov/casc/Overture \\
UCRL-MA-153081 \\
\vspace{1\baselineskip}
\today
\vspace{\baselineskip}
\end{flushleft}

\begin{abstract}
\smesh is a general purpose, interactive, 2D unstructured mesh
generator based on \overt.  It supports three kinds of mesh generation
techniques : structured patches with transfinite interpolation (TFI);
unstructured triangles based on an advancing front technique; and a
Cartesian cutcell/triangle hybrid method.  Meshes are generated in a
generalized ``mulit-block'' manner where each ``block'', or region,
can be one of the three mesh types. Geometry definitions can be
created interactively by placing points and interpolating curves.
Spacing information is provided by both the curve discretization
(which can be stretched) and a user specified preferred grid spacing
for a region.  A mesh optimization procedure is available for the
non-TFI regions for mesh quality improvement.  Each mesh region is
given an unique identifier and an optional string name.  Meshes are
exported to a modified ``ingrid'' format including mesh region
identifiers and names.  Facilities for command scripting and batch
running are available.

\end{abstract}

\tableofcontents

\section{Introduction}
\smesh can be used to generate two dimensional unstructured meshes.
It can be used interactively, as in Figure~\ref{fig:smeshgui}, or
without graphics in a batch mode using scripted commands.  The program
can be invoked on the command line using:
\begin{verbatim}
 smesh [noplot] [nopause] [command_file]
\end{verbatim}
When started, \smesh will either process a command file or, if a file
is not specified, begin an interactive session. The optional command
file is a script of \smesh commands; this script is usually adapted
from the log of a previous interactive session.  \smesh stores all the
commands executed during a run (interactive or batch) in a file called
{\tt smesh.cmd}.  This file can be given to \smesh as a command file
to reproduce the interactive session.  {\tt noplot} starts \smesh
without graphics; the code will then be driven by text input.  When
{\tt nopause} is specified, ``pause'' commands in the command file are
ignored.
\begin{figure}[htb]
\begin{center}
\epsfig{file=\figures/smesh.screenshot.ps,width=10cm}
\end{center}
\caption{Screenshot of \smesh in action}\label{fig:smeshgui}
\end{figure}

Generating a mesh with \smesh proceeds with the following steps:
\begin{itemize}
\item create points and curves
\item optionally choose point distributions on curves
\item use a collection of curves to define and generate a region
\item make more geometry/regions
\item output the mesh to a text file
\end{itemize}
Figure~\ref{fig:maingui} shows the main interface used to perform
these tasks.  Clicking the ``Create Curves'' button brings up the
curve generation interface discussed in Section~\ref{sec:curve}.  A
variety of operations can be performed using mouse clicks, including:
deletion of curves and regions; specifying curve point density and
stretching; specifying text names for regions; and directing the code
to optimize a specific unstructured region.  The region creation
buttons bring up ``wizard'' like interfaces for the generation of
transfinite interpolation ({TFI}) or unstructured regions using a
collection of the already defined curves.  ``Save Mesh'' allows the
user to save the grid in the text format described in
Section~\ref{sec:fileformat}.  Toggle buttons are used to control the
plotting of various pieces of information. For example, ``Plot
Reference Grids'' will plot Cartesian grids bounding each region with
the nominal target grid spacing for each respective region.  Finally,
a text box a the bottom of the window can be used to specify the
default grid spacing for new regions (note this can be overridden when
a region is created).
\begin{figure}[htb]
\begin{center}
\begin{pspicture}(0,0)(14,10)
\rput(7,5){\epsfig{file=\figures/maingui.ps,width=8cm}}
%\psgrid[subgriddiv=2]
\end{pspicture}
\end{center}
\caption{The main command window}\label{fig:maingui}
\end{figure}
\begin{figure}[htb]
\begin{center}
\begin{pspicture}(0,2)(14,9)
\rput(7,5){\psclip{\psframe[linecolor=white](0,4)(14,9.5)}\epsfig{file=\figures/3mesh.ps,width=14cm}\endpsclip}
\rput(.5,8){\makebox(0,0)[l]{cutcell hybrid mesh}}
\rput(4.5,8){\makebox(0,0)[l]{unstructured triangles}}
\rput(9,8){\makebox(0,0)[l]{quadrilaterals using {TFI}}}
\psline[linewidth=1pt]{->}(2.25,7.75)(1,5)
\psline[linewidth=1pt]{->}(6.5,7.75)(5,6)
\psline[linewidth=1pt]{->}(11.25,7.75)(10,6)
%\psgrid[subgriddiv=2]
\end{pspicture}
\end{center}
\caption{The three types of unstructured mesh generation supported by \smesh}\label{fig:3mesh}
\end{figure}

Figure~\ref{fig:3mesh} illustrates the three kinds of unstructured
mesh generation supported by \smesh.  Each one of these mesh regions
begins with the specification of an outer bounding curve (inner curves
are also possible for non-{TFI} regions).  The ``cutcell hybrid''
method first covers the area bounded by the curve with a uniform
Cartesian grid with a user defined spacing.  This Cartesian grid is
then cut by the bounding curve; any grid outside the curve is
discarded.  An advancing front mesh generator connects the curve to
the Cartesian grid with triangles forming the completed mesh.  The
unstructured triangulation in the middle of Figure~\ref{fig:3mesh}
gives the discretized outer curve to an advancing front mesh generator
which fills the region with triangles.  Finally, the {TFI} region
begins with four curves that specify the outer boundary and then uses
transfinite interpolation to compute the mesh points on the interior
of the domain.  These methods are described in more detail in Sections~\ref{sec:tfi},
and ~\ref{sec:unst}.

\section{How To...}
\subsection{Create geometry (points and curves)}\label{sec:curve}
Geometry used as the bounding curves of mesh regions are created using
the interface shown in Figure~\ref{fig:geom}.  Points can be placed
either by interactively clicking with the mouse (after selecting the
``Build Point'' mouse mode toggle) or by inputting the coordinates in
the text box at the bottom.  Once at least two points are placed, a
curve can be created by changing the mouse mode to ``Interpolate
Curve'' and then picking a sequence of points with the mouse.  After
the last point has been chosen, clicking ``Done'' at the bottom of the
graphics window creates a spline curve that interpolates the selected
points.  At this point, the curve could be edited using mouse modes
such as ``Split Curve''; new curves can be created; or the user can
exit the geometry creation interface.
\begin{figure}
\begin{center}
\epsfig{file=\figures/points_curvesgui.ps,width=10cm}
\end{center}
\caption{Geometry creation interface}\label{fig:geom}
\end{figure}
\subsection{Specify grid points and stretching on curves}
Once curves have been created, it is often useful to specify the
number of points on the curve and how points should be distributed.
The main interface (Figure~\ref{fig:maingui}) mouse mode ``specify
number of curve points'' enables the specification of points on the
curve using the mouse.  Clicking on a curve will bring up a small
dialog box with on text entry specifying the number of points.
Typically, using ``auto'' in the text box will instruct \smesh to
automatically determine the spacing given the requested mesh size of a
region using the curve.  Inputting a number greater than zero will
force that number of points to be used on the curve.  Note that once a
curve has been used by a region its point distribution cannot be
changed.

Specifying stretching occurs in a similar manner to point size; one
selects the ``stretch curve points'' as the mouse mode and then clicks
on the curve to alter.  Clicking on a curve will bring up \overt's
stretching interface, described in more detail in \overt's
documentation~\cite{ovdocs}.  
\subsection{Create a {TFI} region}\label{sec:tfi}
Structured regions based on {TFI} are created by selecting four curves
that bound the region; these curve will be referred to as the
``left'', ``right'', ``bottom'' and ``top'', naturally.  When ``Create
{TFI} Region'' is selected, a dialog window appears that asks the user
to select the ``left'' curve.  Adjacent curves can be concatenated to
form longer and more complex boundaries that may be adjacent to more
than one other region.  Selecting adjacent curves will automatically
concatenate them and display the composite curve in bold black.  When
the curve is complete, selecting ``done'' in the dialog window
proceeds to the next curve.  Each curve is selected as described in
the order specified above.  Once all four curves are selected, another
dialog appears that allows the user to specify a text name for the
region and adjust the requested mesh spacing.  Selecting ``Generate'' at the
bottom of this dialog box generates the mesh and returns to the main
interface.
\subsection{Create an unstructured region}\label{sec:unst}
An unstructured region is created by selecting one outer bounding
curve and optional inner curves.  When ``Created Unstructured Region''
is selected, a dialog window requesting the user to choose the outer
bounding curve pops up.  Curve selection proceeds in the same manner
as a {TFI} region; compound curves are selected by clicking on
adjacent curves and finishing by clicking ``done'' on the popup.
Again, the curve under construction is displayed in bold black as
curves are selected.  Once the outer curve has been chosen, a new
dialog will request an inner curve.  Inner curves are selected in the
same manner as the outer curves.  When all the inner curves are
selected, clicking ``done'' in the dialog without choosing a curve
will present a window in which the user can specify a name, change the
spacing or select between the Cartesian cutout hybrid and all
triangular mesh generation techniques.  As before, pushing
``Generate'' creates the mesh and returns to the main interface.  An
unstructured mesh may be optimized (have vertices moved to more
optimal locations) by selecting ``optimize unstructured'' as the mouse
mode in the main interface.  Simply clicking on an unstructured region
will perform one iteration of a mesh optimization procedure on the
region.
\subsection{Output the mesh to a file}\label{sec:fileformat}
Clicking the ``Save Mesh'' button on the main interface will present a
file selection dialog.  Type or choose a filename for the mesh to
write the data to a text file.  At the end of each mesh file, a
description of the file format is presented.  The file format is a
modified version of the ``ingrid'' text file:
\begin{verbatim}
FORMAT SUMMARY ::
element vertices specified counter-clockwise, 
1 based index with 0 specifying a null vertex (vertex 4 of a triangle)
line 1 : Comment Line
line 2 : nRegions nVertices nElements maxNVertsInElement domainDimension rangeDimension
line 3 : 0 x0 y0
line 4 : 1 x1 y1
...
line nVertices+2 : nVertices-1 xNv yNv 
line nVertices+3 : 0 reg0 e0v1 e0v2 e0v3 e0v4
line nVertices+4 : 1 reg1 e1v1 e1v2 e1v3 
...
line nVertices+2+nElements : nElements-1 regNe eNv1 eNv2 eNv3 eNv4
line nVertices+3+nElements : region0ID region0Name
...
line nVerties+3+nElements+nRegions : regionN regionNName
This Summary
\end{verbatim}
\section{Examples}
\subsection{A simple {TFI} region}
A {TFI} region is built using four curves.  In this example,
the curves are built as straight lines connecting four vertices.
Figure~\ref{fig:buildtfi} illustrates the process of constructing
this type of region.
\begin{figure}[htb]
\begin{center}
\begin{pspicture}(0,0)(11,14.5)
\rput(3.,12.5){\epsfig{file=\figures/maketfi_1.ps,width=5cm}}
\rput(3,10){\makebox(0,0)[c]{\footnotesize (a) construct points}}
\rput(8,12.5){\epsfig{file=\figures/maketfi_2.ps,width=5cm}}
\rput(8,10){\makebox(0,0)[c]{\footnotesize (b) interpolate points forming curves}}
\rput(3,7.5){\epsfig{file=\figures/maketfi_3.ps,width=5cm}}
\rput(3,5.25){\makebox(0,0)[c]{\footnotesize (c) select left curve}}
\rput(8,7.5){\epsfig{file=\figures/maketfi_4.ps,width=5cm}}
\rput(8,5.25){\makebox(0,0)[c]{\footnotesize (d) select right curve}}
\rput(3,2.75){\epsfig{file=\figures/maketfi_5.ps,width=5cm}}
\rput(3,.5){\makebox(0,0)[c]{\footnotesize (e) select bottom curve}}
\rput(8,2.75){\epsfig{file=\figures/maketfi_7.ps,width=5cm}}
\rput(8,.5){\makebox(0,0)[c]{\footnotesize (f) select top and generate mesh}}
%\psgrid[subgriddiv=2]
\end{pspicture}
\end{center}
\caption{Construction of a tfi region: (a) construct points; (b) interpolate points
to form bounding curves; (c)-(f) assemble four curves to form the grid (f).}\label{fig:buildtfi}
\end{figure}
\subsubsection{command file}
\begin{verbatim}
Create Curves
Mouse Mode Build Point
new point 2.055071e-01 6.812946e-01
new point 7.907053e-01 7.342878e-01
new point 2.698978e-01 1.025259e-01
new point 8.664590e-01 1.029423e-01
Mouse Mode Interpolate Curve
point for interpolation 1
point for interpolation 0
stop picking
point for interpolation 0
point for interpolation 2
stop picking
point for interpolation 2
point for interpolation 3
stop picking
point for interpolation 3
point for interpolation 1
stop picking
exit
Create TFI Region
select left 1
Done
select right 3
Done
select bottom 2
Done
select top 0
Done
generate
tb Plot Reference Grids 0
Save Mesh
/home/chand/overture/views/tools/2dmesh/tfi.msh
exit
\end{verbatim}
\subsection{A simple unstructured region}
Unstructured regions also begin by defining a set of curves that
define a connected domain.  At least one curve is required to define
the outer boundary.  Multiply connected domains may be created by
creating one or more inner boundary curves.  There are two types of
unstructured meshing techniques available.  One simply fills the
domain with triangles using \overt's advancing front mesh generator.
The second method builds a mixed triangle-quadrilateral mesh by
cutting a Cartesian grid with the bounding curves and then connecting
the curves to the grid with triangles.  These options are show in
Figure~\ref{fig:ustreg}.
\begin{figure}[htb]
\begin{center}
\begin{pspicture}(0,1)(17,6)
\rput(2.5,4){\epsfig{file=\figures/unscurves.ps,width=6.5cm}}
\rput(2.5,4.7){\makebox(0,0)[l]{\footnotesize inner curve}}
\rput(2.5,5.75){\makebox(0,0)[l]{\footnotesize outer curve}}
\rput(2.5,1.5){\makebox(0,0)[c]{\footnotesize (a) bounding curves}}
\rput(8.5,4){\epsfig{file=\figures/unstri_fig.ps,width=6.5cm}}
\rput(8.5,1.5){\makebox(0,0)[c]{\footnotesize (b) advancing front mesh}}
\rput(14.5,4){\epsfig{file=\figures/cutcell_fig.ps,width=6.5cm}}
\rput(14.5,1.5){\makebox(0,0)[c]{\footnotesize (c) cutcell hybrid mesh}}
%psgrid[subgriddiv=2]
\end{pspicture}
\end{center}
\caption{Unstructured regions are built using inner and outer bounding curves (a); they come in 
two flavors: (b) all triangles and (c) mixed quads and triangles}\label{fig:ustreg}
\end{figure}
\subsubsection{command file}
\begin{verbatim}
Create Curves
Mouse Mode Build Point
new point 1.411164e-01 5.980980e-01
new point 3.456383e-01 7.086174e-01
new point 6.107900e-01 6.916550e-01
new point 8.134314e-01 5.745974e-01
new point 8.210067e-01 4.336066e-01
new point 5.312484e-01 2.322357e-01
new point 6.915023e-02 2.534473e-01
Mouse Mode Interpolate Curve
point for interpolation 0
point for interpolation 1
point for interpolation 2
point for interpolation 3
point for interpolation 4
point for interpolation 5
point for interpolation 6
point for interpolation 0
stop picking
Mouse Mode Build Point
new point 3.352156e-01 4.913438e-01
new point 4.573885e-01 5.203880e-01
new point 4.687516e-01 4.123340e-01
new point 3.399701e-01 3.757012e-01
Mouse Mode Interpolate Curve
point for interpolation 7
point for interpolation 8
point for interpolation 9
point for interpolation 10
point for interpolation 7
stop picking
exit
dxdy .02,.02
Create Unstructured Region
select outer 0
Done
select inner 1
Done
Done
generate
tb Plot Reference Grids 0
pause
mm delRegion
delete region 0
Create Unstructured Region
select outer 0
Done
select inner 1
Done
Done
use cutout 0
generate
mm optimize
optimize region 1
optimize region 1
optimize region 1
optimize region 1
\end{verbatim}
\subsection{An example with multiple regions}
This example demonstrates the creation of a mesh with multiple
regions.  Figure~\ref{fig:mreg} shows three regions created using
each of the mesh generation methods available.  The {TFI} region at
the top assembled its ``bottom'' curve from the two horizontal curves
in the middle of the geometry.  The region is adjacent to two
unstructured regions that utilize the middle curves as subsets of
their outer boundaries.  Two circular arc curves form the inner
boundary of an unstructured region at the lower left region of the
mesh.  Information about these regions is placed in the text file
when the mesh is saved.
\begin{figure}[htb]
\begin{center}
\begin{pspicture}(2,0)(12,12)
%\rput(7,9){\epsfig{file=\figures/mregfig_curves.ps,width=9cm}}
\rput(7,9){\psclip{\psframe[linecolor=white](0,1)(9,7.5)}\epsfig{file=\figures/mregfig_curves.ps,width=9cm}\endpsclip}
\rput(7,10.75){\makebox(0,0)[c]{\footnotesize bottom boundary}}
\psline[linewidth=.5pt]{->}(7,10.5)(4.5,9.9)
\psline[linewidth=.5pt]{->}(7,10.5)(8.5,9.9)
%
\rput(3.25,9.25){\makebox(0,0)[l]{\footnotesize outer}}
\psline[linewidth=.5pt]{->}(3.25,9)(2.9,8)
\psline[linewidth=.5pt]{->}(3.25,9)(3.5,6.5)
\psline[linewidth=.5pt]{->}(3.25,9)(6.2,8.9)
\psline[linewidth=.5pt]{->}(3.25,9)(3.25,9.9)
%
\rput(4.5,7){\makebox(0,0)[l]{\footnotesize inner}}
\psline[linewidth=.5pt]{->}(4.3,7)(4.1,8)
\psline[linewidth=.5pt]{->}(4.3,7)(4.9,8)
%
\rput(8.7,8){\makebox(0,0)[c]{\footnotesize outer boundary}}
\psline[linewidth=.5pt]{->}(7.4,8)(6.2,8.9)
\psline[linewidth=.5pt]{->}(8.7,7.8)(8.7,6.5)
\psline[linewidth=.5pt]{->}(8.7,8.2)(8.5,9.8)
\psline[linewidth=.5pt]{->}(10,8)(11.2,8)
%\rput(7,3){\epsfig{file=\figures/mregfig.ps,width=9cm}}
\rput(7,3){\psclip{\psframe[linecolor=white](0,1)(9,7.5)}\epsfig{file=\figures/mregfig.ps,width=9cm}\endpsclip}
%\psgrid[subgriddiv=2]
\end{pspicture}
\end{center}
\caption{A multi-region mesh consisting of regions of each type; note that composite curves
are used to form the ``bottom'' boundary of the top, {TFI} region, and the inner boundary of
the lower left unstructured region.}\label{fig:mreg}
\end{figure}
\subsubsection{command file}
\begin{verbatim}
Create Curves
Mouse Mode Build Point
new point 0 0
new point 2 0
new point 5 0
new point 0 2
new point 2 2
new point 5 2
new point 0 3
new point 5 3
new point 1 .75
new point 1 1.25
Mouse Mode Circular Arc
arc segment 8 9
radius of curvature .25
arc segment 9 8
radius of curvature .25
point for interpolation 0
point for interpolation 1
stop picking
point for interpolation 1
point for interpolation 2
stop picking
point for interpolation 0
point for interpolation 3
stop picking
point for interpolation 3
point for interpolation 6
stop picking
point for interpolation 6
point for interpolation 7
stop picking
point for interpolation 3
point for interpolation 4
stop picking
point for interpolation 4
point for interpolation 5
stop picking
point for interpolation 2
point for interpolation 5
stop picking
point for interpolation 5
point for interpolation 7
stop picking
point for interpolation 1
point for interpolation 4
stop picking
exit
dxdy .1,.1
Create TFI Region
select left 5
Done
select right 10
Done
select bottom 7
select bottom 8
Done
select top 6
Done
generate
tb Plot Reference Grids 0
Create Unstructured Region
select outer 4
select outer 2
select outer 11
select outer 7
Done
select inner 1
select inner 0
Done
Done
generate
Create Unstructured Region
select outer 11
select outer 3
select outer 9
select outer 8
Done
Done
use cutout 0
generate
mm optimize
optimize region 3
optimize region 3
\end{verbatim}
\subsection{Stretched regions}
The distribution of points along curves can be adjusted
by specifying the number of points using the main interface
as well as stretching the points along a curve using \overt's {\tt StretchTransform}
interface~\cite{mappings}.  An example of both these tools is displayed
in Figure~\ref{fig:strfig} for both unstructured and {TFI} regions.  Note,
when a cutcell unstructured region is formed, the stretching information is not
propagated to the interior.
\begin{figure}[htb]
\begin{center}
\begin{pspicture}(0,2)(6,10)
\rput(1.5,6){\psclip{\psframe[linecolor=white](3,1)(6,9)}\epsfig{file=\figures/str_tri.ps,width=10cm}\endpsclip}
\rput(4.5,6){\psclip{\psframe[linecolor=white](3,1)(6,9)}\epsfig{file=\figures/str_tfi.ps,width=10cm}\endpsclip}
%\rput(4.5,6){\epsfig{file=\figures/str_tfi.ps,width=10cm}}
%\psgrid[subgriddiv=2]
\end{pspicture}
\end{center}
\caption{\smesh can be used to generated graded meshes}\label{fig:strfig}
\end{figure}
\subsubsection{command file}
\begin{verbatim}
dxdy .05,.05
Create Curves
new point 0 0
new point .2 0
new point .2 1
new point 0 1
DISPLAY AXES:0 0
Mouse Mode Interpolate Curve
point for interpolation 0
point for interpolation 1
stop picking
point for interpolation 1
point for interpolation 2
stop picking
point for interpolation 2
point for interpolation 3
stop picking
point for interpolation 3
point for interpolation 0
stop picking
exit
DISPLAY AXES:0 0
mm nCurvePts
points on curve 0  11
points on curve 1  25
points on curve 3  25
mm strCurve
stretch points on curve 1
print grid statistics
stretch grid
Stretch r1:exp
STP:stretch r1 exp: min dx .02
stretch grid
STP:stretch r1 exp: cluster at r=0
stretch grid
pause
exit
stretch points on curve 3
Stretch r1:exp
STP:stretch r1 exp: min dx .02
STP:stretch r1 exp: cluster at r=1
stretch grid
pause
close r1 stretching parameters
exit
Create Unstructured Region
select outer 0
select outer 1
select outer 2
select outer 3
Done
Done
use cutout 0
name stretched
generate
tb Plot Reference Grids 0
mm optimize
optimize region 0
optimize region 0
optimize region 0
\end{verbatim}
\bibliographystyle{unsrt} \bibliography{\bib/kkc}

% \makeLLNLBackCover

\end{document}
