\section{TridiagonalSolver: Solve sets of tridiagonal (or pentadiagonal) systems or block tridiagonal systems}
\index{tridiagonal solver}\index{block tridiagonal solver}\index{pentadiagonal solver}

The {\tt TridiagonalSolver} class can be used to solve tridiagonal (or pentadiagonal) systems and 
block tridiagonal systems. Sets of tridiagonal (pentadiagonal) systems can be solved such as those
that are formed when line smoothing is performed on a 2 or 3 dimensional grid or
an ADI type method is used to solve a PDE. Currently only blocks or size 2 or 3 are
implemented. 

The two basic steps to solve a tridiagonal (pentadiagonal) system are to first factor the system
by calling the {\tt factor} member function and then solve the system using {\tt solve}.

There are three types of boundary conditions supported, {\tt normal}, {\tt extended} and
{\tt periodic}. 
A {\tt normal} matrix tridiagonal is of the form
\[
  {\rm normal } = 
        \begin{bmatrix}
               b_0 & c_0 &      &        &        &     \\
               a_1 & b_1 & c_1  &        &        &     \\
                   & a_2 & b_2  & c_2    &        &     \\
                   &     &\ddots&\ddots  & \ddots &      \\ 
                   &     &      & a_{n-1}& b_{n-1} & c_{n-1} \\
                   &     &      &        & a_n     & b_n 
        \end{bmatrix}
\]
The extended matrix allows extra entries on the first and last rows (required
by some PDE boundary conditions)
\[
  {\rm extended} = 
        \begin{bmatrix}
               b_0 & c_0 & a_0  &        &        &     \\
               a_1 & b_1 & c_1  &        &        &     \\
                   & a_2 & b_2  & c_2    &        &     \\
                   &     &\ddots&\ddots  & \ddots &      \\ 
                   &     &      & a_{n-1}& b_{n-1} & c_{n-1} \\
                   &     &      & c_n    & a_n     & b_n 
        \end{bmatrix}
\]
The periodic case is
\[
  {\rm periodic } = 
        \begin{bmatrix}
               b_0 & c_0 &      &        &        & a_0 \\
               a_1 & b_1 & c_1  &        &        &     \\
                   & a_2 & b_2  & c_2    &        &     \\
                   &     &\ddots&\ddots  & \ddots &      \\ 
                   &     &      & a_{n-1}& b_{n-1} & c_{n-1} \\
              c_n  &     &      &        & a_n     & b_n 
        \end{bmatrix}
\]
A {\tt normal} pentadiagonal matrix is of the form
\[
  {\rm normal } = 
        \begin{bmatrix}
               c_0 & d_0 & e_0  &       &        &        &        &     \\
               b_1 & c_1 & d_1  & e_1   &        &        &        &     \\
               a_2 & b_2 & c_2  & d_2   &  e_2   &        &        &     \\
                   & a_3 & b_3  & c_3   &  d_3   &  e_3   &        &     \\
                   &     &\ddots&\ddots & \ddots & \ddots & \ddots &          \\ 
                   &     &      &a_{n-2}& b_{n-2}& c_{n-2}& d_{n-2}& e_{n-2}   \\
                   &     &      &       &a_{n-1} & b_{n-1}& c_{n-1}& d_{n-1}   \\
                   &     &      &       &        &a_{n  } & b_{n  }& c_{n  }   \\
        \end{bmatrix}
\]
The {\tt extended} pentadigonal matrix allows extra equations in the first two and last two rows,
\[
  {\rm extended} = 
        \begin{bmatrix}
               c_0 & d_0 & e_0  & a_0   &  b_0   &        &        &     \\
               b_1 & c_1 & d_1  & e_1   &  a_1   &        &        &     \\
               a_2 & b_2 & c_2  & d_2   &  e_2   &        &        &     \\
                   & a_3 & b_3  & c_3   &  d_3   &  e_3   &        &     \\
                   &     &\ddots&\ddots & \ddots & \ddots & \ddots &          \\ 
                   &     &      &a_{n-2}& b_{n-2}& c_{n-2}& d_{n-2}& e_{n-2}   \\
                   &     &      &e_{n-1}&a_{n-1} & b_{n-1}& c_{n-1}& d_{n-1}   \\
                   &     &      & d_n   &e_{n  } &a_{n  } & b_{n  }& c_{n  }   \\
        \end{bmatrix}
\]
The periodic pentadigonal case is
\[
  {\rm periodic } = 
        \begin{bmatrix}
               c_0 & d_0 & e_0  &       &        &        & a_0    & b_0 \\
               b_1 & c_1 & d_1  & e_1   &        &        &        & a_1 \\
               a_2 & b_2 & c_2  & d_2   &  e_2   &        &        &     \\
                   & a_3 & b_3  & c_3   &  d_3   &  e_3   &        &     \\
                   &     &\ddots&\ddots & \ddots & \ddots & \ddots &          \\ 
                   &     &      &a_{n-2}& b_{n-2}& c_{n-2}& d_{n-2}& e_{n-2}   \\
           e_{n-1} &     &      &       &a_{n-1} & b_{n-1}& c_{n-1}& d_{n-1}   \\
             d_n   & e_n &      &       &        &a_{n  } & b_{n  }& c_{n  }   \\
        \end{bmatrix}
\]
Here is an example code that shows how to use the TridiagonalSolver
(file {\tt Overture/tests/trid.C})
{\footnotesize
\listinginput[1]{1}{\otherStuff/trid.C}
}

\subsection{Member Functions}

\input TridiagonalSolverInclude.tex

