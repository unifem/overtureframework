\section{Constructing A++ Index Objects for Grid Functions: 
   \Index{getIndex}, \Index{getBoundaryIndex}, \Index{getGhostIndex}} 
\label{getIndex}


A number of functions are provided to construct A++ Index 
objects for grid functions. The {\tt getIndex} function
constructs Index objects for the interior points
of a grid function. The function {\tt getBoundaryIndex}
constructs the Index objects corresponding to a
given boundary face. The {\tt getGhostIndex} function
constructs Index objects for a ghost line on a given
face of the grid. 

Recall that a gridFunction can be cell-centred
or vertex-centred in any of the coordinate
directions. Typically grid functions are either
all vertex-centred, all cell-centred or
else face-centred (a face-centred grid function
is vertex-centred along one axis and cell-centred
along the others).
By passing a grid function to {\tt getIndex}, 
{\tt getBoundaryIndex} or {\tt getGhostIndex} one
can be sure to get the Index objects corresponding 
to the ``centred-ness'' of the grid function.
This is important because the ``interior'' points
of a vertex-centred grid function are different
from the interior
points of a cell-centred grid function or the
interior points of a face-centred grid function.
For grid functions that are {\tt faceCenteredAll} (i.e. they have components
that are faceCentered along each axis) there is a {\tt getIndex} function
with an extra argument that species which face centering axis to use.


\subsection{Index functions}\index{OGgetIndex}

Here are the getIndex functions (see also file {\tt \gf/OGgetIndex.h}).
There are at least three flavours of each function. The first takes an {\tt indexArray},
the second takes a grid function and component number and the third takes
a grid function (in which case {\tt component=0} is used to determine
the Index's). The function {\tt getIndex} has an additional 2 flavours.


\noindent A summary of the arguments is as follows:
\begin{itemize}
  \item {\tt indexArray} : an intArray with dimensions (0:1,0:2), such as
        {\tt indexRange}, {\tt gridIndexRange} or {\tt dimension}, that defines
        a region on the grid.
  \item {\tt extra} :  Increase the size of the domain by this many lines. 
  \item {\tt side, axis} : these arguments define which face to use
  \item {\tt floatMappedGridFunction} : Use the cell-centeredness properties of this grid function 
     to determine the index range. Versions of the function also exist for
    {\tt doubleMappedGridFunction} and {\tt intMappedGridFunction}. 
  \item {\tt component} : base the Index objects on this component of
    the grid function (required since different components of a grid function may be centred in
    different ways). The functions where the {\tt component} argument is missing use {\tt component=0}.
    The exception to the above rule is when the grid function 
    is {\tt faceCenteredAll} in which case {\tt component} =0,1 or 2 will indicate whether to return
   Index's for the faceCenteredAxis1, faceCenteredAxis2 or the faceCenteredAxis3 components.
  \item {\tt ghostLine} : return Index objects for this ghost line. Choose
    {\tt ghostline=1} for the first ghost line, {\tt ghostline=2} for the
    second ghost line. ({\tt ghostline=0} will give the boundary).
\end{itemize}

\input OGgetIndexInclude.tex


