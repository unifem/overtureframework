%-----------------------------------------------------------------------
%   OtherStuff :
%      Here we describe other Overture stuff
%
%-----------------------------------------------------------------------

\documentclass{article}

\usepackage{times}  % for embeddable fonts, Also use: dvips -P pdf -G0

\voffset=-1.25truein
\hoffset=-1.truein
\setlength{\textwidth}{7in}      % page width
\setlength{\textheight}{9.5in}    % page height
%\setlength{\textheight}{7.5in}    % page height for xdvi
% \renewcommand{\baselinestretch}{1.5}    % "double" spaced


\usepackage{amsmath}
\usepackage{amssymb}

\usepackage{verbatim}
\usepackage{moreverb}
\usepackage{graphics}    
\usepackage{calc}
\usepackage{ifthen}
% \usepackage{fancybox}

\usepackage{makeidx} % index
\makeindex
\newcommand{\Index}[1]{#1\index{#1}}

% ---- we have lemmas and theorems in this paper ----
\newtheorem{assumption}{Assumption}
\newtheorem{definition}{Definition}

\newcommand{\otherStuff}{/home/henshaw/res/otherStuff}

\newcommand{\primer}{/users/henshaw/res/primer}
\newcommand{\examples}{Overture/examples}
\newcommand{\gf}{/users/henshaw/res/gf}
\newcommand{\mapping}{/users/henshaw/res/mapping}
% \newcommand{\ogshow}{/users/henshaw/res/ogshow}
% \newcommand{\oges}{/users/henshaw/res/oges}
% \newcommand{\cguser}{/users/henshaw/cgap/cguser}
\newcommand{\figures}{/home/henshaw/OvertureFigures}
\newcommand{\grid}{/home/henshaw/res/grid}

\newcommand{\OvertureOverture}{/users/henshaw/Overture/Overture}

% \newcommand{\RA}{realArray}
% \newcommand{\MGF}{MappedGridFunction}
% \newcommand{\RMGF}{realMappedGridFunction}
% \newcommand{\RCGF}{realCompositeGridFunction}
% 
% \newcommand{\DABO}{Differential\-And\-Boundary\-Operators}
% 
% \newcommand{\MG}{Mapped\-Grid}
% \newcommand{\GC}{Grid\-Collection}
% \newcommand{\CG}{Composite\-Grid}
% \newcommand{\MGCG}{Multigrid\-Composite\-Grid}
% 
% \newcommand{\MGO}{MappedGridOperators}
% \newcommand{\GCO}{Grid\-Collection\-Operators}
% \newcommand{\CGO}{Composite\-Grid\-Operators}
% \newcommand{\MGCGO}{Multigrid\-Composite\-Grid\-Operators}

\begin{document}

% -----definitions-----
\input wdhDefinitions.tex

\vspace{5\baselineskip}
\begin{flushleft}
{\Large
Other Stuff for Overture\\
User Guide, Version 1.0\\
}
\vspace{2\baselineskip}
William D. Henshaw
\footnote{
        This work was partially
        supported by grant N00014-95-F-0067 from the Office of Naval
        Research
        } \\
\vspace{\baselineskip}
Centre for Applied Scientific Computing \\
Lawrence Livermore National Laboratory    \\
Livermore, CA, 94551   \\
henshaw@llnl.gov \\
http://www.llnl.gov/casc/people/henshaw \\
http://www.llnl.gov/casc/Overture\\
\vspace{\baselineskip}
\today\\
\vspace{\baselineskip}
UCRL-MA-134292

\vspace{4\baselineskip}

\noindent{\bf Abstract:}
We describe miscellaneous Overture stuff:
\begin{description}
 \item[Overture start and finish functions, Overture global variables]
 \item[sPrintF,sScanF,...] : miscellaneous utility routines.
 \item[getIndex] : the {\tt getIndex} utility routines for generating A++ Index's for operating
     on sub-domains of {\tt MappedGrid}'s.
  \item[Twilight-zone functions]: {\tt OGFunction}, {\tt OGPolyFunction}, {\tt OGTrigFunction} and
      {\tt OGPulseFunction}.
  \item[Database access functions]: functions for reading various objects (such as a {\tt CompositeGrid})
     from a database file, including the function {\tt getFromADataBase}.
  \item[display functions] : formatted display functions for A++ arrays and for writing A++ arrays to files.
  \item[Integrate] : This class provides methods for easily integrating grid functions on domains
      or boundaries of domains.
  \item[TridiagonalSystem solver] : solve tridiagonal and block tridiagonal systems.
  \item[FortranIO] : write unformatted Fortran files.
  \item[Reference Counting]: A description of how reference counting works.
\end{description}
\end{flushleft}

\vfill\eject
\tableofcontents
% \listoffigures

%---------- End of title Page for a Research Report
% \vfill\eject
% \section{Introduction}

%==============================================================================
\vfill\eject
\section{Overture start and finish functions, Overture global variables}
\input OvertureInclude.tex
%==============================================================================



%==============================================================================
\vfill\eject
\section{Miscellaneous Stuff}
\input otherStuffInclude.tex
%==============================================================================

%==============================================================================
\vfill\eject
\input getIndex.tex
%==============================================================================

%=============================================================================
\vfill\eject
\input OGFunction.tex
%=============================================================================

%==============================================================================
\vfill\eject
\input DataBaseAccessFunctions.tex
%==============================================================================

%=============================================================================
\vfill\eject
\section{Display functions for arrays (writing arrays to files)}\index{display!of A++ arrays}
\input displayInclude.tex
%=============================================================================

%=============================================================================
\vfill\eject
\input IntegrateHTML.tex
%=============================================================================

%=============================================================================
\vfill\eject
\input TridiagonalSolverHTML.tex
%=============================================================================

%=============================================================================
\vfill\eject
\input FortranIO.tex
%=============================================================================

%=============================================================================
\vfill\eject
\input referenceCountedObjectsHTML.tex

%=============================================================================


\printindex

\end{document}



