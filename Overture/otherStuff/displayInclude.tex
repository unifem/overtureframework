\subsection{display: display an A++ array}
 
\newlength{\displayIncludeArgIndent}
\begin{flushleft} \textbf{%
int \\ 
\settowidth{\displayIncludeArgIndent}{display(}%
display( const floatArray \& x, const char *label, const char *format\_, const Index *Iv /* =NULL*/ )
}\end{flushleft}
\begin{description}
\item[{\bf Description:}] 
      Another version of display -- pass a format but no FILE
\item[{\bf x (input) :}]  array to display. There are also versions of this routine for int and double arrays.
\item[{\bf label (input):}]  optional header label
\item[{\bf format (input) :}]  an optional format such as "\%6.1e " or "\%11.4e " (note blank at end) 
  that will be used to display each element in the array. The default is "\%11.4e "
\item[{\bf Iv[d] :}]  If Iv is not NULL then print the values x(Iv[0],Iv[1],Iv[2],...) You must supply at
   least nd entries in the array Iv[d] where nd=x.numberOfDimensions();
\end{description}
\subsection{display: save an A++ array in a file}
 
\begin{flushleft} \textbf{%
int \\ 
\settowidth{\displayIncludeArgIndent}{display(}%
display( const floatArray \& x, \\ 
\hspace{\displayIncludeArgIndent}const char *label    = NULL, \\ 
\hspace{\displayIncludeArgIndent}FILE *file           = NULL, \\ 
\hspace{\displayIncludeArgIndent}const char *format\_  = NULL, \\ 
\hspace{\displayIncludeArgIndent}const Index *Iv /* =NULL*/ )
}\end{flushleft}
\begin{description}
\item[{\bf Description:}] 
    Display an A++ array
\item[{\bf x (input) :}]  array to display. There are also versions of this routine for int and double arrays.
\item[{\bf label (input):}]  optional header label
\item[{\bf file (input) :}]  optionally supply a file to print to.
\item[{\bf format (input) :}]  an optional format such as "\%6.1e " or "\%11.4e " (note blank at end) 
  that will be used to display each element in the array. The default is "\%11.4e "
\item[{\bf Iv[d] :}]  If Iv is not NULL then print the values x(Iv[0],Iv[1],Iv[2],...)
\end{description}
\subsection{display an A++ array with DisplayParameters}
 
\begin{flushleft} \textbf{%
int \\ 
\settowidth{\displayIncludeArgIndent}{display(}%
display( const floatArray \& x, const char *label, const DisplayParameters \& displayParameters, \\ 
\hspace{\displayIncludeArgIndent}const Index *Iv /* =NULL*/ )
}\end{flushleft}
\begin{description}
\item[{\bf Description:}] 
      Another version of display -- pass a format but no FILE
\item[{\bf x (input) :}]  array to display. There are also versions of this routine for int and double arrays.
\item[{\bf label (input):}]  optional header label
\item[{\bf format (input) :}]  an optional format such as "\%6.1e " or "\%11.4e " (note blank at end) 
  that will be used to display each element in the array. The default is "\%11.4e "
\item[{\bf displayParameters (input) :}]  provide parameters for display. 
\item[{\bf Iv[d] :}]  If Iv is not NULL then print the values x(Iv[0],Iv[1],Iv[2],...)
\end{description}
\subsection{displayMask}
 
\begin{flushleft} \textbf{%
int  \\ 
\settowidth{\displayIncludeArgIndent}{displayMask(}%
displayMask( const intArray \& mask, \\ 
\hspace{\displayIncludeArgIndent}const aString \& label  =nullString,\\ 
\hspace{\displayIncludeArgIndent}FILE *file  = NULL, \\ 
\hspace{\displayIncludeArgIndent}const Index *Iv /* =NULL*/ )
}\end{flushleft}
\begin{description}
\item[{\bf Description:}] 
 Display the mask array in a MappedGrid in a reasonable way
 The mask array in a MappedGrid is a bit-mapping that is difficult to look at
 if displayed in the formal way. This routine will display the mask in a more
 compact form (although some information is not printed) where each entry printed will mean:
 \begin{description}
   \item[1] : ISdiscretizationPoint
   \item[2] : ISghostPoint
   \item[-1] : ISinterpolationPoint
 \end{description}   
\end{description}
\subsection{displayMask}
 
\begin{flushleft} \textbf{%
int  \\ 
\settowidth{\displayIncludeArgIndent}{displayMask(}%
displayMask( const intSerialArray \& mask, \\ 
\hspace{\displayIncludeArgIndent}const aString \& label  =nullString,\\ 
\hspace{\displayIncludeArgIndent}FILE *file  = NULL, \\ 
\hspace{\displayIncludeArgIndent}const Index *Iv /* =NULL*/ )
}\end{flushleft}
\begin{description}
\item[{\bf Description:}] 
 Display the mask array in a MappedGrid in a reasonable way
 The mask array in a MappedGrid is a bit-mapping that is difficult to look at
 if displayed in the formal way. This routine will display the mask in a more
 compact form (although some information is not printed) where each entry printed will mean:
 \begin{description}
   \item[1] : ISdiscretizationPoint
   \item[2] : ISghostPoint
   \item[-1] : ISinterpolationPoint
 \end{description}   
\end{description}
