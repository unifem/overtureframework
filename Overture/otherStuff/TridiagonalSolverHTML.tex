\section{TridiagonalSolver: Solve sets of tridiagonal systems or block tridiagonal systems}
\index{tridiagonal solver}\index{block tridiagonal solver}

The {\tt TridiagonalSolver} class can be used to solve tridiagonal systems or
block tridiagonal systems. Sets of tridiagonal systems can be solved such as those
that are formed when line smoothing is performed on a 2 or 3 dimensional grid or
an ADI type method is used to solve a PDE. Currently only blocks or size 2 or 3 are
implemented. 

The two basic steps to solve a tridiagonal system are to first factor the system
by calling the {\tt factor} member function and then solve the system using {\tt solve}.

There are three types of boundary conditions supported, {\tt normal}, {\tt extended} and
{\tt periodic}. A {\tt normal} matrix is of the form
\[
  {\rm normal } = 
        \begin{bmatrix}
               b_0 & c_0 &      &        &        &     \\
               a_1 & b_1 & c_1  &        &        &     \\
                   & a_2 & b_2  & c_2    &        &     \\
                   &     &\ddots&\ddots  & \ddots &      \\ 
                   &     &      & a_{n-1}& b_{n-1} & c_{n-1} \\
                   &     &      &        & a_n     & b_n 
        \end{bmatrix}
\]
The extended matrix allows extra entries on the first and last rows (required
by some PDE boundary conditions)
\[
  {\rm extended} = 
        \begin{bmatrix}
               b_0 & c_0 & a_0  &        &        &     \\
               a_1 & b_1 & c_1  &        &        &     \\
                   & a_2 & b_2  & c_2    &        &     \\
                   &     &\ddots&\ddots  & \ddots &      \\ 
                   &     &      & a_{n-1}& b_{n-1} & c_{n-1} \\
                   &     &      & c_n    & a_n     & b_n 
        \end{bmatrix}
\]
The periodic case is
\[
  {\rm periodic } = 
        \begin{bmatrix}
               b_0 & c_0 &      &        &        & a_0 \\
               a_1 & b_1 & c_1  &        &        &     \\
                   & a_2 & b_2  & c_2    &        &     \\
                   &     &\ddots&\ddots  & \ddots &      \\ 
                   &     &      & a_{n-1}& b_{n-1} & c_{n-1} \\
              c_n  &     &      &        & a_n     & b_n 
        \end{bmatrix}
\]
Here is an example code that shows how to use the TridiagonalSolver
(file {\tt Overture/tests/trid.C})
{\footnotesize
\begin{verbatim}
#include "TridiagonalSolver.h"
#include "display.h"

// test the tridiagonal solver

int
main()
{
  ios::sync_with_stdio(); // Synchronize C++ and C I/O subsystems
  Index::setBoundsCheck(on);  //  Turn on A++ array bounds checking


  TridiagonalSolver tri;
  // first solve a single tridiagonal system
  Range I1(0,10);
  RealArray a(I1),b(I1),c(I1),u(I1);
  a=1.;
  b=2.;
  c=1.;
  // choose the rhs so the answer will be 1
  u=4.;
  tri.factor(a,b,c,TridiagonalSolver::periodic);
  tri.solve(u);

  real error = max(abs(u-1.));
  printf(" ****maximum error=%e for the periodic case.\n",error);

  // Now solve a collection of tridiagonal systems
  Range I2(0,2), I3(0,2);
  a.redim(I1,I2,I3);
  b.redim(I1,I2,I3);
  c.redim(I1,I2,I3);
  u.redim(I1,I2,I3);
  
  int base =I1.getBase();
  int bound=I1.getBound();
  for( int i3=I3.getBase(); i3<=I3.getBound(); i3++)
  {
    for( int i2=I2.getBase(); i2<=I2.getBound(); i2++)
    {
      for( int i1=I1.getBase(); i1<=I1.getBound(); i1++)
      {
	a(i1,i2,i3)=  -(i1+i2+i3+1);
	b(i1,i2,i3)= 4*(i1+i2+i3+1);
	c(i1,i2,i3)=-2*(i1+i2+i3+1);
	u(i1,i2,i3)=   (i1+i2+i3+1);
      }
      u(base,i2,i3) -=a(base,i2,i3);
      u(bound,i2,i3)-=c(bound,i2,i3);
    }
  }
  
  tri.factor(a,b,c,TridiagonalSolver::normal,axis1);
  tri.solve(u,I1,I2,I3);
  
  error = max(abs(u-1.));
  printf(" ****maximum error=%e for the normal case.\n",error);

  return 0;
}
\end{verbatim}
}

\subsection{Member Functions}

\input TridiagonalSolverInclude.tex

