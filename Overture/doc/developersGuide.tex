%-----------------------------------------------------------------------
%   OtherStuff :
%      Here we describe other Overture stuff
%
%-----------------------------------------------------------------------

\documentclass{article}
\usepackage{times}  % for embeddable fonts, Also use: dvips -P pdf -G0

\input documentationPageSize.tex

\usepackage{amsmath}
\usepackage{amssymb}

\usepackage{verbatim}
\usepackage{moreverb}
\usepackage{graphics}    
\usepackage{calc}
\usepackage{ifthen}
% \usepackage{fancybox}

% ---- we have lemmas and theorems in this paper ----
\newtheorem{assumption}{Assumption}
\newtheorem{definition}{Definition}

\newcommand{\primer}{/users/henshaw/res/primer}
\newcommand{\GF}{/n/\-c3servet/\-henshaw/\-res/\-gf}
\newcommand{\examples}{Overture/examples}
\newcommand{\gf}{/users/henshaw/res/gf}
\newcommand{\mapping}{/users/henshaw/res/mapping}
\newcommand{\ogshow}{/users/henshaw/res/ogshow}
\newcommand{\oges}{/users/henshaw/res/oges}
\newcommand{\cguser}{/users/henshaw/cgap/cguser}
\newcommand{\figures}{../docFigures}

\newcommand{\OVERTUREOVERTURE}{/n/\-c3servet/\-henshaw/\-Overture/\-Overture}
\newcommand{\OvertureOverture}{/users/henshaw/Overture/Overture}

\newcommand{\RA}{realArray}
\newcommand{\MGF}{MappedGridFunction}
\newcommand{\RMGF}{realMappedGridFunction}
\newcommand{\RCGF}{realCompositeGridFunction}

\newcommand{\DABO}{Differential\-And\-Boundary\-Operators}

\newcommand{\MG}{Mapped\-Grid}
\newcommand{\GC}{Grid\-Collection}
\newcommand{\CG}{Composite\-Grid}
\newcommand{\MGCG}{Multigrid\-Composite\-Grid}

\newcommand{\MGO}{MappedGridOperators}
\newcommand{\GCO}{Grid\-Collection\-Operators}
\newcommand{\CGO}{Composite\-Grid\-Operators}
\newcommand{\MGCGO}{Multigrid\-Composite\-Grid\-Operators}

\begin{document}

% \input list.tex    %  defines Lentry enviroment for documenting functions

% -----definitions-----
\def\R      {{\bf R}}
\def\Dv     {{\bf D}}
\def\bv     {{\bf b}}
\def\fv     {{\bf f}}
\def\Fv     {{\bf F}}
\def\gv     {{\bf g}}
\def\iv     {{\bf i}}
\def\jv     {{\bf j}}
\def\kv     {{\bf k}}
\def\nv     {{\bf n}}
\def\rv     {{\bf r}}
\def\tv     {{\bf t}}
\def\uv     {{\bf u}}
\def\Uv     {{\bf U}}
\def\vv     {{\bf v}}
\def\Vv     {{\bf V}}
\def\xv     {{\bf x}}
\def\yv     {{\bf y}}
\def\zv     {{\bf z}}
\def\lt     {{<}}
\def\grad    {\nabla}
\def\comma  {~~~,~~}
\def\uvd    {{\bf U}}
\def\ud     {{    U}}
\def\pd     {{    P}}
\def\calo{{\cal O}}

\def\nv {{\ff nv }}
\def\ng {{\ff ng }}
\def\bc {{\ff bc}}

\vspace{5\baselineskip}
\begin{flushleft}
{\Large
Overture Developers Guide \\
Version 1.0
}
\vspace{2\baselineskip}
William D. Henshaw
\vspace{\baselineskip}
Centre for Applied Scientific Computing \\
Lawrence Livermore National Laboratory    \\
Livermore, CA, 94551   \\
henshaw@llnl.gov \\
http://www.llnl.gov/casc/people/henshaw \\
http://www.llnl.gov/casc/Overture
\vspace{\baselineskip}
\today
\vspace{\baselineskip}
UCRL-MA-134300
% LA-UR-98-1967

\vspace{4\baselineskip}

\noindent{\bf Abstract:}
This document provides information for developers of Overture software.
\end{flushleft}

\tableofcontents
% \listoffigures


\vfill\eject
\section{Naming Conventions}


\begin{description}
  \item[general identifiers]: are written without underscores, using embedded captials to 
   denote the beginning of new words. Examples {\bf numberOfDimensions}, {\bf getClassName}, 
    {\bf inverseVertexDerivative}. All identifiers other than class names are {\bf not} capitalized.
  \item[class names] : begin with an initial capital, examples: {\bf SquareMapping}, {\bf MappedGridOperators},
   {\bf GridCollection}.
  \item[publically visible identifiers] : such as class names and public member function names should avoid introducing
    short forms. For example, we prefer {\bf numberOfComponentGrids()} to {\bf numGrids()} and
    {\bf realMappedGridFunction} to {\bf realMGF}.
  \item[macro names] : should be in all capitals with words separated by underscores.
  \item[enumerators] : The name of the enumerator should be capitalized (?) while the actual enumerators
    should follow the rules for general identifiers.
\end{description}


\section{Overture Types}

\subsection{Type names}
The following are the suggested type names to use, found in 
{\bf Overture/include/OvertureTypes.h}

\begin{description}
  \item[bool] :  is always defined even on machines that do not yet support the standard.
  \item[Real, real] : is either a float or double, depending on whether Overture is compiled in 
           single precision or double precision. Normally you should never declare a {\bf float}
           or {\bf double}, use {\bf Real} instead so that your code is precision independent.
  \item[IntegerArray] : is a serial (non-distributed array), equal to {\bf intSerialArray} 
       in P++ (the type {\bf intSerialArray} does not exist in A++ ?).
  \item[IntegerDistributedArray, intArray] : is a distributed array.
  \item[RealArray] : is a serial (non-distributed array), equal to {\bf floatSerialArray} or
        {\bf doubleSerialArray}.     
  \item[RealDistributedArray, realArray] : is a distributed array.
\end{description}



\subsection{Enums and constants}
Here are some useful enum's and constants some of which are found in {\bf Overture/include/wdhdefs.h}
\begin{description}
  \item[REAL\_EPSILON] : use instead of {\bf FLT\_EPSILON} or {\bf DBL\_EPSILON} to have precision
     independent code. The other machine constants, such as {\bf REAL\_RADIX}, {\bf REAL\_ROUNDS},
     {\bf REAL\_MANT\_DIG}, {\bf REAL\_MAX}, {\bf REAL\_MIN  }, and {\bf REAL\_MIN\_EXP }, 
      are defined in a similar way
  \item[axis1, axis2, axis3] : are just equal to 0,1,2 and can be used to name the coordinates axes.
  \item[Start, End] are just equal to 0,1 and can be used to name the start and end of a coordinate axis.
  \item[Pi, twoPi] : {\bf Pi}$=\pi$ and {\bf twoPi}$=2\pi$.
  \item[getCPU()] : returns the current cpu count in seconds.
  \item[char* sPrintF(char *s, const char *format, ...)] : like {\bf sprintf} but returns the 
       character string.
  \item[int sScanF(const String \& s, const char *format, ...)] like {\bf sscanf} but removes commas
      from the input String `s' (so that the `format' string can assume there are no commas separating
      input fields )and converts `%e' formats when Overture is compiled in double precision.
  \item
\end{description}

\subsection{Global objects}
Here are some useful global variables some of which are found in {\bf Overture/include/wdhdefs.h}
These can be used as a default arguments, for example.
\begin{description}
  \item[Index nullIndex] : a null Index.
  \item[Range nullRange]  : a null Range.
  \item[String nullString] 
  \item[String blankString] 
  \item[IntegerArray nullIntArray] 
  \item[RealArray nullRealArray] 
  \item[MappingParameters nullMappingParameters]
  \item[GraphicsParameters defaultGraphicsParameters]
  \item[PlotStuffParameters defaultPlotStuffParameters]
  \item[BoundaryConditionParameters defaultBoundaryConditionParameters]
\end{description}



\section{Common pitfalls and useful tricks}


\begin{description}
  \item[avoid scalar indexing of arrays]  where possible. In particular avoid scalar indexing of
    distributed arrays as this will be slow in the parallel environment. 
   Be aware of the {\bf seqAdd}, {\bf where} and {\bf sum} A++ operations.
  \item[evaluate dangling A++ expressions] since an expression is a temporary until it hits
    an equals operator, copy constructor or is evaluated.
    \begin{verbatim}
         floatArray a,b;
         ....
         myFunction( a(I1,I2)+b(I1,I2) );             // ***** WRONG WAY ******
         myFunction( evaluate(a(I1,I2)+b(I1,I2)) );   // the right way
         realArray & c = evaluate(a(I1,I2)+b(I1,I2));
         realArray d = a(I1,I2)+b(I1,I2);             // this is ok since the copy constructor is called
    \end{verbatim}
  \item[watch out for referencing a view] In the example below the array {\bf c} will be a reference
    to the view {\bf a(I)}. It will {\bf NOT} be a copy.
    \begin{verbatim}
         floatArray a;
         ....
         realArray c = a(I);     // *AVOID* This is a reference to a view on most compilers.
         realArray c(a(I));      // This is a reference to a view
         realArray & d = a(I);   // This is a reference to a view
         realArray d; d = a(I);  // This is a copy
    \end{verbatim}
  \item[cast MappedGridFunctions to arrays] : when writing long expressions that use array indexing, so
    the compiler can avoid an extra level of function calls in the parenthesis operator ().
    \begin{verbatim}
        realMappedGridFunction velocity(...);
        ...
        realArray & u = velocity;
        ...
        u(I1,I2,I3,0)=u(I1+1,I2,I3,0)*u(I1+1,I2,I3,1)+...;
    \end{verbatim}
  \item[Reference views for readability] : when a view appears many times in expressions.
    \begin{verbatim}
       const realArray & a1 = b(I1+1,I2,I3);      
       const realArray & a2 = b(I1+1,I2+1,I3);      
       const realArray & a3 = b(I1+1,I2+1,I3+1);      

       c = a1*a2+a3;
       d = a2+a3*a1;
    \end{verbatim}
  \item[Reuse a where mask for efficiency]: 
    An A++ logical expression just builds an intArray holding ones and zeros. Instead of computing the
    expression twice as in
    \begin{verbatim}
       where( a(I)>0. && a(I-1)>7. )
         b(I)=c(I)+3;
       ...
       where( a(I)>0. && a(I-1)>7. )
         c(I)=7;
    \end{verbatim}
    you can just build the expression first and then reuse it as in
    \begin{verbatim}
       intArray mask = a(I)>0. && a(I-1)>7.;
       where( mask )
        b(I)=C(I)+3;
       ...
       where( mask )
        c(I)=7.;      
    \end{verbatim}
  \item[Avoid derivation from most Overture Classes] : avoid the temptation to derive a new class from
    most Overture classes such as A++ array's, MappedGridFunction's, MappedGrid's, GridCollections's etc. 
    Derived classes are usually hard to support and experience shows that many levels of derivation is
    evil. Of course some classes such as Mapping's or the ReferenceCounting class 
    are explicitly meant to be derived from; this is clear from the documentation.
\end{description}

\section{Parallel}

  When compiling with P++, one cannot mix serial and distributed
array operations. The basic rule of thumb is that small arrays are serial and big
arrays (e.g. those that live on a MappedGrid) are distributed.

\end{document}