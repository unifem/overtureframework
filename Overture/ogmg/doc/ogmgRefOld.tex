%=======================================================================================================
% Ogmg Multigrid solver : Reference Guide for developers
%=======================================================================================================
\documentclass[12pt]{article}

\input homeHenshaw

\usepackage{times}  % for embeddable fonts, Also use: dvips -P pdf -G0
\input documentationPageSize.tex

\usepackage{amsmath}
\usepackage{amssymb}

\usepackage{verbatim}
\usepackage{moreverb}
\usepackage{graphics}    
\usepackage{epsfig}    
% \usepackage{calc}
% \usepackage{ifthen}
% \usepackage{float}
% \usepackage{fancybox}

\usepackage{makeidx} % index
\makeindex
\newcommand{\Index}[1]{#1\index{#1}}

\input{pstricks}\input{pst-node}

\begin{document}

% -----definitions-----

\newcommand{\ogen}{\homeHenshaw/res/ogen}
\newcommand{\figures}{\homeHenshaw/Overture/docFigures}
\newcommand{\automg}{\homeHenshaw/papers/automg}
% \newcommand{\automg}{../automg}
\newcommand{\ogmgDir}{\homeHenshaw/Overture/ogmg/doc}
\newcommand{\ogmgDocDir}{\homeHenshaw/Overture/ogmg/doc}
% \newcommand{\ogmgDir}{.}

\newcommand{\Ogen}{{Ogen}}
\newcommand{\Overture}{{Overture}}
\newcommand{\Ogmg}{{Ogmg}}

\newcommand{\figWidth}{.495\linewidth}
\newcommand{\tablefontsize}{\footnotesize}
\newcommand{\clipfig}{}

\input wdhDefinitions.tex


%---------- Title Page for a Research Report----------------------------
\vspace{5\baselineskip}
\begin{flushleft}
{\Large
Reference Guide for Ogmg: A Multigrid Solver for Overlapping Grids, \\
Version 1.00 \\
}
\vspace{3\baselineskip}
William D. Henshaw   \\                    
\vspace{2\baselineskip}
Centre for Applied Scientific Computing \\
Lawrence Livermore National Laboratory    \\
Livermore, CA, 94551   \\
henshaw@llnl.gov \\
http://www.llnl.gov/casc/people/henshaw \\
http://www.llnl.gov/casc/Overture\\
\vspace{2\baselineskip}
\today\\
\vspace{\baselineskip}
UCRL-MA-???????


\vspace{4\baselineskip}

\noindent{\bf Abstract:}
This is the reference guide to \Ogmg, the Overlapping-Grid-MultiGrid-solver.
This document is primarily intended for developers and contains reference notes,
member function descriptions as well current and future work. In addition this
report contains extensive results from running \Ogmg~ on a variety of problems.

The Overlapping-Grid-MultiGrid-solver, \Ogmg, that can be used to obtain solutions 
to elliptic boundary value problems.
% on composite overlapping grids with the multigrid algorithm.  
\Ogmg~ solves problems in two and three space
dimensions on composite overlapping grids. 
Second and fourth-order accurate approximations are supported.
Given an overlapping grid generated from the \Ogen~  grid generator,
\Ogmg~  will generate the coarse grid multigrid levels using an automatic coarsening algorithm.
The equations on the coarse grids can be determined automatically using a Galerkin averaging
procedure.
The multigrid solution algorithm has been optimised for some commonly occuring problems such as
equations defined with the Laplace operator.
Smoothers include Red-Black, Jacobi, Gauss-Seidel, line-zebra and line-Jacobi.
\Ogmg~  is particularly efficient when a majority of the grid points belong to cartesian component grids;
this is often the case when grids become sufficiently fine.
The fourth-order accurate approximations are solved directly with multigrid (as opposed to using
a defect correction scheme). Convergence rates for the fourth-order approximations are often nearly as
good as the convergence rates for second-order discretizations.
% Currently only scalar elliptic boundary value problems can be solved. 
\end{flushleft}

\vfill\eject
\tableofcontents
% \listoffigures

%---------- End of title Page for a Research Report


\vfill\eject
\section{Introduction}\index{multigrid}

\Ogmg~ is a multigrid solver for use with \Overture~\cite{overset96},\cite{OGES}.
\Ogmg~ can solve scalar elliptic problems
on overlapping grids. This reference manual is intended for developers and describes
some of the details of the code structure and organization.
This report also describes work that needs to be completed and potential new features
as well as working notes.


\section{Work in progress notes}

\begin{itemize}
  \item Fourth-order accuracy: To do: Neumann equation BC's in 3D.
  \item Optimize storage for fourth-order curvilinear grids by using jacobian derivatives
        instead of forming the elliptic operator.
  \item Allow sharing of the grid hierarchy amongst different \Ogmg~ solvers.
  \item Clean up the code, further optimizations.
\end{itemize}


\input eigenfunctions


\clearpage
\section{Singular problems}

When the problem being solved is singular, such as the Neumann problem
\begin{align*}
   \Delta u & = f \\
   u_n & =g
\end{align*}
there are some difficulties using multigrid and estimating the errors and
convergence rates. The problem stems from the facts that the solution is
only determined up to a constant and that there is a compatability condition
on $f$ and $g$ that may not be exactly satisfied in the discrete equations. 
If the compatability condition is not satisfied then the residual cannot 
be driven to zero but rather the solution can only be expected to converge
in a generalized sense. 


$$
    A U = F 
$$
If $l^T A =0$ then the compatability condition is $l^TF=0$. 

     
When Oges is used to solve a singular system such as the one given
it solves the related non-singular problem
\begin{equation} \label{aug}
   \left[ \begin{array}{cc}
            A  & \rv  \\
          \rv^T &  0
           \end{array}\right]
   \left[ \begin{array}{c} U \\ \alpha  \end{array}\right]
 = \left[ \begin{array}{c} F \\ 1  \end{array}\right]
\end{equation}
where $\rv=[1,1,\ldots,1]^T$ is the right null vector.
Although the matrix $A$ is singular the augmented matrix
is nonsingular and has a unique solution:
\begin{align*}
   A U &= F - \alpha \rv \\
   \rv^T U &=0 \\
 \alpha &= { \lv^T F \over \lv^t\rv } 
\end{align*}


We could try to solve the above non-singular system with multigrid. The
question arises how to compute $\alpha$ or how to compute iterates
$\alpha^n$ that converge. I don't know how to do this. 
If we knew the left null-vector then we could compute $\alpha$ directly.
The left null vector is expensive to compute and thus we may wish to avoid
computing it (especially in a moving grid application where the left null
vector would be changing as the grid changes.)

Another approach is to use multigrid to iterate as if the problem were
non-singular. If we further set the mean value of the solution to 
be zero at every iteration we would then expect the solution to 
converge. The only problem is that the residual will not go to
zero if the compatibility condition is not satisfied. 
The multigrid solver relies on knowing an estimate for the
norm of the residual in order to cycle properly. We therefore
would like an estimated residual that does go to zero. 

To get this estimate we can define the solution to our singular problem
to be the one that satisfies
\begin{align*}
    \rv^n & = \fv - A \uv^n \\    
    \tilde{\rv}^n &= \rv^n -\alpha \zv^*
\end{align*}
where $\alpha$ minimizes the expression
\[
    \min_\alpha \| A \uv - \fv - \alpha \wv \|
\]
and $\wv$ satisfies
\begin{align*}
     A \wv &= \rv \\
     \rv^T \wv &= 1
\end{align*}


\clearpage
\input numericalResults



\clearpage
\section{References}
\bibliography{\homeHenshaw/papers/henshaw}
\bibliographystyle{siam}

\clearpage
\section{Ogmg Function Descriptions}

\input OgmgInclude



\printindex

\end{document}


