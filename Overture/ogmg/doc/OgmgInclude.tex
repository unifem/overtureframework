\subsection{constructor}
 
\newlength{\OgmgIncludeArgIndent}
\begin{flushleft} \textbf{%
\settowidth{\OgmgIncludeArgIndent}{Ogmg(}% 
Ogmg()
}\end{flushleft}
\begin{description}
\item[{\bf Description:}] 
   Default constructor.
\end{description}
\subsection{constructor}
 
\begin{flushleft} \textbf{%
\settowidth{\OgmgIncludeArgIndent}{Ogmg(}% 
Ogmg( CompositeGrid \& mg, \\ 
\hspace{\OgmgIncludeArgIndent}GenericGraphicsInterface *ps\_  = 0) 
}\end{flushleft}
\begin{description}
\item[{\bf Description:}] 
   
    Build a multigrid solver.
\item[{\bf mg (input) :}]  grid to use
\item[{\bf ps\_ (input) :}]  supply an optional GenericGraphicsInterface object for plotting.
\item[{\bf Notes:}]  
    Here are some notes. See the Ogmg User Guide for further details.
  \begin{description}
    \item[Boundary Conditions]: Ogmg looks at the coefficient matrix to determine if the ghost line
     values are extrapolated or not. If not it assumes there is some sort of neumann or mixed boundary
     conditions and does a few things differently.
    \item[Interpolants]  The multigrid solver needs to interpolate at the different levels.
      If no Interpolant is found to be associated with a CompositeGrid at a given level
      the {\tt updateToMatchGrid} routine will build an Interpolant the first time (and update it
      on subsequent calls). Thus if you have built an Interpolant for any level then you are responsible
      to update it if the grid changes.
  \end{description}
\end{description}
\subsection{setPlotStuff}
 
\begin{flushleft} \textbf{%
void  \\ 
\settowidth{\OgmgIncludeArgIndent}{set(}%
set( GenericGraphicsInterface *ps\_ )
}\end{flushleft}
\begin{description}
\item[{\bf Description:}] 
    Supply a GenericGraphicsInterface object to use for plotting.
\item[{\bf ps\_ (input) :}]  pointer to a GenericGraphicsInterface object.
\end{description}
\subsection{updateToMatchGrid}
 
\begin{flushleft} \textbf{%
void  \\ 
\settowidth{\OgmgIncludeArgIndent}{updateToMatchGrid(}%
updateToMatchGrid( CompositeGrid \& mg\_ )
}\end{flushleft}
\begin{description}
\item[{\bf Description:}] 
    Update the solver to match this grid.
\item[{\bf mg (input) :}]  grid to use
\end{description}
\subsection{getMaximumResidual}
 
\begin{flushleft} \textbf{%
real  \\ 
\settowidth{\OgmgIncludeArgIndent}{getMaximumResidual(}%
getMaximumResidual() const
}\end{flushleft}
\begin{description}
\item[{\bf Description:}]  
   Return the maximum residual from the last solve
\end{description}
\subsection{getNumberOfIterations}
 
\begin{flushleft} \textbf{%
int  \\ 
\settowidth{\OgmgIncludeArgIndent}{getNumberOfIterations(}%
getNumberOfIterations() const
}\end{flushleft}
\begin{description}
\item[{\bf Description:}]  
   Return the number of multigrid iterations (cycles).
\item[{\bf Return value:}]  the number of iterations.
\end{description}
\subsection{sizeOf}
 
\begin{flushleft} \textbf{%
real   \\ 
\settowidth{\OgmgIncludeArgIndent}{sizeOf(}%
sizeOf( FILE *file  =NULL) const 
}\end{flushleft}
\begin{description}
\item[{\bf Description:}]  
   Return number of bytes allocated by Ogmg; Optionally print detailed info to a file

 An estimate of space requirements is  $20N$ (2D) or $40N$ (3D) where $N$ is the number of
  grid points on the finest grid (assumes maximum number of levels so that $.5+.25+.125+...=1$.
 For line smoothers there is addition space required.

\item[{\bf file (input) :}]  optionally supply a file to write detailed info to. Choose file=stdout to
 write to standard output.
\item[{\bf Return value:}]  the number of bytes.
\end{description}
\subsection{setOgmgParameters}
 
\begin{flushleft} \textbf{%
int   \\ 
\settowidth{\OgmgIncludeArgIndent}{setOgmgParameters(}%
setOgmgParameters(OgmgParameters \& parameters\_ )
}\end{flushleft}
\begin{description}
\item[{\bf Description:}] 
 set parameters equal to another parameter object.

\end{description}
\subsection{fullMultiGrid}
 
\begin{flushleft} \textbf{%
int  \\ 
\settowidth{\OgmgIncludeArgIndent}{fullMultigrid(}%
fullMultigrid()
}\end{flushleft}
\begin{description}
\item[{\bf Description:}] 
    Perform the first full multigrid cycle (nested iteration) in order to obtain an initial guess.


                        X
                       /
                  X   X
                 / \ /
                X   X

\end{description}
\subsection{solve}
 
\begin{flushleft} \textbf{%
int  \\ 
\settowidth{\OgmgIncludeArgIndent}{solve(}%
solve( realCompositeGridFunction \& u, realCompositeGridFunction \& f)
}\end{flushleft}
\begin{description}
\item[{\bf Description:}] 
   Solve Au=f with multigrid

\item[{\bf u (input/output) :}]  initial guess on input, answer on output. It is NOT necessary
    that u be defined on all multigrid levels. u may only live on finest level.
    It is best if u satisfies the boundary conditions on input although this is not required.
\item[{\bf f (input) :}]  right hand side for the problem. f should be defined for the finest level.
    As with u, f can only be defined on the finest level if desired.
\end{description}
\subsection{cycle}
 
\begin{flushleft} \textbf{%
int  \\ 
\settowidth{\OgmgIncludeArgIndent}{cycle(}%
cycle(const int \& level, const int \& iteration, real \& maximumDefect, const int \& numberOfCycleIterations )
}\end{flushleft}
\begin{description}
\item[{\bf Description:}] 
   Perform a multigrid cycle. This routine is called recursively.

\item[{\bf maximumDefect (input/output) :}]  on input (if non-negative) 
     this is the current maximum defect, on output this is the new  maximum defect.

\end{description}
\subsection{printStatistics}
 
\begin{flushleft} \textbf{%
void  \\ 
\settowidth{\OgmgIncludeArgIndent}{printStatistics(}%
printStatistics(FILE *file\_  =stdout) const
}\end{flushleft}
\begin{description}
\item[{\bf Description:}] 
   Print performance statistics such as the cpu time required by various routines.

\end{description}
\subsection{setCoefficientArray}
 
\begin{flushleft} \textbf{%
int  \\ 
\settowidth{\OgmgIncludeArgIndent}{setCoefficientArray(}%
setCoefficientArray( realCompositeGridFunction \& coeff,\\ 
const IntegerArray \& bc   =Overture::nullIntArray(),\\ 
const RealArray \& bcData  ==Overture::nullRealArray())
}\end{flushleft}
\begin{description}
\item[{\bf Description:}] 
    Supply the coefficient matrix, and optionally supply boundary conditions.
\item[{\bf matrix (input) :}]  a coefficient matrix defined on all levels.

\item[{\bf bc(0:}] 1,0:2,numberOfComponentGrids) (input): boundary conditions, Ogmg::dirichlet, neumann or mixed.
    If this array is NOT supplied then the boundary conditions, dirichlet, neumann, mixed should be
  given in cg[grid].boundaryCondition(side,axis)
\item[{\bf bcData(0:}] ?,0:1,0:2,numberOfComponentGrids) (input) : data for the boundary conditions.
    For mixed boundary conditions bcData(0:1,side,axis,grid) = (a0,a1) are the coefficients in the 
    mixed condition: a0*u + a1*u.n = g

 Ogmg::updateToMatchGrid should have already been called at this point so that
 the the multigrid-composite grid with extra levels has already been built.
\end{description}
\subsection{setBoundaryConditions}
 
\begin{flushleft} \textbf{%
int  \\ 
\settowidth{\OgmgIncludeArgIndent}{setBoundaryConditions(}%
setBoundaryConditions(const IntegerArray \& bc\_,\\ 
const RealArray \& bcData  =Overture::nullRealArray())
}\end{flushleft}
\begin{description}
\item[{\bf Description:}] 
    Set the boundary conditions. This is a private routine.

 This function assigns the internal arrays:

   boundaryCondition(side,axis,grid) = extrapolate
                                     = equation
   bc(side,axis,grid) = dirichlet
                      = neumann
                      = mixed
 
\end{description}
\subsection{setOrderOfAccuracy}
 
\begin{flushleft} \textbf{%
int  \\ 
\settowidth{\OgmgIncludeArgIndent}{setOrderOfAccuracy(}%
setOrderOfAccuracy(const int \& orderOfAccuracy\_)
}\end{flushleft}
\begin{description}
\item[{\bf Description:}] 
    Set the order of accuracy (2 or 4).
\item[{\bf orderOfAccuracy\_ (input) :}]  

\end{description}
\subsection{interpolate}
 
\begin{flushleft} \textbf{%
int  \\ 
\settowidth{\OgmgIncludeArgIndent}{interpolate(}%
interpolate(realCompositeGridFunction \& u, const int \& grid  =-1 */, int level /* =-1)
}\end{flushleft}
\begin{description}
\item[{\bf Description:}] 

 Interpolate here so we can keep track of the cpu time used.

\item[{\bf grid (input) :}]  interpolate this grid only (if possible)
\end{description}
\subsection{update}
 
\begin{flushleft} \textbf{%
int  \\ 
\settowidth{\OgmgIncludeArgIndent}{update(}%
update( GenericGraphicsInterface \& gi )
}\end{flushleft}
\begin{description}
\item[{\bf Description:}] 
   Update parameters interactively.
\end{description}
\subsection{update}
 
\begin{flushleft} \textbf{%
int  \\ 
\settowidth{\OgmgIncludeArgIndent}{update(}%
update( GenericGraphicsInterface \& gi, CompositeGrid \& cg  )
}\end{flushleft}
\begin{description}
\item[{\bf Description:}] 
   Update parameters interactively. Use this update if you have not already given a
 CompositeGrid to Ogmg (through a constructor or with the updateToMatchGrid function).

\end{description}
\subsection{computeDefectRatios}
 
\begin{flushleft} \textbf{%
void  \\ 
\settowidth{\OgmgIncludeArgIndent}{computeDefectRatios(}%
computeDefectRatios( int level )
}\end{flushleft}
\begin{description}
\item[{\bf Description:}] 
  Compute the defect ratios needed for the automatic sub-smoothing algorithm
  (We actually just compute the l2-norm of the defect here -- the ratio is computed in smooth)
\end{description}
\subsection{smooth}
 
\begin{flushleft} \textbf{%
void  \\ 
\settowidth{\OgmgIncludeArgIndent}{smooth(}%
smooth(const int \& level, int numberOfSmoothingSteps, int cycleNumber )
}\end{flushleft}
\begin{description}
\item[{\bf Description:}] 
     This is the "composite" smooth routine. Smooth on all grids using a possibly different
  smoother for each grid.
\end{description}
\subsection{applyOgesSmoother}
 
\begin{flushleft} \textbf{%
void      \\ 
\settowidth{\OgmgIncludeArgIndent}{applyOgesSmoother(}%
applyOgesSmoother(const int level, const int grid)
}\end{flushleft}
\begin{description}
\item[{\bf Description:}] 
    Jacobi or Gauss-Seidel Smoother.

\item[{\bf smootherChoice (input) :}]  0=jacobi, 1=gauss-seidel
\end{description}
\subsection{smoothJacobi}
 
\begin{flushleft} \textbf{%
void  \\ 
\settowidth{\OgmgIncludeArgIndent}{smoothJacobi(}%
smoothJacobi(const int \& level, const int \& grid, int smootherChoice  = 0)
}\end{flushleft}
\begin{description}
\item[{\bf Description:}] 
    Jacobi or Gauss-Seidel Smoother.

\item[{\bf smootherChoice (input) :}]  0=jacobi, 1=gauss-seidel
\end{description}
\subsection{smoothGaussSeidel}
 
\begin{flushleft} \textbf{%
void  \\ 
\settowidth{\OgmgIncludeArgIndent}{smoothGaussSeidel(}%
smoothGaussSeidel(const int \& level, const int \& grid)
}\end{flushleft}
\begin{description}
\item[{\bf Description:}] 
    Gauss Seidel Smoother. NOT implemented yet.
\end{description}
\subsection{smoothRedBlack}
 
\begin{flushleft} \textbf{%
void  \\ 
\settowidth{\OgmgIncludeArgIndent}{smoothRedBlack(}%
smoothRedBlack(const int \& level, const int \& grid)
}\end{flushleft}
\begin{description}
\item[{\bf Description:}] 
    Red-Black Smoother

  First smooth "red" points, then black points

 \begin{verbatim}
  Two-dimensions:
        shift1  shift2 
          0       0
          1       1
          1       0
          0       1
  Three-dimensions:
        shift1  shift2  shift3 
          0       0      0
          1       1      0
          1       0      1
          0       1      1
          1       0      0
          0       1      0
          0       0      1
          1       1      1
 \end{verbatim}
\end{description}
\subsection{smoothLine}
 
\begin{flushleft} \textbf{%
void  \\ 
\settowidth{\OgmgIncludeArgIndent}{smoothLine(}%
smoothLine(const int \& level, const int \& grid, \\ 
\hspace{\OgmgIncludeArgIndent}const int \& direction, \\ 
\hspace{\OgmgIncludeArgIndent}bool useZebra  =true,\\ 
\hspace{\OgmgIncludeArgIndent}const int smoothBoundarySide  = -1)
}\end{flushleft}
\begin{description}
\item[{\bf Description:}] 
    Line Smoother. Zebra line smoothing.
    
\item[{\bf direction (input) :}]  smooth on lines in this direction, 0,1,2
\item[{\bf useZebra (input) :}]  if true use alternating line zebra smoothing.
\end{description}
\subsection{alternatingLineSmooth}
 
\begin{flushleft} \textbf{%
void   \\ 
\settowidth{\OgmgIncludeArgIndent}{alternatingLineSmooth(}%
alternatingLineSmooth(const int \& level, const int \& grid, bool useZebra  =true)
}\end{flushleft}
\begin{description}
\item[{\bf Description:}] 
    Here we do alternating line smooths; one line smooth in each direction.
\end{description}
\subsection{fineToCoarse(level)}
 
\begin{flushleft} \textbf{%
void  \\ 
\settowidth{\OgmgIncludeArgIndent}{fineToCoarse(}%
fineToCoarse(const int \& level, bool transferForcing  = false)
}\end{flushleft}
\begin{description}
\item[{\bf Description:}] 
     Transfer the defect from the fine grid at 'level' to the coarse grid at 'level+1'
\item[{\bf level (input):}] 
\item[{\bf transferForcing (input) :}]  by default we transfer the defect. For the full multigrid algorithm we
    transfer the forcing to coarser levels (so that we cannot assume the defect to be zero on Dirichlet
    boundaries, for example).

 *** these next comments are probably wrong ****
 \begin{verbatim}
         Fine to Coarse (Restriction) Transfer
         -------------------------------------
            f2 <- Restriction( f1 )
 
  Notes:
 
   (1) Full Weighting at all Interior Nodes with mask() > 0
   (2) Boundary nodes(*) are full weighted in the boundary
       (i.e. 1 dimension less) if mask > 0
   (3) The first line of fictitious points(*) are full weighted
       amongst fictitious points of the first line,
       (i.e. 1 dimension less) if mask(boundary) > 0
   (*) except BC corners or edges where the defect is injected
 
   Philosophy:
     Don't average together defects that come from different types
     of equations. Boundary nodes are assumed to be distinct from
     interior nodes and the first line of fictitious points are
     also assumed to be of a different type. Corners in 2D
     or BC edges in 3D are also assumed to be distinct.
 \end{verbatim}

\end{description}
\subsection{fineToCoarse(level,grid)}
 
\begin{flushleft} \textbf{%
void  \\ 
\settowidth{\OgmgIncludeArgIndent}{fineToCoarse(}%
fineToCoarse(const int \& level, const int \& grid, bool transferForcing  = false)
}\end{flushleft}
\begin{description}
\item[{\bf Description:}] 
     Transfer the defect from the fine grid at 'level' to the coarse grid at 'level+1'
\item[{\bf level,grid (input):}] 

\item[{\bf Notes:}] 
   Full weighting on interior and boundary points where the equation is applied.
  If the BC for ghost points is extrapolation (e.g. if there is a dirichlet BC) then the
  boundary defects are averaged along the boundary. 

\end{description}
\subsection{coarseToFine(level)}
 
\begin{flushleft} \textbf{%
void  \\ 
\settowidth{\OgmgIncludeArgIndent}{coarseToFine(}%
coarseToFine(const int \& level) 
}\end{flushleft}
\begin{description}
\item[{\bf Description:}] 
    Correction: coarse to fine transfer.
 \[
          u[level] += \textbf{Prolongation}[ u[level+1] ]
 \]
\end{description}
\subsection{coarseToFine(level,grid)}
 
\begin{flushleft} \textbf{%
void  \\ 
\settowidth{\OgmgIncludeArgIndent}{coarseToFine(}%
coarseToFine(const int \& level, const int \& grid)
}\end{flushleft}
\begin{description}
\item[{\bf Description:}] 
  Correct a Component Grid
 \[
      u(i,j) = u(i,j) + P[ u2(i,j) ]   \qquad\textrm{( P : Prolongation )}
 \]
  cp21,cp22,cp23 : coeffcients for prolongation, 2nd order
  cp41,cp41,cp43 : coeffcients for prolongation, 4th order

\end{description}
\subsection{defect(level)}
 
\begin{flushleft} \textbf{%
void  \\ 
\settowidth{\OgmgIncludeArgIndent}{defect(}%
defect(const int \& level)
}\end{flushleft}
\begin{description}
\item[{\bf Description:}] 
    Defect computation 

   Fill in  defectMG[level] with f-Lu
\end{description}
\subsection{defect(level,grid)}
 
\begin{flushleft} \textbf{%
void  \\ 
\settowidth{\OgmgIncludeArgIndent}{defect(}%
defect(const int \& level, const int \& grid)
}\end{flushleft}
\begin{description}
\item[{\bf Description:}] 
    Defect computation on a component grid

    Compute defectMG.multigridLevel[level][grid]
\end{description}
\subsection{defect(level,grid)}
 
\begin{flushleft} \textbf{%
real  \\ 
\settowidth{\OgmgIncludeArgIndent}{defectMaximumNorm(}%
defectMaximumNorm(const int \& level, int approximationStride  =1)
}\end{flushleft}
\begin{description}
\item[{\bf Description:}] 
    Compute the maximum norm of the defect 

\item[{\bf WARNING:}]  Calling this routine will change defectMG.multigridLevel[level][grid]

\item[{\bf approximationStride :}]  use approximately this subset of points in each direction. If a given
    direction only has a few points then this value is reduced. If approximationStride=1 then
    this will be the true norm.

\end{description}
\subsection{defect(level,grid)}
 
\begin{flushleft} \textbf{%
real  \\ 
\settowidth{\OgmgIncludeArgIndent}{defectNorm(}%
defectNorm(const int \& level, const int \& grid, int option  =0 */, int approximationStride /* =8)
}\end{flushleft}
\begin{description}
\item[{\bf Description:}] 
    Compute a norm of the defect (optionally approximate, using a subset of the total points)

\item[{\bf WARNING:}]  Calling this routine will change defectMG.multigridLevel[level][grid]

\item[{\bf option:}]  0=get l2-norm, 1=get max-norm
\item[{\bf approximationStride :}]  use approximately this subset of points in each direction. If a given
    direction only has a few points then this value is reduced. If approximationStride=1 then
    this will be the true norm.

\end{description}
\subsection{getDefect}
 
\begin{flushleft} \textbf{%
real  \\ 
\settowidth{\OgmgIncludeArgIndent}{getDefect(}%
getDefect(const int \& level, \\ 
\hspace{\OgmgIncludeArgIndent}const int \& grid, \\ 
\hspace{\OgmgIncludeArgIndent}realArray \& f,     \\ 
\hspace{\OgmgIncludeArgIndent}realArray \& u, \\ 
\hspace{\OgmgIncludeArgIndent}const Index \& I1,\\ 
\hspace{\OgmgIncludeArgIndent}const Index \& I2,\\ 
\hspace{\OgmgIncludeArgIndent}const Index \& I3,\\ 
\hspace{\OgmgIncludeArgIndent}realArray \& defect,\\ 
\hspace{\OgmgIncludeArgIndent}const int lineSmoothOption  = -1,\\ 
\hspace{\OgmgIncludeArgIndent}const int defectOption  = 0,\\ 
\hspace{\OgmgIncludeArgIndent}real \& defectL2Norm, real \& defectMaxNorm )
}\end{flushleft}
\begin{description}
\item[{\bf Description:}] 
    Defect computation on a component grid

   Determine the defect = f - C*u

   This routine knows how to efficiently compute the defect for rectangular
   and non-rectangular grids. It also knows how to compute the defect for
   line smoothers
 Input -
   level,grid
   f,u
   I1,I2,I3
   lineSmoothOption = -1 :
                    = 0,1,2 : compute defect for line solve in direction 0,1,2
   defectOption = 0 : compute defect
                  1  : compute defect and l2 norm
                  2  : compute defect and max norm
 Output -
   defect

\item[{\bf Return Value:}]  l2-norm or max norm of the defect depending on defectOption

\end{description}
\subsection{getGhostLineBoundaryCondition}
 
\begin{flushleft} \textbf{%
FourthOrderBoundaryConditionEnum  \\ 
\settowidth{\OgmgIncludeArgIndent}{getGhostLineBoundaryCondition(}%
getGhostLineBoundaryCondition( int bc, int ghostLine, int grid, int level, \\ 
\hspace{\OgmgIncludeArgIndent}int \& orderOfExtrapolation, aString *bcName  =NULL) const
}\end{flushleft}
\begin{description}
\item[{\bf Description:}] 
   Get the boundary condition to use on a ghost line.
\item[{\bf bc (input) :}]  A boundary condition
\item[{\bf ghostLine,grid,level :}]  get BC for this grid and face
\item[{\bf orderOfExtrapolation (output) :}]  For an extrapolation BC this is the order
\item[{\bf bcName (output):}]  If non-NULL on input, return the name of the BC.
\item[{\bf Ouptut:}]  The BC to use is from the enum in OgmgParameters
 \begin{verbatim}
  enum FourthOrderBoundaryConditionEnum
  {
    useSymmetry,
    useEquationToFourthOrder,
    useEquationToSecondOrder,
    useExtrapolation
  };
 \end{verbatim}
\end{description}
\subsection{useEquationOnGhostLineForDirichletBC}
 
\begin{flushleft} \textbf{%
bool  \\ 
\settowidth{\OgmgIncludeArgIndent}{useEquationOnGhostLineForDirichletBC(}%
useEquationOnGhostLineForDirichletBC(MappedGrid \& mg, int level)
}\end{flushleft}
\begin{description}
\item[{\bf Description:}] 
    Return true if the eqn to 2nd order should be applied on the ghost line
 of a dirichlet boundary.
\item[{\bf mg (input) :}]  grid to use
\end{description}
\subsection{useEquationOnGhostLineForNeumannBC}
 
\begin{flushleft} \textbf{%
bool  \\ 
\settowidth{\OgmgIncludeArgIndent}{useEquationOnGhostLineForNeumannBC(}%
useEquationOnGhostLineForNeumannBC(MappedGrid \& mg, int level)
}\end{flushleft}
\begin{description}
\item[{\bf Description:}] 
    Return true if the eqn to 2nd order should be applied on the ghost line
 of a Neumann or mixed boundary.
\item[{\bf mg (input) :}]  grid to use
\end{description}
\subsection{initializeBoundaryConditions}
 
\begin{flushleft} \textbf{%
int  \\ 
\settowidth{\OgmgIncludeArgIndent}{initializeBoundaryConditions(}%
initializeBoundaryConditions(realCompositeGridFunction \& coeff)
}\end{flushleft}
\begin{description}
\item[{\bf Description:}] 
    Determine the type of boundary condition that is imposed on the **GHOSTLINE**
    for each side of each grid.
 There are 3 possibilities:
 \begin{enumerate}
   \item extrapolation : ghost point is extrapolated. This requires a special formula for the defect.
   \item equation : an equation such as a neumann or mixed boundary condition. This uses basically the
                same formula for the defect, but shifted to be centred on the boundary
   \item combination : a combination of the above two appears on the boundary.
\item[{\bf Notes:}] 
    We check the classify array to determine the type of boundary condition.
 \end{enumerate}
\end{description}
\subsection{applyBoundaryConditions(level,...)}
 
\begin{flushleft} \textbf{%
int  \\ 
\settowidth{\OgmgIncludeArgIndent}{applyBoundaryConditions(}%
applyBoundaryConditions( const int \& level, RealCompositeGridFunction \& u, RealCompositeGridFunction \& f )
}\end{flushleft}
 
\begin{description}
\item[{\bf Description:}] 
    Assign boundary conditions on the **GHOSTLINE**  for each side of each grid.

\end{description}
\subsection{applyBoundaryConditions(level,grid,...)}
 
\begin{flushleft} \textbf{%
int  \\ 
\settowidth{\OgmgIncludeArgIndent}{applyBoundaryConditions(}%
applyBoundaryConditions(const int \& level, \\ 
\hspace{\OgmgIncludeArgIndent}const int \& grid, \\ 
\hspace{\OgmgIncludeArgIndent}RealMappedGridFunction \& u, \\ 
\hspace{\OgmgIncludeArgIndent}RealMappedGridFunction \& f )
}\end{flushleft}
\begin{description}
\item[{\bf Description:}] 
    Assign boundary conditions on the **GHOSTLINE**  for each side of a grid.
  Values on the actual boundary are assumed to be done in the smoothing step since the
 coefficient maxtrix should hold the proper equation there (a Dirichlet BC for example).
\item[{\bf level,grid (input) :}] 
\item[{\bf u (input/output) :}]  apply BC's to this grid function.
\item[{\bf f (input):}]  rhs to the equation (needed to compute the defect for non-extrapolation BC's)

 There are 3 possibilities:
 \begin{enumerate}
   \item extrapolation : ghost point is extrapolated. This requires a special formula for the defect.
   \item equation : an equation such as a neumann or mixed boundary condition. This uses basically the
                same formula for the defect, but shifted to be centred on the boundary
   \item combination : a combination of the above two appears on the boundary.
 \end{enumerate}
\end{description}
\subsection{buildExtraLevels}
 
\begin{flushleft} \textbf{%
int  \\ 
\settowidth{\OgmgIncludeArgIndent}{buildExtraLevels(}%
buildExtraLevels(CompositeGrid \& mg)
}\end{flushleft}
\begin{description}
\item[{\bf Description:}]  
       Build extra multigrid levels. This routine will create coarser levels automatically.
 The tricky part is to determine how to interpolate on the new coarser levels.
 After a grid is coarsened it may no longer have enough interpolation points. We add new interpolation
 points to fill in the gaps. The width of the interpolation stencil is reduced, on a point by point
 basis, if necessary.

\item[{\bf mg (input/output):}]  
\end{description}
