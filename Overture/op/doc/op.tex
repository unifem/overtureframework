%-----------------------------------------------------------------------
% Differential Operators and Boundary Conditions
%-----------------------------------------------------------------------
\documentclass[12pt]{article}
\usepackage[bookmarks=true]{hyperref}

% \input documentationPageSize.tex
\hbadness=10000 
\sloppy \hfuzz=30pt

\usepackage{calc}
% set the page width and height for the paper (The covers will have their own size)
\setlength{\textwidth}{7in}  
\setlength{\textheight}{9.5in} 
% here we automatically compute the offsets in order to centre the page
\setlength{\oddsidemargin}{(\paperwidth-\textwidth)/2 - 1in}
% \setlength{\topmargin}{(\paperheight-\textheight -\headheight-\headsep-\footskip)/2 - 1in + .8in }
\setlength{\topmargin}{(\paperheight-\textheight -\headheight-\headsep-\footskip)/2 - 1in -.2in }

\input homeHenshaw

\usepackage{verbatim}
\usepackage{moreverb}
\usepackage{graphics}    
\usepackage{calc}
\usepackage{ifthen}
% \usepackage{fancybox} % *****WARNING****** inclusion of this causes the TOC to disappear!

\usepackage{amsmath}


\usepackage{array}

\usepackage{makeidx} % index
\makeindex
\newcommand{\Index}[1]{#1\index{#1}}

% ---- we have lemmas and theorems in this paper ----
\newtheorem{assumption}{Assumption}
\newtheorem{definition}{Definition}

\newcommand{\primer}{\homeHenshaw/Overture/primer}
\newcommand{\gf}{\homeHenshaw/Overture/gf}
\newcommand{\op}{\homeHenshaw/Overture/op}
\newcommand{\examples}{Overture/examples}
\newcommand{\mapping}{\homeHenshaw/Overture/mapping}
\newcommand{\ogshow}{\homeHenshaw/Overture/ogshow}
\newcommand{\oges}{\homeHenshaw/Overture/oges}
\newcommand{\figures}{../docFigures}
% \newcommand{\primer}{\homeHenshaw/res/primer}

\newcommand{\OvertureOverture}{\homeHenshaw/Overture/Overture}

\newcommand{\RA}{realArray}
\newcommand{\MGF}{MappedGridFunction}
\newcommand{\RMGF}{realMappedGridFunction}
\newcommand{\RCGF}{realCompositeGridFunction}

\newcommand{\DABO}{Differential\-And\-Boundary\-Operators}

\newcommand{\MG}{Mapped\-Grid}
\newcommand{\GC}{Grid\-Collection}
\newcommand{\CG}{Composite\-Grid}
\newcommand{\MGCG}{Multigrid\-Composite\-Grid}

\newcommand{\MGO}{MappedGridOperators}
\newcommand{\GCO}{Grid\-Collection\-Operators}
\newcommand{\CGO}{Composite\-Grid\-Operators}
\newcommand{\MGCGO}{Multigrid\-Composite\-Grid\-Operators}
\newcommand{\GCF}{GridCollectionFunction}


\begin{document}

% \input list.tex    % defines Lentry enviroment for documenting functions

% -----definitions-----
\input wdhDefinitions
% 
% \def\R      {{\bf R}}
% \def\Dv     {{\bf D}}
% \def\av     {{\bf a}}
% \def\bv     {{\bf b}}
% \def\cv     {{\bf c}}
% \def\fv     {{\bf f}}
% \def\Fv     {{\bf F}}
% \def\gv     {{\bf g}}
% \def\iv     {{\bf i}}
% \def\jv     {{\bf j}}
% \def\kv     {{\bf k}}
% \def\nv     {{\bf n}}
% \def\mv     {{\bf m}}
% \def\rv     {{\bf r}}
% \def\tv     {{\bf t}}
% \def\uv     {{\bf u}}
% \def\Uv     {{\bf U}}
% \def\vv     {{\bf v}}
% \def\Vv     {{\bf V}}
% \def\xv     {{\bf x}}
% \def\yv     {{\bf y}}
% \def\zv     {{\bf z}}
% \def\lt     {{<}}
% \def\grad    {\nabla}
\def\comma  {~~~,~~}
\def\uvd    {{\bf U}}
\def\ud     {{    U}}
\def\pd     {{    P}}
\def\calo{{\cal O}}

\def\ff {\tt} % font for fortran variables

\vspace{5\baselineskip}
\index{operators}
\begin{flushleft}
{\Large
Finite Difference Operators and Boundary Conditions for Overture \\
User Guide, Version 1.00 \\
}
\vspace{2\baselineskip}
Bill Henshaw \\
\vspace{\baselineskip}
Centre for Applied Scientific Computing \\
Lawrence Livermore National Laboratory    \\
Livermore, CA, 94551   \\
henshaw@llnl.gov \\
http://www.llnl.gov/casc/people/henshaw \\
http://www.llnl.gov/casc/Overture  \\
\vspace{\baselineskip}
\today \\
\vspace{\baselineskip}
UCRL-MA-132232
% LA-UR-96-3467

\vspace{4\baselineskip}

\noindent{\bf Abstract:}
We describe some finite difference operators and boundary conditions for
use with the Overture grid functions. Second and fourth order accurate
approximations are available for general curvilinear grids. For rectangular
periodic domains the pseudo-spectral approximations are also available.
\end{flushleft}

\tableofcontents
% \listoffigures

\vfill\eject
%---------- End of title Page for a Research Report

\section{Introduction}

We describe some finite difference operators and boundary conditions for
use with the Overture grid functions. 
The derivative operators allow one to take first and second order derivatives
($\partial_x$, $\partial_y$, $\partial_z$, $\partial_{xx}$, $\partial_{xy}$ etc.)
with second order, fourth order or spectral accuracy.  (Spectral accuracy is
for rectangular periodic domains only). 
The derivative operators can also be used
to generate the matrix (9 point stencil, for example) 
corresponding to a derivative operator. These ``coefficient'' operators can be used
to generate a sparse matrix.

The boundary condition operators define a ``library'' of elementary boundary
condition operations that can be used to implement application specific
boundary conditions. Examples of elementary boundary conditions include
Dirichlet, Neumann and mixed conditions, extrapolation, setting the normal
component of a vector and so on. A solver can apply one or more elementary
boundary conditions to the different sides of a grid.

% To use the boundary conditions one must first
% specify, for each side of each grid, the number and type of each boundary condition.
% Inhomogeneous values for the boundary conditions can be supplied and then the
% {\ff applyBoundaryConditions} function is called to apply each boundary condition
% in the order they were specified. 


The class {\ff \MGO} defines operators
for differentiating MappedGridFunction's and operators for applying
boundary conditions to MappedGridFunction's.

The classes {\ff \GCO}, {\ff \CGO} and {\ff \MGCGO} use the
operators in the class {\ff \MGO} (or a class derived there-of)
to define differential
and boundary condition operators for {\ff \GC}'s, 
{\ff \CG}'s and {\ff \MGCG}'s.

The {\ff \MGO} class can be used to compute spatial derivatives of a
{\ff realMappedGridFunction} including all first and 
second order derivatives
with respect to $x$, $y$ and $z$.

This class can also be used to define boundary conditions
and to evaluate the boundary conditions.

There may be one or more ``flavour'' of this class. One flavour
will define derivatives in the ``standard'' finite difference
manner using the ``mapping method''. Another flavour will define
derivatives using a finite volume approach. Yet other flavours
can be defined (by derivation from this class).

The grid function classes {\ff realMappedGridFunction} and
{\ff realCompositeGridFunction} 
have member functions for differentiation
and applying boundary conditions. 
A {\ff \MGF} has a pointer
to an object of the {\ff \MGO} class.
It uses this object to perform the differentiation or to
apply boundary conditions. To use a different ``flavour'' of
differentiation one must tell the grid function using the
{\ff set\MGO} member function. Similarly a {\ff \GCF}
has a pointer to a {\ff \GCO} and a {\ff CGF} has a 
pointer to a {\ff \CGO}.

\input \primer/otherDocs.tex

% This class defines various spatial derivatives of grid functions including
% all first and second order derivatives. For vertex-centred grids the derivatives
% are defined using finite differences and the ``mapping method''.
% Both second and fourth-order difference approximations are available.
% For cell-centred grids the derivatives are defined in a finite volume
% formulation.
% 
% 
% The classes {\ff typeDifferentiableCompositeGridFunction} and
% {\ff typeDifferentiableMappedGridFunction} ({\ff type=float} or {\ff type=double})
% are grid functions that know how to compute partial derivatives in space
% and how to apply boundary conditions. These classes are useful for writing
% PDE solvers. 
% 
% 
% The Differentiable classes are derived from the corresponding
% non-differentiable grid functions.

\subsection{Differentiation}\index{differentiation}

A number of different approximations are provided for a variety of differential operators. 
Given a vector-valued grid-function $u$, one may evaluate the 
% \setlength{\extrarowheight}{12pt}
% \begin{tabular}{|l|c|c|c|c|}\hline
%    Derivative                    & 2nd-order & 4th-order & 2c & 4c \\ \hline
% $\dfrac{\partial}{\partial x}$ & X & X &    &    \\ \hline
%  $\partial^2 / \partial x\partial y$ & X & X &    &    \\ \hline
% \end{tabular}
first derivatives:
\[
{\partial u \over \partial x}~,~~ {\partial u \over \partial y}~,~~{\partial u\over \partial z}
\]
second-derivatives:
\[
{\partial^2 u\over \partial x^2}~,~~
{\partial^2 u\over \partial y^2}~,~~
{\partial^2 u\over \partial z^2}~,~~
{\partial^2 u\over \partial x \partial y}~,~~
{\partial^2 u\over \partial x \partial z}~,~~
{\partial^2 u\over \partial y \partial z}
\]
as well as other second-order operators:
\[
   \Delta u = {\partial^2 u\over \partial x^2} + {\partial^2 u\over \partial y^2} + {\partial^2 u\over \partial z^2}
\]
\[
   \grad\cdot( s(\xv) \grad)= {\partial\over \partial x}( s(\xv){\partial u\over \partial x} ) +
                              {\partial\over \partial y}( s(\xv){\partial u\over \partial y} )+
                              {\partial\over \partial z}( s(\xv){\partial u\over \partial z} )
\]
\[
  {\partial\over \partial x}( s(\xv) {\partial u\over \partial x})~,~~
  {\partial\over \partial x}( s(\xv) {\partial u\over \partial y})~,~~
  {\partial\over \partial z}( s(\xv) {\partial u\over \partial x})~,~~
  {\partial\over \partial y}( s(\xv) {\partial u\over \partial x})~,~~
   \ldots
\]

There are second-order, fourth-order, sixth-order and eight-order accurate approximations. 
For many operators there are conservative (finite-volume) and non-conservative approximations.


\vskip\baselineskip

There are a number of different ways to evaluate derivatives of a grid function.
\begin{itemize}
\item Use the member function found in the {\ff \MGO}
  object.
\item Use the member function found in the {\ff realMappedGridFunction}
\item Use the member function found in the {\ff realCompositeGridFunction}
\item Use the member function {\ff getDerivatives} found in 
the {\ff \MGO} class to evaluate a set
of derivatives all at once. This is more efficient than the previous approaches.
\item Use the function {\ff derivative} to directly compute the derivative. This is
more efficient than the previous approaches. 
\end{itemize}
Currently the most natural way is not the most efficient because 
it involves extra computation and extra data movement.
All of these approaches are illustrated in the examples that follow.


%=========================================================================================
\vfill\eject
\input MappedGridOperators.tex
%=========================================================================================


%=========================================================================================
\vfill\eject
\input CompositeGridOperators.tex
%=========================================================================================

%=========================================================================================
%  Describe Boundary conditions
\vfill\eject
\input boundary.tex
%=========================================================================================

%=========================================================================================
%  Describe Coefficient matrices for sparse matrix systems
\vfill\eject
\input coefficients.tex
%=========================================================================================


%=========================================================================================
%  Describe the Fourier operators
\vfill\eject
\input fourier.tex
%=========================================================================================



\bibliography{\homeHenshaw/papers/henshaw}
\bibliographystyle{siam}

\printindex

\end{document}


