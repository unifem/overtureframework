
\subsection{MappedGridFunction}\index{MappedGridFunction}

This is a grid function that can be used with a MappedGrid.
The main purpose of the {\ff MappedGridFunction} is to act
like a ``smart'' A++ array. (Not that A++ arrays are not already
pretty smart).
This class is derived from an A++ array so all A++ operations
are defined.
Since it is associated (has a pointer to) a {\ff MappedGrid},
this grid function knows how to dimension itself and how
to update periodic edges when the grid is periodic.
It also knows how to update itself when the MappedGrid is
changed (perhaps the number of points on the grid is increased).
Here the update only involves redimensioning; not assigning
values to the new grid function.

The types of MappedGridFunctions are 
{\ff floatMappedGridFunction},
{\ff doubleMappedGridFunction},
 and {\ff int\-Mapped\-Grid\-Function}.

This is a reference counted class so that there is no need
to keep a pointer to a grid function. Use the {\ff reference}
member function to make one grid function reference another.


\begin{figure} \label{fig:MappedGridFunction}
  \begin{center}
  \includegraphics{\figures /mappedGridFunction.idraw.ps}
  \caption{Class diagram for a MappedGridFunction}
  \end{center}
\end{figure}

A grid function takes some of it's dimensions from the MappedGrid that
it is associated with. The index positions in a MappedGridFunction
corresponding to the dimensions
of the grid are called the {\bf coordinate} positions. A grid function 
can also have one or more {\bf component} positions which indicate
how many values are stored at each grid point.
Consider an example of a MappedGrid in 2D with dimensions {\ff(-1:11,-1:11)}
(these values are stored in the MappedGrid in the {\ff dimension} array).
A MappedGridFunction defined on this MappedGrid could have dimensions
{\ff u(-1:11,-1:11,0:2)}. This grid function {\ff u} has 3 components {\ff (0:2)},
and 
{\ff u.positionOfCoordinate(0)=0}, {\ff u.positionOfCoordinate(1)=1} and
{\ff u.positionOfComponent(0)=2}. 
A MappedGridFunction could also be created as {\ff u(0:1,-1:11,-1:11)} in
which case
{\ff u.positionOfCoordinate(0)=1}, {\ff u.positionOfCoordinate(1)=2} and
{\ff u.positionOfComponent(0)=0}.

% ===========================================================================
In the description of the member functions that follow, {\ff MappedGridFunction}
will stand for one of {\ff floatMappedGridFunction}, {\ff doubleMappedGridFunction},
or {\ff intMappedGridFunction}. In addition, any references to {\ff double} will change to
{\ff float} or {\ff int}.

\input MappedGridFunctionInclude.tex



In the following, ``{\ff type}'' will mean one of {\ff float}, {\ff double} or {\ff int}.
The most commonly used constructor takes a {\ff MappedGrid} and an optional 
sequence of Range's (or int's).
The optional {\ff Range} arguments indicate the positions of the 
coordinates and the positions and dimensions of any components.
\begin{tabbing}
{\ff typeMappedGridFunction(}\={\ff int m=3,xxxxxxxxxxxxxxxxxxxxxxxxxxx}\= \kill
{\ff typeMappedGridFunction()}\>\> default constructor \\
{\ff typeMappedGridFunction(MappedGrid \& mappedGrid, }\> \\
{\ff ~~~~~~~~~~~~~~~~~~~~~~~Range \& R0=nullRange,}\> \>this arg can be an int, or Range \\
{\ff ~~~~~~~~~~~~~~~~~~~~~~~Range \& R1=nullRange,}\> \\
{\ff ~~~~~~~~~~~~~~~~~~~~~~~Range \& R2=nullRange,}\> \\
{\ff ~~~~~~~~~~~~~~~~~~~~~~~  ...         }\> \\
{\ff ~~~~~~~~~~~~~~~~~~~~~~~Range \& R7=nullRange, )  }\> \\
{\ff typeMappedGridFunction(MappedGrid \& mappedGrid, }\> \> Old style declaration, THIS WILL GO AWAY\\
{\ff ~~~~~~~~~~~~~~~~~~~~~~~int numberOfComponents=1,  }\> \\
{\ff ~~~~~~~~~~~~~~~~~~~~~~~int positionOfComponent=default )  }\> \\
\end{tabbing}
Each grid function is dimensioned\index{grid function!dimension} according to the dimensions found
with the {\ff MappedGrid}, using the {\ff dimension} values.
Grid functions can have up to 8 dimensions, the index positions
not used by the coordinate dimensions can be used to store
different components. For example, a {\em vector} grid
functions would use 1 index position for components while a {\em matrix} grid functions would
use two index positions for components. 
Here are some examples
{\footnotesize\begin{verbatim}

        //  R1 = range of first dimension of the grid array
        //  R2 = range of second dimension of the grid array
        //  R3 = range of third dimension of the grid array 

        MappedGrid mg(...);

        Range R1(mg.dimension(Start,axis1),mg.dimension(End,axis1));
        Range R2(mg.dimension(Start,axis2),mg.dimension(End,axis2));
        Range R3(mg.dimension(Start,axis3),mg.dimension(End,axis3));

        Range all;    // null Range is used to specify where the coordinates are

        MappedGridFunction u(mg);                           //  --> u(R1,R2,R3);

        MappedGridFunction u(mg,all,all,all,1);             //  --> u(n,R1,R2,R3,0:1);
        MappedGridFunction u(mg,all,all,Range(1,1));        //  --> u(n,R1,R2,1:1,R3);

        MappedGridFunction u(mg,2,all);                     //  --> u(n,0:2,R1,R2,R3);
        MappedGridFunction u(mg,Range(0,2),all,all,all);    //  --> u(n,0:2,R1,R2,R3);
        MappedGridFunction u(mg,all,Range(3,3),all,all);    //  --> u(n,R1,3:3,R2,R3);

\end{verbatim}
}


\subsection{Examples}\index{MappedGridFunction!examples}

In this example we show how to define a grid function using
a {\ff MappedGrid}, how to assign the grid function with
A++ operations, how to reference one grid function to
another and how to set and get names for the grid function
and its components.
{\footnotesize\begin{verbatim}
  ...
   MappedGrid cg(...);  // here is a mapped grid
   floatMappedGridFunction u(cg),v;
   u=5.;
   Index I(0,10);
   u(I,I)=3.;
   v.reference(u);   // v is referenced to u
   v=7.;             // changes u as well
   v.breakReference();  // v is no longer referenced to u
   v=10.;            // v changed but not u

   // Here is how to dimension a grid function after it has been declared:
   floatMappedGridFunction w;
   Range all;
   int numberOfComponents=2;
   w.updateToMatchGrid( cg,all,all,all,numberOfComponents );  
   ...
   // give names to the grid function and components
   w.setName("w");      // name grid function
   w.setName("w.0",0);  // name component 0
   w.setName("w.1",1);  // name component 1
   cout << w.getName() << "," << w.getName(0) << "," << w.getName(1) << endl;
}
\end{verbatim}
}





\subsection{Grid functions defined on boundaries}\index{grid function!defined on boundaries}

Grid functions can be created so that they are defined on the boundary of a grid, or on 
the boundary and some number of neighbouring grid lines. 
For example, you may want to store the normal vectors on a given boundary of a grid.

To specify which boundary a grid function lives on one must create or update the grid
function with a Range object that was created in a special way. This special Range 
object is used in place of an ``all'' Range object.
Grid functions defined in this way can be correctly updated when the number of 
points on the grid changes -- they will still live on the appropriate boundary.

Here are some examples
(file {\ff \gf/edge.C})
{\footnotesize
\listinginput[1]{1}{\gf/edge.C}
}



\subsection{Grid Functions that hold coefficient matrices}\index{grid function!coefficient matrix}



A MappedGridFunction can be used to hold the coefficients for a sparse matrix.

{\bf ********** NOTE: This section is purely hypothetical. I am recording
some possible ways to interface to the coefficient matrices *****************}

Here is a standard way to store the coefficients:

{\footnotesize
\begin{verbatim}

   ****** this does not work yet *****

   MappedGrid mg(...);
   numberOfStencilCoefficients=pow(3,mg.numberOfDimensions);   // 9 or 27 points
   realMappedGridFunction coeff(numberOfStencilCoefficients,all,all,all);

   coeff.setIsACoefficientMatrix(standardStencil);             // 

   ... fill in coeff matrix here ....

   Index I1,I2,I3;
   getIndex(mg.gridIndexRange,I1,I2,I3);
   realMappedGridFunction u(mg),residual(mg);

   intArray & so = coeff.stencilOffset;    // these values indicate the offsets
   // 
   // Here is how the residual can be computed
   //
   //     residual(I1,I2,I3) = SUM coeff(m,I1,I2,I3) * u(I1+so(0,m),I2+so(1,m),I3+so(2,m))
   //                          m=0 
   //
   residual(I1,I2,I3) = coeff(0,I1,I2,I3)*u(I1+so(0,0),I2+so(1,0),I3+so(2,0))
                       +coeff(1,I1,I2,I3)*u(I1+so(0,1),I2+so(1,1),I3+so(2,1))
                       +coeff(2,I1,I2,I3)*u(I1+so(0,2),I2+so(1,2),I3+so(2,2))
                       +coeff(3,I1,I2,I3)*u(I1+so(0,3),I2+so(1,3),I3+so(2,3))
                       +coeff(4,I1,I2,I3)*u(I1+so(0,4),I2+so(1,4),I3+so(2,4))
                       +coeff(5,I1,I2,I3)*u(I1+so(0,5),I2+so(1,5),I3+so(2,5))
                           .... etc ....


    // 
    //  Here is another way which assumes it is a standard 3x3x3 stencil
    //
    intArray value(Range(-1,1),Range(-1,1),Range(-1,1));
    for( int m3=-1; m3<=1; m3++ )
    for( int m2=-1; m2<=1; m2++ )
    for( int m1=-1; m1<=1; m1++ )
    {
      value(m1,m2,m3)=coeff.coefficientPosition(m1,m2,m3);
      assert( value(m1,m2,m3)>=0 );   // check for any errors in computing the coefficient position
      
    }

    residual(I1,I2,I3) = coeff(value(-1,-1,-1),I1,I2,I3)*u(I1-1,I2-1,I3-1)
                        +coeff(value( 0,-1,-1),I1,I2,I3)*u(I1  ,I2-1,I3-1)
                        +coeff(value(+1,-1,-1),I1,I2,I3)*u(I1+1,I2-1,I3-1)
                        +coeff(value(-1, 0,-1),I1,I2,I3)*u(I1-1,I2  ,I3-1)
                        +coeff(value( 0, 0,-1),I1,I2,I3)*u(I1  ,I2  ,I3-1)
                        +coeff(value(+1, 0,-1),I1,I2,I3)*u(I1+1,I2  ,I3-1)
                             ... etc ...

\end{verbatim}
}
Here is how we handle systems of equations
{\footnotesize
\begin{verbatim}


   ****** this does not work yet *****

   MappedGrid mg(...);
   numberOfComponents=2;
   numberOfStencilCoefficients=pow(3,mg.numberOfDimensions);   // 9 or 27 points

   realMappedGridFunction coeff(numberOfComponents*numberOfStencilCoefficients,all,all,all);

   coeff.setIsACoefficientMatrix(standardStencil,numberOfComponents);             // 

   ... fill in coeff matrix here ....

   Index I1,I2,I3;
   getIndex(mg.gridIndexRange,I1,I2,I3);
   realMappedGridFunction u(mg,all,all,all,2),residual(mg,all,all,all,2);

   intArray & so = coeff.stencilOffset;    // these values indicate the offsets
   intArray & c  = coeff.stencilComponent; // these values indicate the component
   // 
   // Here is how the residual can be computed
   //
   //     residual(I1,I2,I3) = SUM coeff(m,I1,I2,I3) * u(I1+so(0,m),I2+so(1,m),I3+so(2,m))
   //                          m=0 
   //
   residual(I1,I2,I3) = coeff(0,I1,I2,I3)*u(I1+so(0,0),I2+so(1,0),I3+so(2,0),c(0))
                       +coeff(1,I1,I2,I3)*u(I1+so(0,1),I2+so(1,1),I3+so(2,1),c(1))
                       +coeff(2,I1,I2,I3)*u(I1+so(0,2),I2+so(1,2),I3+so(2,2),c(2))
                       +coeff(3,I1,I2,I3)*u(I1+so(0,3),I2+so(1,3),I3+so(2,3),c(3))
                       +coeff(4,I1,I2,I3)*u(I1+so(0,4),I2+so(1,4),I3+so(2,4),c(4))
                       +coeff(5,I1,I2,I3)*u(I1+so(0,5),I2+so(1,5),I3+so(2,5),c(5))
                           .... etc ....
    // 
    //  Here is another way which assumes it is a standard 3x3x3 stencil
    //
    intArray value(Range(-1,1),Range(-1,1),Range(-1,1),numberOfComponents);
    for( int n=0; n<numberOfComponents; n++)
    for( int m3=-1; m3<=1; m3++ )
    for( int m2=-1; m2<=1; m2++ )
    for( int m1=-1; m1<=1; m1++ )
    {
      value(m1,m2,m3,n)=coeff.coefficientPosition(m1,m2,m3,n);
      assert( value(m1,m2,m3,n)>=0 );   // check for any errors in computing the coefficient position
      
    }
    residual(I1,I2,I3) = coeff(value(-1,-1,-1,0),I1,I2,I3)*u(I1-1,I2-1,I3-1,0)  // coeff's for component 0
                        +coeff(value( 0,-1,-1,0),I1,I2,I3)*u(I1  ,I2-1,I3-1,0)
                        +coeff(value(+1,-1,-1,0),I1,I2,I3)*u(I1+1,I2-1,I3-1,0)
                             ... etc ...
                        +coeff(value(-1,-1,-1,1),I1,I2,I3)*u(I1-1,I2-1,I3-1,1)  // coeff's for component 1
                        +coeff(value( 0,-1,-1,1),I1,I2,I3)*u(I1  ,I2-1,I3-1,1)
                        +coeff(value(+1,-1,-1,1),I1,I2,I3)*u(I1+1,I2-1,I3-1,1)

\end{verbatim}
}

Other possible ways to store the coefficients
{\footnotesize
\begin{verbatim}


   ****** this does not work yet *****

   coeff.setIsACoefficientMatrix(fiveOrSevenPointStencil);   // 5 point star in 2d, 7 point star in 3D
   
  
   coeff.setIsACoefficientMatrix(generalStencil);   // user defined stencil

\end{verbatim}
}