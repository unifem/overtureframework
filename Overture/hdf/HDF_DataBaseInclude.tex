\subsection{Constructors}
 
\newlength{\HDFDataBaseIncludeArgIndent}
\begin{flushleft} \textbf{%
\settowidth{\HDFDataBaseIncludeArgIndent}{HDF\_DataBase(}% 
HDF\_DataBase()
}\end{flushleft}
\begin{description}
\item[{\bf Description:}] 
   Default constructor;
\item[{\bf Author:}]  WDH

\end{description}

 
\begin{flushleft} \textbf{%
\settowidth{\HDFDataBaseIncludeArgIndent}{HDF\_DataBase(}% 
HDF\_DataBase(const HDF\_DataBase \& db )
}\end{flushleft}
\begin{description}
\item[{\bf Description:}] 
   Copy constructor (shallow copy).
   Make a copy of the directory. This does not copy the data-base file.
\item[{\bf Author:}]  WDH

\end{description}

 
\begin{flushleft} \textbf{%
\settowidth{\HDFDataBaseIncludeArgIndent}{HDF\_DataBase(}% 
HDF\_DataBase(const GenericDataBase \& db )
}\end{flushleft}
\begin{description}
\item[{\bf Description:}] 
   Copy constructor, this works if db is really a member of this derived class.
\item[{\bf Author:}]  WDH

\end{description}
\subsection{mount}
 
\begin{flushleft} \textbf{%
int   \\ 
\settowidth{\HDFDataBaseIncludeArgIndent}{mount(}%
mount(const aString \& fileName, const aString \& flags )
}\end{flushleft}
\begin{description}
\item[{\bf Description:}] 
   Mount a data-base file.
\item[{\bf fileName (input):}]  the name of the file.
\item[{\bf flags (input):}]  flags to indicate how to access the file, "I" = initialize
   a new file, "W" = open an existing file for reading and writing,
   "R" = open an existing file read-only.
\item[{\bf Return values:}]  0=success, -1=unable to open a file that was supposed to exist,
     1=error in file format for an exitsing file, 2=unknown value for flags 
\end{description}
