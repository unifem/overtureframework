\subsubsection{constructors}
 
\newlength{\OgshowIncludeArgIndent}
\begin{flushleft} \textbf{%
\settowidth{\OgshowIncludeArgIndent}{Ogshow(}% 
Ogshow()
}\end{flushleft}
\begin{description}
\item[{\bf Description:}] 
  default constructor
\item[{\bf Author:}]  WDH
\end{description}

 
\begin{flushleft} \textbf{%
\settowidth{\OgshowIncludeArgIndent}{Ogshow(}% 
Ogshow(const aString \& nameOfShowFile, \\ 
\hspace{\OgshowIncludeArgIndent}const aString \& nameOfDirectory  = ".",\\ 
\hspace{\OgshowIncludeArgIndent}int useStreamMode  =false,\\ 
\hspace{\OgshowIncludeArgIndent}ShowFileOpenOption openOption = openNewFileForWriting )
}\end{flushleft}
\begin{description}
\item[{\bf Description:}] 
  construct a show file
\item[{\bf nameOfShowFile (input) :}]  name of the new show file to create
\item[{\bf nameOfDirectory (input) :}]  directory in the Overlapping grid data base file to use
\item[{\bf useStreamMode (input):}]  if true, save file in streaming mode (compressed)
\item[{\bf openOption (input) :}]  specifies whether to open a new file for writing (openNewFileForWriting) or
    (openOldFileForWriting) open an old file to append to. 
\item[{\bf Author:}]  WDH
\end{description}
\subsubsection{close}
 
\begin{flushleft} \textbf{%
int  \\ 
\settowidth{\OgshowIncludeArgIndent}{close(}%
close()
}\end{flushleft}
\begin{description}
\item[{\bf Description:}] 
     Close a show file.
\end{description}
\subsubsection{cleanup}
 
\begin{flushleft} \textbf{%
int  \\ 
\settowidth{\OgshowIncludeArgIndent}{cleanup(}%
cleanup()
}\end{flushleft}
\begin{description}
\item[{\bf Access:}]  protected
\item[{\bf Description:}] 
     Close and cleanup the show file.
\end{description}
\subsubsection{open}
 
\begin{flushleft} \textbf{%
int  \\ 
\settowidth{\OgshowIncludeArgIndent}{open(}%
open(const aString \& nameOfShowFile, \\ 
\hspace{\OgshowIncludeArgIndent}const aString \& nameOfDirectory  = ".",\\ 
\hspace{\OgshowIncludeArgIndent}int useStreamMode  =false,\\ 
\hspace{\OgshowIncludeArgIndent}ShowFileOpenOption openOption = openNewFileForWriting )
}\end{flushleft}
\begin{description}
\item[{\bf Description:}] 
     Open a show file (close any currently open file).
\item[{\bf nameOfShowFile (input) :}]  name of the new show file to create
\item[{\bf nameOfDirectory (input) :}]  directory in the Overlapping grid data base file to use
\item[{\bf useStreamMode (input):}]  if true, save file in streaming mode (compressed)
\item[{\bf openOption (input) :}]  specifies whether to open a new file for writing (openNewFileForWriting) or
    (openOldFileForWriting) open an old file to append to. 
\end{description}
\subsubsection{getFrame}
 
\begin{flushleft} \textbf{%
HDF\_DataBase*  \\ 
\settowidth{\OgshowIncludeArgIndent}{getFrame(}%
getFrame(const  FrameSeriesID /*= 0*/)
}\end{flushleft}
\begin{description}
\item[{\bf Description:}] 
     Return a pointer to the data base directory holding the current frame.
     You could use this pointer to save additional data in the frame. In the following example
     some extra data in the form of a realArray is saved in the frame. 
    \begin{verbatim}
       Ogshow show(...);
       ...
       show.startFrame();
       realArray myData(10); 
       myData(0)=1.; myData(1)=2.; ...
       show.getFrame()->put(myData,"my data");
       ...
    \end{verbatim}
    This data can be retrieved using the ShowFileReader.
\item[{\bf Return value:}]  Return a pointer to the data base directory holding the current frame, possibly NULL.
\item[{\bf Author:}]  WDH
\end{description}
\subsubsection{setFlushFrequency}
 
\begin{flushleft} \textbf{%
void   \\ 
\settowidth{\OgshowIncludeArgIndent}{setFlushFrequency(}%
setFlushFrequency( const int flushFrequency    = 5)
}\end{flushleft}
\begin{description}
\item[{\bf Description:}] 
     Flush the file every time "flushFrequency" frames have been added.
   In the current implementation "flushing the file" consists of closing the file
   and opening a new file to save new frames in. 

\item[{\bf flushFrequency (input):}]  If positive then the file is "flushed" when every time
    this many new frames have been added.
\item[{\bf Author:}]  WDH
\end{description}
\subsubsection{getFlushFrequency}
 
\begin{flushleft} \textbf{%
int   \\ 
\settowidth{\OgshowIncludeArgIndent}{getFlushFrequency(}%
getFlushFrequency() const
}\end{flushleft}
\begin{description}
\item[{\bf Description:}] 
     Return the flush frequency.
\end{description}
\subsubsection{isFirstFrameInSubFile}
 
\begin{flushleft} \textbf{%
bool  \\ 
\settowidth{\OgshowIncludeArgIndent}{isFirstFrameInSubFile(}%
isFirstFrameInSubFile() const
}\end{flushleft}
\begin{description}
\item[{\bf Description:}] 
     Return true if the current frame is the first frame in the current subFile ( subfile's are
 named fileName.show, fileName.show1, fileName.show2, ...)
   frame will go into a new sub-file. 
\end{description}
\subsubsection{isLastFrameInSubFile}
 
\begin{flushleft} \textbf{%
bool  \\ 
\settowidth{\OgshowIncludeArgIndent}{isLastFrameInSubFile(}%
isLastFrameInSubFile() const
}\end{flushleft}
\begin{description}
\item[{\bf Description:}] 
     Return true if the current frame is the last frame in the current subFile ( subfile's are
 named fileName.show, fileName.show1, fileName.show2, ...)
   frame will go into a new sub-file. 
\end{description}
\subsubsection{setMovingGridProblem}
 
\begin{flushleft} \textbf{%
bool  \\ 
\settowidth{\OgshowIncludeArgIndent}{getIsMovingGridProblem(}%
getIsMovingGridProblem() const
}\end{flushleft}
\begin{description}
\item[{\bf Description:}] 
     Return true if this is a moving grid problem.
\item[{\bf Author:}]  WDH
\end{description}
\subsubsection{setMovingGridProblem}
 
\begin{flushleft} \textbf{%
void  \\ 
\settowidth{\OgshowIncludeArgIndent}{setIsMovingGridProblem(}%
setIsMovingGridProblem( const bool trueOrFalse )
}\end{flushleft}
\begin{description}
\item[{\bf Description:}] 
     Indicate if this is a moving grid problem so that the grid is saved in every frame
\item[{\bf trueOrFalse (input):}]  TRUE is this is a moving grid problem
\item[{\bf Author:}]  WDH
\end{description}
\subsubsection{getNumberOfFrames}
 
\begin{flushleft} \textbf{%
int  \\ 
\settowidth{\OgshowIncludeArgIndent}{getTotalNumberOfFrames(}%
getTotalNumberOfFrames() const
}\end{flushleft}
\begin{description}
\item[{\bf Description:}] 
   return the number of frames that exist in the show file.
\item[{\bf Author:}]  KKC
\end{description}
\subsubsection{getNumberOfFrames}
 
\begin{flushleft} \textbf{%
int  \\ 
\settowidth{\OgshowIncludeArgIndent}{getNumberOfFrames(}%
getNumberOfFrames() const
}\end{flushleft}
\begin{description}
\item[{\bf Description:}] 
   return the number of frames that exist in a particular frame series
\item[{\bf Author:}]  WDH
\end{description}
\subsubsection{getShowFileName}
 
\begin{flushleft} \textbf{%
const aString \&  \\ 
\settowidth{\OgshowIncludeArgIndent}{getShowFileName(}%
getShowFileName() const
}\end{flushleft}
\begin{description}
\item[{\bf Description:}] 
   return the name of the show file.
\item[{\bf Author:}]  WDH
\end{description}
\subsubsection{startFrame}
 
\begin{flushleft} \textbf{%
int  \\ 
\settowidth{\OgshowIncludeArgIndent}{startFrame(}%
startFrame( const int frameNo  = newFrame)
}\end{flushleft}
\begin{description}
\item[{\bf Description:}] 
   start a new frame or write to an existing one
\item[{\bf frameNo (input):}]  by default start a new frame, otherwise open a frame with
   the given value.
\item[{\bf Author:}]  WDH
\end{description}
\subsubsection{endFrame}
 
\begin{flushleft} \textbf{%
int  \\ 
\settowidth{\OgshowIncludeArgIndent}{endFrame(}%
endFrame()
}\end{flushleft}
\begin{description}
\item[{\bf Description:}] 
   End the currently open frame (if any). The main purpose of calling
 this routine is to close a sub-file if this was the last frame in the sub-file.
 This will allow the sub-file to be read programs such as plotStuff.
 WARNING: once a sub-file is closed you can no longer write to a frame in that sub-file.
  This needs to be fixed.
 
\item[{\bf Author:}]  WDH
\end{description}
\subsubsection{saveGeneralComment}
 
\begin{flushleft} \textbf{%
int  \\ 
\settowidth{\OgshowIncludeArgIndent}{saveGeneralComment(}%
saveGeneralComment( const aString \& comment0 )
}\end{flushleft}
\begin{description}
\item[{\bf Description:}] 
   Save a general comment (this comment is associated with the entire show file).
   Multiple comments can be saved by repeatedly calling this function.
\item[{\bf comment0 (input):}]  comment to save.
\item[{\bf Author:}]  WDH
\end{description}
\subsubsection{saveComment}
 
\begin{flushleft} \textbf{%
int  \\ 
\settowidth{\OgshowIncludeArgIndent}{saveComment(}%
saveComment( const int commentNumber, const aString \& comment0 )
}\end{flushleft}
\begin{description}
\item[{\bf Description:}] 
   Save a comment to go in the current frame.
\item[{\bf commentNumber (input):}]  An integer, 0,1,2,.. that numbers the comment
\item[{\bf comment0 (input):}]  comment to save.
\item[{\bf Author:}]  WDH
\end{description}
\subsubsection{saveGeneralParameters}
 
\begin{flushleft} \textbf{%
int  \\ 
\settowidth{\OgshowIncludeArgIndent}{saveGeneralParameters(}%
saveGeneralParameters( ListOfShowFileParameters \& params, const PlaceToSaveGeneralParameters placeToSave )
}\end{flushleft}
\begin{description}
\item[{\bf Description:}] 
   Save parameters that apply to the whole file.
\item[{\bf params (input):}]  A list of parameters to save
\item[{\bf placeToSave (input) :}]  save in the root directory (THEShowFileRoot) or in the current frame (THECurrentFrameSeries).
\item[{\bf Author:}]  WDH
\end{description}
\subsubsection{saveGeneralParameters}
 
\begin{flushleft} \textbf{%
  // Save a named set of parameters\\int  \\ 
\settowidth{\OgshowIncludeArgIndent}{saveParameters(}%
saveParameters(const aString \& nameOfDirectory, ListOfShowFileParameters \& params )
}\end{flushleft}
\begin{description}
\item[{\bf Description:}] 
   Save a named set of parameters in the current frame.
\item[{\bf nameOfDirectory (input) :}]  the name of the (new) directory where to save the parameters
\item[{\bf params (input):}]  A list of parameters to save
 
\item[{\bf Author:}]  WDH
\end{description}
\subsubsection{saveSolution}
 
\begin{flushleft} \textbf{%
int  \\ 
\settowidth{\OgshowIncludeArgIndent}{saveSolution(}%
saveSolution(realMappedGridFunction \& u, \\ 
\hspace{\OgshowIncludeArgIndent}const aString \& name  = "u",\\ 
\hspace{\OgshowIncludeArgIndent}int frameForGrid  = useDefaultLocation)
}\end{flushleft}
\begin{description}
\item[{\bf Description:}] 
   Save a mappedGridFunction in the current frame.
   (for now save a CompositeGridFunction)
\item[{\bf u (input) :}]  grid function to save
\item[{\bf name (input):}]  save in the frame under this name. (Currently if you change this name
   from the default then plotStuff will not find the solution).
\item[{\bf frameForGrid :}]  indicates where in the show file the grid for this solution can be found.
    This grid will saved in this frame if it does not already exist.
     useDefaultLocation : use default location (frame 1), useCurrentFrame : current frame, 
      $>0$ : specify a frame number.
\item[{\bf Author:}]  WDH
\end{description}
\subsubsection{saveSolution}
 
\begin{flushleft} \textbf{%
int  \\ 
\settowidth{\OgshowIncludeArgIndent}{saveSolution(}%
saveSolution(realGridCollectionFunction \& u\_, \\ 
\hspace{\OgshowIncludeArgIndent}const aString \& name  = "u",\\ 
\hspace{\OgshowIncludeArgIndent}int frameForGrid  = useDefaultLocation)
}\end{flushleft}
\begin{description}
\item[{\bf Description:}] 
   Save a realGridCollectionFunction or realCompositeGridFunction in the current frame.
\item[{\bf u (input) :}]  grid function to save
\item[{\bf name (input):}]  save in the frame under this name. (Currently if you change this name
   from the default then plotStuff will not find the solution).
\item[{\bf frameForGrid :}]  indicates where in the show file the grid for this solution can be found.
    This grid will saved in this frame if it does not already exist.
     useDefaultLocation : use default location (frame 1), useCurrentFrame : current frame, 
      $>0$ : specify a frame number. 
\item[{\bf Author:}]  WDH
\end{description}
\subsubsection{newFrameSeries}
 
\begin{flushleft} \textbf{%
FrameSeriesID  \\ 
\settowidth{\OgshowIncludeArgIndent}{newFrameSeries(}%
newFrameSeries(const aString \& name)
}\end{flushleft}
\begin{description}
\item[{\bf Description:}] 
     Create a new frame series with the given name.
 
\item[{\bf name (input) :}]  name of a new frame series
 
\item[{\bf Return value:}]  the frameSeriesID
\end{description}
\subsubsection{setCurrentFrameSeries}
 
\begin{flushleft} \textbf{%
int  \\ 
\settowidth{\OgshowIncludeArgIndent}{getNumberOfFrameSeries(}%
getNumberOfFrameSeries() const
}\end{flushleft}
\begin{description}
\item[{\bf Description:}] 
   Return the number of frame series in this show file.
\end{description}
\subsubsection{getFrameSeriesID}
 
\begin{flushleft} \textbf{%
FrameSeriesID  \\ 
\settowidth{\OgshowIncludeArgIndent}{getFrameSeriesID(}%
getFrameSeriesID(const aString \& name)
}\end{flushleft}
\begin{description}
\item[{\bf Description:}] 
   Return the FrameSeriesID corresponding to a given name.
\item[{\bf name (input) :}]  name of an existing frame series
\item[{\bf Return value:}]  -1 if the name was not found.
\end{description}
\subsubsection{getFrameSeriesName}
 
\begin{flushleft} \textbf{%
const aString\&  \\ 
\settowidth{\OgshowIncludeArgIndent}{getFrameSeriesName(}%
getFrameSeriesName(const FrameSeriesID frameSeries)
}\end{flushleft}
\begin{description}
\item[{\bf Description:}] 
   Return the name of a frame series in this show file.
 
\item[{\bf frameSeries (input) :}]  a frame series ID.
\item[{\bf Return value:}]  A nullString if the frameSeries was not found.
 
\end{description}
\subsubsection{setFrameSeriesName}
 
\begin{flushleft} \textbf{%
int  \\ 
\settowidth{\OgshowIncludeArgIndent}{setFrameSeriesName(}%
setFrameSeriesName(const FrameSeriesID frameSeries, const aString \& name)
}\end{flushleft}
\begin{description}
\item[{\bf Description:}] 
     Assign a new name to an existing frame series.
 
\item[{\bf frameSeries (input) :}]  a frame series ID.
\item[{\bf name (input) :}]  name for the frame series
\item[{\bf Return value:}]  0=success, 1=failure.
 
\end{description}
\subsubsection{getCurrentFrameSeries}
 
\begin{flushleft} \textbf{%
FrameSeriesID  \\ 
\settowidth{\OgshowIncludeArgIndent}{getCurrentFrameSeries(}%
getCurrentFrameSeries() const
}\end{flushleft}
\begin{description}
\item[{\bf Description:}] 
   Return the number of current frame series in this show file.
\end{description}
\subsubsection{setCurrentFrameSeries}
 
\begin{flushleft} \textbf{%
int  \\ 
\settowidth{\OgshowIncludeArgIndent}{setCurrentFrameSeries(}%
setCurrentFrameSeries(const FrameSeriesID frameSeries)
}\end{flushleft}
\begin{description}
\item[{\bf Description:}] 
   Set the current frame series to the given ID.
\item[{\bf frameSeries (input) :}]  a frame series ID.
\item[{\bf Return value:}]  1 if the frameSeries is not valid.
\end{description}
\subsubsection{setCurrentFrameSeries}
 
\begin{flushleft} \textbf{%
FrameSeriesID  \\ 
\settowidth{\OgshowIncludeArgIndent}{setCurrentFrameSeries(}%
setCurrentFrameSeries(const aString \& name)
}\end{flushleft}
\begin{description}
\item[{\bf Description:}] 
    Set the current frame series to be "name". Create a new frame series with this
  name if it does not already exist.
 
\end{description}
