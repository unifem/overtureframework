%-----------------------------------------------------------------------
% Ogshow: Overlapping Grid Show file class
%-----------------------------------------------------------------------
%123456789 123456789 123456789 123456789 123456789 123456789 123456789 12

\documentclass{article}
\usepackage[bookmarks=true]{hyperref}

% \input documentationPageSize.tex
\hbadness=10000 
\sloppy \hfuzz=30pt

\usepackage{calc}
% set the page width and height for the paper (The covers will have their own size)
\setlength{\textwidth}{7in}  
\setlength{\textheight}{9.5in} 
% here we automatically compute the offsets in order to centre the page
\setlength{\oddsidemargin}{(\paperwidth-\textwidth)/2 - 1in}
% \setlength{\topmargin}{(\paperheight-\textheight -\headheight-\headsep-\footskip)/2 - 1in + .8in }
\setlength{\topmargin}{(\paperheight-\textheight -\headheight-\headsep-\footskip)/2 - 1in -.2in }

\input homeHenshaw


\usepackage{verbatim}
\usepackage{moreverb}
\usepackage{graphics}    

\usepackage{makeidx} % index
\makeindex
\newcommand{\Index}[1]{#1\index{#1}}

\newcommand{\ogshow}{\homeHenshaw/Overture/ogshow}

% ---- we have lemmas and theorems in this paper ----
\newtheorem{assumption}{Assumption}
\newtheorem{definition}{Definition}


\begin{document}

\input wdhDefinitions.tex
%

\vspace{5\baselineskip}
\begin{flushleft}
{\Large
Ogshow: Overlapping Grid Show File Class \\
Saving Solutions to be Displayed with plotStuff \\
ShowFileReader: A Class for Reading Solutions from a Show File \\
\vspace{\baselineskip}
User Guide, Version 1.00  \\
}
\vskip2\baselineskip
Bill Henshaw \\
\vspace{\baselineskip}
Centre for Applied Scientific Computing \\
Lawrence Livermore National Laboratory    \\
Livermore, CA, 94551   \\
henshaw@llnl.gov \\
http://www.llnl.gov/casc/people/henshaw \\
http://www.llnl.gov/casc/Overture
%\footnote{
%        This work was partially
%        supported by grant N00014-93-C-0200 from the Office of Naval
%        Research}

\vspace{\baselineskip}
\today
\vspace{\baselineskip}
UCRL-MA-132235
% LA-UR-96-3465

\vspace{4\baselineskip}

\noindent{\bf Abstract:}
We describe the class Ogshow for creating show files with Overture.
This class can be used in solvers in order to save solution and comments
into a data base file (``show file'') that can be later read by plotStuff.
``plotStuff'' can be used graphically display the results found in the
show file.

We also describe the class \Index{ShowFileReader} which can be used to read grids
and solutions from a show file. The ShowFileReader can be used by PDE solvers
to read in initial conditions from soluitions that have previously been
saved in a show file.
\end{flushleft}

\tableofcontents

\vfill\eject
%---------- End of title Page for a Research Report

\section{Introduction}\index{show file}

This class can be used in solvers in order to save solution and comments
into a data base file (``show file'') that can be later read by plotStuff.
``plotStuff'' can be used graphically display the results found in the
show file. The executable plotStuff is found in Overture/bin/plotStuff.

A showfile consists of a sequence of ``frames''. A frame will hold
``items'' that are related.  In a typical case, there will be one
frame for each time when the solution is saved. Each frame will hold
one or more grid functions. Grid functions appearing in different
frames but having the same name will be recognized by plotStuff. This
sequence of grid functions can be used to make a ``movie''.

Comments can also be saved in the showfile.  There are two types of
comments. General comments are associated with the whole showfile and
are printed out when plotStuff first reads the showfile.  There are
also comments that are associated with each solution in the
showfile. These comments are displayed when plotStuff plots the
solution.

Other things can be saved in the showfile, such as a time sequence
of some values. Currently plotStuff only knows how to plot
realCompositeGridFunction's although one can write a subroutine
to tell plotStuff how to plot other things.


% define a boxed item
\newcommand{\boxitem}[1]{\begin{picture}(50,25)     % root in a box
  \put(-25,5){\framebox(50,20){#1}}
 \end{picture}}

%---------------------------------------------------------------
\begin{figure}
\begin{center}
 \begin{picture}(180,220)
 \put(10,+80){
  \put( 80,110){\boxitem{show file}}
  \put(-40, 45){\boxitem{comments}}
  \put( 20, 45){\boxitem{frame1}}
  \put( 80, 45){\boxitem{frame2}}
  \put(140, 45){\boxitem{frame3}}

  \put( 20,  5){\boxitem{comments}}
  \put( 80,  5){\boxitem{comments}}
  \put(140,  5){\boxitem{comments}}

  \put(-40, 70){\line(0,1){20}}
  \put( 20, 70){\line(0,1){20}}
  \put( 80, 70){\line(0,1){45}}
  \put(140, 70){\line(0,1){20}}

  \put(-40,90){\line(1,0){260}}    % long horizonal line

  \put( 20, 30){\line(0,1){20}}
  \put( 80, 30){\line(0,1){20}}
  \put(140, 30){\line(0,1){20}}

  \put( 20,-35){\boxitem{u}}
  \put( 80,-35){\boxitem{u}}
  \put(140,-35){\boxitem{u}}

  \put( 20,-10){\line(0,1){20}}
  \put( 80,-10){\line(0,1){20}}
  \put(140,-10){\line(0,1){20}}

  \put( 20,-75){\boxitem{Grid}}
  \put( 80,-75){\boxitem{(Grid)}}
  \put(140,-75){\boxitem{(Grid)}}

  \put( 20,-50){\line(0,1){20}}
  \put( 80,-50){\line(0,1){20}}
  \put(140,-50){\line(0,1){20}}

  } % end put
 \end{picture}

 \caption{A show file with multiple frames, frames hold solutions at different times. For moving grids,
          a grid is found in each frame, otherwise the grid is only in frame 1}
 \label{figa}
\end{center} 
\end{figure}
%---------------------------------------------------------------

\section{Class Ogshow}


\subsection{Interface}

\input OgshowInclude.tex 

% \subsection{Constructors}
% 
% \begin{tabbing}
% {\ff Ogshow( String \& nameOfOGFile, String \& nameOfDirectory, }\={\ff xxxxxxxxxx}\= \kill
% {\ff Ogshow()} \> Default constructor \\
% {\ff Ogshow( String \& nameOfOGFile, String \& nameOfDirectory,  }\>  create a show file from \\
% {\ff \phantom{Ogshow(} String \& nameOfShowFile )  }\> a composite grid data file  \\
% \end{tabbing}
% 
% \subsection{Member Functions and Member Data}
% 
% \noindent
% \begin{tabbing}
% {\ff xxxxxxxxxxxxxxxxxxxxxxxxxxxxxxxxxxxxxxxxxxxxxxxxxxxxxxxxxx 0123456789} \= 
%     xxxxxxxxxxxxx \= \kill
% {\ff int createShowFile( String \& nameOfOGFile, String \& nameOfDirectory,} \> 
%                  create a show file \\
% {\ff \phantom{ int createShowFile(} String \& nameOfShowFile )} \> \\
% {\ff MultigridCompositeGrid multigridCompositeGrid} \> Here is the grid from the data base file\\
% {\ff CompositeGrid compositeGrid} \> Here is the grid from the data base file (level 0) \\
% {\ff int saveGeneralComment( String \& comment )} \> Save a general comment \\
% {\ff int saveComment( int commentNumber, String \& comment )} \> Save a comment \\
% {\ff int startFrame() } \>  Start a new frame \\
% {\ff int saveSolution( realCompositeGridFunction \& u )} \> save a solution \\
% \end{tabbing}
% 
% The {\ff createShowFile} function would be used if the Ogshow object
% were created with the default constructor.


\subsection{Typical Usage}

Consider an example where a solver is computing some velocity 
field $(u,v)$. The user would like to save $u$, $v$ and the
``Mach number'' $u^2+v^2$ in the show file. Here is how this
might be done: (file ``togshow.C'')

{\footnotesize
\listinginput[1]{1}{\ogshow/togshow.C}
}

In this example we also save some additional information into the data base (the time ``t''). 
The function {\tt getFrame} returns
a pointer to the {\tt HDF\_DataBase} directory in which the current frame is being saved.
The data base function {\tt put} is then used to save a real number. It is also possible
to save arrays and Strings in this way. See the data base documentation for further details.

\subsection{Moving grids}

The only difference with moving grids is that one should say {\tt show.setMovingGridProblem(TRUE)}.
In this case each frame will hold a new grid as well as the solution.

See the moving grid example in the primer documentation for an example of saving a moving grid
in a show file.
% For reasons of efficiency it is important
% that the CompositeGrid associated with the 
% grid function $q$ is initially referenced to
% the CompositeGrid that appears in the show file.
% When the CompositeGridGenerator makes a new
% CompositeGrid it will only break references on
% that arrays that have changed.
% By doing this, arrays in the CompositeGrid that
% have not changed will only be saved once in
% the showfile. 
% {\footnotesize
% \listinginput[1]{1}{/users/henshaw/res/ogshow/tmg.C}
% }

\section{ShowFileReader}

The ShowFileReader class can be used to read in grids and solutions
from a show file. This class can be used, for example,
to read in initial conditions for a solver.

\subsection{Interface}
\input ShowFileReaderInclude.tex

\vfill\eject
\subsection{Example: Using ShowFileReader to Read in Initial Conditions to a PDE Solver}
\index{initial conditions!from a show file}
\index{restart!from a show file}

In this example we show how to mount a show file and read a grid and solution from
the show file. We also show how to interpolate a solution on one CompositeGrid
to a solution on another CompositeGrid using the {\tt interpolateAllPoints} function.
The {\tt interpolateAllPoints} function is described in more detail in the GridFunction
documentation.

This example shows how one could read initial conditions for a PDE solver from a show file.
The CompositeGrid used by the PDE solver does not have to be the same as the CompositeGrid
in the showFile. For example, the grid may be refined or a new component grid added
(file {\tt Overture/examples/readShowFile.C}).
{\footnotesize
\listinginput[1]{1}{\ogshow/readShowFile.C}
}

\printindex

\end{document}
