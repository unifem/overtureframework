\subsection{Constructor}
 
\newlength{\GraphicsParametersIncludeArgIndent}
\begin{flushleft} \textbf{%
  \\ 
\settowidth{\GraphicsParametersIncludeArgIndent}{GraphicsParameters(}%
GraphicsParameters(bool default0)
}\end{flushleft}
\begin{description}
\item[{\bf Description:}] 
   Constructor 

\end{description}
\subsection{isDefault}
 
\begin{flushleft} \textbf{%
bool   \\ 
\settowidth{\GraphicsParametersIncludeArgIndent}{isDefault(}%
isDefault()
}\end{flushleft}
\begin{description}
\item[{\bf Description:}] 
   Return true if this object is a default object. 
 This routine can be used to tell whether a GraphicsParameter object is
 equal to the static object {\tt Overture::defaultGraphicsParameters()} which can
 be used as a default argument in a function call.
\end{description}
\subsection{getObjectWasPlotted}
 
\begin{flushleft} \textbf{%
int  \\ 
\settowidth{\GraphicsParametersIncludeArgIndent}{getObjectWasPlotted(}%
getObjectWasPlotted() const
}\end{flushleft}
\begin{description}
\item[{\bf Description:}] 
    Determine if the object was plotted in the last plotting routine
    that was called.
\item[{\bf Return value:}]  true if an object was plotted, false otherwise.

\end{description}
\subsection{get(aString)}
 
\begin{flushleft} \textbf{%
aString \&  \\ 
\settowidth{\GraphicsParametersIncludeArgIndent}{get(}%
get(const GraphicsOptions \& option, aString \& label) const
}\end{flushleft}
\begin{description}
\item[{\bf Description:}] 
   Return the aString associated with a GraphicsParameter option.
\item[{\bf option (input) :}]  Return the aString associated with this option (if any).
\item[{\bf label (output) :}]  Return the string in this variable
\item[{\bf Return value:}]  the return value is also equal to label.
\end{description}
\subsection{get(int)}
 
\begin{flushleft} \textbf{%
int \&  \\ 
\settowidth{\GraphicsParametersIncludeArgIndent}{get(}%
get(const GraphicsOptions \& option, int \& value) const
}\end{flushleft}
\begin{description}
\item[{\bf Description:}] 
   Return the int associated with a GraphicsParameter option.
\item[{\bf option (input) :}]  Return the int value associated with this option (if any).
\item[{\bf value (output) :}]  Return the value in this variable
\item[{\bf Return value:}]  the return value is also equal to value.
\end{description}
\subsection{get(real)}
 
\begin{flushleft} \textbf{%
real \&   \\ 
\settowidth{\GraphicsParametersIncludeArgIndent}{get(}%
get(const GraphicsOptions \& option, real \& value) const
}\end{flushleft}
\begin{description}
\item[{\bf Description:}] 
   Return the real associated with a GraphicsParameter option.
\item[{\bf option (input) :}]  Return the real value associated with this option (if any).
\item[{\bf value (output) :}]  Return the value in this variable
\item[{\bf Return value:}]  the return value is also equal to value.
\end{description}
\subsection{get(IntegerArray)}
 
\begin{flushleft} \textbf{%
IntegerArray \&  \\ 
\settowidth{\GraphicsParametersIncludeArgIndent}{get(}%
get(const GraphicsOptions \& option, IntegerArray \& values) const
}\end{flushleft}
\begin{description}
\item[{\bf Description:}] 
   Return the IntegerArray associated with a GraphicsParameter option.
\item[{\bf option (input) :}]  Return the IntegerArray value associated with this option (if any).
\item[{\bf value (output) :}]  Return the value in this variable
\item[{\bf Return value:}]  the return value is also equal to value.
\end{description}
\subsection{get(RealArray)}
 
\begin{flushleft} \textbf{%
RealArray \&  \\ 
\settowidth{\GraphicsParametersIncludeArgIndent}{get(}%
get(const GraphicsOptions \& option, RealArray \& values) const
}\end{flushleft}
\begin{description}
\item[{\bf Description:}] 
   Return the RealArray associated with a GraphicsParameter option.
\item[{\bf option (input) :}]  Return the RealArray value associated with this option (if any).
\item[{\bf value (output) :}]  Return the value in this variable
\item[{\bf Return value:}]  the return value is also equal to value.
\end{description}
\subsection{get(Sizes)}
 
\begin{flushleft} \textbf{%
real \&  \\ 
\settowidth{\GraphicsParametersIncludeArgIndent}{get(}%
get(const Sizes \& option, real \& value) const
}\end{flushleft}
\begin{description}
\item[{\bf Description:}] 
   Deterimine the value of a {\it size} parameter
\item[{\bf option (input) :}]  determine the value for this size.
\item[{\bf value (output) :}]  the value.
\item[{\bf Return value:}]  the return value is also equal to value.

\end{description}
\subsection{set(GraphicsOptions, int/real)}
 
\begin{flushleft} \textbf{%
int   \\ 
\settowidth{\GraphicsParametersIncludeArgIndent}{set(}%
set(const GraphicsOptions \& option, real value)
}\end{flushleft}
\begin{description}
\item[{\bf Description:}] 
   Assign a parameter with an int or real value

\end{description}
\subsection{set(GraphicsOptions, IntegerArray)}
 
\begin{flushleft} \textbf{%
int  \\ 
\settowidth{\GraphicsParametersIncludeArgIndent}{set(}%
set(const GraphicsOptions \& option, const IntegerArray \& values)
}\end{flushleft}
\begin{description}
\item[{\bf Description:}] 
   Assign a parameter with that requires an array of int's

\end{description}
\subsection{set(GraphicsOptions, RealArray)}
 
\begin{flushleft} \textbf{%
int  \\ 
\settowidth{\GraphicsParametersIncludeArgIndent}{set(}%
set(const GraphicsOptions \& option, const RealArray \& values)
}\end{flushleft}
\begin{description}
\item[{\bf Description:}] 
   Assign a parameter with that requires an array of real's

\end{description}
\subsection{set(GraphicsOptions, aString)}
 
\begin{flushleft} \textbf{%
int  \\ 
\settowidth{\GraphicsParametersIncludeArgIndent}{set(}%
set(const GraphicsOptions \& option, const aString \& label)
}\end{flushleft}
\begin{description}
\item[{\bf Description:}] 
   Assign a parameter with a aString

\end{description}
\subsection{set(Sizes)}
 
\begin{flushleft} \textbf{%
int  \\ 
\settowidth{\GraphicsParametersIncludeArgIndent}{set(}%
set(const Sizes \& option, real value) set a size
}\end{flushleft}
\begin{description}
\item[{\bf Description:}] 
   Assign a {\it size} parameter

\end{description}
\subsection{setColourTable}
 
\begin{flushleft} \textbf{%
int  \\ 
\settowidth{\GraphicsParametersIncludeArgIndent}{setColourTable(}%
setColourTable(ColourTableFunctionPointer ctf)
}\end{flushleft}
\begin{description}
\item[{\bf Description:}] 
   Provide a function to use for a colour table. This function will then be subsequently used for
   the colour table. The colour table can be reset to one of the provided colour tables
   using the  {\tt GI\_SET\_COLOUR\_TABLE} option. (The function provided here corresponds
   to the {\tt userDefined} colour table).
\item[{\bf ctf (input) :}]  a pointer to a function of the form shown below. 

  Here is an example of a function that defines a colour table
 
 {\footnotesize
 \begin{verbatim}
 void 
 defaultColourTableFunction(const real & value, real & red, real & green, real & blue)
 // =============================================================================================
 // Description: Convert a value from [0,1] into (red,green,blue) values, each in the range [0,1]
 // value (input) :  0 <= value <= 1 
 // red, green, blue (output) : values in the range [0,1]
 // =============================================================================================
 { // a sample user defined colour table function: 
   red=0.;
   green=value;
   blue=(1.-value);
 }
 \end{verbatim}
 }

\end{description}
