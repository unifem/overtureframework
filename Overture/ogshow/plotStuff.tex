%-----------------------------------------------------------------------
% Documentation for the plotStuff post-processor
%
%-----------------------------------------------------------------------
\documentclass{article}

\usepackage[bookmarks=true]{hyperref}

% \input documentationPageSize.tex
\hbadness=10000 
\sloppy \hfuzz=30pt

\usepackage{calc}
% set the page width and height for the paper (The covers will have their own size)
\setlength{\textwidth}{7in}  
\setlength{\textheight}{9.5in} 
% here we automatically compute the offsets in order to centre the page
\setlength{\oddsidemargin}{(\paperwidth-\textwidth)/2 - 1in}
% \setlength{\topmargin}{(\paperheight-\textheight -\headheight-\headsep-\footskip)/2 - 1in + .8in }
\setlength{\topmargin}{(\paperheight-\textheight -\headheight-\headsep-\footskip)/2 - 1in -.2in }

\input homeHenshaw

\usepackage{html} %to have conditional code for latex2html

\usepackage{amsmath}
\usepackage{verbatim}
% \usepackage{moreverb}
\usepackage{graphics}    
\usepackage{epsfig}    
\usepackage{calc}
\usepackage{ifthen}
% \usepackage{fancybox}
\usepackage{supertabular}


\usepackage{makeidx} % index
\makeindex
\newcommand{\Index}[1]{#1\index{#1}}


% ---- we have lemmas and theorems in this paper ----
\newtheorem{assumption}{Assumption}
\newtheorem{definition}{Definition}

\newcommand{\ogshow}{\homeHenshaw/Overture/ogshow}
\newcommand{\figures}{\homeHenshaw/OvertureFigures}


\begin{document}


% -----definitions-----
\input wdhDefinitions

\vspace{\baselineskip}
\begin{flushleft}
{\Large
The plotStuff Graphics Post Processor for Overture\\
User Guide, Version 1.00 \\
}
\vspace{2\baselineskip}
Bill Henshaw \\
\vspace{\baselineskip}
Centre for Applied Scientific Computing \\
Lawrence Livermore National Laboratory    \\
Livermore, CA, 94551   \\
henshaw@llnl.gov \\
http://www.llnl.gov/casc/people/henshaw \\
http://www.llnl.gov/casc/Overture \\
\vspace{\baselineskip}
\today \\
\vspace{\baselineskip}
UCRL-MA-138730 \\

\vspace{4\baselineskip}

\noindent{\bf Abstract:}
This document explains how to use {\tt plotStuff}, a program for plotting
results from Overture programs. {\tt plotStuff} reads data-base files created
by Overture that contain information about the grid and solutions, a ``show file''.
{\tt plotStuff} has features for plotting contours, cutting planes, streamlines,
iso-surfaces and grids. It can also be used to plot various derived quantities
such as derivatives of the solution components. This document describes the class
{\tt DerivedFunctions} which is used to compute these derived quanties.
\end{flushleft}

\tableofcontents
% \listoffigures

\vfill\eject
%---------- End of title Page for a Research Report

\section{Introduction}\index{plotStuff}\index{graphics post processing}\index{show files}

\begin{figure}[htb]
  \begin{center}
   \epsfig{file=\figures /plotStuffScreen.ps,width=.6\linewidth}
  \caption{plotStuff is used to display results from an Overture data-base file (`show file') such as those
     created by the {\bf OverBlown} flow solver}  \label{fig:plotStuff}
  \end{center}
\end{figure}

The {\tt plotStuff} program, usually found in {\tt Overture/bin/plotStuff} can be used to
plot results found in ``show files'' or ``grid files'' created by Overture programs.

A ``show file'' contains a sequence of {\bf frames}, each frame will normally contain a
``solution'' (i.e. a grid function defined on a grid). In the moving grid case
each frame may also contain a new grid. Each solution will consist of a number
of {\tt components}, such as the velocity and pressure. A show file may also
contain one or more {\bf sequence}, which is an array of values such as the maximum residual
over time.

Some of the features of {\tt plotStuff} include
\begin{itemize}
  \item plot contours of various flavours, plot streamlines, plot grids.
  \item show a movie of a time sequence of frames; save hard-copies for making mpeg movies.
  \item plot ``derived types'' from variables in the show file, 
      such as plotting the vorticity given the velocity or plotting derivatives of
      the variables.
\end{itemize}

The {\tt plotStuff} program is basically a front end to the features available in other classes
such as {\tt GenericGraphicsInterface}\cite{PLOTSTUFF}, Ogshow\cite{OGSHOW}, {\tt ShowFileReader}\cite{OGSHOW}
and the {\tt DerivedFunctions} class.  

\section{Typical usage}

The typical usage of {\tt plotStuff} is to start the program up by typing
``{\tt plotStuff fileName.show}'' or just ``{\tt plotStuff fileName}'' since the ``.show'' is optional.
Here {\tt fileName.show} is the name of a show file such as one created by {\bf OverBlown}\cite{OverBlownUserGuide}.
Now one can choose {\tt contour} to plot contours. At this point a new menu appears, this is
the contour plotter menu. By choosing {\tt exit} one returns to the main {\tt plotStuff}
menu. Choosing a new solution, or typing {\tt next} or {\tt previous} will plot contours
of a different solution. One can also plot a different component. At this point one could
choose {\tt grid} or `{\tt stream lines}' to plot the grid or plot streamlines. Note that
the contour plot will not be erased by default. This allows multiple things to be plotted
at once. One can choose {\tt erase} to remove the contour plot before plotting other things.
By choosing the `{\tt derived types}' menu item\index{derived types} on can build new components
from existing ones. For example the vorticity or divergence can be created from a velocity vector.
After these new derived types have been created they will appear in the {\tt plotStuff} component
menus. 

\begin{figure}[htb]
  \begin{center}
   \epsfig{file=\figures /nlr2.ps,width=.7\linewidth}
  \caption{plotStuff can be used to plot more than one quantity on the same figure; here streamlines
     and a grid are plotted. This plot used an option in the grid-plotter to change the
     height of the grid. This computation was performed by the {\bf OverBlown} solver.}
  \end{center}
\end{figure}

\section{The plotStuff menu items}
\input plotStuffMenuInclude

\clearpage
\section{Derived quantities}\index{plotting derived quantities}

Various derived quantities can be plotted with {\tt plotStuff}.
The class {\tt DerivedFunctions}, described in more detail in the next section,
is used to define derived functions for plotting with {\tt plotStuff}. For example
one can compute the vorticity given some velocity components. One can also differentiate components.
There is also a way for users to define new derived quantities by editing the functions in the
{\tt userDefinedDerivedFunction.C} file and re-linking this with {\tt plotStuff}.

Here are some of the derived functions we can create
\begin{description}
  \item[x,y,z,xx,yy,zz,laplacian] : compute first and second derivatives of any components,
     \begin{align*}
    \mbox{x} &= \partial_x u \\
    \mbox{y} &= \partial_{y} u \\
    \mbox{z} &= \partial_{z} u \\
    \mbox{xx} &= \partial_{xx} u \\
    \mbox{yy} &= \partial_{yy} u \\
    \mbox{zz} &= \partial_{zz} u \\
    \mbox{laplacian} &= \Delta u = \grad\cdot\grad u = \partial^2_x u + \partial^2_y u + \partial^2_z u
     \end{align*}
  \item[vorticity] The same as zVorticity, see below. In 2D this is the only non-trivial component of the vorticity.
  \item[xVorticity, yVorticity, zVorticity] Components of the curl of the velocity, $\grad\times\uv$,
     \begin{align*}
         \mbox{xVorticity} &= \partial_z v - \partial_y w  \\
         \mbox{yVorticity} &= \partial_x w - \partial_z u  \\
         \mbox{zVorticity} &= \partial_y u - \partial_x v 
    \end{align*}
  \item[divergence] The dilatation or divergence of the velocity, $\grad\cdot\uv$,
    \[
         \mbox{divergence} = \partial_x u + \partial_y v + \partial_z w
    \]
  \item[mach Number] The ratio of the local magnitude of velocity to the local speed of sound:
      \begin{align*}
        \mbox{mach Number} & = { \sqrt{ u^2 + v^2 + w^2 } \over a } \\
         a  & = \sqrt{ \gamma R_g T } \qquad \mbox{if T is defined} \\
         a  & = \sqrt{ \gamma p / \rho } \qquad \mbox{if p and $\rho$ are defined} \\
      \end{align*}
%  \item[pressure]
  \item[temperature] The temperature as a function of the pressure, density and $\gamma$,
     \[
       \mbox{temperature}  = { \gamma p \over \rho }
     \]
  \item[speed] The magnitude of the velocity, $\sqrt{\uv\cdot\uv}$,
    \[
         \mbox{speed} = \sqrt{ u^2 + v^2 + w^2 }
    \]
  \item[schlieren] Use this quantity with the ``gray'' colour table to generate a {\it Schlieren} image.
     For hardcopy, save the image as ``gray scale'' (from ``output format menu'' in the ``file'' pull-down menu).
     The exponential in the formula below acts to enhance fine scale structure.
    \begin{align*}
         \mbox{schlieren} & = \alpha \exp( -\beta s(\xv) ) \qquad  ( \alpha = 1, ~\beta = 15) \\
          s(\xv) & = (\tilde{s} - \min \tilde{s})/( \max \tilde{s} - \min \tilde{s} ) \qquad (s \in [0,1]) \\
          \tilde{s}(\xv) & = \sqrt{ \rho_x^2 + \rho_y^2 + \rho_z^2 } 
    \end{align*}
    The quantities $\alpha$ and $\beta$ can be changed using the {\tt schlieren parameters} menu.
  \item[minimumScale, r1MinimumScale, r2MinimumScale, r3MinimumScale] Local measures of the minimum scale.
   Where the minimun scale is large, the flow is less well resolved. The quantities 
  r1MinimumScale, r2MinimumScale, and r3MinimumScale are meant to measure how well the flow is
  resolved along each coordinate direction.
   See the discussion in the OverBlown reference guide. I think the definition of r1MinimumScale etc.
   could be simplified to the second expression below (Anders?)
    \begin{align*}
  \mbox{minimumScale} & =  \| \Delta \xv \| \left[ \sum_{m,n} {\partial u_m \over \partial x_n} 
                    ~{1 \over \nu} \right]^{1/2} \\
  \mbox{r[i]MinimumScale} & =  
          \Delta r_i   \left[ \sum_n {\partial x_n \over \partial r_i} \right]^{1/4}
        \left[ \sum_{m,n} {\partial u_m \over \partial x_n}{\partial x_n \over \partial r_i} 
              ~{1 \over \nu} \right]^{1/2}  \\
                &=  \Delta r_i \left[ \sum_{m} {\partial u_m \over \partial r_i}  ~{1 \over \nu} \right]^{1/2} 
    \end{align*}
  \item[user defined] This option can be used to access any user defined derived functions; these
    are defined in the file {\tt userDefinedDerivedFunction.C}.

\end{description}

\vspace{\baselineskip}
The components $u$,$v$,$w$ etc. are determined by looking for the following variables in the show file,
\begin{description}
 \item[uComponent] : the component number of x-component of the velocity $u$.
 \item[vComponent] : the component number of y-component of the velocity $v$.
 \item[wComponent] : the component number of z-component of the velocity $w$.
 \item[pressureComponent] : the component number of the pressure $p$.
 \item[temperatureComponent] : the component number of the temperature $T$.
 \item[densityComponent] : the component number of the density $\rho$.
\end{description}
For example, {\bf OverBlown} saves the incompressible flow components with {\tt pressureComponent=0},
{\tt uComponent=1}, {\tt vComponent=2}, and {\tt wComponent=3}. See the {\bf OverBlown} file {\tt saveShow.C}
for an example of saving this extra info in the show file.

Other variables that may be required in the show file are
\begin{description}
 \item[gamma] : ration of specific heats (if constant).
 \item[Rg] : the gas constant.
 \item[nu] : kinematic viscosity.
\end{description}


\section{The DerivedFunctions class}\index{class DerivedFunctions}

Here is a description of the {\tt DerivedFunctions} class.

\input DerivedFunctionsInclude.tex

\vfill\eject

\bibliography{\homeHenshaw/papers/henshaw}
\bibliographystyle{siam}

\printindex
\end{document}
