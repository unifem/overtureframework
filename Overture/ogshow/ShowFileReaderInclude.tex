\subsubsection{constructor}
 
\newlength{\ShowFileReaderIncludeArgIndent}
\begin{flushleft} \textbf{%
\settowidth{\ShowFileReaderIncludeArgIndent}{ShowFileReader(}% 
ShowFileReader(const aString \& showFileName  = nullString)
}\end{flushleft}
\begin{description}
\item[{\bf Description:}] 
    Create an object for reading show files (generated by Ogshow) and
  optionally supply the name of the show file to open. A show file can
  also be opened with the {\tt open} member function.
\item[{\bf showFileName (input):}]  name of an existing show file (if specified).
\item[{\bf Author:}]  WDH
\end{description}
\subsubsection{close}
 
\begin{flushleft} \textbf{%
int  \\ 
\settowidth{\ShowFileReaderIncludeArgIndent}{close(}%
close()
}\end{flushleft}
\begin{description}
\item[{\bf Description:}] 
    Close an open show file.
\item[{\bf Author:}]  WDH
\end{description}
\subsubsection{getNumberOfFrames}
 
\begin{flushleft} \textbf{%
int  \\ 
\settowidth{\ShowFileReaderIncludeArgIndent}{getNumberOfFrames(}%
getNumberOfFrames() const
}\end{flushleft}
\begin{description}
\item[{\bf Description:}] 
    Returns the number of frames that exist in the show file for the furrent frame series. 
\item[{\bf Author:}]  WDH
\end{description}
\subsubsection{getNumberOfSolutions}
 
\begin{flushleft} \textbf{%
int  \\ 
\settowidth{\ShowFileReaderIncludeArgIndent}{getNumberOfSolutions(}%
getNumberOfSolutions() const
}\end{flushleft}
\begin{description}
\item[{\bf Description:}] 
    Returns the number of Solutions that exist in the show file for the furrent frame series. 
\item[{\bf Author:}]  WDH
\end{description}
\subsubsection{getNumberOfSequences}
 
\begin{flushleft} \textbf{%
int  \\ 
\settowidth{\ShowFileReaderIncludeArgIndent}{getNumberOfSequences(}%
getNumberOfSequences() const
}\end{flushleft}
\begin{description}
\item[{\bf Description:}] 
    Returns the number of Sequences that exist in the show file for the furrent frame series. 
\item[{\bf Author:}]  WDH
\end{description}
\subsubsection{setMaximumNumberOfOpenShowFiles}
 
\begin{flushleft} \textbf{%
void  \\ 
\settowidth{\ShowFileReaderIncludeArgIndent}{setMaximumNumberOfOpenShowFiles(}%
setMaximumNumberOfOpenShowFiles(const int maxNumber )
}\end{flushleft}
\begin{description}
\item[{\bf Description:}] 
 For very large files we may have to reduce the number of files that we allow to be open at any time
 
\end{description}
\subsubsection{getAGrid}
 
\begin{flushleft} \textbf{%
ReturnType  \\ 
\settowidth{\ShowFileReaderIncludeArgIndent}{getAGrid(}%
getAGrid(MappedGrid \& mg, \\ 
\hspace{\ShowFileReaderIncludeArgIndent}int \& solutionNumber, \\ 
\hspace{\ShowFileReaderIncludeArgIndent}int frameForGrid  =useDefaultLocation)
}\end{flushleft}

 
\begin{flushleft} \textbf{%
ReturnType  \\ 
\settowidth{\ShowFileReaderIncludeArgIndent}{getAGrid(}%
getAGrid(GridCollection \& cg, \\ 
\hspace{\ShowFileReaderIncludeArgIndent}int \& solutionNumber, \\ 
\hspace{\ShowFileReaderIncludeArgIndent}int frameForGrid  =useDefaultLocation)
}\end{flushleft}
\begin{description}
\item[{\bf Description:}] 
   Get grid GridCollection or CompositeGrid from a show file. If this a moving grid problem then return the grid
    corresponding to a give solutionNumber.
\item[{\bf cg (output):}]  The grid corresponding to the solution numbered {\tt solutionNumber}. This
   routine will always read in a new grid if a grid is found.
\item[{\bf solutionNumber (input):}]  For moving grid problems only.
   Find the grid corresponding to this solution number. 
   This number should be
   in the range [1,numberOfSolutions], where numberOfSolutions is the value by getNumberOfSolutions()
   If solutionNumber is out of range then the closest valid solution number is chosen, and this value
   is returned in solutionNumber.  Thus if you want to get the last grid in the file choose
   solutionNumber to be any integer that is larger than the number of solutions in the file.
\item[{\bf frameForGrid :}]  indicates where in the show file the grid for this solution can be found.
     useDefaultLocation : use default location (frame 1 for non-moving grids or the current frame
    for moving grids), useCurrentFrame : current frame, $>0$ : specify a frame number. 
\item[{\bf return values:}]  notFound or gridFound.
\item[{\bf Author:}]  WDH
\end{description}
\subsubsection{getHeaderComments}
 
\begin{flushleft} \textbf{%
const aString*  \\ 
\settowidth{\ShowFileReaderIncludeArgIndent}{getHeaderComments(}%
getHeaderComments(int \& numberOfHeaderComments0)
}\end{flushleft}
\begin{description}
\item[{\bf Description:}] 
   Get header comments for the last grid or solution that was found
\item[{\bf numberOfHeaderComments0 (output):}]  The number of comments in the array of Strings 
\item[{\bf return value:}]  An array of aString's with the comments that are associated
    with this solution. (You might use the declaration {\tt const aString *headerComment}).
\item[{\bf Author:}]  WDH
\end{description}
\subsubsection{getASolution}
 
\begin{flushleft} \textbf{%
ReturnType  \\ 
\settowidth{\ShowFileReaderIncludeArgIndent}{getASolution(}%
getASolution(int \& solutionNumber,\\ 
\hspace{\ShowFileReaderIncludeArgIndent}MappedGrid \& mg,\\ 
\hspace{\ShowFileReaderIncludeArgIndent}realMappedGridFunction \& u)
}\end{flushleft}

 
\begin{flushleft} \textbf{%
ReturnType  \\ 
\settowidth{\ShowFileReaderIncludeArgIndent}{getASolution(}%
getASolution(int \& solutionNumber,\\ 
\hspace{\ShowFileReaderIncludeArgIndent}GridCollection \& cg,\\ 
\hspace{\ShowFileReaderIncludeArgIndent}realGridCollectionFunction \& u)
}\end{flushleft}
\begin{description}
\item[{\bf Description:}] 
   Get grid (GridCollection or CompositeGrid) and a grid function (realGridCollectionFunction
   or realCompositeGridFunction) from a show file.
\item[{\bf solutionNumber (input/ouptut):}]  The number of the solution to get. This number should be
   in the range [1,numberOfSolutions], where numberOfSolutions is the value by getNumberOfSolutions()
   If solutionNumber is out of range then the closest valid solution number is chosen, and this value
   is returned in solutionNumber.  Thus if you want to get the last solution in the file choose
   solutionNumber to be any integer that is larger than the number of solutions in the file.
\item[{\bf cg (input/output):}]  The grid corresponding to the solution numbered {\tt solutionNumber}. 
    The grid cg will only be changed under certain circumstances. 
    The grid cg will be created or changed in the following cases:
    \begin{itemize}
      \item cg is a null grid on input.
      \item The show file contains moving grids and solutionNumber is not equal to the
            solutionNumber of the last time this routine was called.
    \end{itemize}
\item[{\bf u (output):}]  The grid function corresponding to the solution numbered {\tt solutionNumber}
\item[{\bf return values:}]  notFound or gridFound or solutionFound or solutionAndGridFound.
\item[{\bf Author:}]  WDH
\end{description}
\subsubsection{getFrame}
 
\begin{flushleft} \textbf{%
HDF\_DataBase*  \\ 
\settowidth{\ShowFileReaderIncludeArgIndent}{getFrame(}%
getFrame(int solutionNumber  = -1)
}\end{flushleft}
\begin{description}
\item[{\bf Description:}] 
     Return a pointer to the data base directory holding a frame; by default the current frame.
     You could use this pointer to get any additional data that has been saved in the frame.
     The current frame for all other calls is also set to the requested frame number.
      In the following example some extra data in the form of a realArray is retrieved.
    \begin{verbatim}
       ShowFileReader show(...);
       ...
       show.getASolution(...);
       realArray myData;
       show.getFrame()->get(myData,"my data");
       ...
    \end{verbatim}
\item[{\bf solutionNumber (input):}]  get the frame for this solution number. If no argument is specified then
    return the current frame. 
\item[{\bf Return value:}]  Return a pointer to the data base directory holding the frame, possibly NULL.
\item[{\bf Author:}]  WDH
\end{description}
\subsubsection{getSequenceNames}
 
\begin{flushleft} \textbf{%
int  \\ 
\settowidth{\ShowFileReaderIncludeArgIndent}{getSequenceNames(}%
getSequenceNames(aString *name, int maximumNumberOfNames)
}\end{flushleft}
\begin{description}
\item[{\bf Description:}] 
     Return the names of the sequences, up to a maximum of maximumNumberOfNames,
\item[{\bf Return value:}]  
\item[{\bf Author:}]  WDH
\end{description}
\subsubsection{getSequence}
 
\begin{flushleft} \textbf{%
int  \\ 
\settowidth{\ShowFileReaderIncludeArgIndent}{getSequence(}%
getSequence(int sequenceNumber,\\ 
\hspace{\ShowFileReaderIncludeArgIndent}aString \& name, RealArray \& time, RealArray \& value, \\ 
\hspace{\ShowFileReaderIncludeArgIndent}aString *componentName1, int maxComponentName1,\\ 
\hspace{\ShowFileReaderIncludeArgIndent}aString *componentName2, int maxComponentName2)
}\end{flushleft}
\begin{description}
\item[{\bf Description:}] 
     Return the data for a sequence.

\item[{\bf name (output) :}]  name of the sequence.
\item[{\bf time (output) :}]  time(0...n-1) - array of n 'time' values or other iteration variable.
\item[{\bf value (output) :}]  value(0...n-1,0..m-1) array of n values for each of m components.
\item[{\bf componentName1 (output) :}]  name1[0..m-1] name for the components.
\item[{\bf maxComponentName1 (input) :}]  maximum number of array elements in the array componentName1.
\item[{\bf componentName2 (output) :}]  names for a second level of components BUT DO NOT USE THIS for now.
\item[{\bf maxComponentName2 (input) :}]  maximum number of array elements in the array componentName2.

\item[{\bf Return value:}]  0 for success.
\item[{\bf Author:}]  WDH
\end{description}
\subsubsection{getGeneralParameters}
 
\begin{flushleft} \textbf{%
int  \\ 
\settowidth{\ShowFileReaderIncludeArgIndent}{getGeneralParameters(}%
getGeneralParameters( const int displayInfo  =1)
}\end{flushleft}
\begin{description}
\item[{\bf Description:}] 
    Get the list of parameters that go with this file.
\item[{\bf displayInfo (input) :}]  if not equal to zero output info about the parameters in the file.
\item[{\bf return value:}]  
\item[{\bf Author:}]  WDH
\end{description}
\subsubsection{getGeneralParameter(int)}
 
\begin{flushleft} \textbf{%
bool  \\ 
\settowidth{\ShowFileReaderIncludeArgIndent}{getGeneralParameter(}%
getGeneralParameter(const aString \& name, int \& ivalue,const  PlaceToSaveGeneralParameters placeToSave ) 
}\end{flushleft}
\begin{description}
\item[{\bf Description:}] 
    Get a general parameter with type `int'
\item[{\bf return value:}]  
\item[{\bf Author:}]  WDH
\end{description}
\subsubsection{getGeneralParameter(int)}
 
\begin{flushleft} \textbf{%
bool  \\ 
\settowidth{\ShowFileReaderIncludeArgIndent}{getGeneralParameter(}%
getGeneralParameter(const aString \& name, real \& rvalue,const   PlaceToSaveGeneralParameters placeToSave ) 
}\end{flushleft}
\begin{description}
\item[{\bf Description:}] 
    Get a general parameter with type `int'
\item[{\bf return value:}]  
\item[{\bf Author:}]  WDH
\end{description}
\subsubsection{getGeneralParameter(int)}
 
\begin{flushleft} \textbf{%
bool  \\ 
\settowidth{\ShowFileReaderIncludeArgIndent}{getGeneralParameter(}%
getGeneralParameter(const aString \& name, aString \& stringValue,const  PlaceToSaveGeneralParameters placeToSave ) 
}\end{flushleft}
\begin{description}
\item[{\bf Description:}] 
    Get a general parameter with type `int'
\item[{\bf return value:}]  
\item[{\bf Author:}]  WDH
\end{description}
\subsubsection{getGeneralParameter(int)}
 
\begin{flushleft} \textbf{%
bool  \\ 
\settowidth{\ShowFileReaderIncludeArgIndent}{getGeneralParameter(}%
getGeneralParameter(const aString \& name,  ParameterType type, int \& ivalue, real \& rvalue, \\ 
\hspace{\ShowFileReaderIncludeArgIndent}aString \& stringValue,const Ogshow::PlaceToSaveGeneralParameters placeToSave ) 
}\end{flushleft}
\begin{description}
\item[{\bf Description:}] 
    Get a general parameter with the given name and type
\item[{\bf name, type (input) :}]  get the parameter value for a parameter with this name and type.
\item[{\bf ivalue (output) :}]  return integer parameters in this variable (if type==intParameter)
\item[{\bf rvalue (output) :}]  return real parameters in this variable (if type==realParameter)
\item[{\bf stringValue (output) :}]  return string parameters in this variable (if type==stringParameter)
\item[{\bf return value:}]  
\item[{\bf Author:}]  WDH
\end{description}
\subsubsection{getListOfGeneralParameters}
 
\begin{flushleft} \textbf{%
ListOfShowFileParameters\&  \\ 
\settowidth{\ShowFileReaderIncludeArgIndent}{ getListOfGeneralParameters(}%
getListOfGeneralParameters( \\ 
\hspace{\ShowFileReaderIncludeArgIndent}const Ogshow::PlaceToSaveGeneralParameters placeToSave  =Ogshow::THECurrentFrameSeries)
}\end{flushleft}
\begin{description}
\item[{\bf Description:}] 
 Return the general parameters
 
\item[{\bf placeToSave (input):}]  define which list to use.
 
\item[{\bf return value:}]  The list of show file parameters
\item[{\bf Author:}]  WDH
\end{description}
\subsubsection{getParameters}
 
\begin{flushleft} \textbf{%
bool  \\ 
\settowidth{\ShowFileReaderIncludeArgIndent}{getParameters(}%
getParameters(const aString \& nameOfDirectory, ListOfShowFileParameters \& params )
}\end{flushleft}
\begin{description}
\item[{\bf Description:}] 
    Get parameters from a given directory
 
\item[{\bf nameOfDirectory (input) :}]  look for the parameters in this directory.
\item[{\bf params (input):}]  The parameters are returned here.
 
\item[{\bf return value:}]  
\item[{\bf Author:}]  WDH
\end{description}
\subsubsection{isAMovingGrid}
 
\begin{flushleft} \textbf{%
bool  \\ 
\settowidth{\ShowFileReaderIncludeArgIndent}{isAMovingGrid(}%
isAMovingGrid()
}\end{flushleft}
\begin{description}
\item[{\bf Description:}] 
   Return TRUE if this is a moving grid problem
\item[{\bf return value:}]  Return TRUE if this is a moving grid problem
\item[{\bf Author:}]  WDH
\end{description}
\subsubsection{open}
 
\begin{flushleft} \textbf{%
int  \\ 
\settowidth{\ShowFileReaderIncludeArgIndent}{open(}%
open(const aString \& showFileName, const int displayInfo  =1)
}\end{flushleft}
\begin{description}
\item[{\bf Description:}] 
    Open a show file that was generated by Ogshow.
\item[{\bf showFileName (input):}]  name of an existing show file.
\item[{\bf displayInfo (input) :}]  if not equal to zero output info about the parameters in the file.
\item[{\bf Author:}]  WDH
\end{description}
\subsubsection{getNumberOfFrameSeries}
 
\begin{flushleft} \textbf{%
int  \\ 
\settowidth{\ShowFileReaderIncludeArgIndent}{getNumberOfFrameSeries(}%
getNumberOfFrameSeries() const 
}\end{flushleft}
\begin{description}
\item[{\bf Description:}] 
     Return the number of frame series in the show file.

\item[{\bf Return value:}]  0 for success.
\item[{\bf Author:}]  KKC
\end{description}
\subsubsection{getFrameSeriesName}
 
\begin{flushleft} \textbf{%
aString  \\ 
\settowidth{\ShowFileReaderIncludeArgIndent}{getFrameSeriesName(}%
getFrameSeriesName( const  FrameSeriesID frame\_series )
}\end{flushleft}
\begin{description}
\item[{\bf Description:}] 
     Return the string name of a particular frame series

\item[{\bf frame\_series (input) :}]  integer specifying the frame series

\item[{\bf Return value:}]  the name of frame\_series on success, an null string ("") on failure
\item[{\bf Author:}]  WDH
\end{description}
